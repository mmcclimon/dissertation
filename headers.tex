%% headers.tex %%

% misc. TeX settings
% ------------------
\widowpenalty=10000
\clubpenalty=10000
\setlength{\parindent}{0pt}
% \setlength{\parskip}{6pt plus 2pt minus 1pt}
% \setlength{\emergencystretch}{3em}  % prevent overfull lines
% \setcounter{secnumdepth}{0}

% page layout
% -----------
\usepackage[margin=1in,headheight=.5in]{geometry}

\usepackage{setspace}
\doublespacing
\raggedright
\parindent=0.5in

% bibliography
% ------------
\usepackage[
  backend=biber,
  notes,
  parentracker=true
]{biblatex-chicago}
\addbibresource{diss.bib}
\addbibresource{discog.bib}

% footnotes
% ------------
\usepackage{sepfootnotes}
\newfootnotes{f}
\newcommand{\fn}{\fnote}
\newcommand{\fntext}{\fnotecontent}

\usepackage[ragged, bottom, splitrule]{footmisc}
\setlength{\skip\footins}{14pt}
\renewcommand{\footnotelayout}{\raggedright}

% fonts
% -----------
\usepackage{amssymb,amsmath}
\usepackage{fixltx2e} % provides \textsubscript

\usepackage{mathspec}
\usepackage{xltxtra,xunicode}
\defaultfontfeatures{Mapping=tex-text,Scale=MatchLowercase}

\setmainfont{Junicode}
\setmathfont(Digits){Junicode}
\setmathsfont(Latin){Junicode}
\setmathrm{Junicode}
\setmathsfont(Symbols){XITS Math}
\fontsize{12pt}{12pt}

\usepackage{subscript}    % for \textsubscript
\usepackage{paralist}

\usepackage[
  leftmargin = 0.5in,
  rightmargin = 0.5in,
  vskip = \partopsep
]{quoting}

% section headings
% ----------------
\usepackage{titlesec}
\titleformat{\chapter}[display]{%
  \normalfont\large\bfseries}
  {\chaptertitlename\ \thechapter}{0pt}{\vspace{-1ex}\Large}{}
\titlespacing*{\chapter}{0pt}{0pt}{1em}

\titleformat*{\section}{\large\bfseries}
\titleformat*{\subsection}{\normalsize\bfseries}

% page header
% -----------
\usepackage{fancyhdr}
\pagestyle{fancy}
% \setlength{\headheight}{14pt}

\lhead{}
\chead{}
\rhead{\thepage}
\lfoot{}
\cfoot{}
%\rfoot{\scriptsize{(final version)}}
\renewcommand{\headrulewidth}{0pt}

% hyperref
% --------
% (Provides PDF table of contents)
\usepackage[unicode,bookmarks,pdfencoding=auto,hyperfootnotes=false]{hyperref}
\urlstyle{same}

% examples/captions
% -----------------
\usepackage{newfloat}
\DeclareFloatingEnvironment[
  name={Example},
  placement=Htbp
]{example}
\newcommand{\exBeg}{\begin{example}}
\newcommand{\exEnd}{\end{example}}

\newcommand{\figBeg}{\begin{figure}}
\newcommand{\figEnd}{\end{figure}}

\usepackage[section]{placeins}

% provide opts to includegraphics as optional arg
\newcommand{\centerGraphic}[2][]{%
  \begin{center}%
    \includegraphics[#1]{#2}%
  \end{center}%
}

\usepackage[
  format = plain,
  labelsep = period,
  font=small,
  margin=0.5in,
  justification = raggedright,
  singlelinecheck = false,
]{caption}


\usepackage{marginnote}
\newcommand{\sidenote}[1]{\marginnote{\singlespacing\footnotesize#1}}

% misc. formatting
% ----------------
\usepackage{datetime}
\yyyymmdddate
\renewcommand{\dateseparator}{-}
\usepackage[chicago]{ellipsis}      % better spaced ellipses

% proposal-specific
% -----------------

% scale degrees /music stuff
\newcommand{\sd}[1]{\ensuremath{\hat{#1}}}
\renewcommand{\flat}{{\fontspec[Scale=MatchUppercase]{Times + Musical}f}}
\renewcommand{\sharp}{{\fontspec[Scale=MatchUppercase]{Times + Musical}s}}
\newcommand{\nat}{{\fontspec[Scale=MatchUppercase]{Times + Musical}n}}

% kerning on these characters sucks, so define some kern lengths
\newcommand{\fklen}{0.5pt}
\newcommand{\sklen}{0.4pt}
\newcommand{\flatkern}{\hspace{\fklen}}
\newcommand{\sharpkern}{\hspace{\sklen}}

\newcommand{\Bflat}{\mbox{B\flatkern\flat\sharpkern}}
\newcommand{\Eflat}{\mbox{E\flatkern\flat\sharpkern}}
\newcommand{\Aflat}{\mbox{A\flatkern\flat\sharpkern}}
\newcommand{\Dflat}{\mbox{D\flatkern\flat\sharpkern}}
\newcommand{\Gflat}{\mbox{G\flatkern\flat\sharpkern}}
\newcommand{\Cflat}{\mbox{C\flatkern\flat\sharpkern}}
\newcommand{\Fflat}{\mbox{F\flatkern\flat\sharpkern}}

% Define a new command, \h, that prints a harmony.
% It substitutes all 's' chars with sharps, and all 'b' chars with flats.
% Also adds some padding before the special chars and wraps them in an mbox{}
% so that they don't get split across lines
%
% Ex: \h{Fsm7b5} --> \mbox{F\hspace{1.25pt}\sharp{}m7\hspace{1.25pt}\flat{}5}
\usepackage{xstring}
\newcommand{\h}[1]{%
  \mbox{%
    \StrSubstitute{#1}{s}{\sharp{}}[\x]%
    \StrSubstitute{\x}{b}{\flat{}}[\x]%
    \expandarg%
    \StrSubstitute{\x}{\flat}{\hspace{\fklen}\flat}[\x]%
    \StrSubstitute{\x}{\sharp}{\hspace{\sklen}\sharp}[\x]%
    \x%
  }%
}

% text shortcuts
\newcommand{\tf}{\mbox{ii--V}}
\newcommand{\tfo}{\mbox{ii--V--I}}
\newcommand{\tsub}{\textsubscript}
\newcommand{\tsup}{\textsuperscript}


% math shortcuts
\newcommand{\lra}{\longrightarrow}
\newcommand{\TFarrow}{\ensuremath{\overset{\mathrm{TF}}{\lra}}}
\newcommand{\intZ}{\ensuremath{\mathbb{Z}}}

%% end headers.tex %%
