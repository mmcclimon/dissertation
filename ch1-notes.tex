\fntext{pja-1}{%
  This is not to say, of course, that there are not jazz compositions that do
  specify these minor details. These compositions are the exception, rather than
  the rule, in the music in which this study is interested. }

\fntext{pja-2}{%
  Miles Davis’s “So What” is probably the most well-known modal jazz piece; it
  is a 32-bar tune in which the first chord, \h{Dm}, lasts 16 bars, moves to
  \h{Ebm} for 8 bars, and back to \h{Dm} for the final 8.  The term also
  describes other similar pieces, including the rest of Davis’s album
  \emph{Kind of Blue}, John Coltrane’s recording of “My Favorite Things,” and
  Herbie Hancock’s “Maiden Voyage.”}

\fntext{pja-3}{%
  % I don’t like this note
  This is the tradition of the “contrafact,” to which I will return later.}

\fntext{pja-3a}{%
  Fake books are collections of lead sheets which were traditionally compiled
  anonymously and sold illegally, in order to avoid paying the copyright
  owners of the compositions they contained. The name “fake book” comes from
  the fact that with the melody and chord changes, jazz musicians can easily
  “fake” a tune they do not know. The most famous jazz fake book is ironically
  titled \emph{The Real Book}, and was compiled in Boston in the early 1970s.
  In recent years, fake books have become mainstream, and most of them have
  now obtained proper copyright permissions. Hal Leonard now publishes the 6th
  edition of \emph{The Real Book}\nocite{realbook} (a nod to the five illegal
  editions); many of the notorious errors in the earlier editions have been
  corrected and it is now available for purchase legally. Further references
  to \emph{The Real Book} in this document refer to this version unless
  otherwise noted. For a history of fake books, see \cite{kernfeld:2006}.}

\fntext{pja-3b}{%
  By the time they become standards, many non-jazz compositions (Tin Pan Alley
  songs or traditional songs like “Back Home Again in Indiana”) have been
  adapted, typically to additional harmonic motion to provide interest during
  solos. Some examples are more radical: the tune “Alice in Wonderland,” for
  example, is known to jazz musicians as a jazz waltz with one chord per bar.
  The main title music in the 1951 film from which it was taken, however, is
  in 4/4 with a relatively slower harmonic rhythm.}

\fntext{pja-3c}{%
  \cite[5--6]{martin:1996}. See also \cite[36--39]{heyer:2011}.}

\fntext{pja-4}{%
  The Thelonious Monk Institute of Jazz maintains a web page on the history of
  jazz education in America at
  \url{http://www.jazzinamerica.org/JazzResources/JazzEducation/Page}.}


\fntext{pja-5}{%
  A noted exception here is Andy Jaffe’s \headlesscite{jaffe:1996}, which
  features an extended bibliography that includes many theoretical works.}

\fntext{pja-6}{%
  See, for example, \cite{gjerdingen:2007}; and \cite{sanguinetti:2012}.}
