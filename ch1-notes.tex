% Chapter 1 notes
% vim:fdm=marker

% Problems of Jazz Analysis {{{
\fntext{pja-1}{%
  This is not to say, of course, that there are not jazz compositions that do
  specify these minor details. These compositions are the exception, rather than
  the rule, in the music in which this study is interested. }

\fntext{pja-2}{%
  Miles Davis’s “So What” is probably the most well-known modal jazz piece; it
  is a 32-bar tune in which the first chord, \h{Dm}, lasts 16 bars, moves to
  \h{Ebm} for 8 bars, and back to \h{Dm} for the final 8.  The term also
  describes other similar pieces, including the rest of Davis’s album
  \emph{Kind of Blue}, John Coltrane’s recording of “My Favorite Things,” and
  Herbie Hancock’s “Maiden Voyage.”}

\fntext{pja-3a}{%
  Fake books are collections of lead sheets which were traditionally compiled
  anonymously and sold illegally, in order to avoid paying the copyright
  owners of the compositions they contained. The name “fake book” comes from
  the fact that with the melody and chord changes, jazz musicians can easily
  “fake” a tune they do not know. The most famous jazz fake book is ironically
  titled \emph{The Real Book}, and was compiled in Boston in the early 1970s.
  In recent years, fake books have become mainstream, and most of them have
  now obtained proper copyright permissions. Hal Leonard now publishes the 6th
  edition of \emph{The Real Book}\nocite{realbook} (a nod to the five illegal
  editions); many of the notorious errors in the earlier editions have been
  corrected and it is now available for purchase legally. Further references
  to \emph{The Real Book} in this document refer to this version unless
  otherwise noted. For a history of fake books, see \cite{kernfeld:2006}.}

\fntext{pja-3b}{%
  By the time they become standards, many non-jazz compositions (Tin Pan Alley
  songs or traditional songs like “Back Home Again in Indiana”) have been
  adapted, typically to additional harmonic motion to provide interest during
  solos. Some examples are more radical: the tune “Alice in Wonderland,” for
  example, is known to jazz musicians as a jazz waltz with one chord per bar.
  The main title music in the 1951 film from which it was taken, however, is
  in 4/4 with a relatively slower harmonic rhythm.}

\fntext{pja-3c}{%
  \cite[5--6]{martin:1996}. See also \cite[36--39]{heyer:2011}.}

\fntext{pja-4}{%
  The Thelonious Monk Institute of Jazz maintains a web page on the history of
  jazz education in America at
  \url{http://www.jazzinamerica.org/JazzResources/JazzEducation/Page}.}


\fntext{pja-5}{%
  A noted exception here is Andy Jaffe’s \headlesscite{jaffe:1996}, which
  features an extended bibliography that includes many theoretical works.}

\fntext{pja-6}}}

% Theoretical Approaches {{{

\fntext{ta-1}{%
  \cite{larson:2009}. This book makes use of much of his earlier work,
  including \headlesscite{larson:1996}; \headlesscite{larson:1998}; and
  \headlesscite{larson:2005}.}

\fntext{ta-2}{%
  See, for example, \cite{arthurs:2011}; \cite{heyer:2012}; and
  \cite{mcfarland:2012}. Arthurs’s work on Mehldau is something of an outlier,
  for reasons explained below.
}

\fntext{ta-3}{%
  Throughout this document, the word “classical” used in this sense is used in
  its generic sense to stand for tonal music that participates in the Western
  art music tradition. This definition is imperfect, but is useful for
  distinguishing the music of Bach, Mozart, Beethoven, Schumann, and Wagner
  from that of Ellington, Basie, Parker, Monk, and Coltrane. I am not
  implying that jazz cannot participate in the Western art music; certainly
  at least some of it does.}

\fntext{ta-4}{%
  \cite[16--17]{martin:2011}. This work draws on earlier work of his own as
  well as that of others, including \cite{forte:1995}; and
  \cite{neumeyer:2009}.}

\fntext{ta-5}{%
  See \cite{mcgowan:2005}; and \headlesscite{mcgowan:2011}.}

\fntext{ta-6}{%
  \cite[76--79 and throughout]{mcgowan:2005}. The dialects are particularly
  clear in the final tonic chords, especially in the case of the characteristic
  tonic major-minor seventh of the blues.}

\fntext{ta-7}{%
  \Cite{mcgowan:2005}. On the distinction between “paleo-” and “neo-”
  Riemannian analysis, see \cite{rings:paleo}.}

\fntext{ta-8}{%
  \cite{martin:1980}; see also \headlesscite{martin:1988}.}

\fntext{ta-9}{%
  % appropriate?
  I have a suspicion that these earlier models of harmony tend reflect jazz
  musician’s intuitive understanding because they are not interested in
  legitimizing jazz analysis for the academy, as I suggest above. Both Strunk
  and Martin’s articles appear in the \emph{Journal of Jazz
  Studies}/\emph{Annual Review of Jazz Studies} (the journal was renamed in
  1981), while more recent articles on jazz harmony have appeared in
  theoretical journals: \emph{Music Theory Spectrum}, \emph{Journal of Music
  Theory}, and the \emph{Dutch Journal of Music Theory}, for example. As jazz
  research has moved from the fringes into the theoretical mainstream, I think
  it has grown increasingly abstract and further removed from jazz practice
  itself. As I mention above, one of the goals of the present study is to show
  how we might narrow this gap while still approaching the music with the
  necessary theoretical rigor.}



% }}}
