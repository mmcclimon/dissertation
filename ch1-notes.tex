\fntext{pja-1}{%
  This is not to say, of course, that there are not jazz compositions that do
  specify these minor details. These compositions are the exception, rather than
  the rule, in the music in which this study is interested. }

\fntext{pja-2}{%
  Miles Davis’s “So What” is probably the most well-known modal jazz piece; it
  is a 32-bar tune in which the first chord, \h{Dm}, lasts 16 bars, moves to
  \h{Ebm} for 8 bars, and back to \h{Dm} for the final 8.  The term also
  describes other similar pieces, including the rest of Davis’s album
  \emph{Kind of Blue}, John Coltrane’s recording of “My Favorite Things,” and
  Herbie Hancock’s “Maiden Voyage.”}

\fntext{pja-3}{%
  % I don’t like this note
  This is the tradition of the “contrafact,” to which I will return later.}

\fntext{pja-4}{%
  The Thelonious Monk Institute of Jazz maintains a web page on the history of
  jazz education in America at
  \url{http://www.jazzinamerica.org/JazzResources/JazzEducation/Page}.}


\fntext{pja-5}{%
  A noted exception here is Andy Jaffe’s \headlesscite{jaffe:1996}, which
  features an extended bibliography that includes many theoretical works.
}

\fntext{pja-6}{%
  See, for example, \cite{gjerdingen:2007}; and \cite{sanguinetti:2012}.
}
