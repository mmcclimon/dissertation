% Chapter 1 notes
% vim:fdm=marker

% Problems of Jazz Analysis {{{
\fntext{pja-1}{%
  This is not to say, of course, that there are not jazz compositions that do
  specify these minor details. These compositions are the exception, rather than
  the rule, in the music in which this study is interested. }

\fntext{pja-2}{%
  Miles Davis’s “So What” is probably the most well-known modal jazz piece; it
  is a 32-bar tune in which the first chord, \h{Dm}, lasts 16 bars, moves to
  \h{Ebm} for 8 bars, and back to \h{Dm} for the final 8.  The term also
  describes other similar pieces, including the rest of Davis’s album
  \emph{Kind of Blue}, John Coltrane’s recording of “My Favorite Things,” and
  Herbie Hancock’s “Maiden Voyage.”}

\fntext{pja-2a}{%
  As Scott DeVeaux puts it, bebop is “both the source of the present \ldots\
  and the prism through which we absorb the past. To understand jazz, one must
  understand bebop.” \headlesscite[3]{deveaux:1997}.}

\fntext{pja-2b}{%
  This is not meant to imply that “chromatic harmony” is not tonal; rather,
  studies that focus specifically on chromatic harmony often differentiate
  themselves from other tonal theoretical traditions.}

\fntext{pja-3a}{%
  Fake books are collections of lead sheets that traditionally were compiled
  anonymously and sold illegally, in order to avoid paying the copyright
  owners of the compositions they contained. The name “fake book” comes from
  the fact that with the melody and chord changes, jazz musicians can easily
  “fake” a tune they do not know. The most famous jazz fake book is ironically
  titled \emph{The Real Book}, and was compiled in Boston in the early 1970s.
  In recent years, fake books have become mainstream, and most of them have
  now obtained proper copyright permissions. Hal Leonard now publishes the 6th
  edition of \emph{The Real Book}\nocite{realbook} (a nod to the five illegal
  editions); many of the notorious errors in the earlier editions have been
  corrected and it is now available for purchase legally. Further references
  to \emph{The Real Book} in this document refer to this version unless
  otherwise noted. For a history of fake books, see \cite{kernfeld:2006}.}

\fntext{pja-3b}{%
  By the time they become standards, many non-jazz compositions (Tin Pan Alley
  songs or traditional songs like “Back Home Again in Indiana”) have been
  adapted, typically to additional harmonic motion to provide interest during
  solos. Some examples are more radical: the tune “Alice in Wonderland”
  (analyzed below), for example, is known to jazz musicians as a jazz waltz
  with one chord per bar.  The main title music in the 1951 film from which it
  was taken, however, is in 4/4 with a relatively slower harmonic rhythm.}

\fntext{pja-3c}{%
  \cite[5--6]{martin:1996}. See also \cite[36--39]{heyer:2011}.}

\fntext{pja-3d}{%
  The “head” is what jazz musicians call the statements of the melody in the
  course of a jazz performance. This melody is typically played once at the
  beginning of a performance and again at the end (often referred to as the
  “out head”).}

\fntext{pja-4a}{%
  In places where I refer to a “tune” generically, I have provided at
  least two references to relatively straight-ahead recorded examples in the
  appendix.}

\fntext{pja-4}{%
  The Thelonious Monk Institute of Jazz maintains a web page on the history of
  jazz education in America at
  \url{http://www.jazzinamerica.org/JazzResources/JazzEducation/Page}.}

\fntext{pja-5}{%
  A noted exception here is Andy Jaffe’s \headlesscite{jaffe:1996}, which
  features an extended bibliography that includes many theoretical works.}

\fntext{pja-5a}{%
  This issue of \emph{Music Theory Online} is a Festschrift in memory of Steve
  Larson (September 2012), guest edited by Stephen Rodgers, Henry Martin, and
  Keith Waters.}

\fntext{pja-6}}}

% Theoretical Approaches {{{

\fntext{ta-1}{%
  \cite{larson:2009}. This book makes use of much of his earlier work,
  including \headlesscite{larson:1996}; \headlesscite{larson:1998}; and
  \headlesscite{larson:2005}.}

\fntext{ta-2}{%
  \cite{strunk:1996}. See also, for example, \cite{arthurs:2011};
  \cite{heyer:2012}; and \cite{mcfarland:2012}. Arthurs’s work on Mehldau is
  something of an outlier, for reasons explained further in note
  \ref{fn:mehldau}.}

\fntext{ta-3}{%
  Throughout this document, the word “classical” used in this sense is used in
  its generic sense to stand for tonal music that participates in the Western
  art music tradition. This definition is imperfect, but is useful for
  distinguishing the music of Bach, Mozart, Beethoven, Schumann, and Wagner
  from that of Ellington, Basie, Parker, Monk, and Coltrane. I do not mean to
  imply that jazz cannot participate in the Western art music tradition; certainly
  at least some of it does.}

\fntext{ta-4}{%
  \cite[16--17]{martin:2011}. This work draws on earlier work of his own as
  well as that of others, including \cite{forte:1995}; and
  \cite{neumeyer:2009}.}

\fntext{ta-5}{%
  See \cite{mcgowan:2005}; and \headlesscite{mcgowan:2011}.}

\fntext{ta-6}{%
  \cite[76--79 and throughout]{mcgowan:2005}. The dialects are particularly
  clear in final tonic chords, especially in the case of the characteristic
  tonic major-minor seventh of the blues.}

\fntext{ta-7}{%
  \Cite{mcgowan:2005}. On the distinction between “paleo-” and “neo-”
  Riemannian analysis, see \cite{rings:paleo}.}

\fntext{ta-8}{%
  \cite{martin:1980}; see also his \headlesscite{martin:1988}.}

\fntext{ta-9}{%
  % appropriate?
  I have a suspicion that these earlier models of harmony tend reflect jazz
  musician’s intuitive understanding because they are not interested in
  legitimizing jazz analysis for the academy, as I suggest above. Both Strunk
  and Martin’s articles appear in the \emph{Journal of Jazz
  Studies}/\emph{Annual Review of Jazz Studies} (the journal was renamed in
  1981), while more recent articles on jazz harmony have appeared in
  music theory journals: \emph{Music Theory Spectrum}, \emph{Journal of Music
  Theory}, and the \emph{Dutch Journal of Music Theory}, for example. As jazz
  research has moved from the fringes into the theoretical mainstream, I think
  it has grown more removed from jazz practice
  itself. As I mention above, one of the goals of the present study is to show
  how we might narrow this gap while still approaching the music with the
  necessary theoretical rigor.}

\fntext{ta-10}{%
  This article appears with only slight changes as the second chapter of
  Larson’s book, \headlesscite{larson:2009}.  Garrett Michaelsen critiques
  Schenkerian analysis in similar ways in
  \headlesscite[7--11]{michaelsen:2013}.}

\fntext{ta-11}{%
  In the opening of \emph{Free Composition}, Schenker refers to improvisation
  as “the ability in which all creativity begins”; \headlesscite{schenker:fc}.
  Schenker’s first \emph{Erlauterungsausgabe} (explanatory edition) was indeed
  of an improvisatory work, Bach’s \emph{Chromatic Fantasy and Fugue}.}

\fntext{ta-12}{%
  The interview was recorded on November 6, 1978. The program is available
  online at
  \url{http://www.npr.org/2010/10/08/92185496/bill-evans-on-piano-jazz}, and
  was released on CD under Evans’s name as well: \emph{Marian McPartland’s
  “Piano Jazz” Radio Broadcast} (The Jazz Alliance TJA-12038-2, 1993).}

\fntext{ta-13}{%
  \label{fn:mehldau}Brad Mehldau, who is the focus of Daniel Arthurs’s work,
  is quite similar to Evans, in that he is a classically-trained white pianist
  with a strong acknowledged influence of the European classical tradition.}

\fntext{ta-14}{%
  Bill Evans made his first recordings as a leader in 1958, but he was not
  widely known until his appearance on Miles Davis’s \emph{Kind of Blue} in
  1959.  Shortly after he left Davis, he won wide acclaim with his trio with
  Scott LaFaro and Paul Motian, whose groundbreaking first album was
  \emph{Portrait in Jazz}, released in 1960.}

\fntext{ta-15}{%
  In particular, value systems that do not stem from the European classical
  tradition. Michaelsen, drawing on George Lewis’s distinction between
  “Afrological” and “Eurological” modes of improvising, suggests that
  Schenkerian analysis is a particularly Eurological method of analysis; his
  own theory is designed to address interaction in improvisation, which is a
  more Afrological value. See \headlesscite[4--5, 11--12]{michaelsen:2013}.
  Lewis first introduces the terms in \headlesscite{lewis:1996}.}

\fntext{ta-16}{%
  \cite[241]{larson:1998}. Though there are many published analyses of “Giant
  Steps,” I am unaware of any that use Schenkerian methods.}

% }}}

% Transformational Theory {{{

\fntext{tt-1}{%
  One of the main subjects of Ring’s book is a \gis{} that describes “heard scale
  degrees,” and does not include harmony at all; see \headlesscite[41--99, and
  throughout]{rings:2011}. Lewin provides several examples of non-harmonic
  \gis{}es in \headlesscite[16--24]{lewin:gmit}.}

\fntext{tt-2}{%
  As evidenced in part by the number of books published in the last several
  years; in addition to Rings, there is \cite{cohn:2012}; and
  \cite{tymoczko:2011}.}

\fntext{tt-3}{%
  \Cite{gollin:1998}. Because the terms are in common use in the
  transformational literature, “Tonnetz” and the plural “Tonnetze” are rendered
  without italics throughout this study.}

\fntext{tt-4}{%
  Gollin refers to the differences between the two-dimensional triadic
  \emph{Tonnetz} and his three-dimensional tetrachordal version as
  “degeneracies” (\headlesscite[200]{gollin:1998}). Child’s cubic
  representation only shows 8 of the possible 24 (0258) tetrachords: those
  related by parsimonious voice leading to a single diminished seventh chord
  (\headlesscite[188]{childs:1998}).}

\fntext{tt-5}{%
  \headlesscite{hook:2007}. Lewin’s definition of a transformational network is in
  \headlesscite[193--97]{lewin:gmit}.}

\fntext{tt-6}{%
  An explanation of this four-dimensional space is found in
  \cite[93--112]{tymoczko:2011}; much of this work is based on
  \cite{tymoczko:2006}; and \cite{cqt:2008}. Hook points out Tymoczko’s
  relationship to transformational theory in his review of Tymoczko’s book,
  \textparagraph13--14.\nocite{hook:2011}%
}

\fntext{tt-7}{%
  In particular, it will be useful to consider the II\h{m7b5} chord (a
  half-diminished seventh) as a different type than the \h{V7} chord (a dominant
  seventh) given their different functional roles in jazz harmony, despite the
  fact that they are of the same set class.}

% }}}

% Lead sheet notation {{{

\fntext{ls-1}{%
  It is worth mentioning that not all jazz musicians read music; jazz is
  largely an aural tradition, and many early jazz musicians did not read,
  instead learning the music by ear. For students learning jazz today,
  however, learning to read chord changes is an essential part of their
  training. For an ethnomusicological survey of jazz musicians' relationships
  with lead sheets, see \cite[71--76]{berliner:1994}.}

\fntext{ls-2}{%
  This lead sheet is taken from the older, illegal \emph{Real Book} (249).
  While the newer Hal Leonard edition maintains most of the original selections,
  in some cases they do not: ``Just Friends'' does not appear until vol.~4 of
  the Hal Leonard \emph{Real Book}.}

\fntext{ls-3}{%
  Jamey Aebersold publishes a free book on his website that serves as an
  introduction to the very basics of jazz; the section titled ``Nomenclature''
  is especially useful for deciphering chord symbols.
  \headlesscite[15]{aebersold:handbook}.}

\fntext{ls-4}{%
  \cite[21]{levine:piano}. Realizing chord symbols is perhaps most important
  for pianists and guitarists (who are most often charged with realizing them in
  performance), and any introductory text on these instruments will be filled
  with voicings to be used in different situations.}

% }}}

% Diatonic Chord Space {{{
  % Intervals and transformations{{{{
\fntext{ds-2a}{%
  \emph{The Real Book} gives these changes in E minor, but most recorded
  performances are in G minor. I have transposed the given changes to reflect
  the most common performance key. Steven Strunk analyzes both “Autumn Leaves”
  and “How My Heart Sings” (analyzed below) as examples of 10--7 linear
  intervallic patterns in \headlesscite[96--97]{strunk:1996}.}

\fntext{ds-2b}{%
  Emile De Cosmo’s etude book titled simply \emph{The Diatonic Cycle} (North
  Bergen, N.J.: EDC Publications, 1970)\nocite{decosmo:1970} gives many
  different solo patterns possible over the diatonic cycle in all twelve major
  and minor keys.}

\fntext{ds-2}{%
  In running text, dominant seventh chords are given with just a “\h{7},” major
  sevenths with “\h{M7},” minor sevenths with “\h{m7},” and half-diminished sevenths
  with “\h{m7b5}”. In a minor key the tonic chord is often played with a
  major seventh, which is indicated “\h{mM7}.” In examples, chord symbols
  typically follow conventions used by \emph{The Real Book}.}

\fntext{ds-2c}{%
  \Cite[Definition 2.3.1 (26)]{lewin:gmit}. \emph{GMIT} contains many terms
  that Lewin renders in all capitals; I have rendered them here in small
  capitals (except when quoting directly) in order to reduce their typographical
  impact.}

\fntext{ds-3}{%
  The tonic harmony is given as it is in \emph{The Real Book}, simply as Gm.
  In performance, a musician might choose to play this chord with a major
  seventh (\h{GmM7}), a sixth (\h{Gm6}), or even a minor seventh (\h{Gm7}; this
  is less likely since minor seventh chords are most commonly ii chords, not
  tonics).}

\fntext{ds-3b}{%
  In chromatic pitch space, we might say that the interval between C4 and A3
  is $-3$, while the interval between C4 and A5 would be $+21$. In chromatic
  pitch class space, however, the interval is calculated mod-12 (because
  octaves are equivalent), and both of these intervals are equal to 9.}

\fntext{ds-4}{%
  Lewin’s definition of a mathematical group is spread over several pages in
  the first chapter of \emph{GMIT} (4--6). For other definitions in works of
  music theory, see \cite[12--13]{rings:2011}, and \cite[Ch. 5]{hook:mst}. Any
  introductory mathematical text on group theory will contain the group
  axioms; see, for example, \cite[10--14]{grossman:1964} and
  \cite[51]{carter:2009}.}

\fntext{ds-5}{%
  The group $\mathcal{C}_7$ can be generated by any of its members, since 7 is
  prime. In general, a cyclic group can be generated by a member only if the two
  are relatively prime. The group $\mathcal{C}_{12}$, for example, can be
  generated only by 1, 5, 7, or 11; put another way, only the chromatic scale
  and circle of fifths (or fourths) cycle through all 12 pitches in the
  chromatic octave before returning to the starting point.}

\fntext{ds-5a}{%
  To avoid confusion, we will use only the \gis{} that measures distance in
  diatonic steps from this point on.}

\fntext{ds-6}{%
  \cite[159]{lewin:gmit}. Henry Klumpenhouwer argues that the dichotomy
  between Cartesian \gis{} thinking and what he calls “anti-Cartesian”
  transformational thinking is the central theme of \emph{GMIT}; see
  \headlesscite{klumpenhouwer:2006}. For more on the “transformational
  attitude,” see \cite[24--29]{rings:2011}; \cite{satyendra:2004}; and
  \cite[172--77]{hook:2007gmit}.}

\fntext{ds-7}{%
  Ramon Satyendra describes the two attitudes as being noun-oriented (\gis{}es)
  vs. verb-oriented (transformations); \headlesscite[102--3]{satyendra:2004}.}

\fntext{ds-8}{%
  \cite[160]{lewin:gmit}, emphasis original. Julian Hook clarifies this point
  in \headlesscite[172--77]{hook:2007gmit}.}

\fntext{ds-9}{%
  The operator $t_k$ is described in as a “generic transposition operator” in
  \cite{hook:2014}. I refer to it here as “diatonic” rather than “generic”
  since I am interested at this point only in its application to the G-minor
  diatonic set S.}

\fntext{ds-10}{%
  The intervals chosen for measurement in the \gis{} do have an impact on the
  associated transformational system: if we had used the descending fifths
  generation as described above, the the operator $t_1$ (not $t_3$) would map
  \h{Am7b5} onto \h{D7}, since the interval in the underlying \gis{} would be
  different.}

  % }}}}

% Analytical Applications {{{{
\fntext{ds-11a}{%
  Lewin’s definition of a transformation network appears is at
  \headlesscite[196]{lewin:gmit}. We will delay an in-depth explanation until
  the next chapter, at which point it will be more relevant.}

\fntext{ds-11}{%
  This observation has not gone unrecognized by jazz musicians: at least one
  recording (Vince Guaraldi, on the album \emph{A Flower is a Lovesome
  Thing}\nocite{guaraldi:flower}) includes the \h{VImaj7} chord in this
  measure, though most do not.}

\fntext{ds-12}{%
  Hearing this parallelism also requires hearing the \h{Gm7} in the third bar
  of the final A section as equivalent to the cadential \h{Gm} in the first.
  The chord symbol \h{Gm} is ambiguous by nature; if a performer wanted to
  bring out this parallelism, she could perhaps play the tonic chords in the
  first A section as minor seventh chords.}

\fntext{ds-13}{%
  The most well-known recording of “Alice in Wonderland” is probably on Bill
  Evans’s \emph{Portrait in Jazz}.\nocite{evans:portrait} These changes are as
  given in \emph{The Real Book}, and as played by Evans. This figure gives the
  changes to the second ending of the first sixteen measures (the first ending
  contains a \tf\ in D minor to return to the opening).}

\fntext{ds-13a}{%
  The interval functions are identical, with the caveat that diatonic steps
  are to be counted in the A-minor collection, not the G-minor.}

\fntext{ds-14}{%
  The \h{Eb7} chord is a tritone substitute for \h{A7}, which is of course the
  dominant of the following \h{Dm} chord. This harmonic motion is not easily
  understood in either the A-minor or C-major diatonic spaces; we will return
  to tritone substitutes in the next chapter.}

\fntext{ds-15}{%
  It is this \h{Am7} that confirms we are in a C-major diatonic space; it
  would be more typical to transpose the \tf\ progression exactly to
  \h{Em7}--\h{A7}.}

\fntext{ds-16}{%
  Steven Rings refers to Lewin’s “intuitions” as “apperceptions”; see
  \headlesscite[17--21]{rings:2011}.}

\fntext{ds-17}{%
  The canonical recording of this tune is again by Bill Evans, on the album
  \emph{How My Heart Sings!}\nocite{evans:heartsings} Evans seems to have a
  propensity towards tunes with diatonic cycles: \emph{Portrait in Jazz} also
  contains a very well-known recording of “Autumn Leaves.”}

\fntext{ds-18}{%
  Exactly which chord functions as the pivot is of little importance, so long
  as it happens before the E dominant seventh. I am inclined to hear the
  \h{FM7} as a pivot (functioning simultaneously as IV and VI), in order to
  keep both \tfo\ progressions (in C major and A minor) intact.}

\fntext{ds-18a}{%
  These changes are again taken from \emph{The Real Book}. The \h{Am7b5} chord
  in the sixth bar of the A$^\prime$ section is not included in some charts of
  this tune, so I have put it in parentheses here. (The older, illegal
  \emph{Real Book} as well as another illegal fake book called simply \emph{The
  Book} both omit this chord.)}

\fntext{ds-19}{%
  Again, this implicit diatonicism has not gone unnoticed; Ahmad Jamal arrives
  emphatically on \h{CM7} in the fourth bar of the bridge on \emph{Jamal at
  the Pershing, Vol.~2}.\nocite{jamal:pershing}}

\fntext{ds-20}{%
  If the linking dominant did not appear at the end of the bridge, we would see
  the succession \h{EM7}--\h{Fm7}. This succession is remarkably similar to
  Lewin’s \textsc{slide} operation: retaining the third of the chord while
  moving the root and fifth by half-step; see \emph{GMIT},
  178.\nocite{lewin:gmit} Here, the seventh is also retained as a common tone;
  we will return to this operation (which we will call \slideS) in the next
  chapter.}

\fntext{ds-21}{%
  Henry Martin analyzes the A section simply as a series of descending fifths
  (taking note of the aberrant root tritone \Dflat--G) ending on C major.
  His analysis assumes a chromatic space, in which the harmonic progressions
  are more difficult to make sense of: he notes that “a certain tonal
  ambiguity pervades this piece” (\headlesscite[15--19]{martin:1988}.).}

\fntext{ds-22}{%
  Hearing the third-relations does have the nice side effect, absent from our
  analysis here, that all of the perceived tonics in the first sixteen bars
  (\Aflat, C, \Eflat, G) spell out the tonic major-seventh chord;
  \cite[19]{martin:1988} makes this observation.}

  % }}}}

% }}}

%%% Local Variables: %%%
%%% mode: LaTeX %%%
%%% TeX-master: "diss" %%%
%%% End: %%%
