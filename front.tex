% Front matter
\documentclass[diss]{subfiles}

\begin{document}
\title{A Transformational Approach to Jazz Harmony}
\author{Michael McClimon}
\date{}
\makeatletter \let\Title\@title \makeatother

\pdfbookmark[0]{Front Matter}{fm:frontmatter}

% Title page {{{
\begin{center}
  \singlespacing
  \thispagestyle{empty}

  \vspace*{\fill}
  \keepthetitle
  \MakeUppercase{\Title}

  \vspace{\baselineskip}
  Michael McClimon

  \vspace{\baselineskip}
  \vspace{\baselineskip}
  \vspace{\baselineskip}
  \vspace{\baselineskip}

  Submitted to the faculty of the University Graduate School \\
  in partial fulfillment of the requirements for the degree \\
  Doctor of Philosophy \\
  in the Jacobs School of Music, \\
  Indiana University \\
  month year

  \vspace{\fill}
\end{center}
\clearpage
% }}}

% Acceptance page {{{
\singlespacing
\begin{center}
  Accepted by the Graduate Faculty, Indiana University, in partial fulfillment
  of the requirements for the degree of Doctor of Philosophy.
\end{center}

{\flushleft
  \vspace{\baselineskip}
  Doctoral Committee
}

{\flushright

\vspace{5em}
\rule{20em}{0.5pt} \\ Julian Hook, Ph.D.

\vspace{5em}
\rule{20em}{0.5pt} \\ Blair Johnston, Ph.D.

\vspace{5em}
\rule{20em}{0.5pt} \\ Marianne Kielian-Gilbert, Ph.D.

\vspace{5em}
\rule{20em}{0.5pt} \\ Brent Wallarab

}


{\flushleft
defense date
}

\clearpage
% }}}

% Copyright page {{{
\begin{center}
  \vspace*{\fill}
  Copyright \copyright\ year \\
  Michael McClimon
  \vspace*{\fill}
\end{center}
\clearpage
% }}}

% Acknowledgements {{{

\phantomsection
\section*{Acknowledgements}
\addcontentsline{toc}{section}{Acknowledgements}

\clearpage

% }}}

% Abstract {{{
\phantomsection
\addcontentsline{toc}{section}{Abstract}
\begin{center}
  Michael McClimon

  \vspace{\baselineskip}
  \MakeUppercase{\Title}
\end{center}

\doublespacing
(abstract goes here)

{\flushright

\vspace{4em}
\rule{20em}{0.5pt} \\ Julian Hook, Ph.D.

\vspace{4em}
\rule{20em}{0.5pt} \\ Blair Johnston, Ph.D.

\vspace{4em}
\rule{20em}{0.5pt} \\ Marianne Kielian-Gilbert, Ph.D.

\vspace{4em}
\rule{20em}{0.5pt} \\ Brent Wallarab

}
% }}}

\clearpage
\phantomsection
\addcontentsline{toc}{section}{Table of Contents}
\singlespacing
\tableofcontents

\clearpage
\phantomsection
\addcontentsline{toc}{section}{List of Examples}
\listofexamples

\clearpage
\phantomsection
\addcontentsline{toc}{section}{List of Figures}
\listoffigures

\clearpage

% Notes to the Reader {{{
\phantomsection
\section*{Notes to the Reader}
\addcontentsline{toc}{section}{Notes to the Reader}

\doublespacing
References to recordings in this dissertation are generally made only by
giving the performer’s name and album title; complete identifying information
can be found in the discography.

When discussing a particular piece of jazz, the words “piece,” “composition,”
and “work” all seem out of place. In general, I have adopted the word “tune”
to mean roughly “the basic structure of a work, including (primarily) its
melody and chord changes.” This is in keeping with the way jazz musicians use
the word: they may refer to a “16-bar tune,' a “rhythm changes tune,” or “one
of my favorite tunes” (all referring to the abstract structure of the tune and
not simply the melody). When I am referencing a \emph{particular}
instantiation of a work (e.g.~Bill Evans’s recording of “Autumn Leaves” from
\emph{Portrait in Jazz}), I will make that clear.

% }}}



\end{document}

% vim: fdm=marker
