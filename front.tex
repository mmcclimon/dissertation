% Front matter
\documentclass[diss]{subfiles}

\begin{document}
\title{A Transformational Approach to Jazz Harmony}
\author{Michael McClimon}
\date{}
\maketitle

\singlespacing

\tableofcontents
\listoffigures
\clearpage

\section{Notes to the Reader}

\doublespacing
References to recordings in this dissertation are generally made only by
giving the performer’s name and album title; complete identifying information
can be found in the discography.

When discussing a particular piece of jazz, the words “piece,” “composition,”
and “work” all seem out of place. In general, I have adopted the word “tune”
to mean roughly “the basic structure of a work, including (primarily) its
melody and chord changes.” This is in keeping with the way jazz musicians use
the word: they may refer to a “16-bar tune,' a “rhythm changes tune,” or “one
of my favorite tunes” (all referring to the abstract structure of the tune and
not simply the melody). When I am referencing a \emph{particular}
instantiation of a work (e.g.~Bill Evans’s recording of “Autumn Leaves” from
\emph{Portrait in Jazz}), I will make that clear.



\end{document}
