% Chapter 2 notes
% vim:fdm=marker

% ii-V space {{{

\fntext{tf-1}{%
  \cite{aebersold:1974}. The Aebersold play-along series is a staple of jazz
  pedagogues; most contain a selection of tunes, along with a CD of a rhythm
  section so that students can practice with a recording. The \tf\ volume is
  number three of well over 100, and includes the phrase “the most important
  musical sequence in jazz!” on the cover.}

\fntext{tf-2}{%
  Despite our analysis in the previous chapter of “All the
  Things You Are” as taking place in a diatonic space, the ubiquity of \tfo\
  progressions means that many jazz musicians are apt to hear strings of \tf{}s,
  even when there is a clear diatonic pattern. Though some jazz musicians
  describe “Autumn Leaves” as a diatonic cycle, many may also describe it as a
  \tfo\ in \Bflat\ followed by a \tfo\ in the relative minor, connected by a
  major seventh chord.}

\fntext{tf-2a}{%
  The triangle on the C chord in this figure indicates a major seventh. We use
  the triangle rather than “\h{maj7}” or “\h{M7}” to save space and reduce
  clutter in the graphical representations.}

\fntext{tf-3}{%
  Lewin’s definition of a transformation network is in
  \headlesscite[196]{lewin:gmit}. For a relatively concise summary, see
  \cite[110--16]{rings:2011}.}

\fntext{tf-4}{%
  \headlesscite[Definition 1.3.1 (3)]{lewin:gmit}.}

\fntext{tf-5}{%
  \headlesscite[195--96 and throughout]{lewin:gmit}. See also
  \cite[6--8]{hook:2007}.}

\fntext{tf-6}{%
  Here and throughout this chapter, pitch classes are represented as mod-12
  integers, with C $= 0$; all calculations are performed mod-12.}

\fntext{tf-6a}{%
  This figure represents what Joseph Straus calls “transformational voice
  leadings” in his study of atonal voice leading; \headlesscite{straus:2003}.}

\fntext{tf-7}{%
  Hook himself makes this clear, noting that for any two sets S and
  T it is possible to define a single-type transformation in the set
  S $\cup$ T. He also notes that even when a single-type
  transformation is possible, ``the cross-type approach is often
  simpler and more natural,'' which certainly seems to be the case
  here. \headlesscite[5n8]{hook:2007}.}


% }}}
