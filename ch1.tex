% Chapter 1
% vim:fdm=marker

\chapter{Introduction}

\section{Problems of Jazz Analysis}
\label{sec:problems-jazz-analysis}
% {{{

When compared to a score by Beethoven (for example), the jazz lead sheet appears
strikingly bare. The Beethoven score specifies nearly everything a performer
needs to know in order to perform it. Though the minor details (dynamics,
articulations, phrasing marks, and the like) will differ from piece to piece,
we can usually depend on the presence of some basic information. It is rare for
traditional scores not to include the instrumentation, for example, and a score
that did not include information like the number of measures or which notes to
play in combination with which which other notes would be very unusual indeed.

And yet, this is the usual state affairs for the jazz lead sheet, which is
probably the most common form of a “jazz score.”\fn{pja-1} Most lead sheets only
include the basic outline of a melody along with a set of “changes” that
prescribe the harmonic structure of a piece. Beyond this most basic instruction,
every other aspect is left up to the performers. Given that most small-group
jazz musicians approach the music first with these lead sheets, it seems
appropriate to focus our analytical attention on harmony, the more salient of
those two elements.

Jazz is essentially a harmonic music. In a typical jazz performance, the melody
of the piece is only heard twice (at the beginning and the end), while the
harmonic structure is heard throughout, determining the structure of the
performance. Each soloist typically plays one or more “choruses,” where each
chorus is understood as a single iteration of the pieces harmonic structure.
In marked contrast to a Beethoven score, jazz compositions usually remain
unspecified when it comes to their contrapuntal structure: performers will
typically improvise counterpoint that fits with the underlying harmonic
framework. Harmony is the main restraining factor of a piece, and its primary
method of coherence.

The word “jazz”---which has been used at various times to describe McKinney’s
Cotton Pickers, Benny Goodman, Sun Ra, John Zorn, Tito Puente, and Brad
Mehldau, among many others---is inescapably vague, so it will be useful at
this point to delimit the terms of this study somewhat. Here I am interested
in in what might be called “tonal jazz,” which begins in the swing area and
continues through hard bop, covering roughly the years 1935--1965. In this
music, functional harmonic progressions are still the norm; “tonal jazz” is
meant in opposition to “modal jazz,” where the rate of harmonic change is
slower and the harmony is mostly non-functional.\fn{pja-2} This includes much
of the music that most people think of when they hear the word “jazz,”
including big-band swing (Count Basie, much of Duke Ellington’s music), bebop
(Charlie Parker, Dizzy Gillespie, Thelonious Monk), and the mainstream jazz
that followed bebop, known variously as “hard bop” or “post-bop” (John
Coltrane, Sonny Rollins, Bill Evans, and many others). I intend the dates to
be flexible, especially on the later end; given the strong influence of the
bebop tradition on jazz and jazz pedagogy, the hard bop style continued to
exist well beyond 1965, and many players today still play in the style.

Now that we have dilineated “jazz,” we should explain exactly what we mean by
“harmony.” Harmony is of course one of the oldest topics in music theory, and
as such has been hotly contested throughout its history. It is often
found in opposition to couterpoint; in this view, counterpoint is concerned
with individual melodic voices, while harmony is concerned with individual
verticalities. In other traditions (most notably the Schenkerian tradition),
harmony is understood to be an outgrowth of counterpoint: verticalities arise
primarily through contrapuntal procedures. Furthermore, study of harmony is
often broken down by genre: “tonal harmony” plays a different role than does
“chromatic harmony” in both theoretical research and pedagogy.

When jazz musicians refer to “harmony,” they are typically referring to the
changes themselves, that is, the chord symbols given on a lead sheet or
arrangement. Even when they are not playing from sheet music, the chord symbol
is the basic unit of harmonic understanding for jazz musicians. The reason for
this is largly practical: a chord symbol is a concise way of referring to a
particular sound, and improvising musicians must be able to understand this
information quickly (when reading) and to recall it easily (when improvising).

% rewrite? {
Since the pioneering work of George Russell in the late 1950s, many jazz%{{{
musicians conceive of an equivalence between a harmony (a chord symbol) and a
scale.\footcite{russell:lcc} The chord symbol \h{Dm7} might imply a D dorian
scale, for example, rather than simply the notes D--F--A--C. Because any of
the notes of this scale will sound relatively good over a \h{Dm7} chord, the
chord symbol makes a convenient shorthand for a particular “way of playing.”
This equivalence between chords and scales will be the focus of chapter 4; for
now it enough to note that understanding jazz harmony often involves more than
understanding relationships between four-voice seventh chords.
%}%}}}

When analyzing jazz harmony, it is often difficult to determine exactly what
one should be analyzing. Lead sheets as circulated in fake books can be highly
inaccurate, and often cannot be relied upon as a single source for any
particular jazz performance, since it is rare that performers play directly
from a lead sheet with no modifications.\fn{pja-3a} In the case of jazz
standards which may have originated elsewhere, we might wonder whether should
we analyze the original sources. In many cases, however, the “jazz standard”
version may be significantly different from the original version, reflecting a
history of adaptation by generations of jazz musicians.\fn{pja-3b} To make
matters worse, this knowledge is often secret knowledge, not written down and
learned only from more experienced musicians.

Many published jazz analyses rely on transcriptions of particular performances
as a way to avoid some of these issues. In general, this solution works well,
and I will certainly make use of them from time to time. This study, however,
is interested in harmony more generally, and transcriptions can confuse
matters somewhat. The kinds of questions I am interested in answering are of
the type “What can we say about harmony in the piece ‘Autumn Leaves’?” and
less often of the type “What can we say about Bill Evans’s use of harmony in
the recording of ‘Autumn Leaves’ from \emph{Portrait in Jazz}?” Furthermore,
even transcriptions are not definitive when it comes to harmony: the pianist
and guitarist might not be playing the same chord; the soloist might have a
different harmony in mind than the rhythm section; or the bass player might
play a bass line in such a way that affects our perception of the chordal
root. Even in the course of a single performance, a group might alter a tune’s
harmonic progression, perhaps preferring some substitutions during solos and
others during the head.

This is a problem without one clear solution, and it may make more sense to
use one method or another depending on the situation. Some compositions have
canonical recordings---Coleman Hawkins’s recording of “Body and Soul,” for
example---and in those circumstances determining the changes is usually
unproblematic. Other compositions are more fluid, and different choruses might
alter the basic structure throughout (substituting a \h{V7b9} for a \h{V13s11}
chord, for example).  In these cases, I am interested in what Henry Martin has
called the “ideal changes”; a hypothetical set of chords that we can use as a
basis for understanding the many variations that might occur in actual
performance.\fn{pja-3c} These changes represent a sort of Platonic model of a
composition; individual performances of “Autumn Leaves” can be seen as
instances of some ideal “\textsc{Autumn Leaves}.” Determining these ideal
changes is often a process of mediating published lead sheets, recorded
versions, and other sources; throughout this study I have tried to clarify
exactly \emph{what} harmony I am analyzing in any given example.


In an attempt to answer some of these problems, this dissertation presents a
transformational model of jazz harmony. While on the surface a
transformational model may seem abstract and far-removed from the concerns of
performing jazz musicians, harmony in jazz fits together nicely with David
Lewin’s famous “transformational attitude.”\footcite[159]{lewin:gmit} A jazz
musician does not typically think of harmonies as a series of points in space,
but rather as a series of “characteristic gestures” between them. Rather than
focusing on an underlying tonality, a jazz musician often tries to “make the
changes”---to fully engage with the sound of each individual harmony.

There is often quite a large gap between the way jazz is most commonly taught
(in jazz studios and pedagogical books) and the way it has traditionally been
understood by music theorists. Another goal of the present study is to use
transformational methods in an attempt to narrow this gap, by bringing
theoretical and mathematical rigor to materials that are often ignored by the
theory community, and by applying established theoretical principles in a way
that corresponds closely with the understanding of jazz musicians.

While we can never claim to know “what jazz musicians think,” we might get
somewhat closer to an answer by examining jazz pedagogical materials. In the
late 1960s, jazz began to be accepted into the academy, and many young jazz
musicians began to learn to play the music in schools, rather than exclusively
from older musicians.\fn{pja-4} To supplement this teaching, a great deal of
pedagogical material has appeared that aims to teach young musicians how to
play jazz.

Unfortunately, there is little interaction between these pedagogical materials
and music theoretical materials. Pedagogical materials, such as the recent
\emph{Berklee Book of Jazz Harmony}\nocite{berklee:harmony}, often do not have
bibliographies or mention recent work in the theoretical literature.\fn{pja-5}
Likewise, most theoretical work does not refer to these harmonic handbooks
which are staples of jazz education. Mark Levine’s \emph{Jazz Theory
Book}\nocite{levine:1995}, widely regarded in the jazz education world as
\emph{the} book on jazz theory, does not appear in any of the bibliographies
in the special jazz issue of \emph{Music Theory Online} (18.3), for example.

In recent years the theory community has embraced pedagogical materials as a
means of uncovering how historical musicians might have thought about their
own music.\fn{pja-6} Though jazz pedagogues do not typically publish articles
in music theory journals or otherwise consider themselves “music theorists,”
per se, the goal of their harmonic textbooks is quite similar to the goals of
pedagogical books in music theory: to teach students how to think in a
particular style of music.

David Lewin points out in his introduction to transformational theory that
when considering a particular musical passage, we often “conceptualize along
with it, as one of its characteristic textural features, a family of directed
measurements, distances, or motions of some sort.”\footcite[16]{lewin:gmit} I
certainly hear these characteristic motions when I listen to jazz, and I think
it is these motions that jazz pedagogues are emphasizing when they
teach students to “make the changes.” Despite its somewhat hostile
mathematical appearance, transformational theory is an effective means of
exploring these families of intuitions. Modeling sets of chord changes as
transformations between harmonic objects, for example, allows the theoretical
discourse to draw on the ways in which jazz musicians teach harmony, and can
bring these two disparate fields somewhat closer together.

% }}}

\section{Theoretical Approaches to Jazz Harmony}
