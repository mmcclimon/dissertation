% Chapter 1

\chapter{Introduction}

\section{Problems of Jazz Analysis}

When compared to a score by Beethoven (for example), the jazz lead sheet appears
strikingly bare. The Beethoven score specifies nearly everything a performer
needs to know in order to perform it. Though the minor details (things like
dynamics, articulations, phrasing marks, and the like) will differ from piece to
piece, we can usually depend on the presence of some basic information. It is
rare for traditional scores not to include the instrumentation, for example, and
a score that did not include information like the number of measures or which
notes to play in combination with which which other notes would be very unusual
indeed.

And yet, this is the usual state affairs for the jazz lead sheet, which is
probably the most common form of a “jazz score.”\fn{pja-1} Most lead sheets only
include the basic outline of a melody along with a set of “changes” that
prescribe the harmonic structure of a piece. Beyond this most basic instruction,
every other aspect is left up to the performers. Given that most small-group
jazz musicians approach the music first with these lead sheets, it seems
appropriate to focus our analytical attention on harmony the more salient of
those two elements.

Jazz is essentially a harmonic music. In a typical jazz performance, the melody
of the piece is only heard twice (at the beginning and the end), while the
harmonic structure is heard throughout, determining the structure of the
performance. Each soloist typically plays one or more “choruses,” where each
chorus is understood as a single iteration of the pieces harmonic structure.
In marked contrast to a Beethoven score, jazz compositions usually remain
unspecified when it comes to their contrapuntal structure: performers will
typically improvise counterpoint that fits with the underlying harmonic
framework. Harmony is the main restraining factor of a piece, and its primary
method of coherence.
