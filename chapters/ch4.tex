%% Chapter 4

\chapter{Chord-Scale Transformations}
\label{chap:chord-scale-transformations}
\addtolof{chap:chord-scale-transformations}

For most jazz musicians, a chord symbol implies not only its root, third, and
seventh, but also a variety of other notes as well (its fifth, ninth, and
potentially other chord members). Instead of conceiving of the extended
harmonies common in jazz as stacks of thirds, musicians often describe harmony
in a more linear fashion, using a scale to stand in for a chord symbol. What
is often called ``chord-scale theory'' is a major part of jazz pedagogy, and
cannot be ignored as we try to approach a general theory of jazz harmony. This
chapter will begin with an introduction to chord-scale theory, both in its
original form and its later pedagogical adaptations, and then move on to
incorporate it into the transformational system developed thus far; finally,
we will see how the theory allows us to make analytical insights about more
than just chord symbols.

\section{George Russell’s \emph{Lydian Chromatic Concept}}
\label{sec:lcc}

The ultimate origin of chord-scale theory is George Russell's \emph{Lydian
  Chromatic Concept of Tonal Organization}, first published in 1953 but
revised several times throughout Russell's life.\fn{lcc-1} Russell was a jazz
pianist, drummer, and a well-known composer and arranger (he is perhaps best
known for his composition ``Cubana Be/Cubana Bop'' for the Dizzy Gillespie big
band); he devised the majority of the \emph{Lydian Chromatic Concept} while
hospitalized for tuberculosis in 1945--46.\fn{lcc-2} The influence of the
\emph{Concept} (as it is often called) is difficult to overstate.
Joachim-Ernst Berendt and Günther Heusmann decribe it as ``the first work
deriving a theory of jazz harmony from the immanent laws of jazz, not from the
laws of European music,'' and the blurbs on the back cover contain praises
from musicians including Gil Evans, Ornette Coleman, Eric Dolphy, and Toru
Takemitsu.\fn{lcc-3}

Despite its importance to jazz theory, though, Russell's work has not received
much attention in music-theoretical scholarship on jazz. Dmitri Tymoczko, for
example, does not mention Russell at all in his survey on the pedagogical use
of chord-scales in jazz, and the only mention of Russell in his book is in a
footnote unrelated to Russell's contributions to chord-scale theory.\fn{lcc-4}
There may be many reasons for this---Russell's serpentine and hard-to-follow
prose are probably not least among them---but regardless, a brief introduction
to Russell's theories as he conceived them will be in order here. The
\emph{Lydian Chromatic Concept} can be divided into two main components, which
we will treat separately in the following sections: Lydian tonal organization and
chord--scale equivalence.

\subsection{Lydian Tonal Organization}
\label{subsec:lydian-org}

Russell's central insight---indeed, the Concept itself---is that the
Lydian scale is more fundamental than the major scale. He offers many
explanations, but is easiest to demonstrate, as he does, with an example.
Figure \ref{lcc:tertian-stacks} reproduces Russell's first example; he
provides the following instructions and explanation:%
%
\begin{quoting}
  \singlespacing
  Sound both of the following chords separately. Try to detect the one which
  sounds a greater degree of unity and finality with its tonical [\emph{sic}] C
  major triad. \ldots\ In tests performed over the years in various parts of the
  world, the majority of people have repeatedly chosen the second chord---the C
  Lydian Scale in its tertian order.\footnote{\citetitle[1]{russell:lcc}.}
\end{quoting}%
%
The lowest note of a stack of six perfect fifths is what Russell calls the
``Lydian tonic''; the B--F tritone in the C major scale ``disrupts the perfect
symmetry of the fifths.''\footnote{\citetitle[4]{russell:lcc}.} He does
concede that C is also understood as tonic in the major scale arrangement, but
that it does not sound resolved, since ``the presence of the Lydian \emph{do}
on the major scale's fourth degree permanently denies [it] that
possibility.''\footnote{Ibid., 4.} For Russell, the major
scale is always in a state of tension, wanting to ``resolve'' to the
Lydian.\fn{lcc-5}

\begin{figure}[tbp]
  \centerGraphic[width=16em]{eps/ch4/tertian-stacks.pdf}
  \caption[Two stacks of thirds, using the major scale and the Lydian
    scale.]{Two stacks of thirds, using the major scale (left) and the Lydian
    scale (right). Russell's Example I:1.}
  \label{lcc:tertian-stacks}
\end{figure}

Closely allied with the Lydian tonic is the concept of tonal gravity, which
Russell calls the fundamental principle of the \emph{Concept}. In a stack of
fifths, tonal gravity flows downward: ``the tone F\sharp\ yields to B as its
tonic---F\sharp\ and B surrender ``tonical'' authority to E, and so on down the
ladder of fifths---the entire stack conferring ultimate tonical authority on
its lowermost tone, C.''\fn{lcc-6} The concept of tonal gravity provides the
justification for the primacy of the Lydian scale, since the major scale
cannot be constructed by generating perfect fifths from its tonic.

Its theoretical justifications aside, the Lydian scale has an almost mystical
quality to Russell, which can sometimes be off-putting. A somewhat longer
passage from the \emph{Concept} will help to illustrate Russell's fascination
with the scale, as well as his usual circuitous mode of presentation (the
emphasis and non-bracketed ellipses are original):
%
\begin{quoting}
  \singlespacing
  The \textsc{Lydian Tonic}, as the musical ``Star-Sun,'' is the seminal
  source of tonal gravity and organization of a Lydian Chromatic scale.
  [\ldots] \textsc{Unity} is the state in which the Lydian Scale exists in
  relation to its I major and VI minor tonic station chords, as well as those
  on other scale degrees. Unity \emph{is} \ldots\ instantaneous completeness
  and oneness in the \emph{Absolute Here and Now} \ldots\ above linear time.

  The Lydian Scale is the musical \emph{passive} force. Its unified tonal
  gravity field, ordained by the ladder of fifths, serves as a theoretical
  basis for tonal organization within the Lydian Chromatic Scale and,
  ultimately, for the entire Lydian Chromatic Concept. There is no ``goal
  pressure'' within the tonal gravity field of a Lydian Scale. The Lydian
  Scale exists as a self-organized \emph{Unity} in relations to its tonic tone
  and tonic major chord. The Lydian Scale implies an evolution to higher
  levels of tonal organization. The Lydian Scale is the true scale of tonal
  unity and the scale which clearly represents the phenomenon of tonal gravity
  itself.\fn{lcc-7}
\end{quoting}

Russell's logic is, of course, circular: the Lydian tonic is by definition the
note that is the bottom of a stack of six perfect fifths, and the principle of
tonal gravity confers a special status on the bottom of a stack of six fifths
(conveniently, the Lydian tonic). Partly for this reason, this part of
Russell's theory has not really been taken seriously by modern scholars. He
never gives a reason, for example, that the stack should not be extended
further: wouldn't the lowest note of a stack of \emph{seven} fifths be imbued
with even more tonal gravity? This complication reappears when Russell later
presents the complete ``Lydian Chromatic Order of Tonal Gravity'' (starting
on F):\footnote{\citetitle[12]{russell:lcc}.} \\
{\centering F C G D A E B C\sharp\ \Aflat\ \Eflat\ \Bflat\ \Gflat \par}

\noindent What should be a perfect fifth from B to F\sharp\ is replaced by a
whole step, so that the minor ninth, F--\Gflat, does not appear until the last
note. (We will return to this order below). This sleight of hand also prevents
there from being more than one succession of six perfect fifths in the series.
If it had continued in perfect fifths, there would be by definition seven
Lydian tonics!

% xxx I'm not sure this paragraph makes sense
Still, though, is there anything worth saving in Russell's ideas? He is
probably right that most people prefer the sound of the stack of thirds on the
right of Figure \ref{lcc:tertian-stacks}, with the F\sharp.\fn{lcc-8} And the
Lydian scale \emph{does} have some practical advantages over the major scale.
As Russell points out, \sharp\sd4 appears before \nat\sd4 in the harmonic
series (he does not mention that \flat\sd7 appears before \nat\sd7\,). The
Lydian scale is also unique in that it is possible to form all twelve interval
types with the tonic. Russell's explanation of this fact is reproduced in
Figure \ref{lcc:lydian-intervals} (the F in the major scale is shown with a
hollow note because it is the true Lydian tonic for the scale).

\begin{figure}[tbp]
  \centerGraphic{eps/ch4/lydian-intervals.pdf}
  \caption[Russell's ``Interval Tonic Justification for the Lydian
  scale.'']{Russell's ``Interval Tonic Justification for the Lydian scale''
    (his Example I:9).}
  \label{lcc:lydian-intervals}
\end{figure}

Though we may not share Russell's preoccupation with the Lydian scale, we
should not let his idiosyncratic views distract us from the more important
points of the theory. Theorists are, though, used to adopting worthwhile
theoretical ideas from authors without assuming their entire worldview. We
regularly practice Schenkerian analysis without adopting Schenker's views on
the superiority of German music, and even 18th-century authors like Kirnberger
accepted Rameau's fundamental bass without necessarily espousing his more
contentious thoughts on harmonic generation or
\emph{subposition}.\footcite[240--41]{lester:1992} And yet, with the
\emph{Concept}, this does not seem to have taken place.\fn{lcc-9} If we
concede this point, though, Russell's ideas prove to be remarkably useful, as
we shall see.

% Later, more on the tonal order and the scales themselves, how they're
% separable from the Lydian
Before moving on to chord-scales proper, we should first examine the scales
themselves, as their initial presentation in the \emph{Concept} is entangled
with the discussion of the nature of the Lydian scale. The scales are
generated (more or less) from the chromatic order of tonal gravity, which is
given again here in its generic form: \\
{\centering I\quad V\quad II\quad VI\quad III\quad VII\quad +IV\quad +V\quad
  \flat{}III\quad \flat{}VII\quad IV\quad \flat{}II\quad \par}

\noindent When taken together, the entire series represents the Lydian
Chromatic Scale, the foundation of the titular Concept.

The Lydian Chromatic (or LC) scale contains eleven ``member scales,'' which are
chosen, Russell says, for three reasons:%
%
\begin{compactenum}[\qquad a.\ ]
    \singlespacing
    \item a scale's capacity to parent chords considered important in the
      development of Western harmony
    \item a scale as being most representative of a tonal level of the Lydian
      Chromatic scale
    \item the historical and/or sociological significance of a
      scale.\footnote{\citetitle[12]{russell:lcc}.}
\end{compactenum}
%
These eleven scales are further divided into seven principal scales and four
horizontal scales. The seven principal scales are derived from the Lydian
Chromatic scale, and are shown in Figure \ref{lcc:principal-scales}. The
scales are given what Russell calls their ``ingoing-to-outgoing'' order in
regards to the F Lydian tonic''; we can read ``ingoing'' and ``outgoing'' as
``consonant'' and ``dissonant,'' respectively.\fn{lcc-10}
The principal scales are more familiar to us under different names, as shown in
Table \ref{lcc:scale-names}.\fn{lcc-11}

\begin{figure}[p]
  \centerGraphic{eps/ch4/principal-scales.pdf}
  \caption{The seven principal scales of the F Lydian Chromatic scale.}
  \label{lcc:principal-scales}
\end{figure}

\begin{table}[p]
  \setlength{\tabcolsep}{12pt}
  \centering
  \begin{tabular}{cc}
   Russell's name  & Other common names \\
   \hline
   \rule[1em]{0ex}{1ex}%
   Lydian & -- \\
   Lydian augmented & 3rd mode of melodic minor \\
   Lydian diminished & 4th mode of harmonic major \\
   Lydian flat seventh & Lydian dominant, acoustic \\
   Auxiliary augmented & whole-tone \\
   Auxiliary diminished & octatonic, diminished (whole--half) \\
   Auxiliary diminished blues & octatonic, diminished (half--whole)
  \end{tabular}
  \caption{Russell's principal scale names and their other common names.}
  \label{lcc:scale-names}
\end{table}

The means by which the LC scale generates the seven principal scales is
explained the diagram reproduced in Figure \ref{lcc:lc-tonal-gravity}.
Russell's explanation of this diagram is somewhat confusing. The term ``tone
order'' is never defined, except to say that the LC scale has five of them (it
is unclear why there is no 8-tone order). The shaded ``consonant nucleus''
describes the fact that all of the standard chord types---major, minor,
seventh, augmented, and diminished---are contained within it.\fn{lcc-12} The
consonant nucleus also provides a (tautological) explanation for the missing
fifth: ``the skipping of the interval of a fifth between the seventh and
eighth tones of the Lydian Chromatic Scale allows the five basic chord
categories of Western Harmony to be assimilated by its Nine-Tone Order,
Semi-Ingoing Level, in the logical order of their development in Western
Harmony and the Lydian Chromatic Scale.''\fn{lcc-13}

\begin{figure}[tbp]
  \centerGraphic{eps/ch4/tonal-gravity.pdf}
  \caption{Russell's ``Lydian Chromatic Order of Tonal Gravity'' (his example
    II:3).}
  \label{lcc:lc-tonal-gravity}
\end{figure}

The other four of the eleven member scales are known as the ``horizontal
scales,'' and are shown in Figure \ref{lcc:horizontal-scales}. For Russell,
``horizontal'' is used in opposition to the ``vertical'' generation of the
Lydian scale; because the major scale is not a stack of perfect fifths, he
considers it to be generated in a different direction. All of the horizontal
scales have \nat\sd4; Russell only includes them because of their ``historical
and/or sociological significance.'' The horizontal scales do not, as we shall
see, generate chords in the same way as the vertical scales, and for Russell
exist in a constant state of tension between the ``false'' tonic and the true
Lydian tonic.

\begin{figure}[tbp]
  \centerGraphic{eps/ch4/horizontal-scales.pdf}
  \caption{The four horizontal scales of the F Lydian Chromatic scale.}
  \label{lcc:horizontal-scales}
\end{figure}

Given that most of Russell's ideas on Lydian tonal organization have
disappeared from modern chord-scale theory, it is reasonable to ask why we
have devoted so much space to them here. My reason is twofold. First, much
modern scholarship does not seriously engage with Russell's ideas, and as a
result most theorists are not familiar with its first incarnation. Because the
original nature of chord-scale theory is tied up with that of Lydian tonal
organization, understanding the former is important in order to make sense of
the latter. Second, and more importantly, one of the goals of this
dissertation is to take jazz musicians' conceptions of harmony seriously:
chord-scale theory is an integral part of the way jazz today is taught, and
therefore understood by many practicing musicians. As we develop a
transformational system of chord-scales later in this chapter, we will seek to
revive some of Russell's initial conception.\fn{lcc-14}

\subsection{Chord--Scale Equivalence}
\label{subsec:chord-scales}

While Russell may have understood Lydian tonal organization to be the most
important part of his new theory, the part that has survived---flourished,
even---is his novel conception of chord--scale equivalence.\fn{lcc-15}
Russell's first mention of the concept explains its inception:%
%
\begin{quoting}
  \singlespacing
  In a conversation I had with Miles Davis in 1945, I asked, ``Miles, what's
  your musical aim?'' His answer, ``to learn all the changes (chords),'' was
  somewhat puzzling to me since I felt---and I was hardly alone in the
  feeling---that Miles played like he already knew all the chords. After
  dwelling on his statement for some months, I became mindful that Miles's
  answer may have implied the need to relate to chords in a new way. This
  motivated my quest to expand the tonal environment of the chord beyond the
  immediate tones of its basic structure, leading to the irrevocable conclusion
  that every traditionally definable chord of Western music theory has its
  origin in a \textsc{parent scale}. In this vertical sense, the term refers to
  that scale which is ordained---by the nature of tonal gravity---to be a
  chord's source of arising, and ultimate vertical completeness; the chord and
  its parent scale existing in a state of complete and indestructible
  chord/scale unity---a
  \textsc{chordmode}.\footnote{\citetitle[10]{russell:lcc}.}
\end{quoting}

What Miles was looking for is essentially a way of determining
what notes he could play over a given chord. Simply knowing the chord tones
no longer seemed to be enough, since various extensions and alterations can
change the sound of the chord. Chord--scale theory is ultimately, then, an
improvisational expedient: a single scale stands in for a chord symbol.
Chord symbols with alterations are represented by different scales, and thus
musicians do not necessarily have to keep track of all of the individual chord
tones.

Russell's later explanation of the concept is uncharacteristically clear:%
%
\begin{quoting}
  \singlespacing
  The chord and its parent scale are an inseparable entity---the reciprocal
  sound of one another. \ldots\ In other words, the complete sound of a chord is
  its corresponding mode within its parent scale. Therefore, the broader term
  \textsc{chordmode} is substituted for what is generally referred to as ``the
  chord.''\fn{lcc-16}
\end{quoting}
%
\noindent It is important to understand that for Russell, the two
terms---chord and scale---are truly equivalent: one does not substitute for
the other, rather one \emph{is} the other. Here, we arrive at the reason for
the inclusion of this material in a dissertation about jazz harmony. If we
take Russell seriously (I am arguing that we should), a harmony is a scale,
and vice versa. The two ideas are inseparable, and a study of harmony in jazz
would be incomplete without a commensurate discussion about scales.

At this point a brief overview of Russell's brand of chord--scale theory is
appropriate. This material accounts for the majority of the length of the
\emph{Concept}, so we will be careful here to avoid going into all of its
painstaking detail. Determining the scale that belongs with a particular chord
is a multi-step process: first, identify the parent Lydian scale; then,
determine the harmonic genre based on the characteristic modes of the Lydian
scale.\fn{lcc-17}

Russell goes through all seven modes of the Lydian scale, identifying the
``principal chords'' of each mode. These chords represent the purest form of
the mode, and the basis for the chord--scale matching process. An overview of
these is given in Table \ref{lcc:lydian-modes}; several things are worth
mentioning here.\fn{lcc-18} First, the order of modes in the table follows
Russell's order of presentation, which (though he does not explain it)
coincides with the frequency of each principal chordmode in jazz practice.
Second, the ``sub-principal chords'' are those which are also representative
of a given mode; they ``do not contain all the tones of [the] relative
Principal Chordmode,'' but they ``still exist in a state of unity with [the]
parent Principal scale.''\footnote{\citetitle[23]{russell:lcc}.} Last, those
modes with \textsc{b} in their names refer to bass notes: the III Major
(III\textsc{b}) group refers to major chords with the third in the bass.

\begin{table}[tbp]
  \centerGraphic{eps/ch4/lydian-modes.pdf}
  \caption{Modes of the C Lydian scale.}
  \label{lcc:lydian-modes}
\end{table}

% xxx this paragraph is no good
The treatment of Mode V here bears special mention. The fifth mode of the
Lydian scale is of course the ordinary major scale, which Russell took great
pains to show earlier was \emph{not} a chord-generating scale.\fn{lcc-19} Its
role as the fifth mode of the Lydian scale is only to act as support for the
consonant Lydian harmony. The principal chordmode for this scale is the same
as that of the Lydian proper, with the fifth in the bass. This chord and its
relatives are by nature unstable (cf.\ a cadential $^6_4$ chord), and this
instability allows Russell to avoid a potential complication of the theory.

With the modes of the Lydian scale taken care of, Russell then goes on to show
how his seven vertical scales gives rise to other kinds of chords. To do so,
he introduces another bit of terminology, the Primary Modal Genre
(\abbrev{PMG}):%
%
\begin{quoting}
  \singlespacing
  A PMG is an assemblage of Principal Chord Families of similar type: a
  Principal Chord Family mansion housing the spectrum of variously colored
  Principal Chord Families of the same essential harmonic
  genre.\footnote{\citetitle[29]{russell:lcc}.}
\end{quoting}
%
\noindent All of the principal chordmodes in Table \ref{lcc:lydian-modes} are
\abbrev{PMG}s, and the six other vertical scales---Lyd.\ augmented, Lyd.\
diminished, Lyd.\ flat seventh, Aux.\ augmented, Aux.\ diminished, and Aux.\
dim.\ blues---generate similar assemblages of chordal types.

Figure \ref{lcc:alternate-pmg} gives an example of how this works in practice.
The left-hand side of the figure shows the second mode of the C auxiliary
diminished scale, and the right-side gives its vertical expression as an
altered dominant chord: \h{D13s9b9b5}. Russell works through all of the modes
of the six other vertical scales, and the chart included with the book lists
almost all of these. The eight \abbrev{PMG}s fall into general categories,
which are shown in Table \ref{lcc:lydian-pmgs} and will be sufficient for our
purposes here.\fn{lcc-20}

\begin{figure}[tbp]
  \centerGraphic{eps/ch4/alternate-pmg.pdf}
  \caption{The second mode of the C auxiliary diminished scale, in scalar and
    tertian formations.}
  \label{lcc:alternate-pmg}
\end{figure}

\begin{table}[tbp]
  \centering
  \vspace{1em}
  \begin{tabular}{cl}
   Primary Modal Tonic & Primary Modal Genre \\
   \hline
   \rule[1em]{0ex}{1ex}%
   I    & major and altered major chords \\
   II   & seventh and altered seventh chords \\
   III  & [I] major and altered [I] major 3\textsc{b} (minor +5) chords \\
   +IV  & minor seventh \flat{}5 / [I] major +4\textsc{b} chords \\
   V    & [I] major and altered [I]5\textsc{b} chords \\
   VI   & minor and altered minor chords \\
   VII  & eleventh \flat{}9 / [I] major 7\textsc{b} chords \\
   +V   &  seventh +5 chords
  \end{tabular}
  \caption[The eight principal modal tonics and their associated modal
    genres.]{The eight principal modal tonics and their associated modal genres
    (Russell's example III:30).}
  \label{lcc:lydian-pmgs}
\end{table}

It is by now, I hope, apparent how the \emph{Concept} can simplify matters
somewhat for an improvising jazz musician. Once the process of matching chord
symbols with scales is learned, it is a relatively simple matter to determine
what notes will sound good over, for example, a \h{D13s9b9b5} chord. Russell
seems to have taken Miles's wish to ``learn all the changes'' to heart; he
takes care to note that indeed \emph{all} of the harmonies of Western music
can be found somewhere in the chart (and notes that many ``non-traditional
harmonic colors'' can be found as
well).\footnote{\citetitle[29]{russell:lcc}.}

At the same time, though, it is probably also apparent that Russell's system
is somewhat more complicated than it needs to be. Most of the source of this
complication is in fact his ideas about Lydian tonal organization.
Russell provides an example of how to find the parent scale for an unadorned
\h{Eb7} chord (a very simple example):
%
\begin{quoting}
  \singlespacing
  Over the roman numerals of the scales of Chart A are listed different chord
  families. For example, over roman numeral II of the Lydian Scale are listed
  7th, 9th, 11th, and 13th chords. They belong to the same family: the (II)
  seventh chord family of a Lydian Chromatic Scale.

  The \h{Eb7} chord is found in this family above roman numeral II of the
  Lydian Scale in the right column of Chart A. The Lydian Scale is therefore
  the parent scale of the \h{Eb7} chord.

  Place the root of the \h{Eb7} chord on roman numeral II, and \Eflat\ becomes
  the second degree of that chord's parent scale.

  Think down a major 2nd interval; if \Eflat\ is the second degree of the
  parent scale, \Dflat\ is the first degree. Therefore \Dflat\ is the tonic
  (root) of the \h{Eb7} chord's parent scale. This tonic is called the Lydian
  tonic. For the \h{Eb7} chord, \Dflat\ is the Lydian Tonic and the parent
  scale is \Dflat\ Lydian.\footnote{\citetitle[59]{russell:lcc}.}
\end{quoting}
%
\noindent This is quite a long process to determine that the most ingoing
(consonant) scale for an \h{Eb7} chord is the second mode of the \Dflat\
Lydian scale. The equivalence of chords and scales is genuinely useful for
improvising musicians, while the Lydian organization is more abstract. Faced
with this situation, jazz musicians made the obvious simplification: over an
\h{Eb7} chord, play the \Eflat\ Mixolydian scale.

Russell's theory becomes interesting, though, when we realize that any of the
member scales of the Lydian tonic can stand in for the ordinary Lydian. Once
you have determined that the parent scale of an \h{Eb7} chord is \Dflat\
Lydian, then it becomes easy to substitute more complicated scales built on
the same Lydian tonic. If you wanted to create a more dissonant (outgoing)
sound, you might instead play the second mode of the \Dflat\ Lydian flat
seventh scale; the second mode of the \Dflat\ auxiliary augmented blues scale
would be more dissonant still. Because the seven principal scales form a
spectrum of consonance to dissonance---as Russell frames it, there is a
progression of unity from ingoing to outgoing---the \emph{Concept} provides a
means of measuring how closely a particular progression or improvisation stays
to a particular Lydian tonic. This idea is the core of what we might borrow
from Russell, and we will return to it when we begin to develop our
transformational system in the next section.


%%% Local Variables:
%%% mode: latex
%%% TeX-master: "../diss"
%%% End:
