%% Chapter 4

\chapter{Chord-Scale Transformations}
\label{chap:chord-scale-transformations}
\addtolof{chap:chord-scale-transformations}
\addtolof[lot]{chap:chord-scale-transformations}
\addtocspace


For most jazz musicians, a chord symbol implies not only its root, third, and
seventh, but also a variety of other notes as well (its fifth, ninth, and
potentially other chord members). Instead of conceiving of the extended
harmonies common in jazz as stacks of thirds, musicians often describe harmony
in a more linear fashion, using a scale to stand in for a chord symbol. What
is often called ``chord-scale theory'' is a major part of jazz pedagogy, and
cannot be ignored as we try to approach a general theory of jazz harmony. This
chapter will begin with an introduction to chord-scale theory, both in its
original form and its later pedagogical adaptations, and then move on to
incorporate it into the transformational system developed thus far; finally,
we will see how the theory allows us to make analytical insights about more
than just chord symbols.

\section{George Russell’s \emph{Lydian Chromatic Concept}}
\label{sec:lcc}

The ultimate origin of chord-scale theory is George Russell's \emph{Lydian
  Chromatic Concept of Tonal Organization}, first published in 1953 but
revised several times throughout Russell's life.\fn{lcc-1} Russell was a jazz
pianist, drummer, and a well-known composer and arranger (he is perhaps best
known for his composition ``Cubana Be/Cubana Bop'' for the Dizzy Gillespie big
band); he devised the majority of the \emph{Lydian Chromatic Concept} while
hospitalized for tuberculosis in 1945--46.\fn{lcc-2} The influence of the
\emph{Concept} (as it is often called) is difficult to overstate.
Joachim-Ernst Berendt and Günther Heusmann decribe it as ``the first work
deriving a theory of jazz harmony from the immanent laws of jazz, not from the
laws of European music,'' and the blurbs on the back cover contain praises
from musicians including Gil Evans, Ornette Coleman, Eric Dolphy, and Toru
Takemitsu.\fn{lcc-3}

Despite its importance to jazz theory, though, Russell's work has not received
much attention in music-theoretical scholarship on jazz. Dmitri Tymoczko, for
example, does not mention Russell at all in his survey on the pedagogical use
of chord-scales in jazz, and the only mention of Russell in his book is in a
footnote unrelated to Russell's contributions to chord-scale theory.\fn{lcc-4}
There may be many reasons for this---Russell's serpentine and hard-to-follow
prose are probably not least among them---but regardless, a brief introduction
to Russell's theories as he conceived them will be in order here. The
\emph{Lydian Chromatic Concept} can be divided into two main components, which
we will treat separately in the following sections: Lydian tonal organization and
chord--scale equivalence.

\subsection{Lydian Tonal Organization}
\label{subsec:lydian-org}

Russell's central insight---indeed, the Concept itself---is that the
Lydian scale is more fundamental than the major scale. He offers many
explanations, but is easiest to demonstrate, as he does, with an example.
Figure \ref{lcc:tertian-stacks} reproduces Russell's first example; he
provides the following instructions and explanation:%
%
\begin{quoting}
  \singlespacing
  Sound both of the following chords separately. Try to detect the one which
  sounds a greater degree of unity and finality with its tonical [\emph{sic}] C
  major triad. \ldots\ In tests performed over the years in various parts of the
  world, the majority of people have repeatedly chosen the second chord---the C
  Lydian Scale in its tertian order. (\lcc 1)
\end{quoting}%
%
The lowest note of a stack of six perfect fifths is what Russell calls the
``Lydian tonic''; the B--F tritone in the C major scale ``disrupts the perfect
symmetry of the fifths'' (\lcc 4). He does concede that C is also understood as
tonic in the major scale arrangement, but that it does not sound resolved,
since ``the presence of the Lydian \emph{do} on the major scale's fourth
degree permanently denies [it] that possibility'' (\lcc 4). For
Russell, the major scale is always in a state of tension, wanting to
``resolve'' to the Lydian.\fn{lcc-5}

\begin{figure}[tbp]
  \centerGraphic[width=16em]{eps/ch4/tertian-stacks.pdf}
  \caption[Two stacks of thirds, using the major scale and the Lydian
    scale.]{Two stacks of thirds, using the major scale (left) and the Lydian
    scale (right). Russell's Example I:1.}
  \label{lcc:tertian-stacks}
\end{figure}

Closely allied with the Lydian tonic is the concept of tonal gravity, which
Russell calls the fundamental principle of the \emph{Concept}. In a stack of
fifths, tonal gravity flows downward: ``the tone F\sharp\ yields to B as its
tonic---F\sharp\ and B surrender ``tonical'' authority to E, and so on down the
ladder of fifths---the entire stack conferring ultimate tonical authority on
its lowermost tone, C'' (\lcc 3).  The concept of tonal gravity provides the
justification for the primacy of the Lydian scale, since the major scale
cannot be constructed by generating perfect fifths from its tonic.

Its theoretical justifications aside, the Lydian scale has an almost mystical
quality to Russell, which can sometimes be off-putting. A somewhat longer
passage from the \emph{Concept} will help to illustrate Russell's fascination
with the scale, as well as his usual circuitous mode of presentation (the
emphasis and non-bracketed ellipses are original):
%
\begin{quoting}
  \singlespacing
  The \textsc{Lydian Tonic}, as the musical ``Star-Sun,'' is the seminal
  source of tonal gravity and organization of a Lydian Chromatic scale.
  [\ldots] \textsc{Unity} is the state in which the Lydian Scale exists in
  relation to its I major and VI minor tonic station chords, as well as those
  on other scale degrees. Unity \emph{is} \ldots\ instantaneous completeness
  and oneness in the \emph{Absolute Here and Now} \ldots\ above linear time.

  The Lydian Scale is the musical \emph{passive} force. Its unified tonal
  gravity field, ordained by the ladder of fifths, serves as a theoretical
  basis for tonal organization within the Lydian Chromatic Scale and,
  ultimately, for the entire Lydian Chromatic Concept. There is no ``goal
  pressure'' within the tonal gravity field of a Lydian Scale. The Lydian
  Scale exists as a self-organized \emph{Unity} in relations to its tonic tone
  and tonic major chord. The Lydian Scale implies an evolution to higher
  levels of tonal organization. The Lydian Scale is the true scale of tonal
  unity and the scale which clearly represents the phenomenon of tonal gravity
  itself. (\lcc 8--9)
\end{quoting}

Russell's logic is, of course, circular: the Lydian tonic is by definition the
note that is the bottom of a stack of six perfect fifths, and the principle of
tonal gravity confers a special status on the bottom of a stack of six fifths
(conveniently, the Lydian tonic). Partly for this reason, this part of
Russell's theory has not really been taken seriously by modern scholars. He
never gives a reason, for example, that the stack should not be extended
further: wouldn't the lowest note of a stack of \emph{seven} fifths be imbued
with even more tonal gravity? This complication reappears when Russell later
presents the complete ``Lydian Chromatic Order of Tonal Gravity,'' given here
starting on F (\lcc 12): \\
{\centering F\quad C\quad G\quad D\quad A\quad E\quad B\quad C\sharp\quad
  \Aflat\quad \Eflat\quad \Bflat\quad \Gflat \par}

\noindent What should be a perfect fifth from B to F\sharp\ is replaced by a
whole step, so that the minor ninth, F--\Gflat, does not appear until the last
note. (We will return to this order below). This sleight of hand also prevents
there from being more than one succession of six perfect fifths in the series.
If it had continued in perfect fifths, there would be by definition seven
Lydian tonics!

% xxx I'm not sure this paragraph makes sense
Still, though, is there anything worth saving in Russell's ideas? He is
probably right that most people prefer the sound of the stack of thirds on the
right of Figure \ref{lcc:tertian-stacks}, with the F\sharp.\fn{lcc-8} And the
Lydian scale \emph{does} have some practical advantages over the major scale.
As Russell points out, \sharp\sd4 appears before \nat\sd4 in the harmonic
series (he does not mention that \flat\sd7 appears before \nat\sd7\,). The
Lydian scale is also unique in that it is possible to form all twelve interval
types with the tonic. Russell's explanation of this fact is reproduced in
Figure \ref{lcc:lydian-intervals} (the F in the major scale is shown with a
hollow note because it is the true Lydian tonic for the scale).

\begin{figure}[tbp]
  \centerGraphic{eps/ch4/lydian-intervals.pdf}
  \caption[Russell's ``Interval Tonic Justification for the Lydian
  scale.'']{Russell's ``Interval Tonic Justification for the Lydian scale''
    (his Example I:9).}
  \label{lcc:lydian-intervals}
\end{figure}

Though we may not share Russell's preoccupation with the Lydian scale, we
should not let his idiosyncratic views distract us from the more important
points of the theory. Theorists are, though, used to adopting worthwhile
theoretical ideas from authors without assuming their entire worldview. We
regularly practice Schenkerian analysis without adopting Schenker's views on
the superiority of German music, and even 18th-century authors like Kirnberger
accepted Rameau's fundamental bass without necessarily espousing his more
contentious thoughts on harmonic generation or
\emph{subposition}.\footcite[240--41]{lester:1992} And yet, with the
\emph{Concept}, this does not seem to have taken place.\fn{lcc-9} If we
concede this point, though, Russell's ideas prove to be remarkably useful, as
we shall see.

% Later, more on the tonal order and the scales themselves, how they're
% separable from the Lydian
Before moving on to chord-scales proper, we should first examine the scales
themselves, as their initial presentation in the \emph{Concept} is entangled
with the discussion of the nature of the Lydian scale. The scales are
generated (more or less) from the chromatic order of tonal gravity, which is
given again here in its generic form: \\
{\centering I\quad V\quad II\quad VI\quad III\quad VII\quad +IV\quad +V\quad
  \flat{}III\quad \flat{}VII\quad IV\quad \flat{}II\quad \par}

\noindent When taken together, the entire series represents the Lydian
Chromatic Scale, the foundation of the titular Concept.

The Lydian Chromatic (or LC) scale contains eleven ``member scales,'' which are
chosen, Russell says, for three reasons:%
%
\begin{compactenum}[\qquad a.\ ]
    \singlespacing
    \item a scale's capacity to parent chords considered important in the
      development of Western harmony
    \item a scale as being most representative of a tonal level of the Lydian
      Chromatic scale
    \item the historical and/or sociological significance of a
      scale. (\lcc 12)
\end{compactenum}
%
These eleven scales are further divided into seven principal scales and four
horizontal scales. The seven principal scales are derived from the Lydian
Chromatic scale, and are shown in Figure \ref{lcc:principal-scales}. The
scales are given what Russell calls their ``ingoing-to-outgoing'' order in
regards to the F Lydian tonic''; we can read ``ingoing'' and ``outgoing'' as
``consonant'' and ``dissonant,'' respectively.\fn{lcc-10}
The principal scales are more familiar to us under different names, as shown in
Table \ref{lcc:scale-names}.\fn{lcc-11}

\begin{figure}[p]
  \centerGraphic{eps/ch4/principal-scales.pdf}
  \caption{The seven principal scales of the F Lydian Chromatic scale.}
  \label{lcc:principal-scales}
\end{figure}

\begin{table}[p]
  \setlength{\tabcolsep}{12pt}
  \centering
  \begin{tabular}{cc}
   Russell's name  & Other common names \\
   \hline
   \rule[1em]{0ex}{1ex}%
   Lydian & -- \\
   Lydian augmented & 3rd mode of melodic minor \\
   Lydian diminished & 4th mode of harmonic major \\
   Lydian flat seventh & Lydian dominant, acoustic \\
   Auxiliary augmented & whole-tone \\
   Auxiliary diminished & octatonic, diminished (whole--half) \\
   Auxiliary diminished blues & octatonic, diminished (half--whole)
  \end{tabular}
  \caption{Russell's principal scale names and their other common names.}
  \label{lcc:scale-names}
\end{table}

The means by which the LC scale generates the seven principal scales is
explained the diagram reproduced in Figure \ref{lcc:lc-tonal-gravity}.\fn{lcc-13}
Russell's explanation of this diagram is somewhat confusing. The term ``tone
order'' is never defined, except to say that the LC scale has five of them (it
is unclear why there is no 8-tone order). The shaded ``consonant nucleus''
describes the fact that all of the standard chord types---major, minor,
seventh, augmented, and diminished---are contained within it.\fn{lcc-12} The
consonant nucleus also provides a (tautological) explanation for the missing
fifth: ``the skipping of the interval of a fifth between the seventh and
eighth tones of the Lydian Chromatic Scale allows the five basic chord
categories of Western Harmony to be assimilated by its Nine-Tone Order,
Semi-Ingoing Level, in the logical order of their development in Western
Harmony and the Lydian Chromatic Scale'' (\lcc 16).

\begin{figure}[tbp]
  \centerGraphic{eps/ch4/tonal-gravity.pdf}
  \caption{Russell's ``Lydian Chromatic Order of Tonal Gravity'' (his example
    II:3).}
  \label{lcc:lc-tonal-gravity}
\end{figure}

The other four of the eleven member scales are known as the ``horizontal
scales,'' and are shown in Figure \ref{lcc:horizontal-scales}. For Russell,
``horizontal'' is used in opposition to the ``vertical'' generation of the
Lydian scale; because the major scale is not a stack of perfect fifths, he
considers it to be generated in a different direction. All of the horizontal
scales have \nat\sd4; Russell only includes them because of their ``historical
and/or sociological significance.'' The horizontal scales do not, as we shall
see, generate chords in the same way as the vertical scales, and for Russell
exist in a constant state of tension between the ``false'' tonic and the true
Lydian tonic.

\begin{figure}[tbp]
  \centerGraphic{eps/ch4/horizontal-scales.pdf}
  \caption{The four horizontal scales of the F Lydian Chromatic scale.}
  \label{lcc:horizontal-scales}
\end{figure}

Given that most of Russell's ideas on Lydian tonal organization have
disappeared from modern chord-scale theory, it is reasonable to ask why we
have devoted so much space to them here. My reason is twofold. First, much
modern scholarship does not seriously engage with Russell's ideas, and as a
result most theorists are not familiar with its first incarnation. Because the
original nature of chord-scale theory is tied up with that of Lydian tonal
organization, understanding the former is important in order to make sense of
the latter. Second, and more importantly, one of the goals of this
dissertation is to take jazz musicians' conceptions of harmony seriously:
chord-scale theory is an integral part of the way jazz today is taught, and
therefore understood by many practicing musicians. As we develop a
transformational system of chord-scales later in this chapter, we will seek to
revive some of Russell's initial conception.\fn{lcc-14}

\subsection{Chord--Scale Equivalence}
\label{subsec:chord-scales}

While Russell may have understood Lydian tonal organization to be the most
important part of his new theory, the part that has survived---flourished,
even---is his novel conception of chord--scale equivalence.\fn{lcc-15}
Russell's first mention of the concept explains its inception:%
%
\begin{quoting}
  \singlespacing
  In a conversation I had with Miles Davis in 1945, I asked, ``Miles, what's
  your musical aim?'' His answer, ``to learn all the changes (chords),'' was
  somewhat puzzling to me since I felt---and I was hardly alone in the
  feeling---that Miles played like he already knew all the chords. After
  dwelling on his statement for some months, I became mindful that Miles's
  answer may have implied the need to relate to chords in a new way. This
  motivated my quest to expand the tonal environment of the chord beyond the
  immediate tones of its basic structure, leading to the irrevocable conclusion
  that every traditionally definable chord of Western music theory has its
  origin in a \textsc{parent scale}. In this vertical sense, the term refers to
  that scale which is ordained---by the nature of tonal gravity---to be a
  chord's source of arising, and ultimate vertical completeness; the chord and
  its parent scale existing in a state of complete and indestructible
  chord/scale unity---a \textsc{chordmode}. (\lcc 10)
\end{quoting}

What Miles was looking for is essentially a way of determining
what notes he could play over a given chord. Simply knowing the chord tones
no longer seemed to be enough, since various extensions and alterations can
change the sound of the chord. Chord--scale theory is ultimately, then, an
improvisational expedient: a single scale stands in for a chord symbol.
Chord symbols with alterations are represented by different scales, and thus
musicians do not necessarily have to keep track of all of the individual chord
tones.

Russell's later explanation of the concept is uncharacteristically clear:%
%
\begin{quoting}
  \singlespacing
  The chord and its parent scale are an inseparable entity---the reciprocal
  sound of one another. \ldots\ In other words, the complete sound of a chord is
  its corresponding mode within its parent scale. Therefore, the broader term
  \textsc{chordmode} is substituted for what is generally referred to as ``the
  chord.'' (\lcc 20--21)
\end{quoting}
%
\noindent It is important to understand that for Russell, the two
terms---chord and scale---are truly equivalent: one does not substitute for
the other, rather one \emph{is} the other.\fn{lcc-16} Here, we arrive at the reason for
the inclusion of this material in a dissertation about jazz harmony. If we
take Russell seriously (I am arguing that we should), a harmony is a scale,
and vice versa. The two ideas are inseparable, and a study of harmony in jazz
would be incomplete without a commensurate discussion about scales.

At this point a brief overview of Russell's brand of chord--scale theory is
appropriate. This material accounts for the majority of the length of the
\emph{Concept}, so we will be careful here to avoid going into all of its
painstaking detail. Determining the scale that belongs with a particular chord
is a multi-step process: first, identify the parent Lydian scale; then,
determine the harmonic genre based on the characteristic modes of the Lydian
scale.\fn{lcc-17}

Russell goes through all seven modes of the Lydian scale, identifying the
``principal chords'' of each mode. These chords represent the purest form of
the mode, and the basis for the chord--scale matching process. An overview of
these is given in Table \ref{lcc:lydian-modes}; several things are worth
mentioning here.\fn{lcc-18} First, the order of modes in the table follows
Russell's order of presentation, which (though he does not explain it)
coincides with the frequency of each principal chordmode in jazz practice.
Second, the ``sub-principal chords'' are those which are also representative
of a given mode; they ``do not contain all the tones of [the] relative
Principal Chordmode,'' but they ``still exist in a state of unity with [the]
parent Principal scale'' (\lcc 23). Last, those modes with \textsc{b} in their
names refer to bass notes: the III Major (III\textsc{b}) group refers to major
chords with the third in the bass.

\begin{table}[tbp]
  \centerGraphic{eps/ch4/lydian-modes.pdf}
  \caption{Modes of the C Lydian scale.}
  \label{lcc:lydian-modes}
\end{table}

% xxx this paragraph is no good
The treatment of Mode V here bears special mention. The fifth mode of the
Lydian scale is of course the ordinary major scale, which Russell took great
pains to show earlier was \emph{not} a chord-generating scale.\fn{lcc-19} Its
role as the fifth mode of the Lydian scale is only to act as support for the
consonant Lydian harmony. The principal chordmode for this scale is the same
as that of the Lydian proper, with the fifth in the bass. This chord and its
relatives are by nature unstable (cf.\ a cadential $^6_4$ chord), and this
instability allows Russell to avoid a potential complication of the theory.

With the modes of the Lydian scale taken care of, Russell then goes on to show
how his seven vertical scales gives rise to other kinds of chords. To do so,
he introduces another bit of terminology, the Primary Modal Genre
(\abbrev{PMG}):%
%
\begin{quoting}
  \singlespacing
  A PMG is an assemblage of Principal Chord Families of similar type: a
  Principal Chord Family mansion housing the spectrum of variously colored
  Principal Chord Families of the same essential harmonic
  genre. (\lcc 29)
\end{quoting}
%
\noindent All of the principal chordmodes in Table \ref{lcc:lydian-modes} are
\abbrev{PMG}s, and the six other vertical scales---Lyd.\ augmented, Lyd.\
diminished, Lyd.\ flat seventh, Aux.\ augmented, Aux.\ diminished, and Aux.\
dim.\ blues---generate similar assemblages of chordal types.

Figure \ref{lcc:alternate-pmg} gives an example of how this works in practice.
The left-hand side of the figure shows the second mode of the C auxiliary
diminished scale, and the right-side gives its vertical expression as an
altered dominant chord: \h{D13s9b9b5}. Russell works through all of the modes
of the six other vertical scales, and the chart included with the book lists
almost all of these. The eight \abbrev{PMG}s fall into general categories,
which are shown in Table \ref{lcc:lydian-pmgs} and will be sufficient for our
purposes here.\fn{lcc-20}

\begin{figure}[tb]
  \centerGraphic{eps/ch4/alternate-pmg.pdf}
  \caption{The second mode of the C auxiliary diminished scale, in scalar and
    tertian formations.}
  \label{lcc:alternate-pmg}
\end{figure}

\begin{table}[tb]
  \centering
  \vspace{1em}
  \begin{tabular}{cl}
   Primary Modal Tonic & Primary Modal Genre \\
   \hline
   \rule[1em]{0ex}{1ex}%
   I    & major and altered major chords \\
   II   & seventh and altered seventh chords \\
   III  & [I] major and altered [I] major 3\textsc{b} (minor +5) chords \\
   +IV  & minor seventh \flat{}5 / [I] major +4\textsc{b} chords \\
   V    & [I] major and altered [I]5\textsc{b} chords \\
   VI   & minor and altered minor chords \\
   VII  & eleventh \flat{}9 / [I] major 7\textsc{b} chords \\
   +V   &  seventh +5 chords
  \end{tabular}
  \caption[The eight principal modal tonics and their associated modal
    genres.]{The eight principal modal tonics and their associated modal genres
    (Russell's example III:30).}
  \label{lcc:lydian-pmgs}
\end{table}

It is by now, I hope, apparent how the \emph{Concept} can simplify matters
somewhat for an improvising jazz musician. Once the process of matching chord
symbols with scales is learned, it is a relatively simple matter to determine
what notes will sound good over, for example, a \h{D13s9b9b5} chord. Russell
seems to have taken Miles's wish to ``learn all the changes'' to heart; he
takes care to note that indeed \emph{all} of the harmonies of Western music
can be found somewhere in the chart, and notes that many ``non-traditional
harmonic colors'' can be found as well (\lcc 29).

At the same time, though, it is probably also apparent that Russell's system
is somewhat more complicated than it needs to be. Most of the source of this
complication is in fact his ideas about Lydian tonal organization.
Russell provides an example of how to find the parent scale for an unadorned
\h{Eb7} chord (a very simple example):
%
\begin{quoting}
  \singlespacing
  Over the roman numerals of the scales of Chart A are listed different chord
  families. For example, over roman numeral II of the Lydian Scale are listed
  7th, 9th, 11th, and 13th chords. They belong to the same family: the (II)
  seventh chord family of a Lydian Chromatic Scale.

  The \h{Eb7} chord is found in this family above roman numeral II of the
  Lydian Scale in the right column of Chart A. The Lydian Scale is therefore
  the parent scale of the \h{Eb7} chord.

  Place the root of the \h{Eb7} chord on roman numeral II, and \Eflat\ becomes
  the second degree of that chord's parent scale.

  Think down a major 2nd interval; if \Eflat\ is the second degree of the
  parent scale, \Dflat\ is the first degree. Therefore \Dflat\ is the tonic
  (root) of the \h{Eb7} chord's parent scale. This tonic is called the Lydian
  tonic. For the \h{Eb7} chord, \Dflat\ is the Lydian Tonic and the parent
  scale is \Dflat\ Lydian. (\lcc 59)
\end{quoting}
%
\noindent This is quite a long process to determine that the most ingoing
(consonant) scale for an \h{Eb7} chord is the second mode of the \Dflat\
Lydian scale. The equivalence of chords and scales is genuinely useful for
improvising musicians, while the Lydian organization is more abstract. Faced
with this situation, jazz musicians made the obvious simplification: over an
\h{Eb7} chord, play the \Eflat\ Mixolydian scale.

Russell's theory becomes interesting, though, when we realize that any of the
member scales of the Lydian tonic can stand in for the ordinary Lydian. Once
you have determined that the parent scale of an \h{Eb7} chord is \Dflat\
Lydian, then it becomes easy to substitute more complicated scales built on
the same Lydian tonic. If you wanted to create a more dissonant (outgoing)
sound, you might instead play the second mode of the \Dflat\ Lydian flat
seventh scale; the second mode of the \Dflat\ auxiliary augmented blues scale
would be more dissonant still. Because the seven principal scales form a
spectrum of consonance to dissonance---as Russell frames it, there is a
progression of unity from ingoing to outgoing---the \emph{Concept} provides a
means of measuring how closely a particular progression or improvisation stays
to a particular Lydian tonic. This idea is the core of what we might borrow
from Russell, and we will return to it when we begin to develop our
transformational system in the next section.

\subsection{Chord--Scale Theory after Russell}
\label{subsec:chord-scale-reception}

Russell's fundamental insight about the nature of chords and scales was
revolutionary in jazz, and now forms the basis for much of modern jazz
pedagogy. Most of the later sources for chord--scale theory, as it has come to
be called, do not contain any mention of the Lydian generation of the tonal
system, or go through the fuss of finding a parent Lydian scale and its
associated \abbrev{PMG}. Some of these texts do not mention Russell at all,
which we might take as evidence that (not unlike Rameau's fundamental bass)
chord--scale equivalence is such a natural way of thinking about music that it
was taken for granted and no longer associated with its original author. Given
the influence of this theory in jazz pedagogy, it will be worthwhile to sketch
a brief outline of the literature here, if only to show how it differs from
Russell's conception.

Every textbook uses slightly different variations on the theory, but Mark
Levine's \emph{Jazz Theory Book} will serve here as a surrogate for the theory
in general.\fn{lcc-21} Levine divides his chapter on chord--scale theory into
four parts: major scale harmony, melodic minor scale harmony, diminished scale
harmony, and whole-tone scale harmony. For each of these families, he
describes the modes of the given scale and the harmonies (chord symbols)
associated with them.

Many later chord--scale theorists describe ``avoid notes'' in scales; these
are notes that are dissonant with the underlying harmony and should generally
be avoided in improvisations except as non-harmonic tones. This is an idea
that is not explicit in the \emph{Concept}, but many of the notes that are
described as ``avoid'' notes can be traced back to the fact that later
theorists do not take the Lydian scale as their starting point (\sd4 is
usually described as an avoid note on a major seventh chord). Levine's
first-choice chord-scales for major scale harmony and melodic minor scale
harmony are shown in Tables \ref{lcc:major-scale-harmony} and
\ref{lcc:minor-scale-harmony}, respectively, with avoid notes listed as
needed.\fn{lcc-22} Levine does not give the modes of the diminished and
whole-tone scales (for obvious reasons), and notes that the diminished scale
represents \h{7b9} and fully-diminished harmonies, while the whole-tone scale
can be played over \h{7s5} and \h{7alt.}\ harmonies.

\begin{table}[tb]
  \centering
  %\vspace{1em}
  \setlength{\tabcolsep}{12pt}
  \begin{tabular}{cl}
   Mode & Chord symbol \\
   \hline
   \rule[1em]{0ex}{1ex}%
   Ionian       & \h{Cmaj7} (avoid \sd4) \\
   Dorian       & \h{Dm7} \\
   Phrygian     & Esus\h{b9} \\
   Lydian       & \h{Fmaj7s4} \\
   Mixolydian   & \h{G7} (avoid \sd4); Gsus \\
   Aeolian      & \h{Amb6} \\
   Locrian      & \h{Bm7b5}
  \end{tabular}
  \caption[Levine's chord--scale description of C major scale
    harmony.]{Levine's chord--scale description of C major scale harmony
    (\emph{Jazz Theory Book}, 34).}
  \label{lcc:major-scale-harmony}
\end{table}

\begin{table}[tb]
  \centering
  \vspace{1em}
  \setlength{\tabcolsep}{12pt}
  \begin{tabular}{ccl}
   Mode & Chord symbol & Mode name \\
   \hline
   \rule[1em]{0ex}{1ex}%
   I    & \h{CmM7}    & minor-major         \\
   II   & Dsus\h{b9}  & --                  \\
   III  & \h{Ebmaj7s5}  & Lydian augmented    \\
   IV   & \h{F7s11}   & Lydian dominant     \\
   V    & \h{CmM7/G}  & --                  \\
   VI   & \h{Am7b5}   & half-diminished, Locrian \sharp{}2  \\
   VII  & \h{B7alt.}  & altered, diminished whole-tone
  \end{tabular}
  \caption[Levine's chord--scale description of C melodic minor scale
    harmony.]{Levine's chord--scale description of C melodic minor scale harmony
    (\emph{Jazz Theory Book}, 56).}
  \label{lcc:minor-scale-harmony}
\end{table}

From this brief description, we can see how Russell's theory is more-or-less
stripped of its philosophical underpinnings and used simply as a pedagogical
and performance tool. Though there are vestiges of the Lydian conception of
tonal space (Levine names the Lydian augmented and dominant scales), what
remains is only the idea of chord--scale equivalence. On the one hand, this is
certainly simpler: gone is the complicated derivation of Lydian parent scales,
in its place a simple one-to-one matching of scales with chord symbols.

At the same time, though, something seems lost. For Russell, Lydian
organization of tonal space was not incidental, but rather was the most
important of his ideas (the title of the book, after all, is \emph{The Lydian
Chromatic Concept of Tonal Organization} and not something like
\emph{Chord--Scale Equivalence and Jazz Improvisation}). Rather than simply
writing off Russell's more unusual ideas as eccentric ramblings, we will aim
in the next section to reincorporate some of them, in an effort to include some
of the ``first jazz theory'' back into modern scholarship.

\section{A Chord--Scale Transformational System}
\label{sec:cst}

% xxx rewrite this paragraph
Now that we have explored Russell's theory in some detail, we can take it as a
basis on which to construct a transformational system. Focusing on
chord-scales as first-class objects will allow us to take seriously the idea
that scales \emph{are} harmony, and will enable analytical observations about
the way improvising musicians understand harmony.

\subsection{Introduction: Scale Theory}
\label{subsec:scale-theory}

First, though, it will be useful to take a brief tour through other analytical
approaches that incorporate scales. Dmitri Tymoczko dedicates much of \emph{A
Geometry of Music} to the study of scales, and applies them analytically to
both twentieth-century music and jazz.\fn{cst-1} He is interested primarily in
voice-leading among scales, and constructs voice-leading spaces among the
diatonic, acoustic, harmonic major and minor, hexatonic, octatonic, and
whole-tone scales.\footcite[135]{tymoczko:2011} His conception of scales
though is somewhat different than Russell's; for Tymoczko, ``a scale is a
ruler,'' and provides a way of measuring musical
distance.\footcite[116]{tymoczko:2011} Russell's conception of scales though,
is a bit different. The \emph{Concept}'s view of scales overlaps somewhat with
Tymoczko's idea of a ``macroharmony'': for Russell, a scale acts more like a
set of notes that are all available for improvisation. In general, Russell is
not interested in common-tone connections between chords or their abstract
structure, and accordingly we will not have much occasion to draw on
Tymoczko's work here.\fn{cst-2}

Other authors have applied chord--scale theory to jazz, though in somewhat
different ways than we will do here. Garret Michaelsen, for example, draws on
Tymoczko's work on scalar voice-leading to construct networks for the music of
Wayne Shorter.\footcite{michaelsen:2012} Michaelsen does take seriously the
notion that chords and scales are equivalent, but his work is more interested
in determining how scalar structure can bring structure to harmony that is not
obviously functional. Stefan Love's work on parsimonious connections is
valuable for teaching students about chord-scales, but falls somewhat short,
since it does not include all of the scales Russell
identifies.\footcite{love:2009}

The work that intersects most closely with our work here is John Bishop's
dissertation, which incorporates chord-scales into a triadic transformational
system.\footcite{bishop:2012} Bishop is influenced by chord--scale theory as
it is taught at the Berklee College of Music, which is different in some ways
than Russell's theory outlined above.\fn{cst-3} He is also interested in
triadic approaches to improvisation; in his theory, chord-scales exist as a
means of generating these triads.\fn{cst-4} In this section we will not
restrict our focus to triads, but will instead consider chord-scales as
coherent wholes.

One of the problems facing any scale theory is the need to account for scales
of different cardinalities. Tymoczko's common-tone theory provides a means of
connecting a means of relating the whole-tone (6 notes), diatonic, acoustic,
harmonic major, (all 7), and octatonic (8) scales via ``split'' and ``merge''
operations, and indeed these four scales account for Russell's six of seven
vertical scales.\footcite[134--35]{tymoczko:2011} Tymoczko does mention the
ascending melodic minor scale, but it does not merit a place in his diagram
since it is not a nearly even 7-note scale. The system does not, though,
account for Russell's ``African-American blues scale'' (which we will call
simply the ``blues scale''), which has either 8 or 10 notes, depending on
whether you count \sd2 and \nat\sd7. While we could incorporate this scale into
the common-tone system---it is two splits and a semitone displacement from an
octatonic collection---chord--scale theory as it is usually taught does not
focus on common tones between chord-scales, but rather on determining what
scale captures the sound of a particular chord.\fn{cst-5}

Russell's Lydian tonic system, despite all of its seemingly unnecessary
complexity, does provide a simple solution to the cardinality problem. Because
all non-diatonic scales have a Lydian scale as their ultimate source (their
parent scale), this means we can understand these other scales as alterations
of some diatonic scale. All of the scales in Figure
\ref{lcc:principal-scales} (p.~\pageref{lcc:principal-scales}), for example,
are derived from the F Lydian diatonic collection: the D melodic minor
collection (Lyd.\ augmented), F harmonic major (Lyd.\ diminished), F acoustic
(Lyd.\ flat seventh), whole-tone (\abbrev{WT}\tsub{1}, aux.\ augmented), and
two octatonic scales (\abbrev{OCT}\tsub{02} and \abbrev{OCT}\tsub{12}, the
auxiliary diminished scales). Russell's four horizontal scales, of which the
blues scale is the most important, also have a Lydian tonic, and can be
understood as still another variation on the Lydian collection.\fn{cst-6}

\subsection{A GIS Proper}
\label{subsec:chord-scale-gis}

This reduction to a single diatonic collection will be our first step in
devising a transformational system for chord-scales. Instead of referring to a
scale's parent Lydian tonic as a Lydian scale, we will instead refer to it by
a key signature: we will call the D Lydian collection 3\sharp, the \Eflat\
Lydian collection 2\flat, and the F Lydian collection simply \nat, and so on.
This notation is in common use and, helpfully, eliminates some of the
awkwardness of having to refer constantly to the Lydian mode. The second mode
of the F Lydian scale is of course the same as the G Mixolydian scale, and
both refer to the collection \nat.

\begin{table}[tbp]
  \centering
  \begin{tabular}{rl}
    0. & Lydian (diatonic) \\
    1. & Lydian augmented \\
    2. & Lydian diminished \\
    3. & Lydian dominant \\
    4. & Whole-tone \\
    5. & Whole-half diminished \\
    6. & Half-whole diminished \\
    7. & Blues scale
  \end{tabular}
  \caption{A scale index inspired by Russell, listed from most consonant to
    most dissonant.}
  \label{cst:scale-index}
\end{table}

It is not yet clear, though, how Russell's other member scales might be
incorporated into this system. To do so, first we will first introduce the
concept of a \emph{scale index}, as shown in Table \ref{cst:scale-index} (they
are numbered from 0 to 7 for reasons that will become clear shortly).
Several things are worth noting about this table that are different from
Russell's presentation. First, some of the scale names
have been changed to reflect their common usage; we no longer need to remember
which of the diminished (octatonic) scales is the ``blues'' variant.\fn{cst-7}
I have maintained Russell's names when they clarify the relationship to the
parent scale: scale 1 remains ``Lydian augmented'' rather than ``melodic
minor,'' since the D melodic minor scale has F Lydian, not D Lydian as its
parent scale (we will return to this point shortly).

There are eight scales in our scale index, but Russell gives eleven member
scales. The reason for this is a practical one: two of the horizontal scales
are simply diatonic modes (the major scale and major flat seventh), and the
third is a major scale with an additional \sharp{}\sd5 (major augmented
fifth), which we can usually understand as a chromatic passing tone. The blues
scale, though, does appear frequently in jazz, and merits its own place
here.\fn{cst-8} This scale is given last in the order because it is one of
Russell's horizontal scales, which are inherently more outgoing than their
vertical companions.

With some mathematical sleight of hand, we can define a chord-scale \gis using
our scale index along with a diatonic collection as above. Elements of this
\gis have the form \scalepair{\emph{diatonic collection}}{\emph{scale name}};
the F Lydian collection is described by the pair \scalepair{\nat}{Lydian},
while \scalepair{2\flat}{Lyd.\ dominant} describes the \Eflat\ acoustic
collection. Creating an ordered-pair \gis of course requires us to show that
both elements are part of a mathematical group. Though this will not be its
final form, we will define this chord-scale \gis more formally
here, with the knowledge that it will be refined before we are finished. It is
important to realize that this \gis is designed to reflect Russell's own views
of chord--scale equivalence. There are only three distinct octatonic
collections, for example, but the \gis contains 24 distinct diminished scales:
whole-half and half-whole diminished scales on all twelve Lydian tonics.

The first element of the pair is a key signature, which have been studied in a
transformational context by Julian Hook.\fn{cst-9} Because we are interested
in collections in jazz (often a non-notated music), we will consider
enharmonically equivalent key signatures (like 6\sharp\ and 6\kern0.8pt\flat) to be
identical. There are, then, only twelve key signatures, operated on by the
sharpwise and flatwise transformations, $s_n$ and $f_n$, which add $n$ sharps
or flats to a key signature, respectively; we might write \mbox{1\sharp
$\xrightarrow{\ s_1\ }$ 2\sharp}, \mbox{2\flat $\xrightarrow{\ f_2\ }$ 4\flat}, or
\mbox{1\flat $\xrightarrow{\ s_3\ }$ 2\sharp}.\fn{cst-10}

The scales in our scale index do not obviously form a group, but we can define
the eight scales to be isomorphic to $\intZ_8$, the integers mod 8 (this
definition is somewhat artificial, and we will relax it shortly). The eight
scales do form a progression from consonance to dissonance, and for Russell it
is true that the whole-tone scale is in some sense further away from the
Lydian tonic than the Lydian augmented scale.\fn{cst-11} They are not,
however, cyclic in any meaningful way: it is not as though, for example, the
blues scale is the most dissonant scale and if you take one more step you
arrive back at the Lydian scale. Nor are the metaphorical distances of
consonance between the scales really consistent: though Russell does not
define them (and will not attempt to do so here), the consonant distance
between the two diminished scales seems much less than the distance between
the whole-tone and whole-half diminished scales.\fn{cst-12}
If we accept these limitations of the scales' group structure, though, we gain
all the benefits of a \gis. Intervals between scales are calculated as
integers mod 8, using the scale labels from Table \ref{cst:scale-table}: the
interval from a Lydian augmented scale to a whole-tone scale if $3$ ($=4-1$),
from a whole-tone scale to a half-whole diminished scale is $2$, and so on.

We will call a transformation between chord-scales $R$, after Russell; these
transformations have the form $R$\scalepair{\emph{signature
    transformation}}{\emph{scale index interval}}. This transformation acts on
elements of the \gis in a pairwise fashion in the usual way. We can understand
a passage like the one in Figure \ref{cst:gis-fiveone} as expressing the
$R$\scalepair{$s_1$}{$0$} transformation: the collection changes but the scale does not.
Figure \ref{cst:gis-blues-lyd} shows a passage which begins with the F blues
scale and resolves to the F Lydian collection, representing the transformation
$R$\scalepair{$e$}{$1$}, where $e$ indicates the identity element.\fn{cst-13}

\begin{figure}[tbp]
  \centerGraphic{eps/ch4/gis-fiveone.pdf}
  \caption{A typical V--I jazz lick, along with its chord-scale \textsc{GIS}
    analysis.}
  \label{cst:gis-fiveone}
\end{figure}

\begin{figure}[tbp]
  \centerGraphic{eps/ch4/gis-blues-lyd.pdf}
  \caption{A resolution from the F blues scale to the F Lydian scale.}
  \label{cst:gis-blues-lyd}
\end{figure}

\FloatBarrier
\subsection{Relaxing the GIS}
\label{subsec:relaxing-gis}

This initial pass at a chord-scale \gis is a useful first approximation, but
there are some aspects of it that are not clear. How, for instance, should we
determine what scales match with what diatonic collections? In some cases the
answer is clear, but in others it is not. In Figure \ref{cst:gis-fiveone}
above, for example, we labeled the \h{G7} chord as \scalepair{\nat}{Lyd.},
rather than, say, \scalepair{1\sharp}{Lyd.}; the \sd4 that would confirm
either is absent. The problem seems to become even more intractable when we
encounter the symmetrical scales: a diminished scale has eight possible parent
diatonic collections.

\begin{table}[p]
  \centerGraphic[angle=90,height=.9\textheight]{eps/ch4/scale-table.pdf}
  \caption{Common chords in the modes of the F Lydian Chromatic scale.}
  \label{cst:scale-table}
\end{table}

Here again, we will turn to George Russell. Table \ref{cst:scale-table}
presents a portion of the foldout chart from the \emph{Lydian Chromatic Concept}
in somewhat simplified form (it may be useful to compare this table with Table
\ref{lcc:lydian-pmgs} on p.~\pageref{lcc:lydian-pmgs}). The top of this table
gives the eight scales in the scale index of the previous section, while the
left side lists the modal tonics.\fn{cst-14} Only modes that give rise to
common chords are shown in this table; notably absent are modes III and V,
which are given by Russell as tonic chords with altered bass notes. The most
common chords and scales (which are also the most ingoing) appear on the left
side of the table, and rarer chords and scales appear nearer the right side.
Alternate names for scales, when they exist, are given in italics in the
appropriate box.

It is important to realize that for Russell the modal degrees (what he calls
primary modal tonics) are roughly equivalent to functional categories. This is
important to us here because it helps us to determine a scale's parent
diatonic collection. All of the chords in the top row of the table are
first-mode scales, and act like tonic chords. The \h{F7} in the top row of the
Lydian \flat{}7 column thus represents a major-minor seventh chord acting as
tonic (Russell actually gives this chord symbol as ``Maj \flat{}7'' or ``Maj
9th \flat{}7''). Likewise, dominant chords appear mostly in mode II, minor
seventh chords appear in mode VI, and half-diminished sevenths in mode
+IV.\fn{cst-15}

Of course, the pairing of modes and scales is still not unique, and is
ultimately a question of analysis. Consider the scale in Figure
\ref{cst:ambiguous-acoustic}a. This is an F acoustic scale, which can appear
as an F Lydian \flat{}7 scale or as the second mode of an \Eflat\ Lydian
augmented scale.\fn{cst-16} The \gis allows us to show this, since ``F
acoustic as tonic'' is a different \gis member than ``F acoustic as
dominant.'' Figure \ref{cst:ambiguous-acoustic}b places the ambiguous scale in
the context of a \tfo progression in \Bflat\ (you might imagine this is the
last four bars of a solo on George Gershwin's ``I Got Rhythm''). Here, the
acoustic scale clearly functions as a dominant, and would be labeled
\scalepair{2\flat}{Lyd.\ aug.}: the parent collection is \Eflat\ Lydian
(2\flat), and this is a mode of the Lydian augmented scale (number 2 in the
scale index of Table \ref{cst:scale-index}). In contrast, Figure
\ref{cst:ambiguous-acoustic}c places the fragment in an F blues \tfo
progression (the end of Charlie Parker's ``Now's the Time,'' perhaps). Because
the collection now functions as a tonic chord, it represents the \gis member
\scalepair{\nat}{Lyd.\ \flat{}7}.

\begin{figure}[tbp]
  \centerGraphic{eps/ch4/ambiguous-acoustic.pdf}
  \caption{An ambiguous acoustic scale and two concrete presentations of it.}
  \label{cst:ambiguous-acoustic}
\end{figure}

The intuitions captured by the chord-scale \gis here are not quite like those
represented by other theories of chord-scales. Indeed, the fact that both
\scalepair{\nat}{Lyd.\ \flat{}7} and \scalepair{2\flat}{Lyd.\ aug.} refer to
the same 7-element set is not immediately apparent in the \gis itself.
Theories that prioritize voice-leading would likely include the F acoustic
collection only once, since the voice leading from this scale to itself is
maximally efficient (no voices move at all). Nor is it enough simply to label
the scale as the F Lydian dominant scale, as this does not capture the
difference in function between the passages in Figure
\ref{cst:ambiguous-acoustic}b--c.

The \gis in fact is one of \emph{functional} or \emph{heard} chord-scales. In
this way, it is perhaps most like Steven Rings's \gis for ``heard scale
degrees.''\footcite[44--50 and throughout]{rings:2011} Rings argues that scale
degrees are a perceived, rather than inherent, quality of music, and thus
listeners have different experiences when hearing ``A4 as \sd1'' and ``A4 as
\sd7.''\footcite[42]{rings:2011} In many ways, this distinction is like the
one at the heart of what Russell was trying to accomplish in the \emph{Lydian
  Chromatic Concept}. For Russell, the F acoustic collection played over a
tonic major-minor seventh chord really \emph{is} a different entity than the
same collection played over a dominant major-minor seventh chord. This is what
Russell means when he writes of ``a state of complete and indestructible
chord/scale unity'' (\lcc 10): if an \h{F7} chord can function in more than
one way, so too can its corresponding scale.

Previous chapters have constructed various musical spaces in which to analyze
passages, and we can turn this \gis into a space as well. Before doing that,
though, we will relax its definition such that the scale indexes are no longer
isomorphic to $\intZ_8$. We noted above that this isomorphism was somewhat
artificial, and the space becomes more intuitive if we simply use the
non-modular integers 0--7 (which we will call $S$ for the time being). The
resulting space, though, runs afoul of the formal requirements for a \gis,
which requires that \ivls form a mathematical group. The set $S$ under
addition does not form a group, since it is not closed ($6+5$
is not a member of $S$\,), it is not associative ($1+1$ and $6+1$ are defined,
but $[1 + 6] + 1$ is not), and elements do not have inverses.

This is of course one of the well-known limitations of \gis{}es: the musical
spaces must be continuous and infinite.\fn{cst-18} A \gis cannot account for
musical spaces that are discontinuous or have ``boundaries,'' which is the
case here: it is not possible to conceive of a scale in our system which is
more ingoing than the Lydian scale, or more outgoing than the blues
scale.\fn{cst-19} Nevertheless, the transformations developed above still seem
to be reasonable reflections of intuitions about the nature of chord-scales.
As Hook notes, ``the narrative portions of Lewin's analyses [in \emph{GMIT}]
generally far transcend the logical consequences of the group structure,'' so
the fact that a mathematical group does not underlie this no-longer-\gis
should not dissuade us from exploring its analytical
potential.\footcite[185]{hook:2007gmit}

Lewin does allow for semigroups of transformations, but the scale index
transformations do not form a semigroup either (a semigroup must still be
closed under the group action). All possible intervals for the scale indexes
(the integers 0--7) are contained in the set $\{-7, -6, \ldots, 6, 7\}$, which
forms neither a group nor a semigroup. It does contain the additive identity
($0$), and every element has an inverse, so we can use this set under addition
in practically the same way. Every interval is well-defined, but all intervals
are not possible from every scale. For example: \mbox{int(Lyd., Whole-tone) =
  $4$}, and \mbox{int(Whole-tone, Lyd.) = $-4$}, but there is no scale which
satisfies $x$ in the equation \mbox{int(Blues, $x$) = $4$}, since there is no
scale that is four levels more outgoing than the blues scale.

With these caveats, we can visualize this space using the diagram in Figure
\ref{cst:polar-gis}. In this figure, the diatonic (Lydian) scale is centrally
located, with more outgoing scales located further toward the outside; these
concentric circles combine with the ordinary circle of fifths to divide each
scale into diatonic wedges.\fn{cst-17} The figure is inspired by Russell's
description of the Lydian as a ``musical `Star-Sun'\,'' (\lcc 8) and the
ultimate source of tonal gravity. More outgoing scales have more gravitational
potential energy, as it were, and are more dissonant with the underlying
diatonic collection. We can use this figure to map the two presentations of
the F acoustic scale of Figure \ref{cst:ambiguous-acoustic}; such a mapping is
given in Figure \ref{cst:polar-acoustic}. This visualization makes clear that
the second presentation (a tonic \h{F7} chord) is more outgoing than the first
(a dominant \h{F7} chord), as well as the shift in underlying diatonic
collection (\Eflat\ Lydian vs.\ F Lydian).

\begin{figure}[tbp]
  \centerGraphic{eps/ch4/polar-gis.pdf}
  \caption{A ``planetary'' model of the chord-scale \textsc{GIS}.}
  \label{cst:polar-gis}
\end{figure}

\begin{figure}[tbp]
  \centerGraphic{eps/ch4/polar-acoustic.pdf}
  \caption[Two presentations of the F acoustic scale in the planetary
  model.]{Two presentations of the F acoustic scale, shown in red, in the
    planetary model (compare Figure \ref{cst:ambiguous-acoustic}).}
  \label{cst:polar-acoustic}
\end{figure}

Figure \ref{cst:polar-acoustic} also reveals that perhaps the objects in the
system could be more informative. The right side of this figure represents a
\tfo in F with the sequence \\
{ \centering
  \scalepair{\flat}{Dia.} $\xrightarrow{R\text{\scalepair{$e$}{$0$}}}$
  \scalepair{\flat}{Dia.} $\xrightarrow{R\text{\scalepair{$s_1$}{$3$}}}$
  \scalepair{\nat}{Lyd.\ \flat{}7}.
  \par
}
\noindent That is, the G Dorian scale and C Mixolydian scale are both
represented by the pair \scalepair{\flat}{Dia.}. On one level, this makes
sense: both scales are modes of the \Bflat\ Lydian (or F major) scale. Still,
since a chord-scale is supposed to represent a ``complete and indestructible
unity,'' it seems appropriate to add some information about the chord into the
system proper. As it stands, information about the chords themselves is
separate from the transformations, and there is no way to distinguish the
progression above (a \tfo in F) from, say, the nonsensical progression
\mbox{\h{Em7b5}--\h{Fmaj7}--\h{Bm7b5}}.

We could solve this problem in several different ways, but the most obvious is
to include the chord symbol itself in the chord-scale representation. This
results in what we will call a chord-scale triple of the form
\mbox{$\langle$\emph{chord symbol}, \emph{diatonic collection}, \emph{scale
    name}$\rangle$}. This construction will allow us to draw on the work done
in the previous chapters developing a system of transformations for chord
symbols; we will still call the resulting transformations $R$, but the first
element of the triple will be a transformation between chord
symbols.\fn{cst-20} The F-major \tfo of Figure \ref{cst:ambiguous-acoustic}c
thus becomes \\
{ \centering \cst{Gm7}{\flat}{Dia.}
  $\xrightarrow{R\text{\rtrans{TF}{$e$}{$0$}}}$ \cst{C7}{\flat}{Dia.}
  $\xrightarrow{R\text{\rtrans{TF\tsub{blues}}{$s_1$}{$3$}}}$
  \cst{F7}{\nat}{Lyd.\ \flat{}7}.
  \par
}
\noindent Formally, the transformations of the last two chapters act on
ordered triples of chord root, third, and seventh, but this fact need not
concern us too much here. Recall that the chord symbol transformations are
cross-type transformations, and thus even if we had not already relaxed the
\gis of the previous section (removing the cyclic group $\intZ_8$), the new
version with chord symbols cannot form a \gis.

While adding chord symbols to the system does clarify matters, it also
complicates them. In particular, the planetary model of Figure
\ref{cst:polar-gis} no longer represents the musical space accurately. The
addition of chord symbols means that a copy of the planetary model exists at
every location a chord symbol appears in the previous chapters.\fn{cst-21}
Such a visual space would be forbiddingly complex---imagine the thirds spaces
of Figures \ref{mts:m3-space} or \ref{maj3:maj3-space} redrawn with the
additional chord-scale models. This is a sacrifice made consciously so that
the chord-scale triples and $R$ transformations are clear in the text. In
practice, we can use either the planetary model or the chord spaces of the
previous chapters as the situation demands, with the knowledge that both
representations exist simultaneously as part of the single conceptual space of
chord-scales. With the final version of the chord-scale transformational
system now in place, the stage is now set to turn toward actual jazz performance,
which we will do in the next section.

% xxx Do we need something here about how the size of the new non-group? It's
% not infinite, but it's also not particularly countable...

\section{Chord-Scales in Analysis}
\label{sec:chord-scale-analysis}

Analyzing jazz performances is inherently more complicated than the
lead-sheet analysis done in previous chapters. Because jazz is primarily an
improvised music, our analyses here will rely on transcriptions, which carry
with them their own set of problems.\fn{csa-1} As Steve Larson notes, any
transcription is also a kind of analysis: in many cases it is not at all clear
how a particular recorded sound should be (or indeed, whether it can be)
rendered in Western musical notation.\footcite[1--2]{larson:2009} In the
transcriptions in this dissertation, I focus primarily on pitches and rhythms,
since they are most relevant to our discussion of jazz harmony. As such, many
of the most important aspects of a performance---dynamics, articulation,
timbre, intonation, and so on---are all absent from the notation.\fn{csa-2}

The three short analyses that follow are all of solos of tunes analyzed in the
first three chapters; this will allow us the opportunity to discover how the
abstract chord progressions are realized in improvised performance. They are
also solos by tenor (and soprano) saxophonists: Rahsaan Roland Kirk, Gene
Ammons, Sonny Stitt, and Joe Henderson. This selection reflects some of my own
preference for saxophonists, but also allows us a basis for comparison:
different instruments have different idiomatic patterns. Notably absent from
this list are Charlie Parker and John Coltrane, undoubtedly the two most
famous saxophonists. As noted in Chapter 1, this dissertation is interested in
jazz harmony in the general sense; by focusing on musicians who do not
commonly appear in works of music theory, we gain insight into the lingua
franca of jazz, rather than the peculiarities of Parker's or Coltrane's
style.\fn{csa-3}

\subsection{Introduction: Rahsaan Roland Kirk, “Blues for Alice”}
\label{subsec:kirk-blues-for-alice}

Multi-saxophonist Rahsaan Roland Kirk's recording of ``Blues for Alice'' from
\emph{We Free Kings} (1961) will serve as a useful introduction both to the
issues of analyzing improvised performance and to the analytical utility of
the chord-scale transformations of the last section. The complete
transcription of the performance can be found on
p.~\pageref{transcription:blues-for-alice}, and the analysis of the chord
progression of this tune is in Section \ref{sec:blues-for-alice}. Kirk often
plays multiple saxophones simultaneously; each instrument is given its own
staff in the transcription.

One of the main problems of analyzing chord-scales is determining exactly
which notes should be taken as part of the scale, and which are simply
embellishing. If the principal argument of this chapter is that scales
\emph{are} harmony, then the question becomes is one of non-harmonic tones.
Sometimes it is obvious that notes are embellishing, but other cases are not
so clear. The G\sharp\ in Figure \ref{csa:non-harmonic-tones}a, for example,
is clearly a chromatic passing tone between G (the fifth of the \h{Cm7} chord)
and A (the third of \h{F7}). Figure \ref{csa:non-harmonic-tones}b presents a
more complicated case: is the \Eflat in the scale, with E\nat{} serving as a
chromatic passing tone, or vice versa? If we choose \Eflat as the main note,
the scale implied is \Bflat Mixolydian, while E\nat{} gives a \Bflat Lydian
dominant scale. The choice has analytical implications, as the two scales
represent two different locations in chord-scale space. It is important to
note that the kind of non-harmonic tones we are discussing here are not quite
like ordinary non-harmonic tones. Larson (and many others) would argue that
\emph{both} the \Eflat and E\nat\ are non-harmonic, since at some deeper
level they would reduce to either D or F.\footcite[5--10]{larson:2009} Because
we have broadened the definition of ``harmony'' to include chord-scales, our
idea of what is ``non-harmonic'' must also change accordingly.

\begin{figure}[tbp]
  \centerGraphic{eps/ch4/non-harmonic-tones.pdf}
  \caption[Two non-harmonic tones in Kirk's solo]{Two non-harmonic tones, from
    mm.~112 and 17, respectively, of Kirk's solo on ``Blues for Alice.''}
  \label{csa:non-harmonic-tones}
\end{figure}

One kind of non-harmonic tone figure merits special attention, which Jerry
Coker calls the ``enclosure,'' where a pitch is approached by half steps on
either side.\footcite[50--54]{coker:elements} Two examples of enclosures
appear in the \tfo progression in Figure \ref{csa:enclosures}. The first
appears before the \Bflat on beat 2 of the first bar, and the second before
the resolution to A at the end of the passage. What is interesting about
enclosures from our perspective here is that usually only one of the neighbors
note is not part of the scale. In the first example, only the B\nat\ is truly
non-harmonic; the A is a part of the (very clear) G Dorian scale. Likewise,
the B\nat\ in the next measure is a chromatic passing tone in the C Mixolydian
scale between C and \Bflat, while G\sharp\ is a lower neighbor to the
following A.

\begin{figure}[tbp]
  \centerGraphic{eps/ch4/enclosures.pdf}
  \caption[Two examples of enclosures in a \tfo progression.]{Two examples of
    enclosures (marked with brackets) in a \tfo progression, from mm.~33--35
    of Kirk's solo.}
  \label{csa:enclosures}
\end{figure}


% here: bebop scale, barline shifts



%%% Local Variables:
%%% mode: latex
%%% TeX-master: "../diss"
%%% End:
