%% Chapter 4

\chapter{Chord-Scale Transformations}

For most jazz musicians, a chord symbol implies not only its root, third, and
seventh, but also a variety of other notes as well (its fifth, ninth, and
potentially other chord members). Instead of conceiving of the extended
harmonies common in jazz as stacks of thirds, musicians often describe harmony
in a more linear fashion, using a scale to stand in for a chord symbol. What
is often called ``chord-scale theory'' is a major part of jazz pedagogy, and
cannot be ignored as we try to approach a general theory of jazz harmony. This
chapter will begin with an introduction to chord-scale theory, both in its
original form and its later pedagogical adaptations, and then move on to
incorporate it into the transformational system developed thus far; finally,
we will see how the theory allows us to make analytical insights about more
than just chord symbols.

\section{George Russell’s \emph{Lydian Chromatic Concept}}
\label{sec:lcc}

The ultimate origin of chord-scale theory is George Russell's \emph{Lydian
  Chromatic Concept of Tonal Organization}, first published in 1953 but
revised several times throughout Russell's life.\fn{lcc-1} Russell was a jazz
pianist, drummer, and a well-known composer and arranger (he is perhaps best
known for his composition ``Cubana Be/Cubana Bop'' for the Dizzy Gillespie big
band); he devised the majority of the \emph{Lydian Chromatic Concept} while
hospitalized for tuberculosis in 1945--46.\fn{lcc-2} The influence of the
\emph{Concept} (as it is often called) is difficult to overstate.
Joachim-Ernst Berendt and Günther Heusmann decribe it as ``the first work
deriving a theory of jazz harmony from the immanent laws of jazz, not from the
laws of European music,'' and the blurbs on the back cover contain praises
from musicians including Gil Evans, Ornette Coleman, Eric Dolphy, and Toru
Takemitsu.\fn{lcc-3}

Despite its importance to jazz theory, though, Russell's work has not received
much attention in music-theoretical scholarship on jazz. Dmitri Tymoczko, for
example, does not mention Russell at all in his survey on the pedagogical use
of chord-scales in jazz, and the only mention of Russell in his book is in a
footnote unrelated to Russell's contributions to chord-scale theory.\fn{lcc-4}
There may be many reasons for this---Russell's circuitous and hard-to-follow
prose are probably not least among them---but regardless, a brief introduction
to Russell's theories as he conceived them will be in order here. The
\emph{Lydian Chromatic Concept} can be divided into two main components, which
we will treat separately in the following sections: Lydian tonal organization and
chord-scale equivalence.

\subsection{Lydian Tonal Organization}
\label{subsec:lydian-org}

%%% Local Variables:
%%% mode: latex
%%% TeX-master: "../diss"
%%% End:
