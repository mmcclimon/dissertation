%% Chapter 4

\chapter{Chord-Scale Transformations}

For most jazz musicians, a chord symbol implies not only its root, third, and
seventh, but also a variety of other notes as well (its fifth, ninth, and
potentially other chord members). Instead of conceiving of the extended
harmonies common in jazz as stacks of thirds, musicians often describe harmony
in a more linear fashion, using a scale to stand in for a chord symbol. What
is often called ``chord-scale theory'' is a major part of jazz pedagogy, and
cannot be ignored as we try to approach a general theory of jazz harmony. This
chapter will begin with an introduction to chord-scale theory, both in its
original form and its later pedagogical adaptations, and then move on to
incorporate it into the transformational system developed thus far; finally,
we will see how the theory allows us to make analytical insights about more
than just chord symbols.

\section{George Russell’s \emph{Lydian Chromatic Concept}}
\label{sec:lcc}

The ultimate origin of chord-scale theory is George Russell's \emph{Lydian
  Chromatic Concept of Tonal Organization}, first published in 1953 but
revised several times throughout Russell's life.\fn{lcc-1} Russell was a jazz
pianist, drummer, and a well-known composer and arranger (he is perhaps best
known for his composition ``Cubana Be/Cubana Bop'' for the Dizzy Gillespie big
band); he devised the majority of the \emph{Lydian Chromatic Concept} while
hospitalized for tuberculosis in 1945--46.\fn{lcc-2} The influence of the
\emph{Concept} (as it is often called) is difficult to overstate.
Joachim-Ernst Berendt and Günther Heusmann decribe it as ``the first work
deriving a theory of jazz harmony from the immanent laws of jazz, not from the
laws of European music,'' and the blurbs on the back cover contain praises
from musicians including Gil Evans, Ornette Coleman, Eric Dolphy, and Toru
Takemitsu.\fn{lcc-3}

Despite its importance to jazz theory, though, Russell's work has not received
much attention in music-theoretical scholarship on jazz. Dmitri Tymoczko, for
example, does not mention Russell at all in his survey on the pedagogical use
of chord-scales in jazz, and the only mention of Russell in his book is in a
footnote unrelated to Russell's contributions to chord-scale theory.\fn{lcc-4}
There may be many reasons for this---Russell's serpentine and hard-to-follow
prose are probably not least among them---but regardless, a brief introduction
to Russell's theories as he conceived them will be in order here. The
\emph{Lydian Chromatic Concept} can be divided into two main components, which
we will treat separately in the following sections: Lydian tonal organization and
chord-scale equivalence.

\subsection{Lydian Tonal Organization}
\label{subsec:lydian-org}

Russell's central insight---indeed, the Concept itself---is that the
Lydian scale is more fundamental than the major scale. He offers many
explanations, but is easiest to demonstrate, as he does, with an example.
Figure \ref{lcc:tertian-stacks} reproduces Russell's first example; he
provides the following instructions and explanation:%
%
\begin{quoting}
  \singlespacing
  Sound both of the following chords separately. Try to detect the one which
  sounds a greater degree of unity and finality with its tonical [\emph{sic}] C
  major triad. \ldots\ In tests performed over the years in various parts of the
  world, the majority of people have repeatedly chosen the second chord---the C
  Lydian Scale in its tertian order.\footcite[1]{russell:lcc}
\end{quoting}%
%
The lowest note of a stack of six perfect fifths is what Russell calls the
``Lydian tonic''; the B--F tritone in the C major scale ``disrupts the perfect
symmetry of the fifths.''\footcite[4]{russell:lcc} He does concede that C is
also understood as tonic in the major scale arrangement, but that it does not
sound resolved, since ``the presence of the Lydian \emph{do} on the major
scale's fourth degree permanently denies [it] that
possibility.''\footcite[4]{russell:lcc} For Russell, the major scale is always
in a state of tension, wanting to ``resolve'' to the Lydian.\fn{lcc-5}

\begin{figure}[tbp]
  \centerGraphic[width=16em]{eps/ch4/tertian-stacks.pdf}
  \caption[Two stacks of thirds, using the major scale and the Lydian
    scale.]{Two stacks of thirds, using the major scale (left) and the Lydian
    scale (right). Russell's Example I:1.}
  \label{lcc:tertian-stacks}
\end{figure}

Closely allied with the Lydian tonic is the concept of tonal gravity, which
Russell calls the fundamental principle of the \emph{Concept}. In a stack of
fifths, tonal gravity flows downward: ``the tone F\sharp\ yields to B as its
tonic---F\sharp\ and B surrender ``tonical'' authority to E, and so on down the
ladder of fifths---the entire stack conferring ultimate tonical authority on
its lowermost tone, C.''\fn{lcc-6} The concept of tonal gravity provides the
justification for the primacy of the Lydian scale, since the major scale
cannot be constructed by generating perfect fifths from its tonic.

Its theoretical justifications aside, the Lydian scale has an almost mythical
quality to Russell, which can sometimes be off-putting. A somewhat longer
passage from the \emph{Concept} will help to illustrate Russell's fascination
with the scale, as well as his usual circuitous mode of presentation (the
emphasis and non-bracketed ellipses are original):
%
\begin{quoting}
  \singlespacing
  The \textsc{Lydian Tonic}, as the musical ``Star-Sun,'' is the seminal
  source of tonal gravity and organization of a Lydian Chromatic scale.
  [\ldots] \textsc{Unity} is the state in which the Lydian Scale exists in
  relation to its I major and VI minor tonic station chords, as well as those
  on other scale degrees. Unity \emph{is} \ldots\ instantaneous completeness
  and oneness in the \emph{Absolute Here and Now} \ldots\ above linear time.

  The Lydian Scale is the musical \emph{passive} force. Its unified tonal
  gravity field, ordained by the ladder of fifths, serves as a theoretical
  basis for tonal organization within the Lydian Chromatic Scale and,
  ultimately, for the entire Lydian Chromatic Concept. There is no ``goal
  pressure'' within the tonal gravity field of a Lydian Scale. The Lydian
  Scale exists as a self-organized \emph{Unity} in relations to its tonic tone
  and tonic major chord. The Lydian Scale implies an evolution to higher
  levels of tonal organization. The Lydian Scale is the true scale of tonal
  unity and the scale which clearly represents the phenomenon of tonal gravity
  itself.\fn{lcc-7}
\end{quoting}

Russell's logic is, of course, circular: the Lydian tonic is by definition the
note that is the bottom of a stack of six perfect fifths, and the principle of
tonal gravity confers a special status on the bottom of a stack of six fifths
(conveniently, the Lydian tonic). Partly for this reason, this part of
Russell's theory has not really been taken seriously by modern scholars. He
never gives a reason, for example, that the stack should not be extended
further: wouldn't the lowest note of a stack of \emph{seven} fifths be imbued
with even more tonal gravity? This complication reappears when Russell later
presents the complete ``Lydian Chromatic Order of Tonal Gravity'' (starting
on F):\footcite[12]{russell:lcc} \\
{\centering F C G D A E B C\sharp\ \Aflat\ \Eflat\ \Bflat\ \Gflat \par}

\noindent What should be a perfect fifth from B to F\sharp\ is replaced by a
whole step, so that the minor ninth (F--\Gflat) does not appear until the last
note. This sleight of hand also prevents there from being more than one
succession of six perfect fifths in the series. (If it had continued in
perfect fifths, there would be by definition seven Lydian tonics!)

% xxx I'm not sure this paragraph makes sense
Still, though, is there anything worth saving in Russell's ideas? He is
probably right that most people prefer the sound of the stack of thirds on the
right of Figure \ref{lcc:tertian-stacks}, with the F\sharp.\fn{lcc-8} And the
Lydian scale \emph{does} have some practical advantages over the major scale.
As Russell points out, \sharp\sd4 appears before \nat\sd4 in the harmonic
series (he does not mention that \flat\sd7 appears before \nat\sd7\,). The
Lydian scale is also unique in that it is possible to form all twelve interval
types with the tonic; Russell's explanation of this fact is reproduced in
Figure \ref{lcc:lydian-intervals}.

\begin{figure}[tbp]
  \centerGraphic{eps/xxx.pdf}
  \caption[All twelve intervals formed with the Lydian tonic.]{All twelve
    intervals formed with the Lydian tonic. Russell's Example I:9.}
  \label{lcc:lydian-intervals}
\end{figure}

Though we may not share Russell's preoccupation with the Lydian scale, we
should not let his idiosyncratic views distract us from the more important
points of the theory. Theorists are, though, used to adopting worthwhile
theoretical ideas from authors without assuming their entire worldview. We
regularly practice Schenkerian analysis without adopting Schenker's views on
the superiority of German music, and even 18th-century authors like Kirnberger
accepted Rameau's fundamental bass without necessarily espousing his more
contentious thoughts on harmonic generation or
\emph{subposition}.\footcite[240--41]{lester:1992} And yet, with the
\emph{Concept}, this does not seem to have taken place.\fn{lcc-9} If we
concede this point, though, Russell's ideas prove to be remarkably useful, as
we shall see.

% Later, more on the tonal order and the scales themselves, how they're
% separable from the Lydian

%%% Local Variables:
%%% mode: latex
%%% TeX-master: "../diss"
%%% End:
