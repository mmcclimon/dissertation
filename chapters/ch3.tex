%% Chapter 3

\chapter{Thirds Spaces}
\label{chap:thirds-spaces}
\addtolof{chap:thirds-spaces}
\addtolof[lot]{chap:thirds-spaces}
\addtocspace

The space developed in the last chapter was organized primarily in descending
fifths, and works well for most tonal jazz. Motion by third, both major and
minor, is also a common (though less frequent) occurrence, and will be the
focus of this chapter. Harmonic motion by thirds is one of the main
emphases of non-jazz transformational theorists; this chapter will allow the
opportunity to explore connections between our approach to jazz harmony and
the existing neo-Riemannian and transformational literature.

\section{Minor-Third Substitutions}
\label{sec:minor-third-subst}

\subsection{Formalism}
\label{sec:m3-formalism}

The most common substitution for the dominant in jazz is undoubtedly the
tritone substitution (discussed in Section \ref{sec:tf-space-tritones}), but
the minor-third substitution is also relatively common, especially in the
bebop era. By way of example, Figure \ref{mts:ladybird-changes} gives the
changes to the opening five bars of Tadd Dameron's ``Lady Bird.'' Normally the
progression in mm.~3--4 would function as a \tf in the key of \Eflat, but in
m.~5 it resolves instead to C. What might have been a \tfo in C
(\h{Dm7}--\h{G7}) does not appear;; the \tf is transposed up a minor third to
become \h{Fm7}--\h{Bb7}.

\begin{figure}[tbp]
  \centerGraphic[width=22em]{eps/ch3/ladybird-changes.pdf}
  \caption{Changes to ``Lady Bird'' (Tadd Dameron), mm.~1--5.}
  \label{mts:ladybird-changes}
\end{figure}

The identical tendency tones shared by tritone-substituted dominants makes
them relatively easy to explain, but minor-third substitution is more
difficult. Jazz harmony textbooks often do not provide an explanation for the
phenomenon: Jerry Coker, for example, simply states that ``the I chord \ldots\ is often
preceded by \h{IV-7} to \h{bVII7}, instead of the usual \h{V7}
chord.''\footcite[82]{coker:elements} The \emph{Berklee Book of Jazz Harmony}
places the \h{bVII7} chord in its chapter on ``modal interchange'' (what might
also be called modal mixture), and notes that its function is ambiguous:
the chord has dominant quality but not dominant function, since it lacks the
leading tone (of the following tonic).\footcite[123--24]{berklee:harmony} Unlike the tritone
substitution, there is no strong voice-leading rationale for the minor-third
substitution; it is simply a progression that bebop players often used, and
that we as analysts must now contend with.

There is, though, a certain similarity between the tritone and minor-third
substitutions: just as the tritone evenly divides the octave, so too the minor
third evenly divides the tritone. In the previous chapter, introducing tritone
substitutes to \tf space effectively divided the space in half, splitting the
complete space into ``front'' and ``back'' sides (recall Figure
\ref{tft:complete-space}). Repeating this process again results in a space
that looks something like Figure \ref{mts:m3-space} (which we will call ``m3
space''). The introduction of minor thirds once again changes the topology of
the space. This is somewhat easier to see by focusing again only on the
centrally-located dominant seventh chords; while the tritone version of \tf
space is topologically equivalent to a Möbius strip, the minor-third version
of the space is equivalent to a torus (see Figure
\ref{mts:m3-torus}).\fn{mts-1}

\begin{figure}[tbp]
  \centerGraphic{eps/ch3/m3-space.pdf}
  \caption[The complete minor-third representation of \tf space (m3 space).]{The
    complete minor-third representation of \tf space (\ii chords omitted on
    rear levels for clarity), or m3 space.}
  \label{mts:m3-space}
\end{figure}


\begin{figure}[tbp]
  \centerGraphic{eps/ch3/m3-torus.pdf}
  \caption{The toroidal center of minor-third space (the m3-torus).}
  \label{mts:m3-torus}
\end{figure}

Figure \ref{mts:m3-torus} looks remarkably similar to a more familiar toroidal
figure common in music theory: the neo-Riemannian Tonnetz. Despite the surface
similarity though, the two are quite different. Like the Tonnetz, the dominant
sevenths at the center of m3 space are arranged into axes of perfect fifths
(verticals), minor thirds (horizontals), and major thirds (northwest--southeast
diagonals, not shown in Figure \ref{mts:m3-torus}).\fn{mts-2} Crucially though, the
vertices in the m3-torus are dominant seventh chords (ordered triples), not
individual notes; the m3-torus does not represent a parsimonious voice-leading
space. The neo-Riemannian Tonnetz can also be drawn to represent triads instead of
individual notes, but the resulting graph (Douthett and Steinbach's
``chicken-wire torus'') no longer resembles the m3-torus here.\fn{mts-3}

The minor-third arrangement of \tf space makes it easy to define a
transformation to represent the minor-third substitution. Because jazz
musicians often refer to the minor-third substitution as the ``backdoor
substitution'',  we will call this transformation BD (for ``backdoor''):%
%
\begin{displaymath}
    \mathrm{BD}(X\,) = Y\text{, where}\ X = (x_r, x_t, x_s) \in \Sdom
    \text{ and}\
    Y = (\,y_r, y_t, y_s) = (x_r + 2, x_s - 4, x_t - 3) \in \Smaj
\end{displaymath}%
%
Musically, there is no compelling reason to define BD such that the third is
calculated from the previous chord's seventh and vice versa; there is no
voice-leading connection between them as there is in the TF transformation.
Defining them this way, however, allows the same function to model both the
transformation from minor to dominant seventh and from dominant to major
seventh.\fn{mts-4} In the space, BD is represented as a diagonal line moving
``frontward'' between two layers; see Figure \ref{mts:bd-transformation}. With
this definition, we can easily understand the progression in mm.~3--5 of
``Lady Bird'': \h{Fm7} \TFarrow\ \h{Bb7} $\xrightarrow{\mathrm{BD}}$\ \h{Cmaj7}.

\begin{figure}[tbp]
  \centerGraphic[width=20em]{eps/ch3/bd-transformation.pdf}
  \caption[The BD transformation in m3 space.]{The BD transformation from
    \caph{Bb7} to \caph{Cmaj7} in minor-third space.}
  \label{mts:bd-transformation}
\end{figure}

\subsection{Analytical Interlude: Joe Henderson, “Isotope”}
\label{sec:henderson-isotope}
\FloatBarrier

Minor-third space, as it has been developed so far, may seem like merely
another arrangement of \tf space. One might reasonably ask why it merits a
section in this chapter, instead of being merely an extension of the space
like those explored in Section \ref{sec:tf-extensions}. In a situation
parallel to that of 19th-century chromatic tonality, jazz after 1960 began to
use more chromatic progressions---\allowbreak especially those built on
thirds---in what still might be called tonal jazz. In these compositions, there
is still a prevailing sense of key, but local harmonic progressions depart
from the descending-fifths norm of earlier jazz.\fn{mts-5} Exploring
minor-third space, and understanding why it merits special discussion, is
easiest to do by using an example: Joe Henderson's composition
``Isotope.''

The solo changes for ``Isotope'' are given in Figure
\ref{mts:isotope-solo-changes}.\fn{mts-6} The tune is a modified 12-bar blues,
and contains the typical lowered seventh of the blues dialect: all of the
chords (even the tonic C chords) are major-minor sevenths. While we could
analyze this set of changes in the ``blues \tf space'' of Section
\ref{sec:other-kinds-tonic}, using the m3-torus of Figure \ref{mts:m3-torus}
will help to highlight how minor-third space can be used in analysis.

\begin{figure}[tbp]
  \centerGraphic[width=22em]{eps/ch3/isotope-solo-changes.pdf}
  \caption{Solo changes for ``Isotope'' (Joe Henderson).}
  \label{mts:isotope-solo-changes}
\end{figure}

\begin{figure}[tbp]
  \centerGraphic{eps/ch3/isotope-torus.pdf}
  \caption{An analysis of ``Isotope''(solo changes) in the m3-torus.}
  \label{mts:isotope-torus}
\end{figure}

An analysis of ``Isotope'' in the m3-torus is given in Figure
\ref{mts:isotope-torus}. The solo changes begin with a tonic major-minor
seventh chord for four bars, before moving to the IV chord in m.~5. Instead of
moving directly from IV to I in m.~7 (which would be typical for a blues),
Henderson moves first to \Bflat, resulting in a variant of the backdoor
progression \h{Bb7}--\h{C7} in mm.~6--7.\fn{mts-7}  The tonic function of \h{C7} is
extended by moving to \h{A7} in m.~8, a minor
third away. What would be a string of $T_5$ operations,
\h{A7}--\h{D7}--\h{G7}--\h{C7}, is disrupted slightly by the tritone
substitution of \h{Ab7} for \h{D7} (represented in the m3-torus by replacing a
chord with the chord two spaces to its left or right). Once the progression
returns to the tonic \h{C7} in m.~11, Henderson uses a complete minor-third
cycle as a turnaround, maintaining tonic function for two measures before
beginning the next chorus.

The logic of the minor-third substitution means that the \h{Bb7} chord in m.~6
is understood as a substitute for true dominant \h{G7}. Likewise, the \h{A7}
chord in m.~8 seems to substitute for the tonic in the same way; in an
ordinary 12-bar blues, mm.~7--8 would both contain tonic. This correspondence
suggests an interesting possibility: the top row of harmonies in Figure
\ref{mts:isotope-torus} all seem to have tonic function, while the two chords
used in the bottom row both act as dominants. The harmonies in the middle row,
then, seem to serve as predominants (or subdominants), appearing in
``Isotope'' just before the \h{G7} and \h{Bb7} chords.\fn{mts-7b}

This functional analysis helps to make sense of the unusual turnaround in the
last two bars of the tune. Turnarounds are inherently prolongational
structures, a way of providing harmonic interest between a chorus-concluding
tonic and the next chorus-beginning tonic.\fn{mts-8} While most turnarounds
use functional harmony (a \tfo progression or some variant), Henderson uses a
non-functional minor-third cycle. Coming as it does at the end of the chorus,
which makes liberal use of minor-third substitutions, we are primed to hear
this cycle as a unique way of maintaining tonic function while avoiding the use
of a functional harmonic progression.

\begin{figure}[tbp]
  \centerGraphic[width=\textwidth]{eps/ch3/isotope-head.pdf}
  \caption{``Isotope,'' head (Joe Henderson).}
  \label{mts:isotope-head}
\end{figure}

This minor-third motion is seen first in the head changes, given in Figure
\ref{mts:isotope-head}. (Because the changes are altered in the head in order
to fit the melody, this figure gives the melody as well.) Most of the
alterations between the head and solo changes occur in the first four bars of
the tune; the remaining differences are relatively insignificant.\fn{mts-11}
While the solo changes give only \h{C7} in the first four bars, the head
changes elaborate this harmony with an alteration of a I--ii--V--I
progression: the II$^7$ chord in m.~2 is preceded by \h{Eb7}. While it would
be easy to write this chord off as an upper-neighbor harmony to the more
structural II chord, doing so would minimize the important role of the
minor-third substitution in the rest of the tune. What appears at first to be
an inconsequential embellishment (substituting \h{Eb7} for \h{C7}) gains in
significance throughout the tune, first becoming realized in the backdoor
progression in mm.~5--7 and reaching its fullest expression in the turnaround
that ends the chorus.

Henderson's tour of the m3-torus in ``Isotope'' is interesting for a number of
reasons. First, he is able to create a sense of tonal function and harmonic
progression while using almost entirely major-minor (``dominant'') seventh chords. This
is a feature common to many blues tunes, but is stronger in ``Isotope'' given
the pervasive use of minor-third substitutions. Second, this is a tune that
does not seem to make much sense in the descending-fifths \tf space of the
previous chapter. While certainly some of the tune makes use of harmonic
motion in fifths, the backdoor progression in mm.~6--7 and the final
turnaround would appear as seemingly random, nonsensical jumps in \tf space.

Finally, the progression of ``Isotope'' is not one that is easily explained
using neo-Riemannian theory as it is usually applied to classical music.
Constructing a Tonnetz usually relies on having two varieties of musical objects
under consideration (commonly major and minor triads, or half-diminished and
dominant seventh chords), while our m3-torus only uses major-minor sevenths.
Neo-Riemannian theories often focus on smooth voice leading; measured in these
terms, the \h{Bb7} and \h{C7} chords of mm.~5--6 are quite distant from one
another, despite their functional equivalence to a V--I motion.\fn{mts-9} The
turnaround, most unusual from a tonal perspective, is actually quite typical
of patterns usually analyzed in neo-Riemannian terms: the major-minor sevenths
found there are all minimal perturbations of a single diminished seventh
chord, and each can be connected to the next with a minimal amount of
voice-leading work (two semitones moving in opposite directions).\fn{mts-10}

\section{Major-Third Spaces}
\label{sec:maj3-spaces}

\subsection{Introduction: Coltrane Changes}
\label{subsec:maj3-intro}

Root motion by major third is one of the most difficult harmonic motions to
explain using traditional tonal methods; it is with this kind of music that
transformational methods have proven to be most useful.\fn{maj3-1} The
increasing use of these progressions in nineteenth-century harmony has a
parallel in jazz, as Keith Waters has shown; in both, the harmonic vocabulary
is familiar, but harmonic progressions are often
unfamiliar.\footcite{waters:2013} Nonfunctional harmony is a defining
feature of some post-bop jazz, including much of the music of Chick Corea,
Herbie Hancock, Wayne Shorter, and others.\fn{maj3-2} This dissertation,
though, is interested specifically in tonal jazz, and we will remain careful
during this discussion to avoid straying too far afield from this goal.

The locus classicus for root motion by major third in jazz is of course John
Coltrane's ``Giant Steps,'' first recorded in 1959 on the album of the same
name.\fn{maj3-3} Much of the use of nonfunctional harmony in jazz that
develops after 1960 can be traced back to ``Giant Steps''; Keith Waters has
outlined this lineage in selected compositions of Wayne Shorter, Bill Evans,
and Herbie Hancock.\footcite{waters:2010} Given this influence, ``Giant
Steps'' will serve here as a useful foil for major-third cycles in jazz more
generally. Though we will delay a proper analysis of the tune until we have
developed some formalism, a short overview will be useful at this point.

\begin{figure}[tbp]
  \centerGraphic[width=22em]{eps/ch3/giant-steps-changes.pdf}
  \caption{Changes to ``Giant Steps'' (John Coltrane).}
  \label{maj3:giant-steps-changes}
\end{figure}

The changes to ``Giant Steps'' are given in Figure
\ref{maj3:giant-steps-changes}. The major-third construction of the tune is
readily apparent: the three tonal centers are B, G, and \Eflat, as evidenced
by the major seventh chords. These local tonics are all preceded by their
dominants (mm.~1--3, 5--7) or by complete \tf progressions (all other
locations). Though the distance between the key centers is unusual, the
individual progressions are not.\fn{maj3-4} ``Giant Steps'' is not exactly
tonal, but neither is it really atonal. When I listen to the piece, at least,
the impression is not one of nonfunctional harmony, but rather of
tonal harmony used in an unconventional way. This distinction is easiest to understand
with a counterexample: Figure \ref{maj3:infant-eyes-changes} gives the changes to
Wayne Shorter's ballad ``Infant Eyes''.\fn{maj3-4a} Here there are no \tf{}s,
and the only V--I progressions occur across formal boundaries. Shorter's use
of harmony \emph{does} seem nonfunctional, and gives the piece a floating
quality that ``Giant Steps'' does not have. Rather, ``Giant Steps'' is
strongly forward-directed: all of the dominant chords push toward their
respective tonics, and although the global tonic may be in question, local
tonic chords are crystal clear.\fn{maj3-4b}

\begin{figure}[tbp]
  \hspace{6em}\includegraphics[width=28em]{eps/ch3/infant-eyes-changes.pdf}
  \caption{Changes to ``Infant Eyes'' (Wayne Shorter).}
  \label{maj3:infant-eyes-changes}
\end{figure}

\begin{figure}[tbp]
  \centerGraphic[width=25em]{eps/ch3/coltrane-substitution.pdf}
  \caption[Coltrane's major-third cycle as a substitution for a \tfo
    progression.]{Coltrane's major-third cycle as a substitution for a \tfo
    progression. (Adapted from Levine, \emph{The Jazz Theory Book}, 359.)}
  \label{maj3:coltrane-substitution}
\end{figure}

\begin{figure}[tbp]
  \hspace{1em}\includegraphics[width=31em]{eps/ch3/countdown-tune-up.pdf}
  \caption{The changes to ``Countdown'' (Coltrane), compared with
    ``Tune Up'' (Miles Davis).}
  \label{maj3:countdown-tune-up}
\end{figure}

The major-third cycle of ``Giant Steps'' is the most well-known example, but
Coltrane first developed the progression as an elaborate substitution over a
standard \tfo progression; this particular set of substitutions is often
referred to as ``Coltrane changes.''\fn{maj3-5} Figure
\ref{maj3:coltrane-substitution} shows how this process works: the goal of the
progression (in this case, \h{Dmaj7}) is shifted to the fourth bar; then, major
seventh chords related by major third are placed on the downbeats (\h{Bbmaj7}
and \h{Gbmaj7}); finally, all of the major sevenths are preceded by their own
dominants.\nocite{levine:1995} This process can clearly be seen in Coltrane's
composition ``Countdown,'' which is based on the changes to Miles Davis's
``Tune Up'' (see Figure \ref{maj3:countdown-tune-up}).\fn{maj3-6}
Coltrane changes can be superimposed over any four-measure \tfo progression,
so they can be found not only in Coltrane's own compositions, but also in his
improvisations on other tunes as well as his reharmonizations of standards
(like ``Body and Soul'').\fn{maj3-7}

\subsection{Developing a Transformational System}
\label{subsec:maj3-transformations}

Because Coltrane changes can be considered a \tf variant, it is logical to
include them in this study, even though we may be slightly pushing the limits
of ``tonal jazz.'' We now have a preliminary understanding of how the substitution
works, but it still remains to incorporate it into the transformational system
under development here. First, though, there is a bit of unfinished business
to take care of: in Section \ref{sec:transformational-theory}, we touched on
several transformational approaches only briefly, promising to return to them
at a point when they would be more relevant. Given the central role of
harmonic motion in thirds in many neo-Riemannian theories, it seems
appropriate to fulfill that promise at this point.

Triads related by major third are important in many of these theories,
but especially those of Richard Cohn. In both of Cohn's models of triadic
space in \emph{Audacious Euphony}---hexatonic cycles and Weitzmann
regions---three M3-related major triads combine with three minor triads to
form a six-chord system.\fn{maj3-8} Thus, the three tonal centers of ``Giant
Steps'' can be contained in a single hexatonic cycle or a single Weitzmann
region.

Understanding the rest of ``Giant Steps'' in terms of one of these systems,
though, leaves much to be desired. Cohn's systems, and others like them, are
fundamentally triadic, which is clearly a problem for understanding jazz, with
its saturation of seventh chords (and beyond). To account for this
incongruity, we must either adapt the music to fit our analytical system, or
adapt our analytical system to fit the music. The first option is clearly a
nonstarter: Figure~\ref{maj3:gs-cube-dance} gives a non-example of ``Giant
Steps'' analyzed in Jack Douthett's ``Cube Dance.''\fn{maj3-9}
%
\begin{figure}[tbp]
  \centerGraphic{eps/ch3/gs-cube-dance.pdf}
  \caption[``Giant Steps'' analyzed in Douthett's Cube Dance.]{``Giant
    Steps,'' mm.~1--5, analyzed in Douthett's Cube Dance. Begin by following the
    solid arrows, then continue with the dashed arrows.}
  \label{maj3:gs-cube-dance}
\end{figure}
%
While this analysis makes the major-third cycle of the tonic chords clear,
reducing the chords to triads loses the detail of the chord qualities (both
major-seventh and dominant-seventh chords become major triads), as well as
their functional relationships. Despite the prominent emphasis on M3-cycles in
these theories, then, using these triadic systems as an analytical basis for
our work here would not seem to be the answer.

There are of course neo-Riemannian theories involving seventh chords, but
these turn out to be not so helpful for our purposes here either. Theories
that include only the (0258) tetrachords---half-diminished and dominant
sevenths---are obviously not suitable for analyzing the multiple seventh-chord
types in jazz. The seventh-chord analogy of ``Cube Dance'' is what Cohn calls
the ``4-Cube Trio,'' which is shown in Figure
\ref{maj3:four-cube-trio}.\fn{maj3-10} As is readily apparent, 4-Cube Trio
does not contain any major seventh chords, so it would also create problems if
pressed into use for analyzing ``Giant Steps.''

% xxx do we really need this 4-cube trio graphic?
\begin{figure}[tbp]
  \centerGraphic{eps/ch3/four-cube-trio.pdf}
  \caption[Cohn's ``4-Cube Trio'']{Cohn's ``4-Cube Trio'' (\emph{Audacious
    Euphony}, Fig.~7.16, 158.) The black triangles indicate minor seventh chords,
    while the hollow stars indicate French sixth chords.}
  \label{maj3:four-cube-trio}
\end{figure}

There is, though, a more fundamental problem with these neo-Riemannian theories
of seventh chords, at least when approaching M3-cycles in jazz. As we have
mentioned, many neo-Riemannian theories focus on
efficient voice-leading, and parsimonious relationships among seventh chords
can understood as minimal perturbations of fully-diminished seventh
chords.\fn{maj3-11} This does not generate major-third cycles (as in the
triadic case), but rather partitions the octave into \emph{minor} thirds. The
three dominant sevenths appearing in ``Giant Steps,'' for example, are in
three different ``towers'' in the 4-Cube Trio, which does not reflect
the organizing influence of major thirds in the way that the triadic Cube
Dance does.

Some theorists have turned to neo-Riemannian theory to explain jazz
progressions, though; closest to our intent here is Matthew Santa's nonatonic
system for analyzing Coltrane.\footcite{santa:2003} In a parallel with Cohn's
hexatonic systems, Santa draws from the nonatonic (or enneatonic) collection
in order to explain ``Giant Steps'' in terms of parsimonious voice leading.
Figure \ref{maj3:nonatonic-cycle} shows one of Santa's cycles, along with
three-voice parsimonious realization. (In this figure, note that the triangle
indicates a major \emph{triad}, not a major seventh chord.)
%
\begin{figure}[tbp]
  \centerGraphic{eps/ch3/nonatonic-cycle.pdf}
  \caption[Matthew Santa's nonatonic cycles.]{Matthew Santa's nonatonic
    cycles: the ``Western'' nonatonic cycle (left), and a three-voice parsimonious
    realization (right). (Adapted from ``Nonatonic Progressions in Coltrane,''
    14.)}
\label{maj3:nonatonic-cycle}
\end{figure}
%
This nonatonic system seems to be a convincing analysis of the opening of
``Giant Steps,'' but it comes up a bit short as a general theoretical system.
First, Santa considers only major triads and incomplete dominant seventh
chords; all major seventh chords are reduced to triads, and minor
sevenths---like the \ii chords of ``Giant Steps''---are simply
ignored.\fn{maj3-13} The cycle in Figure \ref{maj3:nonatonic-cycle} is
generated by the collection \{D, \Eflat, E, F\sharp, G, \Aflat, \Bflat, B,
C\}, but it does not contain all of the triads or incomplete dominant sevenths
in that collection (see Table \ref{maj3:nonatonic-table}). Santa's 4-cycle
system, then, is somewhat misleading, since any triad or dominant seventh can
be located in two different nonatonic collections.

\begin{table}[tbp]
  \centering
  \vspace{1em}
  \begin{tabular}{ccc}
    Major triads  & Minor triads & Incomplete \V \\
    \hline
    \rule[1em]{0ex}{1ex}%
    %
    \h{CM}\;  & \h{Cm}    & \h{C7}\; \\
    \h{EbM}$^\star$ & \h{Ebm}   & \h{Bb7}$^\star$ \\
    \h{EM}\;  & \h{Em}    & \h{E7}\; \\
    \h{GM}$^\star$  & \h{Gm}    & \h{D7}$^\star$ \\
    \h{AbM}\; & \h{Abm}   & \h{Ab7}\; \\
    \h{BM}$^\star$  & \h{Bm}    & \h{Fs7}$^\star$ \\
  \end{tabular}
  \caption[All possible consonant triads and incomplete dominant sevenths in the
    nonatonic collection \{D, \Eflat, E, F\sharp, G, \Aflat, \Bflat, B,
    C\}.]{All possible consonant triads and incomplete dominant sevenths in the
    nonatonic collection \{D, \Eflat, E, F\sharp, G, \Aflat, \Bflat, B,
    C\}. Members of Santa's Western system are marked with a star.}
  \label{maj3:nonatonic-table}
\end{table}

Having brought up all of these approaches only to show their shortcomings,
though, the question remains: what should a transformational system that
includes major-third relations look like? Although the relationship of
M3-cycles and smooth voice leading is valuable, so far in this study we have
focused primarily on functional relationships, and it would seem foolish
to abandon that approach here. As we noted above, ``Giant Steps'' does contain
a major-third cycle, but within that cycle the progressions are functional: it
is locally diatonic, but globally chromatic.

As it turns out, we can once again adapt \tf space in order to show
organization by major third, as shown in Figure \ref{maj3:maj3-space}. This
figure looks very similar to the minor-third organization in Figure
\ref{mts:m3-space}, but the relationships between layers have
changed.\fn{maj3-14} Here, the ``layers'' of the space are arranged in
descending major thirds ($T_8$), while the descending fifths arrangement is
otherwise unchanged. This arrangement means that all of ``Giant Steps''
happens in a single horizontal slice of the space.\fn{maj3-15}
This organization of \tf space reflects our intuitions about the organization
of this tune and others like it: by maintaining the integrity of the \tfo
progressions and instead altering the relationships between them, we can keep
both the local functional progressions important to improvising musicians \emph{and}
reflect the unusual chromatic organization of the tune itself.

\begin{figure}[tbp]
  \centerGraphic{eps/ch3/maj3-space.pdf}
  \caption[A major-third organization of \tf space.]{A major-third
    organization of \tf space (\ii chords omitted on rear levels for clarity).}
\label{maj3:maj3-space}
\end{figure}

\begin{figure}[tbp]
  \centerGraphic{eps/ch3/coltrane-subs-space.pdf}
  \caption{Coltrane changes as a \tfo substitution, shown in M3-space.}
\label{maj3:coltrane-subs-space}
\end{figure}

The arrangement into $T_8$-related ``layers'' also helps to clarify in what
sense Coltrane changes function as a substitution for a \tfo progression
(see Figure \ref{maj3:coltrane-subs-space}). As usual, we could define a
transformation to help explain this substitution. Though the $T_8$ between
major seventh chords is certainly important, the most unusual surface feature
in the substitution is the jump from a major-seventh chord to the dominant
seventh whose root is a minor third higher; it is this harmonic move that
gives the progression its forward momentum. We might call this transformation
CS (for ``Coltrane Substitution''):%
%
\begin{displaymath}
    \mathrm{CS}(X\,) = Y\text{, where}\ X = (x_r, x_t, x_s) \in \Smaj
    \text{ and}\
    Y = (\,y_r, y_t, y_s) = (x_r + 3, x_t + 3, x_s + 2) \in \Sdom
\end{displaymath}%
%
With this transformation, we can understand the entire \tfo substitution
as follows:%
%
\begin{center}
  \h{Dm7}  $\xrightarrow{\hspace{8.25em}\mathrm{TF}\hspace{8.25em}}$ \h{G7} \TFarrow\ \h{Cmaj7}

  \vspace{.5em}
  \emph{becomes}
  \vspace{.5em}

  \h{Dm7}  $\xrightarrow{\mathrm{TF}\ \odot\ T_8\,}$\ \h{Eb7} \TFarrow\
  \h{Abmaj7} \CSarrow\ \h{B7} \TFarrow\
  \h{Emaj7}  \CSarrow\ \h{G7} \TFarrow\
  \h{Cmaj7}
\end{center}

This M3-space is well equipped to show the logic of major-third cycles, though
other kinds of tonal relationships are more difficult to see: both tritone
substitutes and minor-third substitutions are maximally far away in M3-space,
for example. While this is indeed a limitation, the fact that \tf space and
its variants share the same essential features means that each can be substituted
for another as needed for a given analytical situation. The spaces developed
so far---\tf space, its tritone-substituted variant, minor-third space, and
now major-third space---all reflect the basic descending-fifths orientation
of tonal jazz, but each prioritizes a particular secondary
relationship.\fn{maj3-16} Maintaining the same basic structure in our
analytical apparatus allows us to understand a wide range of music as
variations on a basic, functionally harmonic theme. There is no need for a
great switching of context from the logic of tritone substitutions to the logic
of Coltrane changes, nor is there a need to invoke set classes of different
cardinalities to justify dividing the octave into three or four equal parts.
This presentation is developed in a rough parallel with the music itself: jazz
musicians did not (and do not) discard everything they learned from bebop when
approaching a progression like that of ``Giant Steps''; rather, each is part
of a single, coherent through line of tonal jazz.

\subsection{Analytical Interlude: Richard Rodgers/Lorenz Hart, “Have You Met
  Miss Jones?”}
\label{subsec:miss-jones}

Though we could turn to any number of Coltrane's middle-period compositions as
analytical examples to illustrate major-third spaces, we will instead opt for
the tune that is always cited as one of his influences: the Richard Rodgers
and Lorenz Hart standard, ``Have You Met Miss Jones?'' While some of
Coltrane's tunes (like ``Giant Steps'') use M3-cycles almost exclusively,
``Miss Jones'' will allow us the opportunity to see how organization by major
thirds can participate in more typical, fifths-based jazz harmony. The changes
for the tune are given in Figure \ref{maj3:miss-jones-changes}; the analysis
here will proceed in sections.\fn{maj3-17}

\begin{figure}[tbp]
  \centerGraphic[width=22em]{eps/ch3/miss-jones-changes.pdf}
  \caption{Changes to ``Have You Met Miss Jones?'' (Richard Rogers/Lorenz Hart).}
\label{maj3:miss-jones-changes}
\end{figure}

The analysis of ``Miss Jones'' in M3-space is given in Figures
\ref{maj3:miss-jones-a}--\ref{maj3:miss-jones-c}. The A section (Figure
\ref{maj3:miss-jones-a}) is fairly typical, though it does include a
fully-diminished seventh chord, which we have not yet seen in this study. This
\h{Fso7} clearly functions as a passing chord, harmonizing the bass line
F--F\sharp--G. The analysis in M3-space interprets this chord, as jazz
musicians often do, as a \h{D7b9} without a root: functionally, the two chords
are identical, with each leading to the following \h{Gm7}.\footcite[85]{levine:1995} This
\h{Gm7} initiates a home-key \tf in mm.~3--4 that resolves deceptively to
\h{Am7} in m.~5, at which point the piece begins a iii--vi--ii--V turnaround
to return to \h{Fmaj7} for the repeat of the A section.

\begin{figure}[tbp]
  \stepcounter{figure} % This is a new figure, but I still want the 'a'
  \ContinuedFloat
  \centerGraphic{eps/ch3/miss-jones-a.pdf}
  \caption{``Miss Jones,'' A section (mm.~1--8), analyzed in M3-space.}
\label{maj3:miss-jones-a}
\end{figure}

\begin{figure}[tbp]
  \ContinuedFloat
  \centerGraphic{eps/ch3/miss-jones-b.pdf}
  \caption{``Miss Jones,'' second A section and bridge (mm.~9--23).}
\label{maj3:miss-jones-b}
\end{figure}

\begin{figure}[tbp]
  \ContinuedFloat
  \centerGraphic{eps/ch3/miss-jones-c.pdf}
  \caption{``Miss Jones,'' last two bars of bridge and final A section
    (mm.~23--32).}
\label{maj3:miss-jones-c}
\end{figure}

The second A section begins like the first, but the end is altered so that the
bridge can begin on the subdominant, \Bflat\ (Figure \ref{maj3:miss-jones-b}).
The bridge of this tune is its most well-known aspect, and contains the
major-third cycle. This organization by major third is readily apparent in
the space, as the music seems to break free of the descending fifths to
elaborate the subdominant with a sequence that moves into the rear layers of
the space and then returns. After arriving on \Bflat\ in the first bar of the
bridge, the tune moves to a \tfo in \Gflat\ (a major third lower), followed by
a \tfo in D (yet another major third lower). After making its way to the rear
of the space, it begins to work its way back up the chain of thirds, finishing
the bridge with a return to a \tfo in \Gflat. This \h{Gbmaj7} chord moves via a
\slideS transformation to a home-key \tfo, which returns to \h{Fmaj7} to begin
the final A section. This final section (Figure \ref{maj3:miss-jones-c}) is
nearly the same as the first, but altered slightly in the last four bars to
arrive more strongly on tonic in the penultimate bar of the form.

\begin{figure}[tbp]
  \centerGraphic{eps/ch3/miss-jones-bridge.pdf}
  \caption{A transformation network for the bridge of ``Miss Jones.''}
\label{maj3:miss-jones-bridge}
\end{figure}

A more detailed transformation network for the bridge of ``Miss Jones'' is
given in Figure \ref{maj3:miss-jones-bridge}.\fn{maj3-18} In this network,
transformations actualized in the music are shown as solid arrows, while
others are shown with dotted arrows. (As usual, the unlabeled arrows are TF
transformations.) A complete $T_8$-cycle is thwarted when the \h{Dmaj7} chord
moves instead back to \Gflat, but the dotted arrow shows that this move would
have completed the cycle. It also clarifies the return to F major in the last
A section: a larger-scale $T_{11}$ from \Gflat\ to F is accomplished via the
\slideS transformation from \h{Gbmaj7} to \h{Gm7}.

The analysis of ``Have You Met Miss Jones'' in M3-space demonstrates how the
logic of major-third cycles in jazz is not independent from that of standard
fifths-based harmony, but instead an extension of it. Any of our spaces would
illustrate the A sections of ``Miss Jones'' equally well, but the construction
of M3-space allows us to better understand the bridge. In normal \tf space
(Figure \ref{tft:complete-space} on p.~\pageref{tft:complete-space}), the \tfo
progressions related by major third are maximally far apart; analyzing the
bridge in that space would make the $T_8$-related \tfo{}s seem like
nonsensical harmonic motions. Rearranging the basic space as we have done in
this section shows that the M3-cycle participates in a coherent way within the
logic of the otherwise typical harmony of the tune.

\section{Parsimonious Voice-Leading}
\label{sec:parsimony}

Given the importance of parsimonious voice-leading in many current
neo-Riemannian and transformational theories of harmony, it seems prudent to
examine the transformational system we have been developing in the last two
chapters in that light. Several of the transformations we have defined are
indeed parsimonious, moving individual voices efficiently (the \textsc{3rd}
and \textsc{7th} transformations, among others) while others are less so (the
transformation CS from this chapter, for example). Though there is indeed a
great deal of literature on parsimonious voice-leading, all of our
transformations are defined on ordered triples of the form (root, third,
seventh), \emph{not} on triads or seventh chords. As such, we will need to
take a few steps in order to connect our work here with the literature that
deals with these basic chord types directly.

For the moment, let us set aside the ordered-triple representation we have
been using here and return to four-note seventh chords proper. Parsimonious
relationships among seventh chords are shown in Douthett's ``Four Cube Trio''
(recall Figure \ref{maj3:four-cube-trio}); Figure \ref{pvl:four-cube-correct}
redraws a portion this figure to include major seventh chords.\fn{pvl-1}
Because we are interested in this figure's application to jazz, the minor
seventh chords have been labeled with root names, and the French sixth chords
are labeled as dominant seventh chords with flatted fifths (a favorite chord of
Thelonious Monk).\fn{pvl-1b}

\begin{figure}[tbp]
  \centerGraphic[width=33em]{eps/ch3/four-cube-correct.pdf}
  \caption[A portion of Four-Cube-Trio, redrawn to include major seventh
  chords.]{A portion of Four-Cube-Trio, redrawn to include major seventh
    chords. All lines, regardless of style, represent a single voice moving by
    half-step.}
  \label{pvl:four-cube-correct}
\end{figure}

\begin{figure}[tbp]
  \centerGraphic[width=33em]{eps/ch3/four-cube-transformations.pdf}
  \caption{The transformations 3\textsc{rd}, 7\textsc{th},
    \textsc{slide}\textsubscript{7}, TF, and \tft in the Four-Cube-Trio.}
  \label{pvl:four-cube-transformations}
\end{figure}

By way of illustration, Figure \ref{pvl:four-cube-transformations} shows some
of the transformations we have defined in the last two chapters in the
Four-Cube Trio. The most important thing to note about this figure is that it
is \emph{not} a voice-leading graph of the transformations as they have been
defined. Because all of our transformations are defined on ordered triples
(ignoring the fifth of a seventh chord), mapping them into four-note space can
be misleading. In our ordered triple representations, for example, there is no
distinction between half-diminished and minor seventh chords, nor between the
French sixths and the dominant sevenths (i.e., \h{Bb7} is indistinguishable
from \h{Bb7b5}). This figure is included only to demonstrate that some of the
transformations do represent single voice-leadings (the \textsc{3rd},
\textsc{7th}, and \slideS transformations), while others do not. It is also
worth noting that the \slideS transformation is the only one we have defined
in which the voice-leading ascends (indicated by a clockwise motion in the figure).

The parsimonious picture that emerges after we collapse the chords which have
the same ordered-triple representation is much less interesting.\fn{pvl-2}
Most damaging to the structure of the Four-Cube Trio is that the minor seventh
chords become more discriminating: considered as an ordered triple, a minor
seventh chord is only connected to one dominant seventh chord, not two (see
Figure \ref{pvl:minor-seventh-parsimony}). This, plus the collapse of the
French sixths into dominant sevenths, means that every chord is connected to
exactly one other by single-half-step voice-leading. (For reference, a
complete voice-leading roster for the ordered-triple representation is given
in Table \ref{pvl:vl-table}.)

\begin{figure}[tbp]
  \centerGraphic{eps/ch3/minor-seventh-parsimony.pdf}
  \caption[Parsimonious voice leading of the minor seventh
    chord.]{Parsimonious voice leading of the minor seventh chord, with its
      fifth (left) and without (right). Common tones are shown with hollow noteheads.}
  \label{pvl:minor-seventh-parsimony}
\end{figure}

\begin{table}[tbp]
  \centering
  \vspace{1em}
  \begin{tabular}{ccccc}
   Starting Chord  & Common Tones & Voice leading & Result & Transformation \\
   \hline
   \rule[1em]{0ex}{1ex}%
   \h{Cmaj7}  & $r,t$ & $s-1$ & \h{C7} & \textsc{7th} \\
   \h{Cmaj7}  & $r,s$ & $t-1$ & [\h{CmM7}] & [\textsc{3rd}] \\
   \h{Cmaj7}  & $t,s$ & $r+1$ & \h{Csm7} & \slideS \\[2ex]
   %
   \h{C7}  & $r,t$ & $s+1$ & \h{Cmaj7} & $\text{\textsc{7th}}^{-1}$\\
   \h{C7}  & $r,s$ & $t-1$ & \h{Cm7} & \textsc{3rd} \\
   \h{C7}  & $t,s$ &  --   &  -- &  -- \\[2ex]
   %
   \h{Cm7}  & $r,t$ & $s+1$ & [\h{CmM7}] & [$\text{\textsc{7th}}^{-1}$] \\
   \h{Cm7}  & $r,s$ & $t+1$ & \h{C7} & $\text{\textsc{3rd}}^{-1}$   \\
   \h{Cm7}  & $t,s$ & $r-1$ & \h{Bmaj7} & $\text{\textsc{slide}}_7^{-1}$
   \vspace{1em}
  \end{tabular}
  \caption[Parsimonious voice-leading among members of \Smin, \Sdom, and
    \Smaj.]{Parsimonious voice-leading among members of \Smin, \Sdom, and
    \Smaj. In this table, r, t, and s indicate the root, third, and seventh of a
    chord, respectively.}
  \label{pvl:vl-table}
\end{table}

We could generate the entire set of 36 ordered triples
using the single-voice half-step transformations \textsc{7th}, \textsc{3rd},
and $\text{\textsc{slide}}_7^{-1}$:\fn{pvl-3}%
%
\begin{center}
  \h{Cmaj7} $\xrightarrow{\text{7\textsc{th}}\,}$
  \h{C7}  $\xrightarrow{\text{3\textsc{rd}}\,}$
  \h{Cm7} $\xrightarrow{\text{\textsc{slide}}_7^{-1}\,}$
  \h{Bmaj7} $\xrightarrow{\text{7\textsc{th}}\,}$
  \h{B7}  $\ldots\ $
  \h{Dbm7} $\xrightarrow{\text{\textsc{slide}}_7^{-1}\,}$
  \h{Cmaj7}
\end{center}%
%
The resulting graph, however, is simply a circle, and does not mirror the rich
voice-leading network of the Four-Cube Trio. We could redefine all of the
other transformations in terms of these single voice-leadings (\h{Dm7}
\TFarrow\ \h{G7} becomes \h{Dm7}
$\xrightarrow{\text{\textsc{slide}}_7^{-1}\ \odot\ {(\text{7\textsc{th}}\ \odot\
  \text{3\textsc{rd}}\ \odot\ \text{\textsc{slide}}_7^{-1})}^6\ \odot\
  7\text{\textsc{th}}\,}$
\h{G7}), but these decompositions would not seem to give much insight into
tonal jazz---which is, after all, the aim of this study.

Throughout the course of the last two chapters, we have developed a fairly
complete transformational system for jazz harmony. Taking \tf space as our
starting point, we have seen how it can altered in various ways to show
different aspects of standard tonal jazz harmony. To this point, though, we
have focused almost exclusively on chord symbols (via their abstraction into
ordered triples). Of course, jazz harmony involves quite a bit more than
relationships among three-note chords: these basic structures are altered in
various ways by rhythm section members, and we have yet to say anything at all
about the role harmony plays for an improvising musician. To do so, we'll need
to expand our harmonic universe somewhat; the next chapter begins to take the
first steps in that direction.

%%% Local Variables: %%%
%%% mode: latex %%%
%%% TeX-master: "../diss" %%%
%%% End: %%%
