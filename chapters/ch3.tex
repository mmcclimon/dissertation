%% Chapter 3

\chapter{Thirds Spaces and Parsimony}

The space developed in the last chapter was organized primarily in descending
fifths, and works well for most tonal jazz. Motion by thirds, both major and
minor, also occur less frequently and will be the focus of this chapter.
Harmonic motion organized in thirds is one of the main emphases of non-jazz
transformational theorists; this focus will permit the opportunity to
connect developments in jazz harmony with analogous developments in
nineteenth-century harmony.

\section{Minor-third Substitutions}
\label{sec:minor-third-subst}

\subsection{Formalism}
\label{sec:m3-formalism}

The most common dominant substitution in jazz is undoubtedly the tritone
substitution (discussed in Section \ref{sec:tf-space-tritones}), but the
minor-third substitution---sometimes called the backdoor substitution---is also
relatively common, especially in the bebop era. By way of example, Figure
\ref{mts:ladybird-changes} gives the changes to opening five bars of Tadd
Dameron's ``Lady Bird.'' Normally the progression in mm.~3--4 would function
as a \tf in the key of \Eflat, but in m.~5 here it resolves instead to C. The
expected \tf in C (\h{Dm7}--\h{G7}) is transposed up a minor third, becoming
\h{Fm7}--\h{Bb7}.

\begin{figure}[tbp]
  \centerGraphic[width=22em]{eps/ch3/ladybird-changes.pdf}
  \caption{Changes to ``Lady Bird'' (Tadd Dameron), mm.~1--5.}
  \label{mts:ladybird-changes}
\end{figure}

The identical tendency tones shared by tritone-substituted dominants makes
them relatively easy to understand, but minor-third substitution is more
difficult. Jazz harmony textbooks often do not provide an explanation for the
phenomenon: Jerry Coker simply states that ``the I chord \ldots\ is often
preceded by \h{IV-7} to \h{bVII7}, instead of the usual \h{V7}
chord.''\footcite[82]{coker:elements} The \emph{Berklee Book of Jazz Harmony}
places the \h{bVII7} chord in its chapter on ``modal interchange'' (what might
also be called modal mixture), and notes that its function is ambiguous, since
the chord has dominant quality but not dominant function, since it lacks the
leading tone.\footcite[123--24]{berklee:harmony} Unlike the tritone
substitution, there is no strong voice-leading rational for the minor-third
substitution; it is simply a progression that bebop players often used, and
that we as analysts must now contend with.

There is, though, a certain similarity between the tritone and minor-third
substitutions: just as the tritone evenly divides the octave, so too the minor
third evenly divides the tritone. In the previous chapter, introducing tritone
substitutes to \tf space effectively divided the space in half, splitting the
complete space into ``front'' and ``back'' sides. Repeating the process again
results in a space that looks something like Figure \ref{mts:m3-space} (which
we will call ``m3 space''). The introduction of minor thirds once again
changes the topology of the space. This is somewhat easier to see by focusing
again only on the centrally-located dominant seventh chords; while the tritone
version of \tf space is topologically equivalent to a Möbius strip (recall
Figure \ref{tft:mobius-dominants}), the minor-thirds version of the space is
equivalent to a torus (see Figure \ref{mts:m3-torus}).\fn{mts-1}

\begin{figure}[tbp]
  \centerGraphic{eps/ch3/m3-space.pdf}
  \caption[The complete minor-thirds representation of \tf space (m3 space).]{The
    complete minor-thirds representation of \tf space (\ii chords omitted on
    rear levels for clarity), or m3 space.}
  \label{mts:m3-space}
\end{figure}


\begin{figure}[tbp]
  \centerGraphic{eps/ch3/m3-torus.pdf}
  \caption{The toroidal center of m3 space (the m3-torus).}
  \label{mts:m3-torus}
\end{figure}

Figure \ref{mts:m3-torus} looks remarkably similar to a more familiar toroidal
figure common in music theory: the neo-Riemannian Tonnetz. Despite the surface
similarity though, the two are quite distinct. Like the Tonnetz, the dominant
sevenths at the center of m3 space are arranged into axes of perfect fifths
(verticals), minor thirds (horizontals), and major thirds (northwest-southeast
diagonals, not shown in Figure \ref{mts:m3-torus}).\fn{mts-2} Crucially though, the
vertices in the m3-torus are dominant seventh chords (ordered triples), not
individual notes; the m3-torus does not represent a parsimonious voice-leading
space. The neo-Riemannian Tonnetz can also be drawn to represent triads instead of
individual notes, but the resulting graph (Douthett and Steinbach's
``chicken-wire torus'') no longer resembles the m3-torus here.\fn{mts-3}

The minor-thirds arrangement of \tf space makes it easy to define a
transformation to represent the minor-third substitution (which we will call
BD, for ``backdoor''):%
%
\begin{displaymath}
    \mathrm{BD}(X\,) = Y\text{, where}\ X = (x_r, x_t, x_s) \in \Sdom
    \text{ and}\
    Y = (\,y_r, y_t, y_s) = (x_r + 2, x_s - 4, x_t - 3) \in \Smaj
\end{displaymath}%
%
Musically, there is no compelling reason to define BD such that the third is
calculated from the previous chords seventh and vice versa; there is no
voice-leading connection between them as there is in the TF transformation.
Defining them this way, however, allows the same function to model both the
transformation from minor to dominant seventh and from dominant to major
seventh.\fn{mts-4} In the space, BD is represented as a diagonal line moving
``frontward'' between two layers; see Figure \ref{mts:bd-transformation}. With
this definition, we can easily understand the progression in mm.~3--5 of
``Lady Bird'': \h{Fm7} \TFarrow\ \h{Bb7} $\xrightarrow{\mathrm{BD}}$\ \h{CM7}.

\begin{figure}[tbp]
  \centerGraphic[width=20em]{eps/ch3/bd-transformation.pdf}
  \caption[The BD transformation in m3 space.]{The BD transformation from
    \caph{Bb7} to \caph{CM7} in m3 space.}
  \label{mts:bd-transformation}
\end{figure}

\subsection{Analytical Interlude: Joe Henderson, “Isotope”}
\label{sec:henderson-isotope}
\FloatBarrier

Minor thirds space, as it has been developed so far, may seem like merely
another arrangement of \tf space. One might reasonably ask why it merits a
section in this chapter, instead of being merely an extension of the space
like those explored in Section \ref{sec:tf-extensions}. In a situation
parallel to that of 19th-century chromatic tonality, jazz after 1960 began to
use more chromatic progressions---\allowbreak especially those built on
thirds---in what still may be called tonal jazz: in these compositions, there
is still a prevailing sense of key, but local harmonic progressions depart
from the descending-fifths norm of earlier jazz.\fn{mts-5} Exploring
minor-thirds space, and understanding why it merits special discussion, is
best shown by way of an example: Joe Henderson's composition
``Isotope.''\fn{mts-6}

The solo changes for ``Isotope'' are given in Figure
\ref{mts:isotope-solo-changes}.\fn{mts-7} The tune is a modified 12-bar blues,
and contains the typical lowered seventh of the blues dialect: all of the
chords (even the tonic C chords) are major-minor sevenths. While we could
analyze this set of changes in the ``blues \tf space'' of Section
\ref{sec:other-kinds-tonic}, using the m3-torus of Figure \ref{mts:m3-torus}
directly will help to highlight how minor thirds space can be used in analysis.

\begin{figure}[bp]
  \centerGraphic[width=22em]{eps/ch3/isotope-solo-changes.pdf}
  \caption{Solo changes for ``Isotope'' (Joe Henderson).}
  \label{mts:isotope-solo-changes}
\end{figure}

\begin{figure}[tbp]
  \centerGraphic{eps/ch3/isotope-torus.pdf}
  \caption{An analysis of ``Isotope''(solo changes) in the m3-torus.}
  \label{mts:isotope-torus}
\end{figure}

An analysis of ``Isotope'' in the m3-torus is given in Figure
\ref{mts:isotope-torus}. The solo changes begin with a tonic major-minor
seventh chord for four bars, before moving to the IV chord in m.~5. Instead of
moving directly from IV to I in m.~7, Henderson moves first to \Bflat,
resulting in a variant of the backdoor progression \h{Bb7}--\h{C7} in
mm.~6--7. (We could easily define a BD\tsub{blues} transformation, similar to
TF\tsub{blues} from the last chapter; for now, we can simply understand this
variant as BD $\odot$ \textsc{7th}.) This tonic \h{C7} is extended by moving
to \h{A7}, a minor third away. What would be a string of $T_5$ operations,
\h{A7}--\h{D7}--\h{G7}--\h{C7}, is disrupted slightly by the tritone
substitution of \h{Ab7} for \h{D7} (represented in the m3-torus by
replacing a chord with the chord two spaces to its left or right). Once the
progression returns to the tonic \h{C7} in m.~11, Henderson uses a complete
minor-third cycle as a turnaround, maintaining tonic function for two measures
before beginning the next chorus.

% xxx check on this functional analysis

The logic of the minor-third substitution means that the \h{Bb7} chord in m.~6
is understood as a substitute for true dominant \h{G7}. Likewise, the \h{A7}
chord in m.~8 seems to substitute for the tonic in the same way; in an
ordinary 12-bar blues, mm.~7--8 would both contain tonic. This correspondence
suggests an interesting possibility: the top row of harmonies in Figure
\ref{mts:isotope-torus} all seem to have tonic function, while the two chords
used in the bottom row both act as dominants. The harmonies in the middle row,
then, seem to serve as predominants, appearing in ``Isotope'' just before the
\h{G7} and \h{Bb7} chords.

This functional analysis helps to make sense of the unusual turnaround in the
last two bars of the tune. Turnarounds are inherently prolongational
structures, a way of providing harmonic interest between a chorus-concluding
tonic and the next chorus-beginning tonic.\fn{mts-8} While most turnarounds
use functional harmony (a \tfo progression or some variant), Henderson uses a
non-functional minor-third cycle. Coming as it does at the end of the chorus,
which makes liberal use of minor-third substitutions, we are primed to hear
this cycle as a unique way of prolonging (or maintaining) tonic harmony
without making use of functional harmonic progression.

\begin{figure}[tbp]
  \centerGraphic[width=\textwidth]{eps/ch3/isotope-head.pdf}
  \caption{``Isotope,'' head (Joe Henderson).}
  \label{mts:isotope-head}
\end{figure}

This minor-third motion is prefigured in the head changes, given in Figure
\ref{mts:isotope-head}. (Because the changes are altered in the head in order
to fit the melody, this figure gives the melody as well.) Most of the
alterations between the head and solo changes occur in the first four bars of
the tune; the remaining differences are relatively insignificant.\fn{mts-11}
While the solo changes give only \h{C7} in the first four bars, the head
changes elaborate this harmony with an alteration of a I--II--V--I
progression: the II$^7$ chord in m.~2 is preceded by \h{Eb7}. While it would
be easy to write this chord off as a simple upper-neighbor harmony to a more
structural II chord, doing so would minimize the important role of the
minor-third substitution in the rest of the tune. What appears at first to be
an inconsequential embellishment gains in significance throughout the tune,
first becoming realized in the backdoor progression in mm.~5--7 and reaching
its fullest expression in the turnaround that ends the chorus.


Henderson's tour of the m3-torus in ``Isotope'' is interesting for a number of
reasons. First, he is able to create a sense of tonal function and harmonic
progression while using only major-minor (``dominant'') seventh chords. This
is a feature common to many blues tunes, but is stronger in ``Isotope'' given
the pervasive use of minor-third substitutions. Second, this is a tune that
does not seem to make much sense in the descending-fifths \tf space of the
previous chapter. While certainly some of the tune makes use of harmonic
motion in fifths, the backdoor progression in mm.~6--7 and the final
turnaround would appear as seemingly random, non-sensical jumps in the space.

Finally, the progression of ``Isotope'' is not one that is easily explained
using neo-Riemannian theory as it is usually applied to classical music.
Constructing a Tonnetz usually relies on having two varieties of musical objects
under consideration (commonly major and minor triads, or half-diminished and
dominant seventh chords), while our m3-torus only uses major-minor sevenths.
Neo-Riemannian theories often focus on smooth voice leading; measured in these
terms, the \h{Bb7} and \h{C7} chords of mm.~5--6 are quite distant from one
another, despite their functional equivalence to a V--I motion.\fn{mts-9} The
turnaround, most unusual from a tonal perspective, is actually quite typical
of patterns usually analyzed in neo-Riemannian terms: the major-minor sevenths
found there are all minimal perturbations of a single diminished seventh
chord, and each can be connected to the next with a minimal amount of
voice-leading work (two semitones moving in opposite directions).\fn{mts-10}

\section{Major-third Spaces}
\label{sec:maj3-spaces}

\subsection{Introduction: Coltrane Changes}
\label{subsec:maj3-intro}

Root motion by major third is one of the most difficult harmonic motions to
explain using traditional tonal methods; it is with this kind of music that
transformational methods have proven to be most useful.\fn{maj3-1} The
increasing use of these progressions in nineteenth-century harmony has a
parallel in jazz, as Keith Waters has shown; in both, the harmonic vocabulary
is familiar, but harmonic progressions are often
unfamiliar.\footcite{waters:2013} Nonfunctional harmony becomes a defining
feature of some post-bop jazz, including much of the music of Chick Corea,
Herbie Hancock, Wayne Shorter, and others.\fn{maj3-2} This dissertation,
though, is interested specifically in tonal jazz, and we will remain careful
during this discussion to avoid straying too far afield from this goal.

% xxx need some text here about how we're using Giant Steps as a foil for
% talking about M3-cycles more generally
The locus classicus for major-third root motion in jazz is of course John
Coltrane's ``Giant Steps,'' first recorded in 1959 on the album of the same
name.\fn{maj3-3} Much of the nonfunctional harmony in jazz that develops after
1960 can be traced back to ``Giant Steps''; Keith Waters has traced this
lineage in selected compositions of Wayne Shorter, Bill Evans, and Herbie
Hancock.\footcite{waters:2010} Though we will delay a proper analysis of
``Giant Steps'' until we have developed some formalism, a short overview will
be useful at this point.

\begin{figure}[tbp]
  \centerGraphic[width=22em]{eps/ch3/giant-steps-changes.pdf}
  \caption{Changes to ``Giant Steps'' (John Coltrane).}
  \label{maj3:giant-steps-changes}
\end{figure}

The changes to ``Giant Steps'' are given in Figure
\ref{maj3:giant-steps-changes}. The major-third construction of the tune is
readily apparent: the three tonal centers are B, G, and \Eflat, as evidenced
by the major seventh chords. These centers are all preceded by their dominants
(mm.~1--3, 5--7) or by \tf progressions in the local key (all other
locations). Though the distance between the key centers is unusual, the
individual progressions are not.\fn{maj3-4} ``Giant Steps'' is not exactly
tonal, but neither is it truly atonal. When I listen to the piece, at least,
the impression is not one of ``nonfunctional harmony,'' but rather one of
tonal harmony used in an interesting way. This distinction is best made by way
of an example. Figure \ref{maj3:infant-eyes-changes} gives the changes to
Wayne Shorter's ballad ``Infant Eyes'';\fn{maj3-4a} here there are no \tf{}s, and the only
V--I progressions occur across formal boundaries. Shorter's use of harmony
\emph{does} seem nonfunctional, and gives the piece a floating quality that
``Giant Steps'' does not have. Rather, ``Giant Steps'' is strongly
forward-directed: all of the dominant chords push toward their respective
tonics, and although the global tonic may be in question, local tonic chords
are crystal clear.\fn{maj3-4b}

\begin{figure}[tbp]
  \hspace{6em}\includegraphics[width=28em]{eps/ch3/infant-eyes-changes.pdf}
  \caption{Changes to ``Infant Eyes'' (Wayne Shorter).}
  \label{maj3:infant-eyes-changes}
\end{figure}

\begin{figure}[tbp]
  \centerGraphic[width=25em]{eps/ch3/coltrane-substitution.pdf}
  \caption[Coltrane's major-third cycle as a substitution for a \tfo
    progression.]{Coltrane's major-third cycle as a substitution for a \tfo
    progression. (Adapted from Levine, \emph{The Jazz Theory Book}, 359.)}
  \label{maj3:coltrane-substitution}
\end{figure}

\begin{figure}[tbp]
  \hspace{1em}\includegraphics[width=31em]{eps/ch3/countdown-tune-up.pdf}
  \caption{The changes to ``Countdown'' (Coltrane), compared with
    ``Tune Up'' (Miles Davis).}
  \label{maj3:countdown-tune-up}
\end{figure}

The major-third cycle of ``Giant Steps'' is the most well-known example, but
Coltrane first developed the progression as an elaborate substitution over a
standard \tfo progression; this particular set of substitutions is often
referred to as ``Coltrane changes.''\fn{maj3-5} Figure
\ref{maj3:coltrane-substitution} shows how this process works: the goal of the
progression (in this case, \h{DM7}) is shifted to the fourth bar; then, major
seventh chords related by major thirds are placed on the downbeats (\h{BbM7}
and \h{GbM7}); finally, all of the major sevenths are preceded by their own
dominants.\nocite{levine:1995} This process can clearly be seen in Coltrane's
composition ``Countdown,'' which is based on the changes to Miles Davis's
``Tune Up'' (see Figure \ref{maj3:countdown-tune-up}).\fn{maj3-6}
Coltrane changes can be superimposed over any four-measure \tfo progression,
so they can be found not only in Coltrane's own compositions, but also in his
improvisations on other tunes as well as his reharmonizations of standards
(like ``Body and Soul'').\fn{maj3-7}

\subsection{Developing a Transformational System}
\label{subsec:}

Because Coltrane changes can be considered a \tf variant, it is logical to
include them in this study, even though we may be slightly pushing the limits
of ``tonal jazz.'' We now have a preliminary understanding of how the substitution
works, but it still remains to incorporate it into the transformational system
under development here. First, though, there is a bit of unfinished business
to take care of: in Section \ref{sec:transformational-theory}, we touched on
several transformational approaches only briefly, promising to return to them
at a point when they would be more relevant. Given the central role of
harmonic motion in thirds in many neo-Riemannian theories, it seems
appropriate to fulfill that promise at this point.

% here: connections with Cohn/Weitzmann regions, Santa's enneatonic system
Triads related by major thirds are important in many of these theories,
but especially those of Richard Cohn. In both of Cohn's models of triadic
space in \emph{Audacious Euphony}---hexatonic cycles and Weitzmann
regions---three M3-related major triads combine with three minor triads to
form a six-chord system.\fn{maj3-8} Thus, the three tonal centers of ``Giant
Steps'' can be contained in a single hexatonic cycle or a single Weitzmann
region.

Understanding the rest of ``Giant Steps'' in terms of one of these systems,
though, leaves much to be desired. Cohn's systems, and others like them, are
fundamentally triadic, which is clearly a problem for understanding jazz, with
its saturation of seventh chords (and beyond). To account for this
incongruity, we must either adapt the music to fit our analytical system, or
adapt our analytical system to fit the music. The first option is clearly a
nonstarter: Figure~\ref{maj3:gs-cube-dance} gives a non-example of ``Giant
Steps'' analyzed in Jack Douthett's ``Cube Dance.''\fn{maj3-9}
%
\begin{figure}[tbp]
  \centerGraphic{eps/ch3/gs-cube-dance.pdf}
  \caption[``Giant Steps'' analyzed in Douthett's Cube Dance.]{``Giant
    Steps,'' mm.~1--5, analyzed in Douthett's Cube Dance. Begin by following the
    solid arrows, then continue with the dashed arrows.}
  \label{maj3:gs-cube-dance}
\end{figure}
%
While this analysis makes the major-third cycle of the tonic chords clear,
reducing the chords to triads loses the detail of the chord qualities (both
major-seventh and dominant-seventh chords become major triads), as well as
their functional relationships. Despite the prominent emphasis on M3-cycles in
these theories, then, using these triadic systems as an analytical basis for
our work here would not seem to be the answer.

There are of course neo-Riemannian theories involving seventh chords, but
these turn out to be not so helpful for our purposes here either. Theories
that include only the (0258) tetrachords---half-diminished and dominant
sevenths---are obviously not suitable for analyzing the multiple seventh-chord
types in jazz. The seventh-chord analogy of ``Cube Dance'' is what Cohn calls
the ``4-Cube Trio,'' which is shown in Figure
\ref{maj3:four-cube-trio}.\fn{maj3-10} As is readily apparent, 4-Cube Trio
does not contain any major seventh chords, so it would also create problems if
pressed into use for analyzing ``Giant Steps.''

% xxx do we really need this 4-cube trio graphic?
\begin{figure}[tbp]
  \centerGraphic{eps/ch3/four-cube-trio.pdf}
  \caption[Cohn's ``4-Cube Trio'']{Cohn's ``4-Cube Trio'' (\emph{Audacious
    Euphony}, Fig.~7.16, 158.) The black triangles indicate minor seventh chords,
    while the hollow stars indicate French sixth chords.}
  \label{maj3:four-cube-trio}
\end{figure}

There is, though, a more fundamental problem with these neo-Riemannian theories
of seventh chords, at least when approaching M3-cycles in jazz. As we have
mentioned, many neo-Riemannian theories focus on
efficient voice-leading, and parsimonious relationships among seventh chords
can understood as minimal perturbations of fully-diminished seventh
chords.\fn{maj3-11} This does not generate major-third cycles (as in the
triadic case), but rather partitions the octave into \emph{minor} thirds. The
three dominant sevenths appearing in ``Giant Steps,'' for example, are in
three different ``towers'' in the 4-Cube Trio, which does not reflect
the organizing influence of major thirds in the way that the triadic Cube
Dance does.\fn{maj3-12}

Closer to our purposes here is Matthew Santa's nonatonic system for analyzing
Coltrane.\footcite{santa:2003} In a parallel with Cohn's hexatonic systems,
Santa draws from the nonatonic (or enneatonic) collection in order to explain
``Giant Steps'' in terms of parsimonious voice leading. Figure
\ref{maj3:nonatonic-cycle} shows one of Santa's cycles, along with three-voice
parsimonious realization. (In this figure, note that the triangle indicates a
major \emph{triad}, not a major seventh chord.)
%
\begin{figure}[tbp]
  \centerGraphic{eps/ch3/nonatonic-cycle.pdf}
  \caption[Matthew Santa's nonatonic cycles.]{Matthew Santa's nonatonic
    cycles: the ``Western'' nonatonic cycle (left), and a three-voice parsimonious
    realization (right). (Adapted from ``Nonatonic Progressions in Coltrane,''
    14.)}
\label{maj3:nonatonic-cycle}
\end{figure}
%
This nonatonic system seems to be a convincing analysis of the opening of
``Giant Steps,'' but it comes up a bit short as a general theoretical
system. First, Santa considers only major triads and incomplete dominant
seventh chords; all major seventh chords are reduced to triads, and minor
sevenths---like the \ii chords of ``Giant Steps''---are simply
ignored.\fn{maj3-13} The cycle in Figure \ref{maj3:nonatonic-cycle} is
generated by the collection \{D, \Eflat, E, F\sharp, G, \Aflat, \Bflat, B,
C\}, but it does not contain all of the major triads or incomplete dominant
sevenths in that collection. This cycle also contains major triads and dominant
sevenths rooted on E, \Aflat, and C, not to mention minor triads on these
plus \Eflat, G, and B. Santa's 4-cycle system, then, is somewhat misleading,
since any triad or dominant seventh can be located in two different nonatonic
collections.

Having brought up all of these approaches only to show their shortcomings,
though, the question remains: what should a transformational system that
includes major-third relations look like? Although the relationship of
M3-cycles and smooth voice leading is valuable, so far in this study we have
focused primarily on functional relationships, and it would seem foolish
to abandon that approach here. As we noted above, ``Giant Steps'' does contain
a major-third cycle, but within that cycle the progressions are functional: it
is locally diatonic, but globally chromatic.

As it turns out, we can once again adapt \tf space in order to show
organization by major thirds, as shown in Figure \ref{maj3:maj3-space}. This
figure looks very similar to the minor-thirds organization in Figure
\ref{mts:m3-space} (p.~\pageref{mts:m3-space}), but the relationships have
changed.\fn{maj3-14} Here, of course, the ``layers'' of the space are arranged
by major thirds ($T_4$), while the descending fifths arrangement is otherwise
unchanged. This arrangement means that all of ``Giant Steps'' happens in a
single horizontal slice of the space. (In fact, the rest of the figure is
unnecessary for ``Giant Steps''; the piece is easier to understand using a
subgraph of the complete M3-space, as shown in Figure
\ref{maj3:giant-steps-subgraph}.) This organization of \tf space reflects our
intuitions about the organization of ``Giant Steps'' and other tunes like it:
by maintaining the \tfo progressions and instead altering the relationships
between them, we can keep both the functional progressions important to
improvising musicians \emph{and} reflect the unusual chromatic organization of
the tune itself.

\begin{figure}[tbp]
  \centerGraphic{eps/xxx.pdf}
  \caption{A major-thirds organization of \tf space.}
\label{maj3:maj3-space}
\end{figure}

\begin{figure}[tbp]
  \centerGraphic{eps/xxx.pdf}
  \caption{A subgraph of M3-space, which contains the entire progression of
    ``Giant Steps.''.}
\label{maj3:giant-steps-subgraph}
\end{figure}

%%% Local Variables: %%%
%%% mode: latex %%%
%%% TeX-master: "../diss" %%%
%%% End: %%%
