%% Chapter 3

\chapter{Thirds Spaces and Parsimony}

The space developed in the last chapter was organized primarily in descending
fifths, and works well for most tonal jazz. Motion by thirds, both major and
minor, also occur less frequently and will be the focus of this chapter.
Harmonic motion organized in thirds is one of the main emphases of non-jazz
transformational theorists; this focus will permit the opportunity to
connect developments in jazz harmony with analogous developments in
nineteenth-century harmony.

\section{Minor-third Substitutions}
\label{sec:minor-third-subst}

The most common dominant substitution in jazz is undoubtedly the tritone
substitution (discussed in Section \ref{sec:tf-space-tritones}), but the
minor-third substitution---sometimes called the backdoor substitution---is also
relatively common, especially in the bebop era. By way of example, Figure
\ref{mts:ladybird-changes} gives the changes to opening five bars of Tadd
Dameron's ``Lady Bird.'' Normally the progression in mm.~3--4 would function
as a \tf in the key of \Eflat, but in m.~5 here it resolves instead to C. The
expected \tf in C (\h{Dm7}--\h{G7}) is transposed up a minor third, becoming
\h{Fm7}--\h{Bb7}.

\begin{figure}[tbp]
  \centerGraphic[width=22em]{eps/ch3/ladybird-changes.pdf}
  \caption{Changes to ``Lady Bird'' (Tadd Dameron), mm.~1--5.}
  \label{mts:ladybird-changes}
\end{figure}

The identical tendency tones shared by tritone-substituted dominants makes
them relatively easy to understand, but minor-third substitution is more
difficult. Jazz harmony textbooks often do not provide an explanation for the
phenomenon: Jerry Coker simply states that ``the I chord \ldots\ is often
preceded by \h{IV-7} to \h{bVII7}, instead of the usual \h{V7}
chord.''\footcite[82]{coker:elements} The \emph{Berklee Book of Jazz Harmony}
places the \h{bVII7} chord in its chapter on ``modal interchange'' (what might
also be called modal mixture), and notes that its function is ambiguous, since
the chord has dominant quality but not dominant function, since it lacks the
leading tone.\footcite[123--24]{berklee:harmony} Unlike the tritone
substitution, there is no strong voice-leading rational for the minor-third
substitution; it is simply a progression that bebop players often used, and
that we as analysts must now contend with.

There is, though, a certain similarity between the tritone and minor-third
substitutions: just as the tritone evenly divides the octave, so too the minor
third evenly divides the tritone. In the previous chapter, introducing tritone
substitutes to \tf space effectively divided the space in half, splitting the
complete space into ``front'' and ``back'' sides. Repeating the process again
results in a space that looks something like Figure \ref{mts:m3-space} (which
we will call ``m3 space''). The introduction of minor thirds once again
changes the topology of the space. This is somewhat easier to see by focusing
again only on the centrally-located dominant seventh chords; while the tritone
version of \tf space is topologically equivalent to a Möbius strip (recall
Figure \ref{tft:mobius-dominants}), the minor-thirds version of the space is
equivalent to a torus (see Figure \ref{mts:m3-torus}).\fn{mts-1}

\begin{figure}[tbp]
  \centerGraphic{eps/ch3/m3-space.pdf}
  \caption[The complete minor-thirds representation of \tf space (m3 space).]{The
    complete minor-thirds representation of \tf space (\ii chords omitted on
    rear levels for clarity), or m3 space.}
  \label{mts:m3-space}
\end{figure}


\begin{figure}[tbp]
  \centerGraphic{eps/ch3/m3-torus.pdf}
  \caption{The toroidal center of m3 space (the m3-torus).}
  \label{mts:m3-torus}
\end{figure}

Figure \ref{mts:m3-torus} looks remarkably similar to a more familiar toroidal
figure common in music theory: the neo-Riemannian Tonnetz. Despite the surface
similarity though, the two are quite distinct. Like the Tonnetz, the dominant
sevenths at the center of m3 space are arranged into axes of perfect fifths
(verticals), minor thirds (horizontals), and major thirds (northwest-southeast
diagonals, not shown in Figure \ref{mts:m3-torus}).\fn{mts-2} Crucially though, the
vertices in the m3-torus are dominant seventh chords (ordered triples), not
individual notes; the m3-torus does not represent a parsimonious voice-leading
space. The neo-Riemannian Tonnetz can also be drawn to represent triads instead of
individual notes, but the resulting graph (Douthett and Steinbach's
``chicken-wire torus'') no longer resembles the m3-torus here.\fn{mts-3}

The minor-thirds arrangement of \tf space makes it easy to define a
transformation to represent the minor-third substitution (which we will call
BD, for ``backdoor''):%
%
\begin{displaymath}
    \mathrm{BD}(X\,) = Y\text{, where}\ X = (x_r, x_t, x_s) \in \Sdom
    \text{ and}\
    Y = (\,y_r, y_t, y_s) = (x_r + 2, x_s - 4, x_t - 3) \in \Smaj
\end{displaymath}%
%
Musically, there is no compelling reason to define BD such that the third is
calculated from the previous chords seventh and vice versa; there is no
voice-leading connection between them as there is in the TF transformation.
Defining them this way, however, allows the same function to model both the
transformation from minor to dominant seventh and from dominant to major
seventh.\fn{mts-4} In the space, BD is represented as a diagonal line moving
``frontward'' between two layers; see Figure \ref{mts:bd-transformation}. With
this definition, we can easily understand the progression in mm.~3--5 of
``Lady Bird'': \h{Fm7} \TFarrow\ \h{Bb7} $\xrightarrow{\mathrm{BD}}$\ \h{CM7}.

\begin{figure}[tbp]
  \centerGraphic[width=20em]{eps/ch3/bd-transformation.pdf}
  \caption[The BD transformation in m3 space.]{The BD transformation from
    \caph{Bb7} to \caph{CM7} in m3 space.}
  \label{mts:bd-transformation}
\end{figure}

% text

%%% Local Variables: %%%
%%% mode: latex %%%
%%% TeX-master: "../diss" %%%
%%% End: %%%
