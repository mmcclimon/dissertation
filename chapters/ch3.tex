%% Chapter 3

\chapter{Thirds Spaces and Parsimony}

The space developed in the last chapter was organized primarily in descending
fifths, and works well for most tonal jazz. Motion by thirds, both major and
minor, also occur less frequently and will be the focus of this chapter.
Harmonic motion organized in thirds is one of the main emphases of non-jazz
transformational theorists; this focus will permit the opportunity to
connect developments in jazz harmony with analogous developments in
nineteenth-century harmony.

\section{Minor-third Substitutions}
\label{sec:minor-third-subst}

The most common dominant substitution in jazz is undoubtedly the tritone
substitution (discussed in Section \ref{sec:tf-space-tritones}), but the
minor-third substitution---also called the backdoor substitution---is also
relatively common, especially in the bebop era. In this substitution, a
dominant seventh chord is replaced with the dominant seventh whose root is a
minor third above. Often, the ii chord is substituted as well: a \tfo in C
major might become \h{Fm7}--\h{Bb7}--\h{CM7}. The identical tendency tones
shared by tritone substitutes makes them relatively easy to understand, but
minor-third substitution is more difficult. Jazz harmony textbooks often do
not provide an explanation for the phenomenon: Jerry Coker simply states that
``the I chord \ldots\ is often preceded by \h{IV-7} to \h{bVII7}, instead of
the usual \h{V7} chord.''\footcite[82]{coker:elements}



%%% Local Variables: %%%
%%% mode: latex %%%
%%% TeX-master: "../diss" %%%
%%% End: %%%
