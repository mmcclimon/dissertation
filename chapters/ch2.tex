%% Chapter 2

\chapter{\tf Space}


\blankfootnote{Earlier versions of this chapter were presented at
  the annual meetings of the Music Theory Society of the Mid-Atlantic
  (Philadelphia, PA, March 2013) and the Society for Music Theory (Milwaukee,
  WI, November 2014). I am grateful for the many helpful comments and questions
  I received at these conferences (especially those from Stefan Love, Brian
  Moseley, and Keith Waters).}%
%
% xxx rewrite this paragraph
Because most jazz is not purely diatonic, we need to expand our
transformational system to account for more chromatic examples. This chapter
will begin that work, taking the very common \tfo\ progression as its basis.
Most of the spaces we will develop in this chapter will fall under the broad
category of “fifths spaces,” but at the end of the chapter we will have
occasion to return to the diatonic space of Chapter 1 to see how they might be
enhanced with the chromatic spaces introduced here.

% xxx this title is stupid
\section{A Descending Fifths Arrangement}
\label{sec:tf-space-1}
%{{{

\subsection{Formalism}
%{{{

The most common harmonic progression in jazz is undoubtedly the
\mbox{ii\tsup{7}--V\tsup{7}--I\tsup{7}} progression (hereafter, simply \tfo, or
often just \tf). It is the first progression taught in most jazz method books,
and the only small-scale harmonic progression to have an entire Aebersold
play-along volume dedicated to it.\fn{tf-1} The progression is so prevalent
that many jazz musicians describe tunes in terms of their constituent \tf{}s;
a musician might describe the bridge of “All the Things You Are” (shown in
Figure \ref{tf:all-things-bridge}) as being “\tf\ to G, \tf\ to E, then V--I
in F.”\fn{tf-2} Given the importance of the progression for improvising jazz
musicians, it seems natural to use the progression as the basis for developing
a more general transformational model of jazz harmony.

\begin{figure}[tbp]
  \centerGraphic[width=22em]{eps/ch2/all-things-bridge.pdf}
  \caption{The bridge of “All the Things You Are” (Jerome Kern).}
  \label{tf:all-things-bridge}
\end{figure}

\begin{figure}[tbp]
  \centerGraphic[width=12em]{eps/ch2/single-tfo-network.pdf}
  \caption[A transformation network for a single \tfo\ progression.]{A
  transformation network for a \tfo\ in C major: \caph{Dm7}--\caph{G7}--\caph{CM7}.}
  \label{tf:single-tfo-network}
\end{figure}

Figure \ref{tf:single-tfo-network} shows a transformation network for a single
\tfo\ progression; we will begin by developing the formal apparatus for this
progression, after which we can begin to combine \tfo\ progressions to form a
larger musical space.\fn{tf-2a} This figure, with its combination of
general Roman numerals and specific key centers, is designed to reflect how
jazz musicians tend to talk about harmony; we might read this network as “a
\tfo\ in C.” The combination of Roman numerals and key areas bears some
similarity to Fred Lerdahl’s chordal-regional space, but Figure
\ref{tf:single-tfo-network} is a transformation network, while
chordal-regional space is strictly a spatial
metaphor.\footcite[96--97]{lerdahl:2004}

We have encountered one transformation network already (in Figure
\ref{ds:autumn-leaves-bridge}), but we have yet to define the concept
formally. Transformation networks are a major part of David Lewin’s project in
\emph{GMIT}, and have been thoroughly covered in the literature, so we will
not rehearse the formalism at any length here.\fn{tf-3} A transformation
network consists of objects of some kind (here, they are chords), represented
as vertices in a graph, along with some relations (transformations) between
them, represented as arrows. In Lewin's definition, all of the objects in a
transformation network must be members of a single set $S$, and the
transformations must be functions from $S$ into $S$ itself.\fn{tf-4} The
transformations in Figure \ref{ds:autumn-leaves-bridge} in the previous
chapter are indeed Lewinnian transformations (mappings in the G-minor diatonic
set), but the \tfo\ transformation network is more complex.

The transformation TF in Figure \ref{tf:single-tfo-network} is in fact a
\emph{cross-type} transformation, as defined by Julian
Hook.\footcite{hook:2007} Hook expands Lewin’s definition of a transformation
network to include objects of different types, necessary to define
transformations in the \tfo\ progression. The progression contains musical
objects of three types of diatonic seventh chords: minor, dominant,
and major sevenths (in the key of C major, the progression is
\h{Dm7}--\h{G7}--\h{CM7}). Using Hook's relaxed definition, we are free to
define transformations from any set of objects to any other; to understand the
figure above, we need to define the transformation TF such that it maps
ii\tsup{7} chords to V\tsup{7} chords, and V\tsup{7} chords to I\tsup{7}
chords.

\begin{figure}[btp]
  \centerGraphic[width=15em]{eps/ch2/single-tfo-graph.pdf}
  \caption{The underlying transformation graph for a single \tfo\ progression.}
  \label{tf:single-tfo-graph}
\end{figure}

Before defining the transformations, however, we first need to define the sets
themselves. To help with this, Figure \ref{tf:single-tfo-graph} shows the
underlying transformation graph of the transformation network in Figure
\ref{tf:single-tfo-network}. Throughout \emph{GMIT}, Lewin is careful to
distinguish transformation graphs from transformation networks: a graph is an
abstract structure, showing only relations (transformations) between
unspecified set members, while a network realizes a graph, specifying the
actual musical objects under consideration.\fn{tf-5} Because cross-type
transformation graphs contain objects of different types, a node in a
cross-type transformation graph must be labeled with the set from which the
node contents may be drawn (even in the abstract transformation
graph).\footcite[7]{hook:2007} In Figure \ref{tf:single-tfo-graph}, the nodes
are labeled simply $\Smin$, $\Sdom$, and $\Smaj$, which we can understand as the
sets of minor, dominant, and major seventh chords, respectively.

While at its core the \tfo\ progression contains three types of seventh
chords, in reality a jazz musician might add any number of extensions or
alterations to this basic structure. Given this practice, defining the
archetypal progression as being composed of four-note set classes (seventh
chords) seems unnecessarily restrictive. In order to allow for some freedom in
the chord qualities, we will consider only chordal roots, thirds, and
sevenths; these pitches are sufficient to distinguish the three chord
qualities in a \tfo.\fn{tf-5a}

In this chapter, we will represent a chord with an ordered triple $X = (x_r,
x_t, x_s)$, where $x_r$ is the root of the chord, $x_t$ the third, and $x_s$
the seventh. The definitions of the three sets are as follows:\fn{tf-6}
% xxx do these definitions need numbers?
{\novspace%
\begin{align*}
  \Smin &= \{ (x_r, x_t, x_s)\ |\ x_t - x_r = 3; x_s - x_r = 10 \} &
    \text{ii\tsup{7} chords} \\ %
  %
  \Sdom &= \{ (x_r, x_t, x_s)\ |\ x_t - x_r = 4; x_s - x_r = 10 \} &
    \text{V\tsup{7} chords} \\ %
  %
  \Smaj &= \{ (x_r, x_t, x_s)\ |\ x_t - x_r = 4; x_s - x_r = 11 \} &
    \text{I\tsup{7} chords}
\end{align*}}
%
The definitions are intuitive, and have clear musical relevance: ii\tsup{7}
chords have a minor third (interval 3) and minor seventh (interval 10),
V\tsup{7} chords have a major third and minor seventh (intervals 4 and 10),
and I\tsup{7} chords have a major third and major seventh (intervals 4 and
11). Defining the chords this way rather than as four-note set classes offers
the great advantage of flexibility. Using the ordered-triple representation,
the progressions \h{Dm7}--\h{G7}--\h{Cmaj7} and
\h{Dm9(b5)}--\h{G7b13s9b9}--\h{Cmaj7s11} are understood as equivalent, since
the roots, thirds, and sevenths are the same: both progressions are
represented $(2, 5, 0)$--\allowbreak$(7, 11, 5)$--\allowbreak$(0, 4, 11)$.
Because the sets are defined in pitch-class space, the three sets all have
cardinality $12$: each pitch class is the root of exactly one ii\tsup{7},
V\tsup{7}, and I\tsup{7} chord.

With the space of the nodes defined, we can now formulate the transformation
representing a \tfo, which we will call simply ``TF'':
%
{\novspace%
\begin{alignat*}{3}
    \mathrm{TF}(X\,) = Y\text{, where}\ X = (x_r, x_t, x_s) \in \Smin
    & \text{ and}\ &
    Y &= (\,y_r, y_t, y_s) = (x_r + 5, x_s - 1, x_t) \in \Sdom \\
    %
    \mathrm{TF}(Y\,) = Z\text{, where}\ Y = (\,y_r, y_t, y_s) \in \Sdom
    & \text{ and} &
    Z &= (z_r, z_t, z_s) = (\,y_r + 5, y_s - 1, y_t) \in \Smaj
\end{alignat*}}%
%
Again, these definitions are designed to be musically relevant; the
voice-leading diagram in Figure~\ref{tf:tfo-voice-leading} illustrates this
more clearly.\fn{tf-6a} The root of the second chord is a fifth below the root
of the first ($y_r = x_r + 5$), the third of the second chord is a semitone
below the seventh of the first ($y_t = x_s - 1$), and the seventh of the
second chord is a common tone with the third of the first ($y_s = x_t$). In
Lewin's transformational language, if a jazz musician is ``at a ii\tsup{7}
chord'' and wishes to ``get to a V\tsup{7} chord,'' the transformation that
will do the best job is TF: ``move the root down a fifth and the seventh down
a semitone to become the new third.'' (Recall that we may also write
\mbox{ii\tsup{7} \TFarrow\ V\tsup{7}}, rather than TF(ii\tsup{7}) = V\tsup{7}.) Note
that the transformation TF is also valid between V\tsup{7} and I\tsup{7} as
well (the second equation above, involving sets $\Sdom$ and $\Smaj$). TF is
both one-to-one and onto for sets of ordered triples; it maps each ii\tsup{7}
to a unique V\tsup{7}, and each V\tsup{7} to a unique I\tsup{7}. As such, its
inverse ($\mathrm{TF}^{-1}$) is well defined, and allows motion backwards
along the arrows shown in the transformation graph in Figure
\ref{tf:single-tfo-graph}.

\begin{figure}[htbp]
  \centerGraphic{eps/ch2/tfo-voice-leading.pdf}
  \caption{Voice leading in the \tfo\ progression.}
  \label{tf:tfo-voice-leading}
\end{figure}

It is worth mentioning here that TF and \tft\ (which we will define in the
next section) are well-defined operations for any ordered triple of members of
the integers mod-12 (i.e., a member of the set $\intZ_{12} \times \intZ_{12}
\times \intZ_{12}$). There is nothing mathematically incorrect
about the statement $(0, 1, 2) \TFarrow (5, 1, 1) \TFarrow (10, 0, 1)$, for
example, but this succession has little musical relevance for the applications
under consideration here. Because Hook does not formally define what he means
by a ``type,'' the formulation allows for situations like this one, in which
the three types are all members of a single larger set.\fn{tf-7} The advantage
for defining TF as a cross-type transformation is that the content of a single
node in the transformation graph is restricted to members of a 12-element set
of specific ordered-triple configurations.

\begin{figure}[tbp]
  \centerGraphic{eps/ch2/trans-graph-large.pdf}
  \caption[A transformation graph and network for a small portion of \tf\
    space]{A transformation graph (left) and transformation network (right)
    for a small portion of \tf\ space.}
  \label{tf:trans-graph-large}
\end{figure}

With this understanding of the transformations involved in a single \tfo\
progression, we can continue to see how we might connect multiple progressions
in order to form a larger \tf\ space. Because root motion by descending
fifth is extremely common in jazz, we might consider connecting \tfo\
progressions by descending fifth; Figure \ref{tf:trans-graph-large}
illustrates this arrangement both as a transformation graph and a
transformation network.  This descending fifths arrangement means that all of
the chords sharing a root are aligned vertically (directly below \h{GM7} is
\h{G7}, which is itself above \h{Gm7}). This arrangement allows us to define
two more transformations, which we will call simply \textsc{7th} and
\textsc{3rd}:

\vspace{0.5\baselineskip}
\h{7}\textsc{th}$(L) = M$, where $L = (l_r, l_t, l_s) \in \Smaj$ and $M=(m_r, m_t,
m_s)=(l_r, l_t, l_s-1) \in \Sdom$

\h{3}\textsc{rd}$(M) = N$, where $M = (m_r, m_t, m_s) \in \Sdom$ and $N=(n_r, n_t,
n_s)=(m_r, m_t-1, m_s) \in \Smin$
\vspace{0.5\baselineskip}

\noindent Like the TF transformation, the \textsc{7th} and \textsc{3rd}
transformations have clear musical relevance: each lowers the given note by a
semitone. Although adjacent progressions are connected by descending fifth,
the $T_5$ labels connecting adjacent ii\tsup{7} chords and I\tsup{7} chords
are shown in gray in the graph (and omitted in the network, and most later
examples), since these chords are not often directly connected in jazz.

\begin{figure}[thb]
  \centerGraphic{eps/ch2/tf-circle-fifths.pdf}
  \caption{The complete \tf\ space, arranged around the circle of fifths.}
  \label{tf:tf-circle-fifths}
\end{figure}

By extending the network of Figure \ref{tf:trans-graph-large}, we arrive at
the entirety of \tf\ space, as shown in Figure \ref{tf:tf-circle-fifths}.
Because \tf\ space includes cross-type transformations, it does not easily
form a Lewinnian \gis.\fn{tf-8} Considered more generally, though, it is easy
to see that by considering a single \tfo\ progression as a unit, \tf\ space
maps cleanly onto ordinary pitch-class space. As Figure
\ref{tf:trans-graph-large} makes clear, we can consider the \tfo\ in C as
being three perfect fifths above the \tfo\ in \Eflat\ (or put
transformationally, the $T_3$ operation transforms a \tfo\ in C to one in
\Eflat). This formulation does not allow us a means to say, for example, that
“the ii chord in C is $x$ units away from the V chord in \Eflat,” but because
\tfo{}s are rarely split up, falling back on normal pitch-class distance is
sufficient in most situations.\fn{tf-9}

% xxx I could go into ii-V space considered as a graph here...I don’t think
% it’s really all that necessary?


%}}}

\FloatBarrier

\subsection{Analytical Interlude: Lee Morgan, “Ceora”}
\label{sec:ceora-analysis}
%{{{

% xxx This whole section should probably be rewritten

Though we will return to the formalism a bit later, we have defined enough of
\tf\ space at this point to see how it might be useful in analysis. To do so,
we will look at Lee Morgan's composition ``Ceora,'' first recorded on the
album \emph{Cornbread} from 1965. The changes for the A section are given in
Figure \ref{tf:ceora-changes-1}, and the accompanying moves in \tf\ space are
shown in Figure \ref{tf:ceora-space-1}.\fn{tf-10} ``Ceora'' is in the key of
\Aflat\ major, and begins with the progression I--ii--V--I in the first three
bars, staying within a single horizontal slice of \tf\ space. This is followed
immediately by a \tfo\ progression in \Dflat, a fifth lower (mm.~4--5).

\begin{figure}[thbp]
  \centerGraphic[width=22em]{eps/ch2/ceora-changes-1.pdf}
  \caption{Changes for the A section of ``Ceora'' (Lee Morgan).}
  \label{tf:ceora-changes-1}
\end{figure}

\begin{figure}[htbp]
  \centerGraphic{eps/ch2/ceora-space-1.pdf}
  \caption{The A section of ``Ceora'' in \tf\ space.}
  \label{tf:ceora-space-1}
\end{figure}

At this point, we might expect the \tf{}s to continue in descending
fifths, but the potential ii\tsup{7} chord in \Gflat\ is substituted with
\h{Dm7}, its tritone substitute, which then resolves as a \tf\ in C.\fn{tf-11}
Instead of resolving to C major in m.~7, this \tf\ resolves instead to C
\emph{minor}: both the seventh and third of the expected \h{CM7} are lowered
to become \h{Cm7}. (This progression is extremely common, and is one of the
principal means of maintaining harmonic motion in the course of a jazz tune.)

A similar progression in \Bflat\ follows, leading to a \h{Bbm7} chord in m.~9.
We then hear a \tf\ progression in the tonic in mm.~9--10, but the
expected \h{AbM7} does not materialize; the \h{Eb7} chord moves instead to
\h{Cm7} as ii of \Bflat\ (a northwesterly move in the space). This
progression repeats in mm.~11--13, leading once again to the \h{Dm7} chord
first heard in m.~5. The repeated upward motions in the space have the
effect of ramping up the tonal tension in the passage; not only do the
dominants fail to resolve as expected, but their stepwise rising motion take
the music far away from the tonic \Aflat. To release this harmonic tension,
the \tf\ in C resolves at m.~15 to C minor, at which point the harmonic rhythm
doubles and the progression follows the normal descending fifths pattern to
reach the tonic that begins the B section in m.~17.

The B section of ``Ceora'' (shown in Figure \ref{tf:ceora-changes-2}) follows
much of the same trajectory as the A section until the last four bars; the
only differences are the addition of the \h{b5} in the \h{Cm7} and the \h{s9}
in the \h{F7} in mm.~11--12 of the section. Because we have defined chords and
transformations only in terms of chordal roots, thirds, and sevenths, neither
of these changes affect our transformational reading of the passage. Instead
of ramping up to ii\tsup{7} of C as in the A section, the \tf\ in \Bflat\
resolves to \h{Bbm7} in m.~12. This \h{Bbm7} becomes the ii chord of a \tfo\
in tonic, resolving in m.~15 of the section. A final \tf\ in the last measure
provides additional harmonic interest, and functions as a turnaround to lead
smoothly back to \h{AbM7} to begin the next chorus.

\begin{figure}[htbp]
  \centerGraphic[width=22em]{eps/ch2/ceora-changes-2.pdf}
  \caption{Changes for the B section of ``Ceora'' (Lee Morgan).}
  \label{tf:ceora-changes-2}
\end{figure}

At this point, we have successfully mapped all of the chords in ``Ceora'' to
their associated locations in \tf\ space; it is reasonable to ask, though,
whether this mapping of chords to space locations should even count as
``analysis.'' After all, \tf\ space contains each minor, dominant, and major
seventh chord exactly once, so we did not even need to make any decisions as
to where in the space a particular chord should go. Have we, in fact, learned
anything about ``Ceora'' from our exploration of \tf\ space?

The answer, I think, is yes. Although we could criticize \tf\ space for being
simply a particular arrangement of common harmonic progressions in jazz,
similar arrangements have proven themselves useful in many areas of music
theory: the circle of fifths, the neo-Riemannian Tonnetz, the pitch-class
``clock face,'' and countless others.\fn{tf-12} One of the benefits of \tf\
space is that it allows us to easily visualize harmonic motions in jazz. The
succession of chord symbols that make up the changes to ``Ceora'' may make
immediate sense to an experienced jazz musician, but \tf\ space allows others
to to make sense of these relationships more clearly. The fact that our
analysis in \tf\ space may seem obvious is in fact a feature, not a bug; such
a criticism reveals that \tf\ space, with all its mathematical formalism,
can clarify information that may otherwise remain hidden in the raw data of
the chord symbols.\fn{tf-13}

%}}}

%}}}



\section{Tritone Substitutions}
\label{sec:tf-space-tritones}
%{{{

\subsection{Formalism}
%{{{

There is an important aspect of jazz harmony that has not yet been considered in
our discussion of \tf\ space. Crucial to harmony beginning in the bebop era is
the tritone substitution: substituting a dominant seventh chord for the
dominant seventh whose root is a tritone away.\fn{tft-1} Because
tritone-substituted dominants are functionally equivalent, both the
progressions \h{Dm7}--\h{G7}--\h{Cmaj7} and \h{Dm7}--\h{Db7}--\h{Cmaj7} may be
analyzed as \tfo\ progressions in the key of C.

This functional equivalence means that a tritone-substituted dominant can
act as a shortcut to an otherwise distant portion of \tf\ space. In the
circle-of-fifths arrangement of Figure \ref{tf:tf-circle-fifths}, keys related
by tritone are maximally far apart (diametrically opposed on the circle), but
in jazz practice, \h{G7} and \h{Db7} are functionally identical (both
dominant-function chords in C major). To account for this progression in our
space, we need to somehow bring these chords closer together; one solution is
to connect two segments of the space by $T_6$ in a sort of ``third
dimension,'' as shown in Figure \ref{tft:complete-space}. The topology of this
space is more complicated than the ordinary circle of fifths, however. Once a
progression reaches the bottom of the ``front'' side of the figure, it
reappears at the top of the ``back'' side (\Gflat\ at the bottom is listed
again as \h{Fs} at the top); likewise, progressions disappearing off the
bottom of the back side reappear at the top of the front side (C major is
given in both locations).

\begin{figure}[thb]
  \centerGraphic[width=\textwidth]{eps/ch2/complete-space.pdf}
  \caption{The complete \tf space, showing tritone substitutions.}
  \label{tft:complete-space}
\end{figure}

This arrangement of key centers is topologically equivalent to a Möbius strip,
which is somewhat easier to see by focusing only on the dominant seventh
chords, as shown in Figure \ref{tft:mobius-dominants}.\fn{tft-2a} By wrapping this figure
into a circle and gluing the ends together with a half-twist (so that \h{C7}
and \h{Gb7} match up), we arrive at the desired Möbius strip. Though
the underlying topology is easier to visualize this way, it is difficult to
include all of the other progressions (the \tf{}s themselves) in this diagram,
so we will continue to use the ``three-dimensional'' version of Figure
\ref{tft:complete-space}, with the understanding that this topology remains in
effect. In any case, the arrangement of keys into the front and back sides is
arbitrary, and may be repositioned as necessary; it is often convenient to
have the tonic key (when there is one) centrally located at the front of the
space.

\begin{figure}[htbp]
  \centerGraphic{eps/ch2/mobius-dominants.pdf}
  \caption{The Möbius strip at the center of \tf\ space.}
  \label{tft:mobius-dominants}
\end{figure}

While we could navigate this space using only the transformations TF and
$T_6$, it is convenient to define another transformation to help with a common
progression like \h{Dm7}--\h{Db7}--\h{Cmaj7}. We will call this
transformation \tft, to highlight its relationship to the more normative TF:%
%
\begin{alignat*}{3}
    \mathrm{TF}_\mathrm{T}(X\,) = Y\text{, where}\ X = (x_r, x_t, x_s) \in \Smin
    & \text{ and}\ &
    Y &= (\,y_r, y_t, y_s) &= (x_r - 1, x_s + 5, x_t + 6) \in \Sdom \\
    %
    \mathrm{TF}_\mathrm{T}(Y\,) = Z\text{, where}\ Y = (\,y_r, y_t, y_s) \in \Sdom
    & \text{ and} &
    Z &= (z_r, z_t, z_s) &= (\,y_r - 1, y_s + 5, y_t + 6) \in \Smaj
\end{alignat*}%
%
The \tft\ transformation represents a tritone substitution, but it
transforms bass motion by fifth into bass motion by semitone (\emph{not} bass
motion by tritone); the voice-leading diagram in Figure
\ref{tft:voice-leading-tft} clarifies the relationship with the ordinary TF. Because TF and $T_6$ commute,
\tft\ can be considered as either TF followed by $T_6$, or vice versa. With
this new transformation, we can understand the progression
\h{Abm7}--\h{Db7}--\h{CM7} as a substituted \tfo\ in C: \h{Abm7} \TFarrow\
\h{Db7} \TFTarrow\ \h{CM7}.

\begin{figure}[htbp]
  \centerGraphic{eps/ch2/voice-leading-tft.pdf}
  \caption{Voice leading in the TF and \tft\ transformations, compared.}
  \label{tft:voice-leading-tft}
\end{figure}

The introduction of tritone substitutes complicates the space somewhat; Figure
\ref{tft:tft-space-extract} shows a transformation network of the same portion
of the space as in Figure \ref{tf:trans-graph-large}, but with some chords
replaced with their tritone substitutes (shown in green).\fn{tft-2b} The relationship
between a substituted dominant in G major (\h{Ab7}) and the unsubstituted ii\tsup{7}
chord in C (\h{Abm7}) is still, of course, a \textsc{3rd} transformation. The
substituted V\tsup{7} in C moving to the diatonic V\tsup{7} in F changes the
transposition from a descending fifth to a descending half-step, as indicated
by the $T_{11}$ arrow.

\begin{figure}[tbp]
  \centerGraphic[width=.6\textwidth]{eps/ch2/tft-space-extract.pdf}
  \caption{A transformation network for a small portion of \tf\ space, with
    tritone substitutions.}
  \label{tft:tft-space-extract}
\end{figure}

Perhaps most interesting in this tritone-substituted space is the new
relationship between a major seventh chord and the substituted ii\tsup{7} in
the progression a fifth below (in this figure, between \h{GM7} and \h{Abm7}).
Normally there is no voice-leading connection between these two chords, but
with the substituted ii\tsup{7}, the third and seventh are both held as common
tones, and the root and fifth of the chord both ascend by semitone (from
G--B--D--\h{Fs} to \h{Ab}--\h{Cb}--\h{Eb}--\h{Gb}).\fn{tft-3} Because of its
similarity to the standard \textsc{slide} transformation, with the addition of
the common tone seventh, we will call this transformation \slideS:\fn{tft-3a}
%
{\novspace%
\begin{displaymath}
    \text{\slideS}(X\,) = Y\text{, where}\ X = (x_r, x_t, x_s) \in \Smaj
     \text{ and}\
    Y = (\,y_r, y_t, y_s) = (x_r + 1, x_t, x_s) \in \Smin
\end{displaymath}}%
%
In jazz this progression occurs frequently when moving between key
centers related by half step, though it is uncommon in classical
music.\fn{tft-4} We have encountered this transformation once already: the
motion from \h{DbM7} to \h{Dm7} in mm.~5--6 of ``Ceora'' is indeed a typical
\slideS\ transformation (see Figure \ref{tft:slide7-ceora}).

\begin{figure}[htbp]
  \centerGraphic{eps/ch2/slide7-ceora.pdf}
  \caption{The \textsc{slide}\tsup{7} transformation from \protect\caph{DbM7} to
    \protect\caph{Dm7} in mm.~5--6 of ``Ceora.''}
  \label{tft:slide7-ceora}
\end{figure}

% xxx more on slide7 here, the idea of it as an involution (UTTs)?

%}}}


\subsection{Analytical Interlude: Charlie Parker, “Blues for Alice”}
\label{sec:blues-for-alice}
%{{{

Equipped with these new tritone-substitution transformations, we can now
analyze somewhat more complicated music; Charlie Parker’s “Blues for Alice”
will serve as a useful first example (the changes are given in Figure
\ref{tft:blues-alice-changes}).\fn{tft-5} The essential structure of the blues is
present: the tune arrives on a subdominant in m.~5, and on a home-key \tf in
m.~9 of the twelve-bar form.

\begin{figure}[tbp]
  \centerGraphic[width=22em]{eps/ch2/blues-alice-changes.pdf}
  \caption{Changes to ``Blues for Alice'' (Charlie Parker).}
  \label{tft:blues-alice-changes}
\end{figure}

\begin{figure}[tbp]
  \centerGraphic[width=\textwidth]{eps/ch2/blues-alice-space.pdf}
  \caption[An analysis of ``Blues for Alice'' in \tf space]{An analysis of
    ``Blues for Alice'' in \tf space: mm.~1--5 (left), mm.~6--11 (right).}
  \label{tft:blues-alice-space}
\end{figure}


Parker elaborates this basic structure with a series of stepwise descending
\tf progressions (see Figure \ref{tft:blues-alice-space}). The first of
these is a diatonic descent: m.~2 jumps from the tonic F  major to a \tf in D,
which resolves (via the \textsc{7th} and \textsc{3rd} transformations) to a
\h{Dm7} chord as the ii\tsup{7} of C major. We first saw this progression in
``Ceora'' (mm.~6--7), where we noted that it was a very common way of
maintaining harmonic motion; instead of a \tf resolving to its tonic, it resolves
to the minor seventh chord with the same root. Because this progression is so
common, we might consider defining it as its own transformation, which we will
call EC (for ``evaded cadence''). Unlike TF, EC is only useful as a
transformation from V\tsup{7} chords to ii\tsup{7} chords:
%
{\novspace%
\begin{displaymath}
  \mathrm{EC}(X\,) = Y\text{, where}\ X = (x_r, x_t, x_s) \in \Sdom
   \text{ and}\
  Y = (\,y_r, y_t, y_s) = (x_r + 5, x_s - 2, x_t - 1) \in \Smin
\end{displaymath}}%
%
EC is of course equivalent to TF $\odot$ \textsc{7th} $\odot$ \textsc{3rd},
but only when the starting chord is a V\tsup{7} chord (a member of $\Sdom$).
In \tf space, EC can be represented by starting on a dominant, then following
one arrow to the right and two arrows downward. The structure of the space
immediately shows that EC is impossible beginning on a ii\tsup{7} chord; we
can follow a single arrow to the right, but there is only one downward arrow
from a V\tsup{7} chord.\fn{tft-7}

This pattern of stepwise descending \tf{}s continues until arriving at the
subdominant \Bflat\ in m.~5, which includes the standard blues alteration of
the lowered seventh.\fn{tft-6} The intuition that this \h{Bb7} is in fact a
stable harmony, rather than a descending-fifth transposition of \h{F7}, can be
captured somewhat in our transformational labels. Instead of labeling this progression
\mbox{\h{F7} $\xrightarrow{T_5}$ \h{Bb7}}, we might instead label it as
\mbox{\h{F7} $\xrightarrow{\mathrm{TF}\ \odot\ \text{\textsc{7th}}}$ \h{Bb7}};
this designation expresses the notion that we hear the \h{Bb7} chord as a
resolution to a stable chord (the TF transformation) that has merely been
inflected with the lowered seventh (the \textsc{7th} transformation).
Combining this with the rest of mm.~2--5, it is easy to construct a
transformation network:

\begin{center}
  \h{Em7b5} \TFarrow\ \h{A7} \ECarrow\
  \h{Dm7}   \TFarrow\ \h{G7} \ECarrow\
  \h{Cm7}   \TFarrow\ \h{F7} $\xrightarrow{\mathrm{TF}\ \odot\ \text{\textsc{7th}}}$
  \h{Bb7}
\end{center}

After this \h{Bb7} chord, Parker uses a \emph{chromatic} stepwise pattern of \tf{}s
(mm.~5--9), which we can understand as a tritone-substituted version of the
earlier descending fifths pattern:

\begin{center}
  \h{Bbm7} \TFarrow\ \h{Eb7} \ECTarrow\
  \h{Am7}  \TFarrow\ \h{D7}  \ECTarrow\
  \h{Abm7} \TFarrow\ \h{Db7} \ECTarrow\
  \h{Gm7}
\end{center}

\noindent (Here, EC\tsub{T} is the tritone-substituted variant of EC,
equivalent to \tft $\odot$ \textsc{7th} $\odot$ \textsc{3rd}, applied to a
dominant seventh chord.) Once this sequence arrives on \h{Gm7} as the ii chord
of the tonic F, there is a \tfo progression in the home key. After the
resolution in m.~11, the progression moves backwards through fifths space to
begin a VI--ii--V turnaround to F major to begin the next chorus.\fn{tft-8}

So far, we have not said very much about the first two chords in ``Blues for
Alice'': \h{FM7}--\h{Em7}. In \tf space, these chords are relatively far
apart (4 edges): \h{FM7} $\xrightarrow{\text{\textsc{7th}}}$ \h{F7} $\xrightarrow{T_6}$
\h{B7} $\xrightarrow{T_5}$ \h{E7} $\xrightarrow{\text{\textsc{3rd}}}$ \h{Em7}.
Because \tf space prioritizes functional relationships, chord progressions
that are close in terms of voice leading often appear quite distant in the
space. In reality, a musician would probably \emph{not} think of this move as
being distant, since the two chords are so close to one another
in pitch space: to rephrase again in Lewinnian terms, the ``characteristic
motion'' that does the best job in taking us from \h{FM7} to \h{Em7} is
something like ``move the root down a half step and both the third and seventh
down a whole step.'' This transformation is easy enough to define, but would
not be part of \tf space proper; inevitably, the space cannot tell us
everything we want to know about jazz harmony. The space is designed to show
typical harmonic motions, so progressions that do not seem to lie well in \tf
space demand other explanations (and indeed, often they are voice-leading
explanations).

% return to this later, to talk about this compared to diatonic space:
% FM7 --> Em7 is very close to the voice leading of the bridge (if you infer a
% Bb7 chord, then you get FM7 ---> Bb7 ---(ECt)---> Em7


%}}} analytical applications

%}}}

\section{A Few Extensions}
\label{sec:tf-extensions}
%{{{

\subsection{Minor Tonic Chords}
%{{{

As it has been developed thus far, \tf space has a glaring omission: it
requires that all tonic chords be major sevenths. Certainly there are
jazz tunes in minor keys, and thus a need to account for the tonic minor. We
have already seen \tf progressions that resolve to minor chords---we called
that transformation EC in the previous section---but the only minor chords
in the space are ii chords, not tonics. One of the advantages of \tf space is
that it is easily extended to account for harmonic features specific to
particular situations; in this section, we will do just that to allow for stable
minor tonic chords.

The minor \tfmo progression is usually played as
\mbox{\h{ii7b5}--\h{V7s9}--\h{i\thinspace{}mM7}}.\fn{tfe-1} Because
we are working with ordered triples of only roots, thirds, and sevenths in
this chapter, the alteration of the fifth and ninth have no effect on our
definitions of \Smin and \Sdom (defined in Section \ref{sec:tf-space-1}
above). We do, however, need to formally define the set of minor-major seventh
chords (chords with a minor third and major seventh):

\vspace{.5\baselineskip}
  $\SmM = \{ (x_r, x_t, x_s)\ |\ x_t - x_r = 3; x_s - x_r = 11 \}$
\vspace{.5\baselineskip}

Now that we have defined the set of minor-major seventh chords, we can explore
how this set interacts with the three we already know. The \textsc{3rd}
transformation works intuitively, and transforms a major seventh chord to a
minor-major seventh with the same root:

\vspace{.5\baselineskip}
  \h{3}\textsc{rd}$(X\,) = Y$, where $X = (x_r, x_t, x_s) \in \Smaj$ and
  $Y=(\,y_r, y_t, y_s)=(x_r, x_t-1, x_s) \in \SmM$
\vspace{.5\baselineskip}

\noindent Likewise, the \textsc{7th} transformation transforms a
minor-major seventh into a minor-minor seventh with the same root:

\vspace{.5\baselineskip}
  \h{7}\textsc{th}$(Y\,) = Z$, where $Y = (\,y_r, y_t, y_s) \in \SmM$ and
  $Z=(z_r, z_t, z_s)=(\,y_r, y_t, y_s-1) \in \Smin$
\vspace{.5\baselineskip}

\noindent It will also be useful to define versions of the TF and \tft
transformations that transform a dominant seventh into a \emph{minor} tonic,
equivalent to TF $\odot$ \textsc{3rd} or \tft $\odot$ \textsc{3rd}. We will
call them simply ``tf'' and ``\tfmt'' (the lowercase here is meant to parallel
the use of lowercase letters to indicate minor triads):
%
{\novspace%
\begin{alignat*}{3}
    \mathrm{tf}\,(X\,) = Y\text{, where}\ X = (x_r, x_t, x_s) \in \Sdom
    & \text{ and}\ &
    Y &= (\,y_r, y_t, y_s) = (x_r + 5, x_s - 2, x_t) \in \SmM \\
    %
    \mathrm{tf}_\mathrm{T}(X\,) = Y\text{, where}\ X = (x_r, x_t, x_s) \in \Sdom
    & \text{ and}\ &
    Y &= (\,y_r, y_t, y_s) = (x_r - 1, x_s + 4, x_t + 6) \in \SmM
\end{alignat*}}%
%
Note that unlike the standard TF and \tft transformations, tf and
\tfmt only transform V\tsup{7} chords to I\tsup{7} chords; the same
transformations do \emph{not} hold for ii\tsup{7} to V\tsup{7}.

\begin{figure}[tbp]
  \centerGraphic{eps/ch2/tf-minor-tonics.pdf}
  \caption{A small portion of \tf space, including minor tonic chords.}
  \label{tfe:tf-minor-tonics}
\end{figure}

Figure \ref{tfe:tf-minor-tonics} shows a small portion of \tf space that
includes minor tonic chords. Because most jazz tunes do not contain
exclusively minor chords, this figure gives both major and minor tonic chords
in every key. The transformations defined in the previous paragraph are
readily apparent in the space, with the exception of the \textsc{7th}
transformation from a minor-major seventh to a minor seventh---\h{GmM7} moving
to \h{Gm7} as ii\tsup{7} of F major, for example. Though we will not do so
here, determining how to fill in the figure with tritone substitutions, or to
conform it around the circle of fifths in the manner of Figure
\ref{tf:tf-circle-fifths}, is easy enough to imagine (if not to draw, given
the added complexity of the minor-major sevenths).

By way of a brief example, Figure \ref{tfe:solar-changes} gives the changes
for Miles Davis's ``Solar.'' This tune is in C minor, though that is not
immediately apparent from the changes themselves; in the canonical recording
of this piece (from Davis's own \emph{Walkin'}\,), the C minor chords are
played as minor-major sevenths, and the piece ends on a \h{CmM7}
chord.\nocite{davis:walkin} The fact that the only tonic chord appears in the
opening bar of the form gives performances of this tune even more of cyclical
quality than is usual in jazz. By not arriving on a tonic at the end of the
short 12-bar form, Davis achieves a formal overlap at every chorus: the
opening tonic serves simultaneously as the harmonic resolution of the previous
chorus and the formal beginning of the next.

\begin{figure}[tbp]
  \centerGraphic[width=22em]{eps/ch2/solar-changes.pdf}
  \caption{Changes to ``Solar'' (Miles Davis).}
  \label{tfe:solar-changes}
\end{figure}

The analysis in \tf space is mostly unremarkable, but it is given in Figure
\ref{tfe:solar-space}. Note that this figure has replaced \h{CM7} at the top
of the space with a minor tonic, \h{CmM7}, and as such there is no arrow given
between the C-minor tonic and V\tsup{7}/F. This analysis, though, is not
possible in the \tf space of the previous section, since the C-minor harmony
of the first bar is decidedly \emph{not} a ii\tsup{7} chord (it would be ii of
\Bflat, and there is no \Bflat\ major harmony in the piece at all).

\begin{figure}[tbh]
  \centerGraphic{eps/ch2/solar-space.pdf}
  \caption{An analysis of ``Solar'' in \tf space, with C minor tonic.}
  \label{tfe:solar-space}
\end{figure}

%}}}

% other kind of tonics (general ResI transformation)

\subsection{Other Kinds of Tonic Chords}
\label{sec:other-kinds-tonic}
%{{{

We have now solved the problem of tonic chords that happen to be minor-major
seventh chords, but in fact the problem is more general: it would be nice to
have some way of allowing for any kind of tonic chord we might find in real
music. As mentioned in Section \ref{sec:theoretical-approaches}, James McGowan
has argued for what he calls three ``dialects of consonance'' in jazz
(extended tones we might consider consonant): the added sixth, the minor
seventh, and the major seventh.\footcite[76--79]{mcgowan:2005} Both of the
approaches in this chapter so far have focused only on the major seventh, when it
appears atop both major and minor triads. Many Tin Pan Alley tunes end with
tonic add-6 chords (which appear nowhere in \tf space), and as we noted in our
discussion of ``Blues for Alice'', it is very common for a blues tonic to be a
major-minor seventh chord (which appear only as V\tsup{7} chords in the
space).

The solution to this shortcoming of the space is to introduce some general
transformation (which we might call ``\textsc{ResI}'') that could be redefined
as needed for each style.\fn{tfe-2} The generic space would then appear as it
does in Figure \ref{tfe:generic-space}. This space is still arranged in
perfect fifths, and the basic shape of the \tfo progressions is still present,
but the quality of the tonic chords is unspecified. Before using this space in
analysis, of course, we must actually define what we mean by a ``tonic chord''
in a given situation. Because \tf space contains cross-type transformations,
this means we need to define both the \emph{set} of tonic chords and the
transformation \textsc{ResI}, from \Sdom to the set of tonics.\fn{tfe-2a} (By defining
\textsc{ResI} to be equivalent to TF and defining tonic chords to be members
of \Smaj, for example, the generic space here becomes the specific layout of
\tf space first presented in Figure \ref{tft:complete-space}.)

\begin{figure}[tbp]
  \centerGraphic[width=.65\textwidth]{eps/ch2/generic-space.pdf}
  \caption{A generic version of \tf space, with unspecified tonic chords.}
  \label{tfe:generic-space}
\end{figure}

Again, it will be easiest to demonstrate exactly how this generic space can be
actualized by means of an example. In our analysis of ``Blues for Alice'' in
Section \ref{sec:blues-for-alice} above, we noted that the \h{Bb7} chord in
m.~5 served as the resolution of the \tf in the preceding bar, but contained
the lowered seventh, which is typical for the blues. There, we tried to
capture the intuition that the \h{Bb7} was stable by labeling the
transformation as TF $\odot$ \textsc{7th}: a resolution merely inflected with
the lowered seventh. This transformation, though, still results in the \h{Bb7}
as a dominant seventh chord (it appears in the space only as V\tsup{7} of F).

The generic \textsc{ResI} transformation offers a better solution, in that we
can define a ``blues TF,'' which resolves a V\tsup{7} to a tonic major-minor
seventh:%
%
\begin{alignat*}{2}
  S_{\mathrm{IMm7}} &= \{ (x_r, x_t, x_s)\ |\ x_t - x_r = 4; x_s - x_r = 10 \} & \\
  %
  \mathrm{TF}_{\mathrm{blues}}(X\,) &= Y\text{, where}\ X = (x_r, x_t, x_s)
    \in \Sdom \text{ and}\ Y &=&\ (\,y_r, y_t, y_s) \\
     &                     &=&\ (x_r+5, x_s-1, x_t-1) \in S_{\mathrm{IMm7}}
\end{alignat*}
%
Note that TF\tsub{blues} is equivalent to $T_5$, but is defined in a way to
demonstrate its similarity to TF (see the voice leading in Figure
\ref{tfe:tf-blues-voice-leading}; TF\tsub{blues} is undefined on ii\tsup{7}
chords). We have also defined the set $S$\tsub{IMm7}, the set of tonic
major-minor seventh chords; this seems intuitive, but is somewhat
complicated. $S$\tsub{IMm7} is exactly equivalent to \Sdom---in the language
of set theory, they are the \emph{same set}. The difference between them is
not structural, but interpretive: $S$\tsub{IMm7} is the set of \emph{tonic}
major-minor seventh chords, while \Sdom is the set of \emph{dominant}
major-minor sevenths. This distinction allows us to capture the difference
between \h{Bb7} as a stable resolution (as it is in m.~5 of ``Blues for
Alice'') and \h{Bb7} as V\tsup{7} of \Eflat\ (as in m.~8 of ``Solar'', for
example).

\begin{figure}[tbp]
  \centerGraphic{eps/ch2/tf-blues-voice-leading.pdf}
  \caption{Voice leading in the TF\tsub{blues} transformation.}
  \label{tfe:tf-blues-voice-leading}
\end{figure}

This sort of interpretive analysis lies at the heart of Steven Rings's work in
\emph{Tonality and Transformation}; the \gis{}es he develops there are
designed to capture the intuitions that collections of pitches can be
heard (or experienced) differently in different contexts. We can adapt this
work slightly to capture the intuition that tonic major-minor sevenths are
experienced differently than dominant major-minor sevenths; Rings would say
that the two sets have different \emph{qualia}.\footcite[41--43 (and
throughout)]{rings:2011}

The tonal \gis Rings develops in his second chapter consists of ordered pairs
of the form (scale degree, acoustic signal); as he has it, ``the notation
(\sd7, \emph{x}) \ldots\ represents the apperception: `scale degree seven
inheres in acoustic signal \emph{x}.'\,''\footcite[44]{rings:2011} Rings goes
on to describe sets of these ordered pairs, which we will use to capture our
intuitions about the varying roles of the \h{Bb7} chord, as shown
below:\footcite[55]{rings:2011}
%
\begin{displaymath}
  \setlength{\arraycolsep}{3em}
  \begin{array}{cc}
    \begin{Bmatrix}
      (\text{\sd4}, \text{\Aflat}) \\
      (\text{\sd7}, \text{D}) \\
      (\text{\sd5}, \text{\Bflat})
    \end{Bmatrix}
    &
    \begin{Bmatrix}
      (\text{\flat\sd7}, \text{\Aflat}) \\
      (\text{\sd3},      \text{D}) \\
      (\text{\sd1},      \text{\Bflat})
    \end{Bmatrix}
    \\
    \text{\h{Bb7} as dominant} & \text{\h{Bb7} as tonic}
  \end{array}
\end{displaymath}
%
Here, the left figure (read bottom to top as root, third, seventh) represents
\h{Bb7} as a dominant seventh of \Eflat\ (i.e., with \sd5, \sd7, and \sd4)
while the right figure represents the same three pitch classes as a tonic
major-minor seventh (with \sd1, \sd3, and \flat\sd7).\fn{tfe-3} Rings's system
of heard scale degrees allows us to distinguish between the sets
$S$\tsub{IMm7} and \Sdom: the \h{Bb7} in m.~5 of ``Blues for Alice'' is a
member of $S$\tsub{IMm7}, while the \h{Bb7} in m.~8 of ``Solar'' is a member
of \Sdom.

\begin{figure}[tbp]
  \centerGraphic[width=30em]{eps/ch2/blues-tf-space.pdf}
  \caption{A small portion of ``blues \tf space.''}
  \label{tfe:blues-tf-space}
\end{figure}

With the distinction between tonic and dominant minor-major sevenths worked
out, we can now specify the generic space of Figure \ref{tfe:generic-space} to
create what we might call a ``blues \tf space''; a small portion of this space
is shown in Figure \ref{tfe:blues-tf-space}. This space, though, presents
another complication: the top arrow marked with a question mark represents a
transformation  from \h{Bb7} as tonic to \h{Bb7} as dominant.
The pitch classes remain the same, but the \emph{quale} of the chord
changes from tonic to dominant, so this transformation is not simply the
identity ($T_0$).

Because this transformation is one of \emph{quale}, we can turn to Rings's
tonal \gis for an explanation. Intervals in this \gis are measured with
ordered pairs, like the elements themselves: the first element is a
scale-degree interval (measured upward), and the second is a pitch-class
interval.\footcite[46--48]{rings:2011} ``Pivot intervals'' are those intervals
where the second element of the pair is $0$.\footcite[58--66]{rings:2011} In
the situation here, we have what Rings would call a ``pivot fifth'' between
the two \h{Bb7} chords:
%
\begin{displaymath}
  \setlength{\arraycolsep}{3em}
    \begin{Bmatrix}
      (\text{\flat\sd7}, \text{\Aflat}) \\
      (\text{\sd3},      \text{D}) \\
      (\text{\sd1},      \text{\Bflat})
    \end{Bmatrix}
    %
    \xrightarrow[\text{ ``pivot fifth'' }]{(5\mathrm{th}, 0)}
    %
    \begin{Bmatrix}
      (\text{\sd4}, \text{\Aflat}) \\
      (\text{\sd7}, \text{D}) \\
      (\text{\sd5}, \text{\Bflat})
    \end{Bmatrix}
\end{displaymath}
%
The pitch-class interval here is 0, since both chords contain \Bflat, D, and
\Aflat, and the scale-degree interval is a 5th (\sd1 to \sd5, \sd3 to \sd7, and
\flat\sd7 to \sd4). With this transformation defined, we can now more fully
realize our intuitions about the short passage in ``Blues for Alice'' (mm.~4--6):
\begin{center}
  \ldots\ \h{Cm7} \TFarrow\ \h{F7} $\xrightarrow{\mathrm{TF}_{\mathrm{blues}}\ }$
  \h{Bb7} $\xrightarrow{\mathrm{ pivot\ 5th}\ \odot\ 3\text{\textsc{rd} }}$
  \h{Bbm7} \TFarrow\ \h{Eb7} \ldots
\end{center}

%}}}

\subsection{Interaction with Diatonic Spaces}

The preceding consideration of other kinds of tonic chords has taken us
relatively far afield from our starting point, and indeed these extensions are
not necessary to understand most tonal jazz. For many purposes, the
conventional space developed in Sections \ref{sec:tf-space-1} and
\ref{sec:tf-space-tritones} will be sufficient. What is missing in our
treatment so far, though, is the concept of a global tonic. This dissertation,
after all, is interested in tonal jazz, and most of this music (and certainly
all of the examples in this chapter) is in a key. To this point, we have
acknowledged this fact only by mentioning the key of a particular tune in our
analytical commentary, or circling the tonic chord in a representation of the
space. Defining \tf space as a fully chromatic space has many advantages: it
is rare that every chord in a tune can be understood in a single key, and it
is convenient not to have to continually switch between diatonic collections.
Moreover, chromatic spaces are much more regular than their diatonic
counterparts: chromatic step size is consistent, while diatonic step size
varies between one and two half steps.\fn{tfe-4}

Still, given the exploration of diatonic transformational systems in Section
\ref{sec:diatonic-spaces}, it seems wise to consider what a diatonic \tf space
might look like. We first made the space chromatic by arranging individual
\tfo progressions in descending fifths (recall Figure
\ref{tf:tf-circle-fifths}). We could instead arrange the space according to
the \emph{diatonic} circle of fifths, as shown in Figure
\ref{tfe:tf-diatonic-fifths}. This space looks much like the chromatic space,
with the exception of the diminished fifth between \sd4 and \sd7, where the
transformational structure of the chromatic space breaks down. The change of
the descending perfect fifth ($T_5$) to a diminished fifth ($T_6$) means that
all of the transformations linking these two key areas must all be combined
with $T_1$: \h{C7} $\xrightarrow{3\text{\textsc{rd}}\ \odot\ T_1}$ \h{Csm7},
\h{C7} $\xrightarrow{T_6\;}$ \h{Fs7}, and
\h{FM7} $\xrightarrow{7\text{\textsc{th}}\ \odot\ T_1}$ \h{Fs7}.\fn{tfe-5}
% the \textsc{3rd} transformation becomes \textsc{3rd} $\odot\ T_1$,
% from \h{C7} (as V\tsup{7} of F) to \h{Csm7} (ii\tsup{7} of B); \textsc{7th}
% becomes \textsc{7th} $\odot\ T_1$ from \h{FM7} to \h{Fs7}; and the $T_5$
% linking the dominants becomes $T_6$.\fn{tfe-5}

\begin{figure}[tbp]
  \centerGraphic{eps/ch2/tf-diatonic-fifths.pdf}
  \caption{A portion of \tf space, conformed to the white-key diatonic circle
    of fifths.}
  \label{tfe:tf-diatonic-fifths}
\end{figure}

As noted in Chapter 1, diatonic space (the cyclic group $\mathcal{C}_7$) can
be generated by any of its members, while chromatic space ($\mathcal{C}_{12}$)
can only be generated by the members 1, 5, 7, or 11 (half-steps and perfect
fourths/fifths). Diatonic \tf space, then, offers the interesting possibility
of departing from the fifths-based space used so in this chapter, in
favor of some other organization of the space (since any interval we might
choose will generate the entire space). To see how such an organization
might allow us to capture different kinds of analytical insights, I want to
return to briefly to Lee Morgan's ``Ceora.''

In the analysis of ``Ceora'' in Section \ref{sec:ceora-analysis}, we saw that
the whole tune takes place in four key areas: \Dflat, C, \Bflat, and the tonic
\Aflat. Given this organization, we might consider arranging \tf space in
descending diatonic steps, as shown in Figure \ref{tfe:ab-diatonic-space}.
(The entire figure could be wrapped around a circle so that the identical \tfo
progressions in \Aflat\ at the top and bottom of the figure line up.) This
arrangement into steps means that the key areas used in the tune are adjacent
in the space; in the chromatic space of Figure \ref{tf:ceora-space-1}, they
were separated by an intervening fifth.

\begin{figure}[tbp]
  \centerGraphic{eps/ch2/ab-diatonic-space.pdf}
  \caption{An \Aflat{}-major diatonic \tf space, arranged in descending steps.}
  \label{tfe:ab-diatonic-space}
\end{figure}

This figure is structurally a bit different than the other spaces we have been
exploring in this chapter, so it will be helpful to examine it in some detail
before returning to ``Ceora.'' The arrangement into descending steps means
that we can no longer align all of the seventh chords sharing a root; only the
major and minor seventh chords sharing a root are adjacent in the
space.\fn{tfe-5a} The \h{GM7} and \h{Gm7} (as ii\tsup{7}/F) chords are close to
one another, for example, but \h{G7} (V\tsup{7}/C) is farther removed. The key
areas in this figure are not connected by $T_5$, but instead by $t_6$: all of
the roots of the major seventh chords (reading down the right side of the
figure) are members of the 4-flat diatonic collection.\fn{tfe-5b} Though the
diatonic distance between key areas is consistent, the chromatic distance
varies: there are two points in the space connected by half steps rather than
whole steps.

As we saw above, the tritone appearing in diatonic space alters the
transformational structure somewhat: transformations spanning this tritone
must must be combined with $T_1$. Here, the relationship between most
I\tsup{7} and ii\tsup{7} chords is the transformation \textsc{7th}~$\odot$
\textsc{3rd}, but between the keys of \Dflat\ and C (as well as \Aflat\ and
G), it is \textsc{7th}~$\odot$ \textsc{3rd}~$\odot\ T_1$. This transformation
is in fact equivalent to the transformation \slideS (see the detail in Figure
\ref{tfe:diatonic-space-detail}); this diatonic origin is one of the reasons
the \slideS transformation is so common in tonal jazz.

\begin{figure}[tbp]
  \centerGraphic[width=30em]{eps/ch2/diatonic-space-detail.pdf}
  \caption{Detail of diatonic \tf space, showing the \textsc{slide}\tsup{7}
    transformation between key centers related by half-step.}
  \label{tfe:diatonic-space-detail}
\end{figure}

All of ``Ceora'' takes place in a relatively small portion of diatonic \tf
space; Figure \ref{tfe:ceora-space-diatonic} gives an analysis of the A
section in this space (the changes can be found in Figure
\ref{tf:ceora-changes-1}).\fn{tfe-6} The analysis of course looks very similar
to our analysis in Section \ref{sec:ceora-analysis}, but the stepwise
arrangement of the space helps us to show different analytical insights. Moves in
the space that are relatively close in the chromatic fifths arrangement in
Figure \ref{tf:ceora-space-1} appear much larger in this arrangement (the move
from \h{AbM7} to \h{Ebm7}, marked ``4'' in this figure), and vice versa (the
\slideS from \h{DbM7} to \h{Dm7}, marked ``7'').

\begin{figure}[tbp]
  \centerGraphic{eps/ch2/ceora-space-diatonic.pdf}
  \caption{An analysis of ``Ceora'' in diatonic \tf space.}
  \label{tfe:ceora-space-diatonic}
\end{figure}

It is worth noting at this point that although we have adapted \tf space to
show aspects of diatonicism, \tf space is still chromatic. The transformations
are still defined on ordered triples of mod-12 (not mod-7) integers, and
nothing in the \tfo progressions themselves has changed. We used the guiding
influence of a diatonic collection in this section only to choose the key
centers we showed in particular representation of the space. This use reflects
the construction of jazz itself; tunes are often globally diatonic (in a key),
while locally chromatic, using \tfo progressions to tonicize other key areas
to a much greater extent than is usually seen in classical music. The reason
for this is largely practical. The head-solos-head form of most jazz means
that we hear the same progression repeated many times (Morgan's recording of
``Ceora'' runs about 6\,½ minutes, for example), and using only pitches from the
\Aflat-major diatonic collection would quickly become boring.

This arrangement of \tf space, combining chromatic and diatonic operations, is
mathematically complicated. As Steven Rings notes, transformation networks
involving both chromatic and diatonic operations violate Lewin's formal
definition of a transformation network, since they act on different
sets.\footcite[98--99]{rings:2011} The underlying transformation graph is not
path consistent, since two different paths in the space can combine to form
$t_6$: \slideS $\odot$ TF $\odot$ TF (\h{AbM7} to \h{GM7}, for example), and
(\textsc{7th} $\odot$ \textsc{3rd}) $\odot$ TF $\odot$ TF (\h{GM7} to
\h{FM7}).\fn{tfe-7} Put another way, putting \h{GM7} in the top row of Figure
\ref{tfe:ab-diatonic-space} while leaving the transformational labels
unchanged does not work: obeying the $t_6$ arrow requires
\h{FM7} to occupy the row below, but following the other path would give
\h{FsM7}. The graph is, however, realizable: it is possible, as Figure
\ref{tfe:ab-diatonic-space} attests, to fill in the nodes such that the arrows
\emph{do} make sense.\footcite[29]{hook:2007} Both Rings and Hook have shown
that transformation networks that are not path consistent---like the diatonic
\tf space developed in this section---can nevertheless be analytically
productive.

\subsection{Summary}

By using the \tfo as the basis of the transformational spaces developed in
this chapter, we can now understand a large swath of tonal jazz harmony.
Because this progression is so ubiquitous, many jazz tunes can be well
understood using only the spaces here (perhaps with some adaptations, as
suggested in this final section). Treating chords as ordered triples of root,
third, and seventh allowed us to define transformations in a way that is still
valid when the actual form of a chord might differ greatly from instance to
instance. This is a useful abstraction, and we will continue to use it in the
next chapter, where we will also consider relationships among our sets of
ordered triples more generally.

%}}} extensions

%%% Local Variables:
%%% TeX-master: "../diss"
%%% End:
% vim:fdm=marker
