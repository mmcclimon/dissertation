% Chapter 5

\chapter{Rhythm Changes}
\label{chap:rhythm-changes}
\addtocspace
\addtolof{chap:rhythm-changes}
\addtolof[lot]{chap:rhythm-changes}

The discussion of chord-scale transformations in the previous chapter
concludes the theoretical portion of this dissertation; this final chapter
will synthesize this theoretical framework in a series of three longer
analyses. All three of the tunes here---Thelonious Monk's ''Rhythm-a-ning,''
Sonny Stitt's ``The Eternal Triangle,'' and George Coleman's ``Lo-Joe''---are
instances of a harmonic archetype known as ``rhythm changes,'' named from
their origin in George Gershwin's ``I Got Rhythm.''\fn{rcg-1} Because tunes
that use Rhythm changes all share a common harmonic origin, they are an ideal
means to investigate jazz harmony. A complex set of standard substitutions and
harmonic patterns have emerged over the many years jazz musicians have been
playing rhythm changes; the three analyses in this chapter will allow us to
compare these musicians' manipulation of the basic harmonic framework.

\section{Rhythm Changes in General}
\label{sec:rhythm-changes-general}

It is hard to overestimate the influence of Rhythm changes on jazz practice;
along with the blues, it is one of the most common harmonic types in the bebop
era and beyond.\footcite[13]{owens:1995} David Baker lists more than 150
rhythm tunes in \emph{How to Learn Tunes}; some of the most well-known
of these are reproduced in Table
\ref{rcg:rhythm-tunes}.\footcite[42--44]{baker:tunes} Before moving on to the
three analyses in the following sections, it will be useful to examine the
form itself, along with some of its more common substitutions.

\begin{table}[tbp]
  \setlength{\tabcolsep}{12pt}
  \centering
  \begin{tabular}{ll}
    Title  & Composer \\
    \hline
    \rule[1em]{0ex}{1ex}%
    Anthropology        & Charlie Parker/Dizzy Gillespie \\
    Cotton Tail         & Duke Ellington \\
    52nd Street Theme   & Thelonious Monk \\
    The Flintstones     & Hoyt Curtain \\
    Jumpin' at the Woodside & Count Basie \\
    Moose the Mooche    & Charlie Parker \\
    Oleo                & Sonny Rollins \\
    The Serpent's Tooth & Miles Davis \\
    Tiptoe              & Thad Jones \\
    Wail                & Bud Powell
  \end{tabular}
  \caption{Several Rhythm tunes and their composers.}
  \label{rcg:rhythm-tunes}
\end{table}

\subsection{Substitution Sets}
\label{subsec:substitution-sets}

``I Got Rhythm'' is, like many jazz standards, a 32-bar AABA form; the basic
progression is shown in Figure \ref{rcg:rhythm-basic}.\fn{rcg-2} As Andy Jaffe
notes, its changes are ``not the least bit astonishing''; the tune is a fairly
basic set of turnarounds and dominant cycles.\footcite[149]{jaffe:1996}
Indeed, this feature is one of the reasons for its popularity: the harmonic
framework is something of a blank slate, and allows room for alteration in a
way that more specific sets of changes (Parker's ``Blues for Alice,'' for
example) do not. Another thing that is immediately apparent is the quick
harmonic rhythm in the A sections, which allows soloists the opportunity to
show off as they navigate the rapidly moving changes.\fn{rcg-3}

\begin{figure}[tbp]
  \centerGraphic[width=24em]{eps/ch5/rhythm-basic.pdf}
  \caption[The basic changes to George Gershwin's ``I Got Rhythm.'']{%
    The basic changes to George Gershwin's ``I Got Rhythm'' (taken from
    Levine, \emph{The Jazz Theory Book}, 238).}\nocite{levine:1995}
  \label{rcg:rhythm-basic}
\end{figure}

The ``mix-and-match'' aspect of harmonic progression is crucially important to
the genre of Rhythm tunes.\fn{rcg-4} Mark Levine explains the problem
nicely:%
%
\begin{quoting}
  \singlespacing
  When a musician calls a Rhythm tune like ``Oleo,'' there's no discussion of
  which version of the changes to play. As with the blues, jazz musicians
  freely mix many versions of Rhythm changes on the spot, as they improvise.
  Playing Rhythm changes is a little like knowing several tunes and playing
  them all at once; that's why ``Rhythm'' tunes are harder to play at first
  than a tune with only a single set of changes.\footcite[241]{levine:1995}
\end{quoting}
%
Given this background, we will proceed with the analytical discussion in
segments: the A sections can each be broken into four-bar halves, while the
bridge is typically treated as a single eight-bar unit.

The first four bars of the Rhythm A section serve to prolong the tonic \Bflat;
Figure \ref{rcg:rc-first-four} gives a number of possible harmonizations of
this section.\fn{rcg-5} Letter \emph{a} gives the original Rhythm changes,
while letter \emph{b} shows what is by far the most common substitution,
substituting \h{Dm7} for \h{Bb} in the third bar; this replaces the
I--vi--ii--V turnaround in the last two bars with a iii--vi--ii-V instead. If
the \h{Bb} harmony is voiced with a major seventh and major ninth
(\Bflat--D--F--A--C, a very common voicing), then we can understand the
substitution of \h{Dm7} as a simple omission of the root. Letter \emph{c}
transforms many of the minor seventh chords into dominant sevenths that lead
more strongly into the following harmonies.

\begin{figure}[tbp]
  \centerGraphic{eps/ch5/rc-first-four.pdf}
  \caption{Several harmonizations of Rhythm changes, mm.~1--4.}
  \label{rcg:rc-first-four}
\end{figure}

\begin{figure}[tbp]
  \centerGraphic{eps/ch5/rc-first-four-space.pdf}
  \caption{The first four bars of Rhythm changes in \tf space.}
  \label{rcg:rc-first-four-space}
\end{figure}

Figure \ref{rcg:rc-first-four-space} shows the relevant portion of \tf space
for these first three harmonizations, along with a few annotations. The
standard harmonization of figure \ref{rcg:rc-first-four}a can be seen by
following the blue arrows. The substitution of \h{Dm7} in letter \emph{b} is
represented in the space by the red arrows: in this reading, first follow the
blue arrows until arriving at \h{F7}, then follow the red arrows until \h{Gm7}
where the blue arrows continue to the tonic \h{Bb}. The minor-to-dominant
substitutions of letter \emph{c} are not shown in the space, but are easy
enough to imagine: both \h{Gm7} and \h{Dm7} are transformed by the
\textsc{3rd}$^{-1}$ operation, and each is replaced by the chord immediately
to its north in \tf space (a substitution which results in the transformation
\mbox{\h{G7} \ECarrow\ \h{Cm7}} across the bar lines at the end of mm.~1 and
3).

The harmonization in Figure \ref{rcg:rc-first-four}d is still more complex.
The tritone substitution of \h{Db7} for \h{G7} in m.~3 is by now familiar, but
the \h{G7} in m.~2 has been replaced with a passing diminished seventh chord.
As we first saw in the analysis of ``Have You Met Miss Jones?'' in Section
\ref{subsec:miss-jones}, fully-diminished sevenths in jazz can be understood
as \h{V7b9} chords missing their roots. The \h{Bo7} here, then, is a logical
substitution for \h{G7b9}, and the \h{Cso7} in the following bar can be
understood as the same substitution of an implied \h{A7b9} chord (the dominant
of the following D minor), resulting in a chromatically ascending bass line in
the first two bars. Miles Davis's composition ``The Serpent's Tooth'' (the
opening of which is shown in Figure \ref{rcg:serpents-tooth}) uses a variation
of this progression. Davis also includes a minor-third substitution in m.~4,
substituting \h{Ebm7}--\h{Ab7} for the diatonic \h{Cm7}--\h{F7}.

\begin{figure}[tbp]
  \centerGraphic{eps/ch5/serpents-tooth.pdf}
  \caption{``The Serpent's Tooth'' (Miles Davis), mm.~1--4.}
  \label{rcg:serpents-tooth}
\end{figure}

The last two harmonizations in Figure \ref{rcg:rc-first-four} are somewhat
different in nature; while any of the substitutions of letters
\emph{a}--\emph{d} can be swapped in and out at will, those in letters
\emph{e}--\emph{f} usually appear as units. Letter \emph{e} harmonizes the
first four bars with a simple cycle of dominant seventh chords (a favorite
technique of Thelonious Monk, and one we will see in the analysis of
``Rhythm-a-ning'' below). In contrast to the relatively compact arrangement of
letters \emph{a}--\emph{c} in \tf space, this cycle traverses nearly the
entire space before arriving at the tonic \h{Bb}.\fn{rcg-6} Letter \emph{f} is
the harmonization from Jimmy Heath's composition ``C.T.A.,'' and features a
lament-bass pattern from \h{Bb} down to \h{F7}, repeated twice.

\begin{figure}[tbp]
  \centerGraphic{eps/ch5/rc-next-four.pdf}
  \caption{Several harmonizations of Rhythm changes, mm.~5--8.}
  \label{rcg:rc-next-four}
\end{figure}

The last four bars of the Rhythm A section contain a shift to the subdominant
in the first two bars, followed by a turnaround in the last two; Figure
\ref{rcg:rc-next-four} gives several common harmonizations of this passage.
Once again, letter \emph{a} reproduces the original changes; a seventh is
added to the tonic \Bflat, tipping it towards an \Eflat that resolves plagally
(via minor iv) back to tonic before a standard vi--ii--V turnaround. This
plagal motion in the second bar is often substituted with a backdoor
progression, as seen in \emph{b} (which also precedes the \h{Bb7} in the first
bar with its own \ii) and \emph{d} (which elides the \h{Eb} and \h{Ebm}
harmonies). Letter \emph{c} makes the substitution of \h{Dm7} for \h{Bb} in
the third bar, as seen in Figure \ref{rcg:rc-first-four}, and includes
\h{Eo7} as a substitution for \h{Eb7b9}.\fn{rcg-7}

\begin{figure}[tbp]
  \centerGraphic{eps/ch5/rc-bridge.pdf}
  \caption{Several harmonizations of the Rhythm bridge, mm.~17--24.}
  \label{rcg:rc-bridge}
\end{figure}

\begin{figure}[tbp]
  \centerGraphic{eps/ch5/rc-bridge-space.pdf}
  \caption{The four Rhythm bridge harmonizations of Figure \ref{rcg:rc-bridge}
    in \tf space.}
  \label{rcg:rc-bridge-space}
\end{figure}

The Rhythm bridge is usually recognizable because of the drastic slowing of
the harmonic rhythm; again, Figure \ref{rcg:rc-bridge} gives several common
harmonizations, and Figure \ref{rcg:rc-bridge-space} shows them in \tf space.
The standard bridge (letter \emph{a}) is a simple cycle of dominants,
beginning on the III chord. The most common substitutions here are tritone
substitutions of every other chord, as shown in \emph{b}--\emph{c}. The other
common option is to insert \ii chord before each of the dominants, as shown in
\emph{d}, decomposing each $T_5$ transformation into TF $\bullet$ \textsc{3rd}.
(This procedure could of course be repeated with the tritone-substituted
versions in \emph{b}--\emph{c} as well.) Other less conventional
harmonizations are also possible; ``The Eternal Triangle'' and ``Lo-Joe'' both
use specialized bridges to which we will return in later sections.

It should be apparent from this discussion that Rhythm tunes can vary widely
in their harmonic particulars. The mix-and-match nature of their construction
means that the chords used by an ensemble can change even over the course of a
single performance: a rhythm section might prefer one harmonization of the
bridge during a saxophone solo and opt for another during a piano solo, for
example. The harmonizations given in Figures \ref{rcg:rc-first-four},
\ref{rcg:rc-next-four}, and \ref{rcg:rc-bridge} have only begun to scratch the
surface; because most of the tune consists of turnarounds, any of the
countless possible turnarounds could be used instead.\fn{rcg-8} It is easy to
imagine a Rhythm tune that makes use of the descending minor-third turnaround
of Henderson's ``Isotope'' or fast-moving Coltrane changes over the bridge.

\subsection{Harmonic Substitution vs.\ Chord-Scale Elaboration}

Before moving on to the three analyses proper, it will be helpful to return to
an issue first mentioned in the last chapter in connection with Kirk's solo on
``Blues for Alice.'' In many cases, it is not clear whether a particular
improvised passage should be heard as a harmonic substitution or as an
outgoing chord-scale choice over the original harmony. In the case of
non-Rhythm tunes, we can usually rely on the head to provide the authoritative
changes for the tune, and likely choose to hear that particular set of changes
throughout the performance. Rhythm changes, though, bring the problem to the
fore, since we cannot depend on a single set of canonical changes.

By way of a short illustration, consider again the melodic passage that opens
Miles Davis's ``Serpent's Tooth'' (shown in Figure \ref{rcg:serpents-tooth}).
If this were an improvised passage, it seems likely that the first choice of
harmonies would \emph{not} be those used by Davis, given the clear
outlines of both \h{G7} and \h{A7} chords in the second halves of
mm.~1--2. It is also possible to hear this passage as a series of outgoing
scale choices over the standard diatonic progression, hearing the
C\sharp--E--G fragment as part of a diminished scale over \h{F7}, for example.
Three possible hearings of these first two bars are shown in Figure
\ref{rcg:serpents-ambiguous}, which gives locations in chord-scale space for
each harmony. They are shown here in ingoing-to-outgoing order: \emph{a} uses
only diatonic scales, \emph{b} uses the same collections but hears the Lydian
diminished scales over the diminished seventh chords, while \emph{c}
emphasizes more widely shifting diatonic collections and scale choices.

\begin{figure}[tbp]
  \centerGraphic{eps/ch5/serpents-ambiguous.pdf}
  \caption{Three possible hearings of the opening of ``The Serpent's Tooth.''}
  \label{rcg:serpents-ambiguous}
\end{figure}

While this prismatic approach to analysis may have seemed excessive for the
relatively insignificant passages in the last chapter where it was used, it
will take a central role in our study of Rhythm changes. Because the harmonic
structure of the tunes is so fluid, it is impossible to claim with any
certainty that a particular set of changes constitutes some Platonic
\textsc{tune}, in the same way that we might be able to for ``Autumn Leaves''
or ``All the Things You Are.'' To fix a set of definitive changes for a
particular passage is to misrepresent the fundamental nature of Rhythm tunes
in jazz practice; the changes are often ill-defined even among the players
themselves (as the above quotation from Levine attests). Engaging
with a single Rhythm tune, then, constitutes an engagement with an entire
genre of tunes, with all its attendant history.\fn{rcg-9} Transformational
theory, with its ability to refract a passage into many possible
interpretations, offers us a way in to this rich network of harmonic
possibilities inherent to the genre.

\section{Thelonious Monk, “Rhythm-A-Ning”}
\label{sec:rhythm-a-ning}

\subsection{Head}

Thelonious Monk's ``Rhythm-A-Ning'' is a basic Rhythm tune, and as such will
be an illustrative first example. The head of the tune is shown in Figure
\ref{ran:head-melody} as it appears in the \emph{Thelonious Monk Fake
  Book}.\footcite[]{sickler:fakebook} The source recording for this lead sheet
is Monk's album \emph{Criss-Cross} (1963); we will analyze a different
performance below, but the differences in the head are
insignificant.\nocite{monk:crisscross} What is noteworthy about this lead
sheet is that there are no changes given in the head, indicating only that the
solos are to be played over Rhythm changes.\fn{ran-1} This speaks not only to
the ubiquity of the form, but also to its fluidity, since a single definitive
version is not given.

\begin{figure}[tbp]
  \centerGraphic{eps/xxx.pdf}
  \caption{Thelonious Monk, ``Rhythm-A-Ning,'' head.}
  \label{ran:head-melody}
\end{figure}

Nevertheless, there are a few aspects of the head that might have an impact on
a soloist's harmonic choices. The first is the arpeggiation of an \Eflat major
triad in the second bar of the A sections. None of the common sets of changes
in Figure \ref{rcg:rc-first-four} use \Eflat in the second bar, but this
plagal motion is essential to the tune. (The harmonization \h{Cm7}--\h{Cso7}
fits the melody, but does not appear in Monk's recordings, where the bassist
consistently arrives on \Eflat on the downbeat of the second bar.) The other
important feature of the tune is the whole-tone ascent at the end of the
bridge. Monk is well-known for his propensity towards the whole-tone scale,
and we will see this manifest below in his solo on the tune.

Instead of the \emph{Criss-Cross} recording, we will instead focus our
analytical attention on a live recording from 1958, \emph{Thelonious in
  Action}.\nocite{monk:action} This recording is attractive for a number of
reasons. First, tenor saxophonist Johnny Griffin takes eleven full choruses on
the tune, allowing the opportunity to analyze a somewhat longer section of
music than we did in the previous chapter. Second, after Griffin's second
chorus, Monk does not play at all, leaving only the bass and drums to
accompany the tenor saxophone.\fn{ran-2} This, combined with the ambiguity of
the solo changes, provides something of a blank harmonic slate, leaving
Griffin's improvised lines to do the bulk of the harmonic work.

\subsection{Johnny Griffin's Solo}

%%% Local Variables:
%%% mode: latex
%%% TeX-master: "../diss"
%%% End:
