% Chapter 5

\chapter{Rhythm Changes}
\label{chap:rhythm-changes}
\addtocspace
% xxx uncomment next line once a figure is added
\addtolof{chap:rhythm-changes}
\addtolof[lot]{chap:rhythm-changes}

The discussion of chord-scale transformations in the previous chapter
concludes the theoretical portion of this dissertation; this final chapter
will synthesize this theoretical framework in a series of three longer
analyses. All three of the tunes here---Thelonious Monk's ''Rhythm-a-ning,''
Sonny Stitt's ``The Eternal Triangle,'' and George Coleman's ``Lo-Joe''---are
instances of a harmonic archetype known as ``rhythm changes,'' named from
their origin in George Gershwin's ``I Got Rhythm.''\fn{rcg-1} Because tunes
that use rhythm changes all share a common harmonic origin, they are an ideal
means to investigate jazz harmony. A complex set of standard substitutions and
harmonic patterns have emerged over the many years jazz musicians have been
playing rhythm changes; the three analyses in this chapter will allow us to
compare these musicians' manipulation of the basic harmonic framework.

\section{Rhythm Changes in General}
\label{sec:rhythm-changes-general}

It is hard to overestimate the influence of rhythm changes on jazz practice;
along with the blues, it is one of the most common harmonic types in the bebop
era and beyond.\footcite[13]{owens:1995} David Baker lists more than 150
rhythm changes tunes in \emph{How to Learn Tunes}; some of the most well-known
of these are reproduced in Table
\ref{rcg:rhythm-tunes}.\footcite[42--44]{baker:tunes} Before moving on to the
three analyses in the following sections, it will be useful to examine the
form itself, along with some of its more common substitution sets.

\begin{table}[tbp]
  \setlength{\tabcolsep}{12pt}
  \centering
  \begin{tabular}{ll}
    Title  & Composer \\
    \hline
    \rule[1em]{0ex}{1ex}%
    Anthropology        & Charlie Parker/Dizzy Gillespie \\
    Cotton Tail         & Duke Ellington \\
    52nd Street Theme   & Thelonious Monk \\
    The Flintstones     & Hoyt Curtain \\
    Jumpin' at the Woodside & Count Basie \\
    Moose the Mooche    & Charlie Parker \\
    Oleo                & Sonny Rollins \\
    Serpent's Tooth     & Miles Davis \\
    Tiptoe              & Thad Jones \\
    Wail                & Bud Powell
  \end{tabular}
  \caption{Several rhythm changes tunes and their composers.}
  \label{rcg:rhythm-tunes}
\end{table}

``I Got Rhythm'' is, like many jazz standards, a 32-bar AABA form; the basic
progression is shown in Figure \ref{rcg:rhythm-basic}.\fn{rcg-2} As Andy Jaffe
notes, its changes are ``not the least bit astonishing''; the tune is a fairly
basic set of turnarounds and dominant cycles.\footcite[149]{jaffe:1996}
Indeed, this feature is one of the reasons for its popularity: the harmonic
framework is something of a blank slate, and allows room for alteration in a
way that more specific sets of changes (Parker's ``Blues for Alice,'' for
example) do not. Another thing that is immediately apparent is the quick
harmonic rhythm in the A sections, which allows soloists the opportunity to
show off as they navigate the rapidly moving changes.\fn{rcg-3}

\begin{figure}[tbp]
  \centerGraphic[width=24em]{eps/ch5/rhythm-basic.pdf}
  \caption[The basic changes to George Gershwin's ``I Got Rhythm''.]{%
    The basic changes to George Gershwin's ``I Got Rhythm'' (taken from
    Levine, \emph{The Jazz Theory Book}, 238).}\nocite{levine:1995}
  \label{rcg:rhythm-basic}
\end{figure}

The ``mix-and-match'' aspect of harmonic progression is crucially important to
the genre of rhythm changes tunes.\fn{rcg-4} Mark Levine explains the problem
nicely:%
%
\begin{quoting}
  \singlespacing
  When a musician calls a Rhythm tune like ``Oleo,'' there's no discussion of
  which version of the changes to play. As with the blues, jazz musicians
  freely mix many versions of Rhythm changes on the spot, as they improvise.
  Playing Rhythm changes is a little like knowing several tunes and playing
  them all at once; that's why ``Rhythm'' tunes are harder to play at first
  than a tune with only a single set of changes.\footcite[241]{levine:1995}
\end{quoting}
%
Given this background, we will proceed with the analytical discussion in
segments: the A sections can each be broken into four-bar halves, while the
bridge is typically treated as a single eight-bar unit.

%%% Local Variables:
%%% mode: latex
%%% TeX-master: "../diss"
%%% End:
