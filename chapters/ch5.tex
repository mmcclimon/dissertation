% Chapter 5

\chapter{Rhythm Changes}
\label{chap:rhythm-changes}
\addtocspace
\addtolof{chap:rhythm-changes}
\addtolof[lot]{chap:rhythm-changes}

The discussion of chord-scale transformations in the previous chapter
concludes the theoretical portion of this dissertation; this final chapter
will synthesize that theoretical framework in a series of three longer
analyses. All three of the tunes here---Thelonious Monk's ``Rhythm-a-ning,''
George Coleman's ``Lo-Joe'', and Sonny Stitt's ``The Eternal Triangle''---are
instances of a harmonic archetype known as ``Rhythm changes,'' so named for
their origin in George Gershwin's ``I Got Rhythm.''\fn{rcg-1} Because tunes
that use Rhythm changes all share a common harmonic framework, they are an ideal
means to investigate jazz harmony. A complex set of standard substitutions and
harmonic patterns have emerged over the many years jazz musicians have been
playing Rhythm changes; the three analyses in this chapter will allow us to
compare these musicians' manipulation of this basic harmonic framework.

\section{Rhythm Changes in General}
\label{sec:rhythm-changes-general}

It is hard to overestimate the importance of Rhythm changes on jazz practice;
along with the blues, it is one of the most common harmonic types in the bebop
era and beyond.\footcite[13]{owens:1995} David Baker lists more than 150
Rhythm tunes in \emph{How to Learn Tunes}; some of the most well-known
of these are reproduced in Table
\ref{rcg:rhythm-tunes}.\fn{rcg-1a} Before moving on to the
analyses in the following sections, it will be useful to examine the
form itself, along with some of its more common harmonic substitutions.

\begin{table}[tbp]
  \setlength{\tabcolsep}{12pt}
  \centering
  \begin{tabular}{ll}
    Title  & Composer \\
    \hline
    \rule[1em]{0ex}{1ex}%
    Anthropology        & Charlie Parker/Dizzy Gillespie \\
    Cotton Tail         & Duke Ellington \\
    52nd Street Theme   & Thelonious Monk \\
    The Flintstones     & Hoyt Curtain \\
    Jumpin' at the Woodside & Count Basie \\
    Moose the Mooche    & Charlie Parker \\
    Oleo                & Sonny Rollins \\
    The Serpent's Tooth & Miles Davis \\
    Tiptoe              & Thad Jones \\
    Wail                & Bud Powell
  \end{tabular}
  \caption{A selection of Rhythm tunes and their composers.}
  \label{rcg:rhythm-tunes}
\end{table}

\subsection{Substitution Sets}
\label{subsec:substitution-sets}

``I Got Rhythm'' is, like many jazz standards, a 32-bar AABA form; the basic
progression is shown in Figure \ref{rcg:rhythm-basic}.\fn{rcg-2} As Andy Jaffe
notes, its changes are ``not the least bit astonishing''; the tune is a fairly
basic set of turnarounds and dominant cycles.\footcite[149]{jaffe:1996}
Indeed, this feature is one of the reasons for its popularity: the harmonic
framework is something of a blank slate, and allows room for alteration in a
way that more specific sets of changes (like Parker's ``Blues for Alice,'' for
example) do not. Another thing that is immediately apparent is the quick
harmonic rhythm in the A sections, which allows soloists the opportunity to
show off as they navigate the rapidly moving changes.\fn{rcg-3}

\begin{figure}[tbp]
  \centerGraphic[width=24em]{eps/ch5/rhythm-basic.pdf}
  \caption[The basic changes to George Gershwin's ``I Got Rhythm.'']{%
    The basic changes to George Gershwin's ``I Got Rhythm'' (taken from
    Levine, \emph{The Jazz Theory Book}, 238).}\nocite{levine:1995}
  \label{rcg:rhythm-basic}
\end{figure}

Fundamental to the genre of Rhythm tunes is their ``mix-and-match'' nature;
each part of the form has many different sets of changes, from which the
performers may choose freely.\fn{rcg-4} Mark Levine explains this issue
succinctly:%
%
\begin{quoting}
  \singlespacing
  When a musician calls a Rhythm tune like ``Oleo,'' there's no discussion of
  which version of the changes to play. As with the blues, jazz musicians
  freely mix many versions of Rhythm changes on the spot, as they improvise.
  Playing Rhythm changes is a little like knowing several tunes and playing
  them all at once; that's why ``Rhythm'' tunes are harder to play at first
  than a tune with only a single set of changes.\footcite[241]{levine:1995}
\end{quoting}
%
Given this background, the analytical discussion below will proceed in
segments: the A sections can each be broken into four-bar halves, while the
bridge is typically treated as a single eight-bar unit.

The first four bars of the Rhythm A section serve to establish the tonic \Bflat;
Figure \ref{rcg:rc-first-four} gives a number of possible harmonizations of
this section.\fn{rcg-5} Letter \emph{a} gives the original Rhythm changes,
while \emph{b} shows what is by far the most common substitution,
replacing \h{Bb} with \h{Dm7} in the third bar; this changes the
I--vi--ii--V turnaround in the last two bars into a iii--vi--ii-V instead. If
the \h{Bb} harmony is voiced with a major seventh and major ninth
(\Bflat--D--F--A--C, a very common voicing), then we can understand the
substitution of \h{Dm7} as a simple omission of the root. Letter \emph{c}
goes further, transforming many of the minor seventh chords into dominant
sevenths that serve to lead more strongly into the following harmonies.

\begin{figure}[tbp]
  \centerGraphic{eps/ch5/rc-first-four.pdf}
  \caption{Several harmonizations of Rhythm changes, mm.~1--4.}
  \label{rcg:rc-first-four}
\end{figure}

\begin{figure}[tbp]
  \centerGraphic{eps/ch5/rc-first-four-space.pdf}
  \caption{The first four bars of Rhythm changes in \tf space.}
  \label{rcg:rc-first-four-space}
\end{figure}

Figure \ref{rcg:rc-first-four-space} shows the relevant portion of \tf space
for these first three harmonizations, along with a few annotations. The
standard harmonization of figure \ref{rcg:rc-first-four}a can be seen by
following the blue arrows. The substitution of \h{Dm7} in letter \emph{b} is
represented in the space by the red arrows: in this reading, first follow the
blue arrows until arriving at \h{F7}, then follow the red arrows until \h{Gm7}
where the blue arrows continue to the tonic \h{Bb}. The minor-to-dominant
substitutions of letter \emph{c} are not shown in the space, but are easy
enough to imagine: both \h{Gm7} and \h{Dm7} are transformed by the
\textsc{3rd}$^{-1}$ operation, and each is replaced by the chord immediately
to its north in \tf space (a substitution which results in the evaded cadence
transformation, \mbox{\h{G7} \ECarrow\ \h{Cm7}}, across the bar lines at the
end of mm.~1 and 3).

The harmonization in Figure \ref{rcg:rc-first-four}d is still more complex.
The tritone substitution of \h{Db7} for \h{G7} in m.~3 is by now familiar, but
the \h{G7} in m.~2 has been replaced with a passing diminished seventh chord.
As we first saw in the analysis of ``Have You Met Miss Jones?'' in Section
\ref{subsec:miss-jones}, fully-diminished sevenths in jazz can often be understood
as \h{V7b9} chords missing their roots. The \h{Bo7} here, then, is a logical
substitution for \h{G7b9}, and the \h{Cso7} in the following bar can be
understood as the same substitution of an implied \h{A7b9} chord (the dominant
of the following D minor), resulting in a chromatically ascending bass line in
the first two bars. Miles Davis's composition ``The Serpent's Tooth'' (the
opening of which is shown in Figure \ref{rcg:serpents-tooth}) uses a variation
of this progression. Davis also includes a minor-third substitution in m.~4,
substituting \h{Ebm7}--\h{Ab7} for the diatonic \h{Cm7}--\h{F7}.

\begin{figure}[tbp]
  \centerGraphic{eps/ch5/serpents-tooth.pdf}
  \caption{``The Serpent's Tooth'' (Miles Davis), mm.~1--4.}
  \label{rcg:serpents-tooth}
\end{figure}

The last two harmonizations in Figure \ref{rcg:rc-first-four} are somewhat
different in nature; while any of the substitutions of letters
\emph{a}--\emph{d} can be swapped in and out at will (the first two bars
\emph{a} followed by the last two bars of \emph{d}, for example), those in
letters \emph{e}--\emph{f} usually appear intact. Letter \emph{e} harmonizes
the first four bars with a cycle of dominant seventh chords (a favorite
technique of Thelonious Monk, and one we will see in the analysis of
``Rhythm-a-ning'' below). In contrast to the relatively compact arrangement of
letters \emph{a}--\emph{c} in \tf space, this cycle traverses nearly the
entire space before arriving at the tonic \h{Bb}.\fn{rcg-6} Letter \emph{f} is
the harmonization from Jimmy Heath's composition ``C.T.A.,'' and features a
repeated lament-bass pattern from \h{Bb} down to \h{F7}.

\begin{figure}[tbp]
  \centerGraphic{eps/ch5/rc-next-four.pdf}
  \caption{Several harmonizations of Rhythm changes, mm.~5--8.}
  \label{rcg:rc-next-four}
\end{figure}

The last four bars of the Rhythm A section contain a shift to the subdominant
in the first two bars, followed by a turnaround in the last two; Figure
\ref{rcg:rc-next-four} gives several common harmonizations of this passage.
Once again, letter \emph{a} reproduces the original changes: a seventh is
added to the tonic \Bflat, tipping it towards an \Eflat chord that resolves
plagally (via minor iv) back to tonic before a vi--ii--V turnaround. This
plagal motion in the second bar is often substituted with a backdoor
progression, \h{Ab7}--\Bflat, as seen in \emph{b} (which also precedes the
\h{Bb7} in the first bar with a \ii chord in \Eflat) and \emph{d} (which elides the
\h{Eb} and \h{Ebm} harmonies). Letter \emph{c} makes the substitution of
\h{Dm7} for \h{Bb} in the third bar and includes \h{Eo7} as a substitution for
\h{Eb7b9}.\fn{rcg-7}

\begin{figure}[tbp]
  \centerGraphic{eps/ch5/rc-bridge.pdf}
  \caption{Several harmonizations of the Rhythm bridge, mm.~17--24.}
  \label{rcg:rc-bridge}
\end{figure}

\begin{figure}[tbp]
  \centerGraphic{eps/ch5/rc-bridge-space.pdf}
  \caption{The four Rhythm bridge harmonizations of Figure \ref{rcg:rc-bridge}
    in \tf space.}
  \label{rcg:rc-bridge-space}
\end{figure}

The Rhythm bridge is usually recognizable because of the drastic slowing of
the harmonic rhythm; again, Figure \ref{rcg:rc-bridge} gives several common
harmonizations, and Figure \ref{rcg:rc-bridge-space} shows them in \tf space.
The standard bridge (letter \emph{a}) is a simple cycle of dominants,
beginning on the III chord; we will call this the ``4-cycle bridge.'' The most
common substitutions here are tritone substitutions of every other chord, as
shown in \emph{b} and \emph{c}. The other common option is to insert a \ii chord
before each of the dominants, as shown in \emph{d}, decomposing each $T_5$
transformation into \mbox{TF $\bullet$ \textsc{3rd}}. (This procedure could of course
be combined with the tritone-substituted versions in \emph{b} and \emph{c} as
well.) Other less conventional harmonizations are also possible; ``The Eternal
Triangle'' and ``Lo-Joe'' both use specialized bridges which we will see
in later sections.

It should be apparent from this discussion that Rhythm tunes can vary widely
in their harmonic particulars. The mix-and-match nature of their construction
means that the chords used by an ensemble can change even over the course of a
single performance: a rhythm section might prefer one harmonization of the
bridge during a saxophone solo and opt for another during a piano solo, for
example. The harmonizations given in Figures \ref{rcg:rc-first-four},
\ref{rcg:rc-next-four}, and \ref{rcg:rc-bridge} have only begun to scratch the
surface; because most of the tune consists of turnarounds, any of the
countless possible turnarounds could be used instead.\fn{rcg-8} It is easy to
imagine a Rhythm tune that makes use of the descending minor-third turnaround
of Henderson's ``Isotope'' in the A sections or fast-moving Coltrane changes
over the bridge.

\subsection{Harmonic Substitution vs.\ Chord-Scale Elaboration}

Before moving on to the three analyses proper, it will be helpful to return to
an issue first mentioned in the last chapter in connection with Rahsaan Roland
Kirk's solo on ``Blues for Alice.'' In many cases, it is not clear whether a
particular improvised passage should be heard as a harmonic substitution or as
an outgoing chord-scale choice over a more basic harmony. In the case of
non-Rhythm tunes, we can usually rely on the head to provide the authoritative
changes for the tune, and it is likely likely that we choose to hear that
particular set of changes throughout the performance. Rhythm changes, though,
bring this problem to the fore, since we cannot depend on a single set of
canonical changes.

By way of a short illustration, consider again the melodic passage that opens
Miles Davis's ``Serpent's Tooth'' (first shown with Davis's original
progression in Figure \ref{rcg:serpents-tooth}). If this were an improvised
passage, it seems likely that the first choice of harmonies would \emph{not}
be those used by Davis, given the clear outlines of both \h{G7} and \h{A7}
chords in the second halves of mm.~1--2. It is also possible to hear this
passage as a series of outgoing scale choices over a standard diatonic
progression, hearing the C\sharp--E--G fragment as part of a diminished scale
over \h{F7}. Three possible hearings of these first two bars are
shown in Figure \ref{rcg:serpents-ambiguous}, which gives locations in
chord-scale space for each harmony. They are shown here in ingoing-to-outgoing
order: \emph{a} uses only diatonic scales, \emph{b} uses the same collections
but hears the Lydian diminished scales over the diminished seventh chords,
while \emph{c} emphasizes more widely shifting diatonic collections and scale
choices.

\begin{figure}[tbp]
  \centerGraphic{eps/ch5/serpents-ambiguous.pdf}
  \caption{Three possible hearings of the opening of ``The Serpent's Tooth.''}
  \label{rcg:serpents-ambiguous}
\end{figure}

While this prismatic approach to analysis may have seemed excessive for the
relatively insignificant passages in the last chapter where it was used, it
will take a central role in our study of Rhythm changes. Because the harmonic
structure of the tunes is so fluid, it is impossible to claim with any
certainty that a particular set of changes constitutes some Platonic
\textsc{tune}, in the same way that we might be able to for ``Autumn Leaves''
or ``All the Things You Are.'' To fix a set of definitive changes for a
particular passage is to misrepresent the fundamental nature of Rhythm tunes
in jazz practice; the changes are often ill-defined even among the players
themselves (as the above quotation from Levine attests). Engaging
with a single Rhythm tune, then, constitutes an engagement with an entire
genre of tunes, with all their attendant history.\fn{rcg-9} Transformational
theory, with its ability to refract a passage into many possible
interpretations, offers us a way into this rich network of harmonic
possibilities inherent to the genre.

\section{Thelonious Monk, “Rhythm-a-ning”}
\label{sec:rhythm-a-ning}

\subsection{Head}

Thelonious Monk's ``Rhythm-a-ning'' is a basic Rhythm tune, and as such will
be an illustrative first example. The head of the tune is shown in Figure
\ref{ran:head-melody} as it appears in the \emph{Thelonious Monk Fake
  Book}.\footcite[]{sickler:fakebook} The source recording for this lead sheet
is from Monk's album \emph{Criss-Cross} (1963); we will analyze a different
performance below, but the differences in the head are
insignificant.\nocite{monk:crisscross} What is noteworthy about this lead
sheet is that there are no changes given in the A sections; there is only an
indication that the solos are to be played over Rhythm changes.\fn{ran-1} This
speaks not only to the ubiquity of the form, but also to its fluidity, since a
single definitive version is not given.

\begin{figure}[tb]
  \centerGraphic{eps/ch5/ran/head-melody.pdf}
  \caption{Thelonious Monk, ``Rhythm-a-ning,'' head.}
  \label{ran:head-melody}
\end{figure}

Nevertheless, there are a few aspects of the head that might have an impact on
a soloist's harmonic choices. The first is the arpeggiation of an \Eflat major
triad in the second bar of the A sections. None of the common sets of changes
in Figure \ref{rcg:rc-first-four} use \Eflat in the second bar, but this
plagal motion is essential to the tune.\fn{ran-0}  The other
important feature of the tune is the whole-tone ascent at the end of the
bridge. Monk is well-known for his propensity towards the whole-tone scale,
and we will see this manifest below in his solo on the tune.

Instead of the \emph{Criss-Cross} recording, we will instead focus our
analytical attention on a live recording made in 1958, on the album \emph{Thelonious in
  Action}.\fn{ran-a}\nocite{monk:action} This recording is attractive for a number of
reasons. First, tenor saxophonist Johnny Griffin takes eleven full choruses on
the tune, allowing the opportunity to analyze a somewhat longer selection of
music than we did in the previous chapter. Second, after Griffin's second
chorus, Monk does not play at all, leaving only the bass and drums to
accompany the tenor saxophone.\fn{ran-2} This, combined with the ambiguity of
Rhythm changes, provides something of a blank harmonic slate, leaving
Griffin's improvised lines to do the bulk of the harmonic work.

\subsection{Johnny Griffin’s Harmonic Strategies}
\FloatBarrier

When approached with the fast-moving harmonies in the Rhythm A sections, Johnny
Griffin's preferred strategy seems to be to ignore them: he frequently uses
harmonic generalizations in the A sections. Often these generalizations are
diatonic, using the 2\flat\ collection; Figure \ref{ran:jg-diatonic-gen} gives
a representative example from chorus \cnum{8A-2}. While we could perhaps imply
a diatonic set of chord changes like those in Figure
\ref{rcg:rc-first-four}a--b, the rising arpeggios harmonizing the top line
F$_5$--F$_6$ seem to take precedence over any particular harmonization.
Similar rising diatonic patterns can be found in mm.~1--4 (chorus \cnum{1A-1})
and mm.~169--72 (\cnum{6A-2}).

\begin{figure}[tbp]
  \centerGraphic{eps/ch5/ran/jg-diatonic-gen.pdf}
  \caption[Diatonic harmonic generalization in Johnny Griffin's solo on
    ``Rhythm-a-ning.'']{Diatonic harmonic generalization in Johnny Griffin's
    solo on ``Rhythm-a-ning'' (mm.~233--36, 4:01).}
  \label{ran:jg-diatonic-gen}
\end{figure}

Other times, Griffin plays passages that are nearly diatonic, but altered
somewhat to fit an underlying harmony. Figure \ref{ran:jg-diatonic-alt} gives
an example from chorus \cnum{9A-1} (chorus \cnum{11A-1} is similar). In this
passage, the line is mostly diatonic, with the exception of the B\nat\ and
\Aflat in m.~259, implying a \h{G7b9} harmony. This passage, unlike the
diatonic ascent in Figure \ref{ran:jg-diatonic-gen}, fits better with a
diatonic chord progression, as shown in the upper transformation network.
While it is certainly possible to hear a 2\flat\ diatonic swath throughout
these four bars (represented in the lower network), hearing the half-note harmonic rhythm
brings out the contrast between G minor in the first bar and the altered
G dominant seventh in the third.

\begin{figure}[tbp]
  \centerGraphic{eps/ch5/ran/jg-diatonic-alt.pdf}
  \caption[Altered diatonic generalization Griffin's solo.]{%
    Altered diatonic generalization in Griffin's solo (mm.~257--60,
    4:22), with two possible transformation networks.}
  \label{ran:jg-diatonic-alt}
\end{figure}

The most common harmonic generalization Griffin uses is the \Bflat blues
scale, which often appears in the last A section of a chorus. The clearest
example of this is also the first, at the end of his third chorus; this
passage is reproduced in Figure \ref{ran:jg-blues-gen}, and similar clear
statements of the blues scale can be found in \cnum{4A-3}, \cnum{8A-3},
\cnum{10A-3}, and \cnum{11A-3}. Because Griffin generalizes the A sections so
frequently, the two-bar harmonic rhythm of the bridge often sounds like an
acceleration of harmonic activity rather than its usual role as a relaxation
of the half-note harmonic rhythm of the A sections. This blues generalization
in the last A section of a chorus, then, helps to increase the contrast to the
dominant cycle of the bridge.


\begin{figure}[tbp]
  \centerGraphic{eps/ch5/ran/jg-blues-gen.pdf}
  \caption[Griffin's blues generalization in a final A section.]{%
    Griffin's blues generalization in a final A section (mm.~89--96, 1:55).}
  \label{ran:jg-blues-gen}
\end{figure}

The blues scale also provides an explanation for Griffin's seemingly unusual
implication of \h{Dbm} at the end of the first chorus, shown in Figure
\ref{ran:jg-blues-subset}. It is not immediately apparent how to understand the
passage in mm.~25--28 (see Figure \ref{ran:dflat-possibilities}): Griffin
could be superimposing \h{Dbm} over a \Bflat diatonic progression, implying
\h{Bbm7b5}, or using a \Bflat half--whole diminished scale generalization.
Given his inclination for the blues scale in the last A section of the tune,
though, my own hearing leans towards this \h{Dbm} triad as a subset of the
\Bflat blues scale. Griffin also emphasizes the pitches \Dflat and \Aflat at
the end of his second and tenth choruses; the former implies \Dflat
major, while the latter leans more clearly toward \Bflat.

\begin{figure}[tbp]
  \centerGraphic{eps/ch5/ran/jg-blues-subset.pdf}
  \caption[A \protect\h{Dbm} triad as a \protect\h{Bb} blues subset in
  Griffin's solo.]{%
    A \Dflat{}m triad as a \Bflat blues subset in Griffin's solo (mm.~25--32, 0:58).}
  \label{ran:jg-blues-subset}
\end{figure}

\begin{figure}[tbp]
  \centerGraphic{eps/ch5/ran/dflat-possibilities.pdf}
  \caption[Four possible harmonic contexts for a \Dflat{}m triad.]{%
    Four possible harmonic contexts for a \Dflat{}m triad: a pure triad; the
    upper tones of a \protect\caph{Bbm7b5}; a diminished scale subset; and a
    blues scale subset.}
  \label{ran:dflat-possibilities}
\end{figure}

Griffin does not always generalize the A sections; he
sometimes plays the half-note harmonic rhythm of the tune itself. The
clearest example of this occurs in chorus \cnum{7A-2}, which is reproduced in Figure
\ref{ran:jg-change-running}. While it seems clear that Griffin hears the
half-note harmonic changes here, the melodic patterns he plays are harmonically
ambiguous, owing to their limited range. While the progression shown in
\emph{a} is most likely---it combines the passing diminished seventh in the
second bar with the applied dominant of C in the third---we might hear the
harmonization at \emph{b} instead, with \h{A7} in the place of \h{Cso7} and a
tritone substitution for \h{G7b9}. Still other, less conventional, hearings
are possible; letter \emph{c} shows a hearing that moves to \Eflat, by first
moving to \h{Eb7} in the second bar (a nod towards the head's tilt towards the
subdominant in the same formal location), and then via an applied dominant to
a modified \tfo in \Eflat.

\begin{figure}[tbp]
  \centerGraphic{eps/ch5/ran/jg-change-running.pdf}
  \caption[Half-note harmonic rhythm in an A section of Griffin's solo.]{%
    Half-note harmonic rhythm in an A section of Griffin's solo (mm.~201--4,
    3:33).}
  \label{ran:jg-change-running}
\end{figure}

\begin{figure}[tbp]
  \centerGraphic{eps/ch5/ran/change-running-bass.pdf}
  \caption[Griffin's improvised solo line, along with Ahmed Abdul-Malik's bass
  line.]{Griffin's improvised solo line, along with Ahmed Abdul-Malik's bass line
    (mm.~201--4, 3:33). Bass sounds as written.}
  \label{ran:change-running-bass}
\end{figure}

Admittedly, this last hearing may be difficult to discern, not least because
it is so far from the typical first four bars of Rhythm changes. Another
important reason, and one we have yet to consider, is the role of the other
band members in shaping harmony. Indeed, the principal role of the rhythm
section (here, piano and bass) is to provide the harmonic framework for the
soloists. Since Monk does not play during Griffin's solo, we might look to
bassist Ahmed Abdul-Malik's line during these four bars, which is shown in
Figure \ref{ran:change-running-bass}.\fn{ran-3} Abdul-Malik does not seem to use the
half-note harmonic rhythm here, and instead plays a generalization in the
first two bars, walking up the \Bflat major scale (or Lydian, depending on
whether the \Eflat or E\nat\ is heard as the chromatic pitch). The strong
tonic--dominant motion from \Aflat to \Eflat in the third bar seems to imply
some \Aflat harmony, perhaps as a backdoor substitution to the downbeat \Bflat
in the fourth bar. He does gesture towards the home-key \tf in the last bar:
we might hear the \Bflat--G on beats 2--3 as a weak arpeggiation of \h{Cm7},
and the final C as a representative of \h{F7} (which resolves to \Bflat in the
next bar). While this bass line certainly provides insight into Abdul-Malik's
conception of the harmony of these four bars, it does not necessarily tell us
anything more about \emph{Griffin}'s harmonic understanding; it is entirely
possible (and common, as here) that all band members do not share exactly the
same harmonic framework, especially in a Rhythm tune.\fn{ran-3a}

As we have noted before, our job as listeners and analysts is not necessarily
to decide on a set of definitive changes. This ambiguity is a critical part of
understanding exactly what jazz harmony \emph{is}, and carries with it
important epistemological questions---questions, incidentally, which relate to
my own suspicion of the Schenkerian analysis of jazz first sketched in Section
\ref{sec:theoretical-approaches}. If we take Abdul-Malik's bass line as
\emph{the} harmony, are we then to understand some of Griffin's note choices
as incorrect? Or vice versa, if Griffin's solo line represents the true
version of the harmony, why does Abdul-Malik choose to ignore it? In
``Rhythm-a-ning,'' do the changes for the head (determined by whom?) hold
through all of the solos, or is the harmonic framework considered anew in
every chorus? In order to make a Schenkerian voice-leading sketch of the
passage in Figure \ref{ran:jg-change-running}, we would be forced to contend with
these issues, since determining what pitches are consonant (or more
structural) is dependent on being able to identify the underlying harmony
unambiguously.\fn{ran-4} My own contention is that the realities of jazz
performance necessitate a more fluid theoretical conception of harmony; the
prismatic transformational approach allows us to make these
kinds of distinctions by presenting multiple transformation networks
(representing multiple harmonic hearings) of a single passage.

Compared to his strategies for the A sections, Griffin's bridges are
much less varied. In most choruses, he uses the standard 4-cycle bridge;
Figure \ref{ran:jg-standard-bridge} gives an example from his final chorus. In
this passage, Griffin repeats the rising arpeggio, altering it in each two-bar
phrase to fit with the descending-fifths harmonic pattern.\fn{ran-5} He often
provides additional harmonic interest by changing the scale to lead more
strongly to the following harmony (not unlike the technique Joe Henderson used
in the first four bars of his solo on ``Isotope,'' examined in Section
\ref{subsec:isotope-solo}). Figure \ref{ran:jg-bridge-altered} gives a passage
from Griffin's second chorus; here, both the \h{D7} and \h{G7} gain a \flat{}9
in their last two beats, while the \h{F7} gets a \sharp{}5 (or \flat{}13) in
its final bar, which acts as a common-tone connection with the \Bflat blues
scale that follows in chorus \cnum{2A-3}. The other common alteration Griffin
makes is the tritone substitution, as shown in Figure
\ref{ran:jg-bridge-tritones}.


\begin{figure}[tbp]
  \centerGraphic{eps/ch5/ran/jg-standard-bridge.pdf}
  \caption[A 4-cycle bridge from Griffin's last chorus.]{%
    A 4-cycle bridge from Griffin's last chorus (mm.~337--44, 5:30).}
  \label{ran:jg-standard-bridge}
\end{figure}

\begin{figure}[tbp]
  \centerGraphic{eps/ch5/ran/jg-bridge-altered.pdf}
  \caption[A 4-cycle bridge, with chord-scale elaborations that lead
  more strongly toward the following harmony.]{%
    A 4-cycle bridge, with chord-scale elaborations that lead more
    strongly toward the following harmony (mm.~49--56, 1:19).}
  \label{ran:jg-bridge-altered}
\end{figure}

\begin{figure}[tbp]
  \centerGraphic{eps/ch5/ran/jg-bridge-tritones.pdf}
  \caption[The bridge from Griffin's eighth chorus, with tritone
  substitutions.]{%
    The bridge from Griffin's eighth chorus, with tritone substitutions shown
    in green (mm.~241--48, 4:08).}
  \label{ran:jg-bridge-tritones}
\end{figure}

\FloatBarrier
\subsection{Monk’s Solo Harmony}

While Griffin's tenor saxophone solo displays a number of interesting harmonic
formations, Thelonious Monk's own solo on ``Rhythm-a-ning'' exhibits a few
more, and is worth a brief visit here. Monk only plays three choruses on the
tune, and his first is characteristically sparse. Throughout all three A
sections of this first chorus (chorus 12 in the transcription), he plays only
pitches from the \Bflat pentatonic collection---the same collection he uses to
comp behind Griffin's first three choruses.\fn{ran-6} The recurring rhythmic
motive is altered slightly so that it fits the harmonies of the standard
4-cycle bridge in mm.~369--76, before returning to the \Bflat pentatonic
collection for the final eight bars of the chorus.

Beginning in chorus 13, Monk consistently plays an 8-chord dominant cycle in
the A sections (the harmonization first seen in Figure
\ref{rcg:rc-first-four}e). Because this harmonization is so distinct from the
ordinary Rhythm A section, Monk simply arpeggiates each chord to avoid
blurring the overall progression. Playing a winding bebop line through the
dominant cycle might risk the coherence of the substitution, especially if the
bass player did not pick up on this harmonization and played a \Bflat diatonic
bass line.\fn{ran-7} Figure \ref{ran:monk-cycle-a} reproduces chorus
\cnum{13A-1}, showing the dominant cycle in the first four bars, followed by a
\Bflat blues harmonic generalization in the next four.

\begin{figure}[tbp]
  \centerGraphic{eps/ch5/ran/monk-cycle-a.pdf}
  \caption[Monk's dominant-cycle A section and blues generalization.]{%
    Monk's dominant-cycle A section and blues generalization from chorus
    13A\tsub{1} (mm.~385--92, 6:12).}
  \label{ran:monk-cycle-a}
\end{figure}

These dominant-cycle A sections are always paired with bridges that use the
whole-tone scales. The head of ``Rhythm-a-ning'' uses the whole-tone scale in
the bridge (clearly over the \h{F7}, and implied over \h{C7} as well), and its
use in Monk's solo helps to provide coherence to the performance as a whole.
The bridge from chorus 14 is shown in Figure \ref{ran:monk-wt-bridge}; though
the whole-tone collection shifts between adjacent dominant seventh chords, the
passage is harmonically consistent. This uniformity is easy to see in the
chord-scale analysis: every change of harmony is represented by the
transformation \rtrans{$T_5$}{$f_1$}{$0$}.

\begin{figure}[tbp]
  \centerGraphic{eps/ch5/ran/monk-wt-bridge.pdf}
  \caption[Monk's bridge from chorus 14, using the whole-tone scale.]{%
    Monk's bridge from chorus 14, using the whole-tone scale (mm.~433--40, 6:53).}
  \label{ran:monk-wt-bridge}
\end{figure}

This brief analysis of ``Rhythm-a-ning'' has illustrated that no single set of
changes can adequately describe this tune. While in a more standard tune like
``Autumn Leaves,'' individual chords might change slightly (adding an
extension or using a tritone substitution), the basic progression remains
intact; rarely do we encounter a situation like that of Monk's 8-cycle A
section, where an entire set is replaced with another. This mix-and-match
approach to harmony is an essential element of Rhythm tunes. Though the dozens
of Rhythm contrafacts are all based on a single chord progression, the wide
range of harmonic approaches means that no two Rhythm tunes sound exactly alike.

\section{George Coleman, “Lo-Joe”}
\label{sec:lo-joe}

To this point in this study, the analyses have focused on harmony as
reflected in a lead sheet, or how particular performances confirm (or
contradict) these given lead-sheet harmonies. This approach naturally requires
a lead sheet to exist in the first place, which is not always the case. Even in
cases where one does exist, there are several reasons a jazz musician
might want to create a lead sheet anew: it may be inaccurate (often the case
with fake books); it may not reflect a particular recording the musician wants
to emulate (John Coltrane's recording of ``Body and Soul,'' for example, does
not use the standard changes); or the musician may simply want to practice ear
training.

In these cases, the transformational approach to harmony can be used ``in
reverse,'' so to speak; rather than analyzing how a soloist elaborates on a
given set of changes, we can take the raw material of a recording and deduce a
likely set of changes. This section will do just that, using George Coleman's
composition ``Lo-Joe,'' recorded on the album \emph{Amsterdam After Dark}
(1979).\fn{lj-1} ``Lo-Joe'' was recorded somewhat later than the other tunes
analyzed here, and as such is somewhat more harmonically adventurous. It is
recognizably a Rhythm tune, though with a highly altered bridge, and in the
key of \Dflat rather than the usual \Bflat.\fn{lj-2}

Before beginning with the analysis, a few disclaimers are in order. While a
transcription of the head (complete with piano and bass parts) can be found in
Appendix B on p.~\pageref{transcription:lo-joe}, I want to emphasize the fact
that a full transcription is in general \emph{not} necessary to create a lead
sheet, and is provided here only as an expedient to writing about the process.
A skilled jazz musician would likely transcribe only the melody, and determine
the harmonies simply by ear, without necessarily writing anything down. Next,
there is the question of whether or not Coleman and his bandmates ever played
from a lead sheet at all; might we be manufacturing a somehow ``false'' lead
sheet rather than ``reconstructing'' one? This question is not important for
our purposes here, since lead sheets are such a common way of conveying jazz
tunes. Even if Coleman did not give his bandmates a lead sheet, he must have
had \emph{some} means of communicating the harmonic progression of the tune,
and a lead sheet is the canonical way to notate this kind of progression in
jazz.

The opening of melody of ``Lo-Joe,'' shown in Figure \ref{lj:opening-melody},
appears to be straightforward, outlining mostly major triads and major seventh
chords. The resulting succession of harmonies, though, does not seem to
reflect any of the usual Rhythm openings, or indeed any ordinary jazz
progression at all. When combined with the ensemble, though, it becomes clear
that the melody consists primarily of upper extensions to harmonies.\fn{lj-2a}
Figure \ref{lj:opening-ens} gives the same passage from the second A section
along with the piano and bass parts.\fn{lj-3} On the downbeats of the first
and third bars, bassist Sam Jones plays a \Dflat, and pianist Hilton Ruiz
plays an identical voicing. Combined with the knowledge of standard Rhythm A
sections, we can be relatively confident that the harmony here is the tonic
\h{Dbmaj7}; Coleman's opening figure in ascending fourths is then understood
as outlining the 13th, 9th, and 5th of this chord.\fn{lj-4}

\begin{figure}[tbp]
  \centerGraphic{eps/ch5/lj/opening-melody.pdf}
  \caption{George Coleman, ``Lo-Joe,'' melody, mm.~1--4.}
  \label{lj:opening-melody}
\end{figure}

\begin{figure}[tbp]
  \centerGraphic{eps/ch5/lj/opening-ens.pdf}
  \caption[The first four bars of the A section of ``Lo-Joe,'' with
  ensemble.]{%
    The first four bars of the A section of ``Lo-Joe,'' with ensemble
    (mm.~9--12).}
  \label{lj:opening-ens}
\end{figure}

The third-bar tonic is preceded, as usual, by a \tf progression, but the
melody over the \h{Ab7} contains the pitches E\nat, A\nat, and B\nat. These
pitches can be understood as the \sharp{}5 (or \flat{}13), \flat{}9, and
\sharp{}9, of the chord, which are all representative extensions of the
altered chord (hereafter, \alt).\fn{lj-5} Depending on our analytical
priorities, we might analyze this harmony as \h{Ab7alt} (in which case the
extended tones are first-class chord members) or as a diatonic \h{Ab7} in
which the melody notes were part of an outgoing scale choice:
\cst{Ab7}{$4$\sharp}{Lyd.\ dim.}.\fn{lj-5a} In either case, understanding the
\h{Ab7} helps to understand the harmony in the previous bar, which is a
\h{Bb7}, again with \sharp{}5 in Ruiz's voicing and \flat{}9 in the melody
(and, in section \cnum{A-3}, with \sharp{}9 in the voicing). The first two bars of the
A section, then, contain a standard I--VI--ii--V progression, with both
dominant sevenths played as \alt chords.

The next two bars are perhaps the most difficult in the entire piece; Figure
\ref{lj:a-third-fourth} provides these two bars from all three A sections of the
head. While the rhythm section pitches are mostly consistent, the harmony is
not so clear. The starting and ending points are stable, and correspond with
ordinary Rhythm changes, with \h{Dbmaj7} in bar 3 of the A section moving to
\h{Db7} in bar 5. Beat three following the \h{Dbmaj7} seems to be \h{Bm7}
(arpeggiated in the melody), but at this point Ruiz seems to double the
harmonic rhythm, playing four chords in the last bar of this passage. Jones's bass
line is also very consistent here, but it is unclear whether this acceleration
of the harmonic rhythm is real or only a surface elaboration.\fn{lj-6}

\begin{figure}[tbp]
  \centerGraphic{eps/ch5/lj/a-third-fourth.pdf}
  \caption[The third and fourth bars of all three A sections of ``Lo-Joe.'']{%
    The third and fourth bars of all three A sections of ``Lo-Joe.'' The
    melody is shown on the top staff, and each grouping of three staves below
    contains piano and bass parts. Each group ends when Jones's bass reaches
    the tonic D\protect\flat.}
  \label{lj:a-third-fourth}
\end{figure}

\begin{figure}[tbp]
  \centerGraphic{eps/ch5/lj/third-fourth-analyses.pdf}
  \caption[Possible interpretations of the third and fourth bars of the A
  sections.]{%
    Possible interpretations of the preceding passage.}
  \label{lj:third-fourth-analyses}
\end{figure}

Figure \ref{lj:third-fourth-analyses} gives several interpretations of this
passage. All three of the interpretations begin with \h{Dbmaj7} and end with
\h{Db7}, with \h{Bm7} on beat 3 of the first bar. Letter \emph{a} conforms
most strongly with the melody, but the progression is unusual: there is a
chromatic slipping effect from \h{Bm7} to \h{Bbm7} that prefigures the
chromatic motion to the tonic via a tritone-substituted dominant,
\h{D7}--\h{Dbmaj7}. Letter \emph{b} is a hearing with doubled harmonic rhythm
that follows the bass line. The progression here makes more harmonic sense,
featuring mostly \tf progressions, though the melodic support for some of the
chords---the \h{A7} and both \h{E7}s---is weak at best. Letter \emph{c}
focuses on the piano line, returning to two chords per bar. In this hearing
the top notes of the voicings in the second bar are heard in both cases as
\sharp{}9 moving to \flat{}9: a logical hearing, but one that is not strongly
supported by the melody or bass line. None of these hearings seem to fit the
music perfectly, but each does fit some aspect of it. As we have seen before,
it is of little analytical use to decide on a single ``true'' analysis, though
it does have consequences for our imaginary lead sheet author (who must put a
set of changes with this melody). This may be a passage in which a lead sheet
is not sufficient; no single fixed interpretation can adequately capture the
essence of this harmonic motion. Only in their interaction (and in the
recording itself) does a full picture of the harmony emerge.

\begin{figure}[tbp]
  \centerGraphic{eps/ch5/lj/a-sect-endings.pdf}
  \caption{A section endings in ``Lo-Joe.''}
  \label{lj:a-sect-endings}
\end{figure}

The second half of the A section is much simpler, though the first A section
ends differently than the other two. (This is typical of Rhythm tunes, and
would probably be notated on a lead sheet as first and second endings, as in
``Rhythm-a-ning'' in Figure \ref{ran:head-melody}.) The fifth bar of the A
section contains a \h{Db7} chord, which moves to the subdominant \h{Gbmaj7} in
the following measure. The second half of m.~6 moves to some kind of \Cflat
chord, though Ruiz's piano voicings in the head are unhelpful in determining
its quality (in the solos it is usually played as a dominant seventh). The
first A section then moves to a tritone-substituted turnaround, while the
other two double the harmonic rhythm to arrive on tonic in the eighth bar (see
Figure \ref{lj:a-sect-endings}).

The bridge of ``Lo-Joe'' is its most distinctive feature, and is given with
chord changes in Figure \ref{lj:bridge}. This bridge is clearly inspired by
the last half of the bridge of ``Eternal Triangle'' (discussed in the next
section), and its sequential nature is helpful to our imagined lead-sheet
author: once a single bar is determined, it can simply be transposed to all of
the others. Here, each bar contains a single \tf progression, made explicit in
the bass and with basic, three- and four-note voicings in the piano.\fn{lj-7}

    \begin{figure}[p]
      \centerGraphic{eps/ch5/lj/bridge.pdf}
      \caption{``Lo-Joe,'' bridge (mm.~17--24).}
      \label{lj:bridge}
    \end{figure}

    \begin{figure}[p]
      \centerGraphic{eps/ch5/lj/bridge-trans.pdf}
      \caption[Several possible transformation networks of the bridge.]{%
        Several possible transformation networks of the bridge (first four bars).
        Unlabeled arrows indicate the TF transformation.}
      \label{lj:bridge-trans}
    \end{figure}

This bridge is phenomenologically rich: the \tf progressions themselves are
clear, but the connections between them admit of multiple possibilities (as a
listener, or for an improvising musician). Figure \ref{lj:bridge-trans} gives
several possible transformation networks for the first half of bridge of
``Lo-Joe". The analysis at \emph{a} is a ``horizontal'' one: do a \tf, move up
a tritone and do another, move down a half-step and do it again, and so on.
Letter \emph{b} emphasizes the descending fifth motion: play a \tf\ and its
tritone transposition, then move down a fifth and repeat. This hearing
respects the descending fifths present in the standard Rhythm bridge, and also
reflects the organization of \tf space, shown (in letter \emph{c}). Letter
\emph{d} highlights the tritone relationships between bars, and also
encourages hearing the \textsc{3rd} transformation connection between
\h{F7}--\h{Fm7} and \h{B7}--\h{Bm7} chords. Hearing the bridge this way allows
hearing as if the music is bouncing back and forth between two normal Rhythm
bridges, one in \Dflat, the other in G. Network \emph{e} emphasizes the
half-step relationships, and encourages a connection between the first and
last pairs of chords and the central two.

It may seem as though transformations themselves have not played an important
role in reconstructing a lead sheet for ``Lo-Joe.'' This reconstruction,
though, has taken place against the background of the musical spaces developed
in the earlier chapters of this study (all of which have transformations as
their logical basis). These musical spaces provide a mostly-unseen structuring
principle to the analytical work in this section. Because the spaces were
developed to demonstrate functional jazz harmony, to recognize that ``Lo-Joe''
uses functional harmony is to recognize that it likely reflects an orderly
representation in (say) \tf space. This, combined with the knowledge of Rhythm
changes in general, means that we could easily reject the triadic analysis in
Figure \ref{lj:opening-melody} as a nonsensical jazz progression. This notion
of syntax is one that is often implicit in the construction of the spaces, but
comes to the fore when used for the kinds of harmonic determination done
here.\fn{lj-8}

\section{Sonny Stitt, “The Eternal Triangle”}
\label{sec:eternal-triangle}

\subsection{Harmonic Peculiarities}

The final Rhythm tune of this chapter was the inspiration for the bridge of
``Lo-Joe'': Sonny Stitt's ``The Eternal Triangle.'' The canonical recording
appears on Dizzy Gillespie's album \emph{Sonny Side Up} (1957), featuring
Stitt along with Sonny Rollins, both on tenor
saxophone.\nocite{gillespie:sonnyside} The album is widely regarded as one of
the best ``jam session'' albums in jazz, and ``Eternal Triangle'' is often
singled out as the standout performance of the record.\fn{et-1} This two-tenor
format will allow us the opportunity to explore more deeply the role of
interaction between players in shaping harmony.

First, though, a brief analysis of the tune itself is in order. The head of
``Eternal Triangle'' is shown in Figure \ref{et:head-melody}; the A sections
are standard Rhythm changes, featuring fast-moving bebop melodic lines. The B section,
though, is unique to this tune, and features \tf progressions descending
by half-step. We might imagine this bridge as being derived from the standard
Rhythm bridge, as shown in Figure \ref{et:bridge-derivation}. In the first
step, the typical III--VI--II--V is compressed into the second half of the
bridge. To preserve the correct length, Stitt extends the fifths cycle
backward by two chords to \h{E7}, maintaining the original harmonic rhythm of
one chord every two bars, increasing the harmonic work done by the bridge (and
consequently, the area of \tf space it traverses). In the next step of the
derivation, each dominant seventh is replaced by a \tf progression;
finally, every other \tf progression is replaced with its tritone
substitute, resulting in the chromatic descent of the bridge itself.

\begin{figure}[tbp]
  \centerGraphic{eps/ch5/et/head-melody.pdf}
  \caption{Sonny Stitt, ``The Eternal Triangle,'' head.}
  \label{et:head-melody}
\end{figure}


\begin{figure}[tbp]
  \centerGraphic[width=\textwidth]{eps/ch5/et/bridge-derivation.pdf}
  \caption{Derivation of the bridge of ``Eternal Triangle'' from a standard
    Rhythm bridge.}
  \label{et:bridge-derivation}
\end{figure}

Because the A sections of ``Eternal Triangle'' use typical harmonies, both Rollins's
and Stitt's solos display many of the same solo approaches we saw in Griffin's solo
on ``Rhythm-a-ning'' above. Both players use harmonic
generalizations of various types: diatonic (choruses \cnum{2A-3}, \cnum{5A-3},
\cnum{8A-2}, and \cnum{11A-1}, for example); blues (\cnum{4A-1}, \cnum{9A-2},
\cnum{11A-3}); and other scales (the half-whole diminished scale in
\cnum{9A-1} and \cnum{12A-2}). Stitt in particular emphasizes the half-note
harmonic rhythm of the A sections, often playing bebop lines that change
accidentals frequently to highlight the harmonic shifts (see choruses
\cnum{6A-3} and \cnum{10A-2}, for example). Many other common harmonic devices
can also be found, including tritone substitutions (mm.~133) and \abbrev{CESH}
(mm.~33--35, 193, and 395--96), in which a chromatic melodic line embellishes
a single unchanging harmony (see p.~\pageref{fn:cesh}n\ref{fn:cesh}).

\begin{figure}[tbp]
  \vspace{1em}
  \centerGraphic{eps/ch5/et/sr-side-slipping.pdf}
  \caption[Side-slipping in Sonny Rollins's solo.]{%
    Side-slipping in Sonny Rollins's solo (mm.~45--46, 1:15).}
  \label{et:sr-side-slipping}
\end{figure}

One harmonic aspect of Rollins's solo does deserve special mention, as it has
not yet appeared in the analytical examples. Figure \ref{et:sr-side-slipping}
shows the fifth and sixth bars of an A section, where we would normally expect
the harmonies \h{Bb7}--\h{Eb7}. Here, Rollins clearly arpeggiates a \h{Bm7}
instead; this is a feature which is often called ``side-slipping'' or
``side-stepping.''\fn{et-2} The overall harmony of this bar is \h{Bb7}
(sometimes with its preceding \ii), but here Rollins plays a harmony a half-step
away. The motion from \h{Bb7} to \h{Bm7} is distant in \tf space, though is
closely related to the \slideS transformation \mbox{(\h{Bb7}
  $\xrightarrow{7\text{\textsc{th}}^{-1}\ \bullet\ \text{\slideS}\,}$
  \h{Bm7})}.\fn{et-3} It is also a convincing way of playing ``outside,''
which is what jazz musicians call improvised lines that do not seem to connect
with the underlying harmony.\fn{et-4} Outside playing becomes an important
feature of more modern jazz improvisations, but it also features prominently
in this recording when Rollins and Stitt begin trading (a section to which we
will return below).

\begin{figure}[tbp]
  \centerGraphic{eps/ch5/et/sr-non-seq.pdf}
  \caption[A non-sequential bridge from Sonny Rollins's solo.]{%
    A non-sequential bridge from Sonny Rollins's solo (mm.~113--20, 2:12).}
  \label{et:sr-non-seq}
\end{figure}

Given that the bridge of ``Eternal Triangle'' is its most interesting feature,
it will be worthwhile to examine the solo strategies of Rollins and Stitt
separately before moving on to discuss how the two interact with each other.
These strategies can be broken down into two basic types: sequential and
non-sequential solo approaches. The non-sequential solo approach to this bridge
is less common, and a single example should suffice. Figure
\ref{et:sr-non-seq} gives the bridge from Rollins's fourth chorus;
because the harmonies are so fast-moving, he uses mostly diatonic scales.
Though the melodic line is not sequential, Rollins's
shifting diatonic palette works in conjunction with the chromatically
descending \tf progressions; the chord-scale analysis highlights
that the $f_7$ signature transformation is equivalent to $T_{11}$.\fn{et-4a}

Because the bridge is made up of these chromatically descending \tf{}s, the most
obvious approach for an improviser is to play sequentially, repeating a single
pattern over each \tf progression. This approach is seen most often in the
second half of the bridge, where the harmonic rhythm doubles; Figure
\ref{et:sr-seq-bridge} gives two examples of this strategy from Sonny Rollins's
solo. In \emph{a}, from the third chorus, Rollins plays the same
quarter-note pattern in all four bars, highlighting the $T_{11}$s of the
progression itself.\fn{et-5} Letter \emph{b}, from the fifth chorus, is
more complicated, but the basic idea is the same. Here, Rollins begins on the
ninth of the minor seventh chord and descends to the third of the dominant
seventh (diatonically, in the key of the dominant). The rhythm here is more
varied, and linking material is inserted in the third bar, but the sequential
pattern is still clear.

\begin{figure}[tbp]
  \centerGraphic{eps/ch5/et/sr-seq-bridge.pdf}
  \caption[Two sequential bridges from Rollins's solo.]{%
    Two sequential bridges from Rollins's solo (mm.~85--88, 1:48 and
    mm.~149--52, 2:42).}
  \label{et:sr-seq-bridge}
\end{figure}

Stitt seems to prefer sequential bridges more than Rollins does, but Stitt
often introduces a metric shift as well; Figure \ref{et:ss-bridge-shift} gives
a representative example from chorus 7 (similar passages can be found in
choruses 9, 10, and 12). Here, Stitt plays a descending bebop scale over every
dominant seventh chord. The first two of these are three-beat patterns,
creating a metrical grouping dissonance (G$3/4$).\fn{et-6} When Stitt repeats
the pattern on the C bebop scale, it should end on the downbeat of the third
bar. As if realizing he has arrived too early---after all, the \h{C7} chord
does not \emph{begin} until the third bar---Stitt extends the scale another
two beats, resulting in a five-beat scale that descends an entire octave. With
five beats remaining in the bridge after the end of this pattern, he repeats
it using the B bebop scale; a final A\nat\ in the last beat acts as an
enclosure to the tonic \Bflat that begins the next A section. While the
harmonic rhythm of ``Eternal Triangle'' normally doubles in the last half of
the bridge, Stitt's frequent metric shifts accelerate it even further, giving
his solos even more momentum into the final A section of each chorus.

\begin{figure}[tbp]
  \centerGraphic{eps/ch5/et/ss-bridge-shift.pdf}
  \caption[A sequential bridge with metric shift from Sonny Stitt's solo.]{%
    A sequential bridge with metric shift, from Sonny Stitt's solo
    (mm.~213--16, 3:34).}
  \label{et:ss-bridge-shift}
\end{figure}

Though the strictly sequential patterns are mostly restricted to the last half
of the bridge, the first half can also support sequential patterns; Figure
\ref{et:ss-seq-first-half} shows an example from the eighth chorus. Because
the harmonic rhythm is slower at the beginning of the bridge, the sequences
are usually somewhat looser. Here, Stitt plays a decorated \h{Bm7} arpeggio,
followed by the same pattern a bar later over \h{Bbm7}. (While in the passage
in Figure \ref{et:ss-bridge-shift} Stitt ignored the \ii chords, here he seems
to ignore the \h{E7} instead.) The pattern breaks in the middle of the third
bar, where Stitt moves toward a standard diatonic pattern for the final
\h{Eb7} chord.

\begin{figure}[tbp]
  \centerGraphic{eps/ch5/et/ss-seq-first-half.pdf}
  \caption[A loosely sequential bridge from Stitt's solo.]{%
    A loosely sequential bridge from Stitt's solo (mm.~241--44, 3:57).}
  \label{et:ss-seq-first-half}
\end{figure}

\FloatBarrier
\subsection{Interactional Elements}

With a few exceptions (including ``Lo-Joe'' in the previous section), the
analyses so far have been interested primarily in a soloist's
improvised line, paying only passing attention to the fact that these
improvisations occur within a framework of group improvisation. Sonny
Rollins's solo on ``Eternal Triangle,'' for example, does not take place in
isolation; he is supported by the rhythm section, and in later choruses he and
Stitt ``trade,'' alternating 4- or 8-bar segments of improvised melodies.
Understanding this interaction is crucial to understanding the performance as
a whole. This section will focus on these moments of interaction in ``Eternal
Triangle,'' acknowledging the role that interaction plays in harmony, and vice
versa.

One of the first models of interaction in jazz is found in Robert Hodson's
\emph{Interaction, Improvisation, and Interplay in Jazz}. He uses a semiotic
model borrowed from Jean-Jacques Nattiez, in which a work of music is both a
product (a score/sound) and a set of processes: both the poietic process of
composition and the esthesic process of a
listener.\footcite[14--15]{hodson:2007} Hodson adapts this model for jazz
performance, since a jazz musician simultaneously creates the sound and
listens to the other band members' sounds. The two separate components of
poiesis and esthesis in Nattiez's model form something of a feedback loop for
jazz musicians, since their musical utterances are often shaped by those of
their fellow musicians (while at the same time, potentially influencing those
musicians themselves).\footcite[15--16]{hodson:2007}

Garrett Michaelsen, in his dissertation on interaction in jazz, critiques
Hodson on this model. He argues that while Hodson's focus on the ensemble as a
whole (via the esthesic process) is valuable, Hodson does not go far
enough. In his analyses, Hodson places himself in an improviser's perspective,
and as Michaelsen notes, ``it is \ldots\ unclear how this vantage point
enables musical analysis because it leaves no room for the outside observer's
non-poietic perspective.''\footcite[24--25]{michaelsen:2013} Instead,
Michaelsen offers a listener-based approach, in which a particular auditory
stream (a bass line, for example) might be heard as influencing another (like
an improvised saxophone line).\fn{et-7} As with harmony, there is often not a
single ``correct'' analysis of a given interaction, and so Michaelsen's
listener-based approach fits nicely with the prismatic approach to harmony
taken here.

Michaelsen discusses harmony only in passing (usually in connection with a
particular musical example), while Hodson dedicates an entire chapter to the
role of interaction in harmony.\fn{et-8} Many of the questions that concern
Hodson are the same as those we have confronted during the course of this study:
\begin{quoting}
  \singlespacing
  How can one reconcile the disparity between different versions, both written
  down and performed, of the ``same'' harmonic progression? Does it even need
  to be reconciled? Some scholars criticize the effort to reconcile these
  variants as an attempt to force a Western ideology of coherence---and a
  modernist ontology of the piece---onto a music to which it doesn't really
  apply. But, if this kind of [harmonic] coherence is not a part of jazz, then
  why do jazz musicians talk about a soloist ``making the changes,'' or an
  improvised line as either ``making sense'' or not? There must be \emph{some}
  criteria for musical coherence.\footcite[53]{hodson:2007}
\end{quoting}
%
Hodson answers these questions by borrowing a linguistic metaphor from Noam
Chomsky: he argues that jazz musicians play the ``deep structure'' of a tune,
which might be realized in any number of ways---and can be revealed by
analyzing musical interaction. Rhythm tunes, he argues, can be generated from
the deep structure of ``I Got Rhythm'': its A sections consist of
prolongations of \Bflat, while the bridge begins off tonic and contains a
motion back towards it. This appeal to linguistics does provide some way of
understanding the myriad of Rhythm harmonizations, but it leaves something to
be desired: it is not as though \emph{any} set of chord changes can appear in
the Rhythm A section, provided it starts and ends with \Bflat. Again, the
transformational approach to harmony developed here allows us a means to
specify the ways in which this deep structure is modified, and an
interactional analysis of ``Eternal Triangle'' seems a natural way of
exploring these modifications.

As a first step in that direction, consider Figure \ref{et:rhythm-sect-11},
from the end of chorus 11. In the final A section of the chorus, Stitt plays a
very strong blues generalization. Though this is a common choice (especially
in \cnum{A-3} sections), this particular occurrence is marked by the strong
emphasis on the blues in the rhythm section parts as well.\fn{et-9} After the
fast-moving harmonies of the bridge (the end of which is given in the
transcription), the group's convergence on eight bars of blues has a striking
effect.

\begin{figure}[tbp]
  \centerGraphic{eps/ch5/et/rhythm-sect-11.pdf}
  \caption[The end of chorus 11, including rhythm section.]{%
    The end of chorus 11, including rhythm section (mm.~341--53, 5:19; bass
    sounds as written).}
  \label{et:rhythm-sect-11}
\end{figure}

The instigating factor for this blues generalization might well have been
Tommy Bryant's decision to play a tonic pedal in m.~345. In a Rhythm tune like
``Eternal Triangle,'' the constant half-note harmonic rhythm can become
tedious, and a pedal point is one of the most effective ways a bass player can
counteract this tendency. This is the first time in the 5½ elapsed minutes of the
recording Bryant uses such a pedal, and Stitt and Bryant's brother Ray on piano
are very likely to have noticed.

Stitt responds to this tonic pedal by playing an emphatic blues lick with a
prominent \Aflat.\fn{et-10} This pitch, \flat\sd7, is unlikely to occur over
any of the common harmonies of the first bar of the Rhythm A section (except
perhaps as a \flat{}9 over \h{G7}), and its repetition as a long note in the
second bar cements its status as a member of the \Bflat blues scale.
Pianist Ray Bryant, hearing the bass pedal along with Stitt's blues lick, then
launches into a series of blues voicings.\fn{et-11} Combined with the tonic pedal,
Bryant's diminished chords and neighboring $^6_4$ chord of m.~348 give the
section a strong blues feel that both reinforces and is reinforced by Stitt's
solo line.

This group interaction is what Michaelsen would call strongly
``convergent,'' in that all three of the members involved play music that
supports the others.\fn{et-12} The convergence continues in the last four bars
of the section, as Tommy Bryant moves away from the tonic pedal to outline the
blues-inflected progression \h{Bb}--\h{Bb7/D}--\h{Eb}--\h{Eo7}--\h{F7}. In
m.~351, he arrives on a dominant pedal at the same moment that Stitt also
concludes his line on F, all reinforced by Ray Bryant's strong left-hand F in
the piano.\fn{et-13} This convergence on the dominant (combined with Charlie
Persip's drum fill) provides a strong push into the following chorus, where
the three fall back into their usual, less convergent, roles.

The most clearly interactional moments of ``Eternal Triangle'' come between
the two saxophonists themselves after the trading begins in chorus 14. Stitt
and Rollins ``trade fours'' for three choruses, then ``trade eights'' for
three more to end the saxophone soloing. The fours supply many good examples
of harmonic interaction; chorus \cnum{15A-2} (shown in Figure
\ref{et:trading-c15-divergent}) gives a representative example. Here, Rollins
plays a diatonic figure in the first two bars, then side-slips in the next two
to play a figure that outlines B major rather than \Bflat. Stitt's response is
at once both convergent and divergent: he enters on the same pitch that
Rollins did and plays a very similar figure in his first bar (marked with an
\textit{x} in Figure \ref{et:trading-c15-divergent}), but his starting pitch
is F\nat, dissonant with the B major triad Rollins is playing at the same
time. The last A section of the same chorus (shown in Figure
\ref{et:trading-c15-convergent}) illustrates a more convergent interaction.
Here, Rollins begins with a motive outlining the pitches G--\Gflat--F--D.
Stitt seamlessly picks up this line on the downbeat of m.~477, continuing the
motive for another four bars, at which point Rollins reclaims it for a bar
before moving on to new material in the beginning of the next chorus.

\vspace{5em}
\begin{figure}[htb]
  \centerGraphic{eps/ch5/et/trading-c15-divergent.pdf}
  \caption[A harmonic interaction between Rollins and Stitt from chorus 15.]{%
    A harmonic interaction between Rollins (red notes) and Stitt (blue notes)
    from chorus 15 (mm.~457--63, 6:52).}
  \label{et:trading-c15-divergent}
\end{figure}

\begin{figure}[htb]
  \centerGraphic{eps/ch5/et/trading-c15-convergent.pdf}
  \caption[A convergent interaction between Rollins and Stitt from later in
  chorus 15.]{%
    A convergent interaction from later in chorus 15 (mm.~473--81, 7:05).}
  \label{et:trading-c15-convergent}
\end{figure}

\newpage
\FloatBarrier
\subsection{Extended Analysis: Trading Eights, Harmony, and Interaction}

For the final analysis in this dissertation, I want to take a more detailed
look at the way harmony functions over a longer period of time, using the
final three choruses of saxophone soloing on ``Eternal Triangle.'' These are
the choruses where Stitt and Rollins trade eights; since each soloist gets an
entire section of a chorus, he has more harmonic leeway than in the rapid
four-bar segments of the preceding three choruses. Since these are the final
three choruses of their solos, this portion is also where the ``saxophone
dueling'' comes to a head. In it, the suggestions of outside playing initially
suggested in Rollins's very first chorus (recall Figure
\ref{et:sr-side-slipping}) reach a final realization, before winding down
again in the final chorus.

The eight-bar trading begins at the beginning of chorus 17, after a
non-musical interaction: someone (probably Stitt) can be heard on the
recording asking ``keep going?'' Stitt's first eight bars are a typical
diatonic A section, with only incidental chromaticism. Rollins's response in
chorus \cnum{17A-2} is more outside, and is shown in Figure
\ref{et:sr-outside-17a2}. The most blatant chromaticism here is in the second
bar, where we would normally expect \h{Cm7}--\h{F7} or \h{Cm7}--\h{Cso7}. As
is usually the case with outside playing, it is not clear how we should
interpret Rollins's note choices here, though the chord-scale triples below
the staff give a few suggestions. After this second bar he returns to more
inside playing, though with a few more outgoing scale choices than usual: he
plays both \sharp{}5 and \flat{}9 over the \h{F7} in m.~524.

\begin{figure}[tbp]
  \centerGraphic{eps/ch5/et/sr-outside-17a2.pdf}
  \caption[Rollins's outside playing in the second A section of chorus 17.]{%
    Rollins's outside playing in the second A section of chorus 17
    (mm.~521--24, 7:44).}
  \label{et:sr-outside-17a2}
\end{figure}

Stitt's ensuing bridge is again typical, using only the expected diatonic
collections. That he chooses not to play outside on the bridge illustrates a
basic principle of jazz harmony, in that progressions that are less common are
generally played more inside. The bridge of ``Eternal Triangle'' is its most
distinctive feature, so soloists typically play improvisations that highlight
these harmonies, whereas the A sections of the tune are more typical and admit
of greater elaboration. Playing far outside the changes on the bridge risks
obscuring the harmonic progression that is an essential feature of this
particular Rhythm tune (and perhaps, in an amateur performance, giving the
impression that outside playing will be mistaken for not \emph{knowing} the
changes).\fn{et-14}

\begin{figure}[tbp]
  \centerGraphic{eps/ch5/et/sr-outside-17a3.pdf}
  \caption[Rollins's increased chromaticism in the final A section of chorus
  17.]{%
    Rollins's increased chromaticism in the final A section of chorus 17
    (mm.~537--42, 7:56).}
  \label{et:sr-outside-17a3}
\end{figure}

In the final A section of the chorus, Rollins increases the chromaticism even
further, as shown in Figure \ref{et:sr-outside-17a3}. His first bar seems to
imply a motion from \h{E7} to A; while we might understand the E harmony as a
tritone substitution for the tonic \Bflat, the A chord in the second half of
the first bar does not make sense as an ordinary substitution in any of the
usual harmonizations of the A section. Given that the \h{A7} harmony seems to
continue into the second bar, we might instead hear this outside playing as a
downward side-slipping, substituting an A diatonic collection for the tonic
\Bflat. Side-slipping is one of Rollins's preferred methods of playing
outside: we observed it in his first chorus, and he repeats the
technique in chorus \cnum{14A-3} (mm.~441--42). Rollins returns to more
typical harmonies after these first two bars, though he still emphasizes
dissonant tones: m.~540, for example, features strongly accented dissonances,
with the local \sd4 (Levine's ``avoid note'') appearing on beats 1 and 3. All
of this chromaticism combines to form an improvised line even more outside
than Rollins's first eight bars of trading, as though his own sense of harmony
is being slowly detached from that of both his own rhythm section and from the
A section of the tune itself.

In the beginning of the next chorus, Stitt seems to take the outside-playing
bait, launching into a dominant-cycle A section very much like Monk's solo on
``Rhythm-a-ning.'' This harmonic move seems to take the rhythm section by
surprise, and draws our attention to the interaction not only between the two
soloists, but between them and the rhythm section as well (a transcription
including the piano and bass parts is given in Figure
\ref{et:rhythm-sect-c18}). Faced with this unexpected dominant cycle, pianist
Ray Bryant's solution is simply to stop playing, while bassist Tommy Bryant
instead plays a \Bflat pedal, as if to stress the tonic in the midst of
Stitt's cycle. Since the 8-chord dominant cycle lasts only four bars, all
three band members all return to a typical A-section harmonic structure in m.~549.

\begin{figure}[p]
  \centerGraphic[page=1]{eps/ch5/et/rhythm-sect-c18.pdf}
  \caption[Transcription of chorus 18, including rhythm section.]{%
    Transcription of chorus 18, including rhythm section (mm.~545--76, 8:03).}
  \label{et:rhythm-sect-c18}
\end{figure}

\begin{figure}[p]
  \renewcommand{\theContinuedFloat}{ (continued)}
  \ContinuedFloat
  \centerGraphic[page=2]{eps/ch5/et/rhythm-sect-c18.pdf}
  \caption[]{Complete transcription of chorus 18.}
\end{figure}

\begin{figure}[tp]
  \renewcommand{\theContinuedFloat}{ (continued)}
  \ContinuedFloat
  \centerGraphic[page=3,clip,trim=0pt 3.1in 0pt 0pt]{eps/ch5/et/rhythm-sect-c18.pdf}
  \caption[]{Complete transcription of chorus 18.}
\end{figure}

Perhaps anticipating another outside response from Rollins, Ray Bryant does
not immediately begin playing in the next A section, and Tommy Bryant opts
this time for a dominant pedal. In the face of harmonic uncertainty, this
approach from the rhythm section makes sense: a dominant pedal in the bass
will work to build tension no matter what Rollins decides to play, and Ray
Bryant's wait-and-see approach prevents any harmonic clashes. As it turns out,
Rollins does play a non-diatonic line, and again it is not entirely clear what
harmonic framework he has in mind. His line in the first four bars of chorus
\cnum{18A-2} is a loose, descending, motivic repetition that recalls the
``C.T.A.'' harmonization of Figure \ref{rcg:rc-first-four}f (with its
whole-step descent). While the motive and direction of the line are relatively
clear, we might also hear this line, in conjunction with Ray Bryant's comping
in the last two bars, as a series of outgoing scale choices on a more standard
progression, as shown in Figure \ref{et:sr-desc-motive}. Like Stitt, Rollins
returns to a more diatonic approach in the last four bars of the section,
setting up the way for Stitt's bridge.

\begin{figure}[tbp]
  \centerGraphic{eps/ch5/et/sr-desc-motive.pdf}
  \caption[One possible interpretation of Rollins's descending line.]{%
    One possible interpretation of Rollins's descending line (mm.~553--56,
    8:09).}
  \label{et:sr-desc-motive}
\end{figure}

The bridge of chorus 18 proceeds normally in all three instruments: Stitt's
scale choices are almost completely diatonic, Ray Bryant uses standard 3-note
voicings, and Tommy Bryant plays a bass line emphasizing the $T_{11}$ root
motion. While the bridge of ``Eternal Triangle'' usually functions as the
locus of harmonic activity in a chorus, the bridge here has an almost calming
effect after the harmonic disruption in the first half of the chorus. The
tension of all of the outside playing seems to disappear as all of the band
members (and we as listeners) relax into the bridge, relatively confident that
it will progress as expected.

This harmonic tranquility does not last long, though, as Ray Bryant instigates
another dominant cycle at the beginning of chorus \cnum{18A-3}. He enters
emphatically on an \h{Fs7} chord, and includes the bass note (typically
omitted by pianists) in his left hand, as if to demonstrate that he understood
Stitt's cycle in \cnum{18A-1} and is willing to support it for this A section.
His brother Tommy catches the cycle almost immediately: after playing a \Bflat
on the downbeat of m.~569, he makes his way to a B\nat\ by beat 3, and
continues with the cycle all the way through the first four bars. This time,
though, Rollins does not follow along, playing the \Bflat Lydian scale
for the first two bars. Ray Bryant's enthusiasm for the dominant cycle fades
quickly, hearing that Rollins does not follow along; by the third bar the
piano voicings are nearly inaudible, returning only in the final bar (where
\h{C7}--\h{F7} is a characteristic choice regardless of the particular
harmonization used).

\begin{figure}[tbp]
  \centerGraphic{eps/ch5/et/ss-side-slipping.pdf}
  \caption[Stitt's side-slipping at the beginning of chorus 19.]{%
    Stitt's side-slipping at the beginning of chorus 19 (mm.~577--79, 8:29).}
  \label{et:ss-side-slipping}
\end{figure}

After a more common diatonic ending in the last four bars of chorus 18, Stitt
begins the next chorus with a side-slipping gesture, arpeggiating a \h{Csm7}
chord before the expected \h{Cm7} in the second bar of the chorus (see Figure
\ref{et:ss-side-slipping}). After this brief moment, though, Stitt plays more
or less diatonically until the end of this A section.

At the beginning of chorus \cnum{19A-2}, the side-slipping occurring
sporadically throughout the saxophone solos reaches its culmination in a
remarkable moment of ensemble convergence. Stitt ends his eight bars with a
tonic triadic descent, ending on \Bflat. At the same moment, both Ray and
Tommy Bryant land on a \h{Bmaj7} chord, turning Stitt's tonic \Bflat into a
chordal major seventh (see the transcription in Figure
\ref{et:ens-side-slipping}).\fn{et-15} This side-slipped B major lasts only
two bars, and by m.~587 the rhythm section moves back to \Bflat. For his part,
Rollins plays a repeating motive over all eight bars of this section
consisting of the pitches \Eflat, D, and \Bflat. Over the \h{Bmaj7}, this lick
emphasizes the major third and seventh of the chord, but when the rhythm
section slips back to \Bflat, Rollins is left emphasizing the dissonant \sd4.
This approach is certainly not as outside as Stitt's dominant cycle of chorus
\cnum{18A-1}, but the dissonance still lends a sense of ``outsidedness'' to the
section as a whole. In the final bar of his A section, Rollins resolves the
\Eflat to D, after which the bridge and final A section proceed almost
completely diatonically.

\begin{figure}[tbp]
  \centerGraphic{eps/ch5/et/ens-side-slipping.pdf}
  \caption[Ensemble arrival on \protect\h{Bmaj7} at chorus
  \protect\cnum{19A-2}.]{%
    Ensemble arrival on \protect\caph{Bmaj7} at chorus
    {\protect\ssliningfont19A\tsub{2}} (mm.~583--88, 8:33).}
  \label{et:ens-side-slipping}
\end{figure}

Heard as a unit, the final three choruses form a progression from inside to
outside playing and back again. Stitt's diatonic playing in chorus 17 is
challenged by Rollins's suggestions of outside playing, weakly at first in
\cnum{17A-2} but more strongly in \cnum{17A-3}. Stitt takes the suggestion in
the opening of chorus 18, playing a dominant cycle that takes the rhythm
section by surprise. Rollins responds not with a dominant cycle, but with a
motivic response that seems almost completely disconnected from typical Rhythm
harmony. The rhythm section tries to anticipate a dominant cycle in the
chorus's final A section, only to be disregarded by Rollins. Amid the harmonic
confusion, Stitt begins chorus 19 with some mild side-slipping before
returning to more inside playing. This side-slipping is answered strongly by
the rhythm section, which begins the following A section a half-step off on
\h{Bmaj7}. After a mildly dissonant eight bars from Rollins, the bridge and
final A section return inside to conclude the saxophone trading and make space
for Dizzy Gillespie's trumpet solo that follows. The harmony in these choruses
is far more nuanced than the chord changes of the lead sheet might suggest; it
is a dynamic, ever-changing attribute, revealed not only in the interaction
among the musicians themselves, but also between the musicians and us as
listeners.

\section{Concluding Remarks}

As the end of this study approaches, it may be useful to return to the
fundamental question: what \emph{is} jazz harmony, really? Or, as Hodson
states the problem, ``what exactly are you analyzing when you analyze jazz
harmony?''\footcite[52]{hodson:2007} Does the harmony exist in the chord
symbols on a lead sheet? In the voicings of a pianist or walking line of a
bassist? Does harmony exist in a tune itself, independent of any particular
performance of it? If a performer's harmony conflicts with the chord symbol,
which one is ``correct''?

These are fundamental questions, and questions that do not often arise in the
study of notated music. The answer, I think, is that harmony in jazz is all of
these things, and more. A single harmony can be captured by a chord symbol,
but this chord symbol is only part of the story. A transcription of a piano
voicing for that symbol does not represent a final solution either, since in
jazz a repeating harmonic framework forms the basis of a performance, and
pianists do not usually play identical voicings for the same chord throughout.
We may be tempted to turn towards the soloist as the arbiter of harmonic
identity, but as we have seen, even a seemingly clear progression in a solo
instrument can be obscured by the choices of a rhythm section.

It is my hope that the transformational approach developed throughout this
dissertation is better equipped to deal with this chimerical nature of jazz
harmony. Chapter 1 began by treating harmony in something of a ``clean-room''
fashion, grouping chord symbols into diatonic sets without worrying too much
about what the chord symbols actually represented. Chapters 2 and 3 took steps
towards clarifying this nature, representing chord symbols in their most basic
form of root, third, and seventh. This abstraction allowed us to explore
connections between harmonies that do not share a diatonic collection, using
variations on a basic musical space of \tfo progressions. In Chapter 4, our
conception of harmony expanded from three-note chords into many-note scales,
as we drew on the work of George Russell to create a chord-scale space. This
work permitted a change of focus from chord symbols and abstract tunes to jazz
performance itself, investigating the way in which harmony can function for
improvising performers. This final chapter has continued this work by looking
through the lens of Rhythm changes, itself a harmonic archetype, variously
instantiated in countless Rhythm contrafacts, each of which is elaborated in
individual performances.

In this last chapter, the transformations themselves receded into the
background somewhat, used only as a tool for discussing harmony in service
of other analytical points. This is intentional, and as Julian Hook notes in
his review of Lewin's \emph{GMIT}, ``transformation theory is a large and
varied toolbox; there are only some minimal instructions for using the tools,
and no designs at all for what one can build with
it.''\footcite[166]{hook:2007gmit} For all the focus on harmony, it is often
the case that harmony is not the most interesting aspect of a particular
passage. Just as we tell our students that a Roman numeral analysis does not
mark the end of the analytical process, neither does a completed chord-scale
labeling mean that we can check off a passage as ``analyzed'' and move on to
the next chorus. This dissertation has developed a flexible set of tools for
analyzing harmony---\allowbreak tools that can be used as necessary whenever a
need to discuss harmony arises.

Given that harmony touches nearly every aspect of jazz performance and
analysis, the foundational work here might be applied or extended in any
number of ways. It is easy to imagine other transformations that might appear
more commonly in other (especially later) jazz repertoires, where either the
ordered-triple or chord-scale approaches might be fruitfully applied. Dmitri
Tymoczko has argued that jazz is a ``modernist synthesis,'' and that jazz
musicians ``act as custodians of a tradition of advanced tonal
thinking.''\footcite[389]{tymoczko:2011} Though I tend to disagree with this
historical view of jazz, it may well be the case that the techniques developed
here might lend insight into the musics Tycmozko identifies on either
historical side of common-practice jazz: the tonality of the impressionists on
one side and that of the minimalists on the other.

Even within the common-practice jazz era of this dissertation, there is room
for expansion. The solo analyses done here only begin to scratch the surface,
and a detailed investigation of chord-scale choices among different performers
might well lead to some meaningful distinction between, for example, a Johnny
Griffin solo and a Joe Henderson solo. Certainly every soloist has their own
style, and the way in which they interpret harmony is often an essential
component of that style. It is also easy to imagine that saxophonists in
general have a different kind of harmonic language than do trumpeters,
trombonists, pianists, or banjo players; chord-scale transformations could
help to tease out these distinctions.

The analyses here have focused exclusively on small-group jazz, since there
are fewer moving parts to manage, but harmony is of course present in (nearly)
all jazz performance. One of my own interests lies in the implications of
harmony for composers and arrangers of big-band music. When an arranger like
Thad Jones or Jim McNeely sits down to arrange a jazz standard, they typically
do so with a knowledge of harmony earned from experience as a player. The fact
that large ensemble music is typically composed and written down means that
the author maintains a tighter control over the harmony than in an improvised
setting; this allows the opportunity for a detailed shaping of harmony over
the course of a tune. These standards are often reharmonized in interesting
ways, and the transformational approach is ideally suited to discussing these
arrangements in dialogue with their original sources.

One of the aims of this dissertation has been to take seriously the manner in
which jazz musicians themselves discuss harmony. Since these musicians do not
often speak technically about harmony, I have used pedagogical materials as a
way of getting at this ``insider's view.'' This focus has often been implicit:
there has been no extended literature review of pedagogical texts, nor was
there a need to use these texts to the exclusion of other, more academic,
treatments of jazz harmony. When a choice arose, however, I usually opted to
cite Jerry Coker, Mark Levine, or Jamey Aebersold rather than Henry Martin,
Dmitri Tymoczko, or Steve Larson. This choice was made not to disparage the
important work of other theorists working on jazz harmony, but rather as a
means to acknowledge the real work on harmony in the jazz community (in
addition to the theory community).

At the outset of this study, I suggested that a transformational approach to
jazz harmony might constitute a set of analytical values different from the
Eurological values of Schenkerian analysis (as often applied to jazz). The
prismatic style of transformational analysis, borrowed from Steven Rings, has
helped to enable this alternate value system. Nearly all of the analyses here
have avoided taking a single synthetic view of a passage, opting instead for a
perspective in which multiple, sometimes conflicting, analyses can be
considered individually in turn. This multifaceted approach is intended to
reflect the nature of jazz harmony itself, and its application can enable us
to narrow the gap between bring jazz theory and jazz practice.


%%% Local Variables:
%%% mode: latex
%%% TeX-master: "../diss"
%%% End:
