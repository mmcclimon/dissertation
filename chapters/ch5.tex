% Chapter 5

\chapter{Rhythm Changes}
\label{chap:rhythm-changes}
\addtocspace
\addtolof{chap:rhythm-changes}
\addtolof[lot]{chap:rhythm-changes}

The discussion of chord-scale transformations in the previous chapter
concludes the theoretical portion of this dissertation; this final chapter
will synthesize this theoretical framework in a series of three longer
analyses. All three of the tunes here---Thelonious Monk's ''Rhythm-a-ning,''
Sonny Stitt's ``The Eternal Triangle,'' and George Coleman's ``Lo-Joe''---are
instances of a harmonic archetype known as ``rhythm changes,'' named from
their origin in George Gershwin's ``I Got Rhythm.''\fn{rcg-1} Because tunes
that use Rhythm changes all share a common harmonic origin, they are an ideal
means to investigate jazz harmony. A complex set of standard substitutions and
harmonic patterns have emerged over the many years jazz musicians have been
playing rhythm changes; the three analyses in this chapter will allow us to
compare these musicians' manipulation of the basic harmonic framework.

\section{Rhythm Changes in General}
\label{sec:rhythm-changes-general}

It is hard to overestimate the influence of Rhythm changes on jazz practice;
along with the blues, it is one of the most common harmonic types in the bebop
era and beyond.\footcite[13]{owens:1995} David Baker lists more than 150
Rhythm tunes in \emph{How to Learn Tunes}; some of the most well-known
of these are reproduced in Table
\ref{rcg:rhythm-tunes}.\footcite[42--44]{baker:tunes} Before moving on to the
three analyses in the following sections, it will be useful to examine the
form itself, along with some of its more common harmonic substitutions.

\begin{table}[tbp]
  \setlength{\tabcolsep}{12pt}
  \centering
  \begin{tabular}{ll}
    Title  & Composer \\
    \hline
    \rule[1em]{0ex}{1ex}%
    Anthropology        & Charlie Parker/Dizzy Gillespie \\
    Cotton Tail         & Duke Ellington \\
    52nd Street Theme   & Thelonious Monk \\
    The Flintstones     & Hoyt Curtain \\
    Jumpin' at the Woodside & Count Basie \\
    Moose the Mooche    & Charlie Parker \\
    Oleo                & Sonny Rollins \\
    The Serpent's Tooth & Miles Davis \\
    Tiptoe              & Thad Jones \\
    Wail                & Bud Powell
  \end{tabular}
  \caption{Several Rhythm tunes and their composers.}
  \label{rcg:rhythm-tunes}
\end{table}

\subsection{Substitution Sets}
\label{subsec:substitution-sets}

``I Got Rhythm'' is, like many jazz standards, a 32-bar AABA form; the basic
progression is shown in Figure \ref{rcg:rhythm-basic}.\fn{rcg-2} As Andy Jaffe
notes, its changes are ``not the least bit astonishing''; the tune is a fairly
basic set of turnarounds and dominant cycles.\footcite[149]{jaffe:1996}
Indeed, this feature is one of the reasons for its popularity: the harmonic
framework is something of a blank slate, and allows room for alteration in a
way that more specific sets of changes (Parker's ``Blues for Alice,'' for
example) do not. Another thing that is immediately apparent is the quick
harmonic rhythm in the A sections, which allows soloists the opportunity to
show off as they navigate the rapidly moving changes.\fn{rcg-3}

\begin{figure}[tbp]
  \centerGraphic[width=24em]{eps/ch5/rhythm-basic.pdf}
  \caption[The basic changes to George Gershwin's ``I Got Rhythm.'']{%
    The basic changes to George Gershwin's ``I Got Rhythm'' (taken from
    Levine, \emph{The Jazz Theory Book}, 238).}\nocite{levine:1995}
  \label{rcg:rhythm-basic}
\end{figure}

The ``mix-and-match'' aspect of harmonic progression is crucially important to
the genre of Rhythm tunes.\fn{rcg-4} Mark Levine explains the issue
nicely:%
%
\begin{quoting}
  \singlespacing
  When a musician calls a Rhythm tune like ``Oleo,'' there's no discussion of
  which version of the changes to play. As with the blues, jazz musicians
  freely mix many versions of Rhythm changes on the spot, as they improvise.
  Playing Rhythm changes is a little like knowing several tunes and playing
  them all at once; that's why ``Rhythm'' tunes are harder to play at first
  than a tune with only a single set of changes.\footcite[241]{levine:1995}
\end{quoting}
%
Given this background, we will proceed with the analytical discussion in
segments: the A sections can each be broken into four-bar halves, while the
bridge is typically treated as a single eight-bar unit.

The first four bars of the Rhythm A section serve to prolong the tonic \Bflat;
Figure \ref{rcg:rc-first-four} gives a number of possible harmonizations of
this section.\fn{rcg-5} Letter \emph{a} gives the original Rhythm changes,
while letter \emph{b} shows what is by far the most common substitution,
substituting \h{Dm7} for \h{Bb} in the third bar; this replaces the
I--vi--ii--V turnaround in the last two bars with a iii--vi--ii-V instead. If
the \h{Bb} harmony is voiced with a major seventh and major ninth
(\Bflat--D--F--A--C, a very common voicing), then we can understand the
substitution of \h{Dm7} as a simple omission of the root. Letter \emph{c}
transforms many of the minor seventh chords into dominant sevenths that lead
more strongly into the following harmonies.

\begin{figure}[tbp]
  \centerGraphic{eps/ch5/rc-first-four.pdf}
  \caption{Several harmonizations of Rhythm changes, mm.~1--4.}
  \label{rcg:rc-first-four}
\end{figure}

\begin{figure}[tbp]
  \centerGraphic{eps/ch5/rc-first-four-space.pdf}
  \caption{The first four bars of Rhythm changes in \tf space.}
  \label{rcg:rc-first-four-space}
\end{figure}

Figure \ref{rcg:rc-first-four-space} shows the relevant portion of \tf space
for these first three harmonizations, along with a few annotations. The
standard harmonization of figure \ref{rcg:rc-first-four}a can be seen by
following the blue arrows. The substitution of \h{Dm7} in letter \emph{b} is
represented in the space by the red arrows: in this reading, first follow the
blue arrows until arriving at \h{F7}, then follow the red arrows until \h{Gm7}
where the blue arrows continue to the tonic \h{Bb}. The minor-to-dominant
substitutions of letter \emph{c} are not shown in the space, but are easy
enough to imagine: both \h{Gm7} and \h{Dm7} are transformed by the
\textsc{3rd}$^{-1}$ operation, and each is replaced by the chord immediately
to its north in \tf space (a substitution which results in the transformation
\mbox{\h{G7} \ECarrow\ \h{Cm7}} across the bar lines at the end of mm.~1 and
3).

The harmonization in Figure \ref{rcg:rc-first-four}d is still more complex.
The tritone substitution of \h{Db7} for \h{G7} in m.~3 is by now familiar, but
the \h{G7} in m.~2 has been replaced with a passing diminished seventh chord.
As we first saw in the analysis of ``Have You Met Miss Jones?'' in Section
\ref{subsec:miss-jones}, fully-diminished sevenths in jazz can be understood
as \h{V7b9} chords missing their roots. The \h{Bo7} here, then, is a logical
substitution for \h{G7b9}, and the \h{Cso7} in the following bar can be
understood as the same substitution of an implied \h{A7b9} chord (the dominant
of the following D minor), resulting in a chromatically ascending bass line in
the first two bars. Miles Davis's composition ``The Serpent's Tooth'' (the
opening of which is shown in Figure \ref{rcg:serpents-tooth}) uses a variation
of this progression. Davis also includes a minor-third substitution in m.~4,
substituting \h{Ebm7}--\h{Ab7} for the diatonic \h{Cm7}--\h{F7}.

\begin{figure}[tbp]
  \centerGraphic{eps/ch5/serpents-tooth.pdf}
  \caption{``The Serpent's Tooth'' (Miles Davis), mm.~1--4.}
  \label{rcg:serpents-tooth}
\end{figure}

The last two harmonizations in Figure \ref{rcg:rc-first-four} are somewhat
different in nature; while any of the substitutions of letters
\emph{a}--\emph{d} can be swapped in and out at will, those in letters
\emph{e}--\emph{f} usually appear as units. Letter \emph{e} harmonizes the
first four bars with a simple cycle of dominant seventh chords (a favorite
technique of Thelonious Monk, and one we will see in the analysis of
``Rhythm-a-ning'' below). In contrast to the relatively compact arrangement of
letters \emph{a}--\emph{c} in \tf space, this cycle traverses nearly the
entire space before arriving at the tonic \h{Bb}.\fn{rcg-6} Letter \emph{f} is
the harmonization from Jimmy Heath's composition ``C.T.A.,'' and features a
lament-bass pattern from \h{Bb} down to \h{F7}, repeated twice.

\begin{figure}[tbp]
  \centerGraphic{eps/ch5/rc-next-four.pdf}
  \caption{Several harmonizations of Rhythm changes, mm.~5--8.}
  \label{rcg:rc-next-four}
\end{figure}

The last four bars of the Rhythm A section contain a shift to the subdominant
in the first two bars, followed by a turnaround in the last two; Figure
\ref{rcg:rc-next-four} gives several common harmonizations of this passage.
Once again, letter \emph{a} reproduces the original changes; a seventh is
added to the tonic \Bflat, tipping it towards an \Eflat that resolves plagally
(via minor iv) back to tonic before a standard vi--ii--V turnaround. This
plagal motion in the second bar is often substituted with a backdoor
progression, as seen in \emph{b} (which also precedes the \h{Bb7} in the first
bar with its own \ii) and \emph{d} (which elides the \h{Eb} and \h{Ebm}
harmonies). Letter \emph{c} makes the substitution of \h{Dm7} for \h{Bb} in
the third bar, as seen in Figure \ref{rcg:rc-first-four}, and includes
\h{Eo7} as a substitution for \h{Eb7b9}.\fn{rcg-7}

\begin{figure}[tbp]
  \centerGraphic{eps/ch5/rc-bridge.pdf}
  \caption{Several harmonizations of the Rhythm bridge, mm.~17--24.}
  \label{rcg:rc-bridge}
\end{figure}

\begin{figure}[tbp]
  \centerGraphic{eps/ch5/rc-bridge-space.pdf}
  \caption{The four Rhythm bridge harmonizations of Figure \ref{rcg:rc-bridge}
    in \tf space.}
  \label{rcg:rc-bridge-space}
\end{figure}

The Rhythm bridge is usually recognizable because of the drastic slowing of
the harmonic rhythm; again, Figure \ref{rcg:rc-bridge} gives several common
harmonizations, and Figure \ref{rcg:rc-bridge-space} shows them in \tf space.
The standard bridge (letter \emph{a}) is a simple cycle of dominants,
beginning on the III chord; we will call this the ``4-cycle bridge.'' The most
common substitutions here are tritone substitutions of every other chord, as
shown in \emph{b}--\emph{c}. The other common option is to insert \ii chord
before each of the dominants, as shown in \emph{d}, decomposing each $T_5$
transformation into TF $\bullet$ \textsc{3rd}. (This procedure could of course
be repeated with the tritone-substituted versions in \emph{b}--\emph{c} as
well.) Other less conventional harmonizations are also possible; ``The Eternal
Triangle'' and ``Lo-Joe'' both use specialized bridges to which we will return
in later sections.

It should be apparent from this discussion that Rhythm tunes can vary widely
in their harmonic particulars. The mix-and-match nature of their construction
means that the chords used by an ensemble can change even over the course of a
single performance: a rhythm section might prefer one harmonization of the
bridge during a saxophone solo and opt for another during a piano solo, for
example. The harmonizations given in Figures \ref{rcg:rc-first-four},
\ref{rcg:rc-next-four}, and \ref{rcg:rc-bridge} have only begun to scratch the
surface; because most of the tune consists of turnarounds, any of the
countless possible turnarounds could be used instead.\fn{rcg-8} It is easy to
imagine a Rhythm tune that makes use of the descending minor-third turnaround
of Henderson's ``Isotope'' or fast-moving Coltrane changes over the bridge.

\subsection{Harmonic Substitution vs.\ Chord-Scale Elaboration}

Before moving on to the three analyses proper, it will be helpful to return to
an issue first mentioned in the last chapter in connection with Kirk's solo on
``Blues for Alice.'' In many cases, it is not clear whether a particular
improvised passage should be heard as a harmonic substitution or as an
outgoing chord-scale choice over the original harmony. In the case of
non-Rhythm tunes, we can usually rely on the head to provide the authoritative
changes for the tune, and likely choose to hear that particular set of changes
throughout the performance. Rhythm changes, though, bring the problem to the
fore, since we cannot depend on a single set of canonical changes.

By way of a short illustration, consider again the melodic passage that opens
Miles Davis's ``Serpent's Tooth'' (shown in Figure \ref{rcg:serpents-tooth}).
If this were an improvised passage, it seems likely that the first choice of
harmonies would \emph{not} be those used by Davis, given the clear
outlines of both \h{G7} and \h{A7} chords in the second halves of
mm.~1--2. It is also possible to hear this passage as a series of outgoing
scale choices over the standard diatonic progression, hearing the
C\sharp--E--G fragment as part of a diminished scale over \h{F7}, for example.
Three possible hearings of these first two bars are shown in Figure
\ref{rcg:serpents-ambiguous}, which gives locations in chord-scale space for
each harmony. They are shown here in ingoing-to-outgoing order: \emph{a} uses
only diatonic scales, \emph{b} uses the same collections but hears the Lydian
diminished scales over the diminished seventh chords, while \emph{c}
emphasizes more widely shifting diatonic collections and scale choices.

\begin{figure}[tbp]
  \centerGraphic{eps/ch5/serpents-ambiguous.pdf}
  \caption{Three possible hearings of the opening of ``The Serpent's Tooth.''}
  \label{rcg:serpents-ambiguous}
\end{figure}

While this prismatic approach to analysis may have seemed excessive for the
relatively insignificant passages in the last chapter where it was used, it
will take a central role in our study of Rhythm changes. Because the harmonic
structure of the tunes is so fluid, it is impossible to claim with any
certainty that a particular set of changes constitutes some Platonic
\textsc{tune}, in the same way that we might be able to for ``Autumn Leaves''
or ``All the Things You Are.'' To fix a set of definitive changes for a
particular passage is to misrepresent the fundamental nature of Rhythm tunes
in jazz practice; the changes are often ill-defined even among the players
themselves (as the above quotation from Levine attests). Engaging
with a single Rhythm tune, then, constitutes an engagement with an entire
genre of tunes, with all its attendant history.\fn{rcg-9} Transformational
theory, with its ability to refract a passage into many possible
interpretations, offers us a way in to this rich network of harmonic
possibilities inherent to the genre.

\section{Thelonious Monk, “Rhythm-A-Ning”}
\label{sec:rhythm-a-ning}

\subsection{Head}

Thelonious Monk's ``Rhythm-A-Ning'' is a basic Rhythm tune, and as such will
be an illustrative first example. The head of the tune is shown in Figure
\ref{ran:head-melody} as it appears in the \emph{Thelonious Monk Fake
  Book}.\footcite[]{sickler:fakebook} The source recording for this lead sheet
is Monk's album \emph{Criss-Cross} (1963); we will analyze a different
performance below, but the differences in the head are
insignificant.\nocite{monk:crisscross} What is noteworthy about this lead
sheet is that there are no changes given in the A sections; there is only an
indication that the solos are to be played over Rhythm changes.\fn{ran-1} This
speaks not only to the ubiquity of the form, but also to its fluidity, since a
single definitive version is not given.

\begin{figure}[tb]
  \centerGraphic{eps/ch5/ran/head-melody.pdf}
  \caption{Thelonious Monk, ``Rhythm-A-Ning,'' head.}
  \label{ran:head-melody}
\end{figure}

Nevertheless, there are a few aspects of the head that might have an impact on
a soloist's harmonic choices. The first is the arpeggiation of an \Eflat major
triad in the second bar of the A sections. None of the common sets of changes
in Figure \ref{rcg:rc-first-four} use \Eflat in the second bar, but this
plagal motion is essential to the tune. (The harmonization \h{Cm7}--\h{Cso7}
fits the melody, but does not appear in Monk's recordings, where the bassist
consistently arrives on \Eflat on the downbeat of the second bar.) The other
important feature of the tune is the whole-tone ascent at the end of the
bridge. Monk is well-known for his propensity towards the whole-tone scale,
and we will see this manifest below in his solo on the tune.

Instead of the \emph{Criss-Cross} recording, we will instead focus our
analytical attention on a live recording from 1958, on the album \emph{Thelonious in
  Action}.\fn{ran-a}\nocite{monk:action} This recording is attractive for a number of
reasons. First, tenor saxophonist Johnny Griffin takes eleven full choruses on
the tune, allowing the opportunity to analyze a somewhat longer section of
music than we did in the previous chapter. Second, after Griffin's second
chorus, Monk does not play at all, leaving only the bass and drums to
accompany the tenor saxophone.\fn{ran-2} This, combined with the ambiguity of
the solo changes, provides something of a blank harmonic slate, leaving
Griffin's improvised lines to do the bulk of the harmonic work.

\subsection{Johnny Griffin’s Solo}
\FloatBarrier

When dealing with the fast-moving harmonies in the Rhythm A sections, Johnny
Griffin's preferred strategy seems to be to ignore them: he frequently uses
harmonic generalizations in the A sections. Often these generalizations are
diatonic, using the 2\flat\ collection; Figure \ref{ran:jg-diatonic-gen} gives
a representative example from chorus \cnum{8A-2}. While we could perhaps imply
a diatonic set of chord changes like those in Figure
\ref{rcg:rc-first-four}a--b, the rising arpeggios harmonizing the top line
F$_5$--F$_6$ seem to take precedence over any particular harmonization.
Similar rising diatonic patterns can be found in mm.~1--4 (chorus \cnum{1A-1})
and mm.~169--72 (\cnum{6A-2}).

\begin{figure}[tbp]
  \centerGraphic{eps/ch5/ran/jg-diatonic-gen.pdf}
  \caption[Diatonic harmonic generalization in Johnny Griffin's solo on
    ``Rhythm-A-Ning.'']{Diatonic harmonic generalization in Johnny Griffin's
    solo on ``Rhythm-A-Ning'' (mm.~233--36, 4:01).}
  \label{ran:jg-diatonic-gen}
\end{figure}

Other times, Griffin plays passages that are nearly diatonic, but altered
somewhat to fit an underlying harmony. Figure \ref{ran:jg-diatonic-alt} gives
an example from chorus \cnum{9A-1} (chorus \cnum{11A-1} is similar). In this
passage, the line is mostly diatonic, with the exception of the G\nat\ and
\Aflat in m.~259, implying a \h{G7b9} harmony. This passage, unlike the
diatonic ascent in Figure \ref{ran:jg-diatonic-gen}, fits better with a
diatonic chord progression, as shown in the upper transformation network.
While it is certainly possible to hear a 2\flat\ diatonic swath throughout
these four bars (the lower network), hearing the half-note harmonic rhythm
brings out the contrast between \h{Gm7} in the first bar and the altered
\h{G7} in the third.

\begin{figure}[tbp]
  \centerGraphic{eps/ch5/ran/jg-diatonic-alt.pdf}
  \caption[Altered diatonic harmonic generalization Griffin's solo.]{%
    Altered diatonic harmonic generalization in Griffin's solo (mm.~257--60,
    4:22), with two possible transformation networks.}
  \label{ran:jg-diatonic-alt}
\end{figure}

The most common harmonic generalization Griffin uses is the \Bflat blues
scale, which often appears in the last A section of a chorus. The clearest
example of this is also the first, at the end of his third chorus; this
passage is reproduced in Figure \ref{ran:jg-blues-gen}, and similar clear
statements of the blues scale are found in \cnum{4A-3}, \cnum{8A-3},
\cnum{10A-3}, and \cnum{11A-3}. In this formal location, the use of a harmonic
generalization serves as a contrast to the cycle of dominants in the bridge
(which Griffin does not generalize).

\begin{figure}[tbp]
  \centerGraphic{eps/ch5/ran/jg-blues-gen.pdf}
  \caption[Griffin's blues generalization in a final A section.]{%
    Griffin's blues generalization in a final A section (mm.~89--96, 1:55).}
  \label{ran:jg-blues-gen}
\end{figure}

The blues scale also provides an explanation for Griffin's seemingly unusual
implication of \h{Dbm} at the end of the first chorus (shown in Figure
\ref{ran:jg-blues-gen}). It is not immediately apparent how to understand the
passage in mm.~25--28 (see Figure \ref{ran:dflat-possibilities}): Griffin
could be superimposing \h{Dbm} over a \Bflat diatonic progression, implying
\h{Bbm7b5}, or using a \Bflat half--whole diminished scale generalization.
Given his inclination for the blues scale in the last A section of the tune,
though, my own hearing leans towards this \h{Dbm} triad as a subset of the
\Bflat blues scale. Griffin also emphasizes the pitches \Dflat and \Aflat at
the end of his second and tenth choruses; the former implies \Dflat
major, while the latter is more clearly based in \Bflat.

\begin{figure}[tbp]
  \centerGraphic{eps/ch5/ran/jg-blues-subset.pdf}
  \caption[A \protect\h{Dbm} triad as a \protect\h{Bb} blues subset in
  Griffin's solo.]{%
    A \Dflat{}m triad as a \Bflat blues subset in Griffin's solo (mm.~25--32, 0:58).}
  \label{ran:jg-blues-subset}
\end{figure}

\begin{figure}[tbp]
  \centerGraphic{eps/ch5/ran/dflat-possibilities.pdf}
  \caption[Four possible harmonic contexts for a \Dflat{}m triad.]{%
    Four possible harmonic contexts for a \Dflat{}m triad: a pure triad; the
    upper tones of a \protect\caph{Bbm7b5}; a diminished scale subset; and a
    blues scale subset.}
  \label{ran:dflat-possibilities}
\end{figure}

Griffin does not always play harmonic generalizations over the A sections; he
sometimes uses the half-note harmonic rhythm of the tune itself. The
clearest example of this occurs in chorus \cnum{7A-2}, which is reproduced in Figure
\ref{ran:jg-change-running}. While it seems clear that Griffin hears the
half-note harmonic changes here, the melodic patterns he plays are harmonically
ambiguous, owing to their limited range. While the progression shown in
\emph{a} is most likely---it combines the passing diminished seventh in the
second bar with the applied dominant of C in the third---we might hear the
harmonization at \emph{b} instead, with \h{A7} in the place of \h{Cso7} and a
tritone substitution for \h{G7b9}. Still other, less conventional, hearings
are possible; letter \emph{c} shows a hearing that moves to \Eflat, by first
moving to \h{Eb7} in the second bar (a nod towards the head's tilt towards the
subdominant in the same formal location), and then via an applied dominant to
a modified \tfo in \Eflat.

\begin{figure}[tbp]
  \centerGraphic{eps/ch5/ran/jg-change-running.pdf}
  \caption[Half-note harmonic rhythm in an A section of Griffin's solo.]{%
    Half-note harmonic rhythm in an A section of Griffin's solo (mm.~201--4,
    3:33).}
  \label{ran:jg-change-running}
\end{figure}

\begin{figure}[tbp]
  \centerGraphic{eps/ch5/ran/change-running-bass.pdf}
  \caption[Griffin's improvised solo line, along with Ahmed Abdul-Malik's bass
  line.]{Griffin's improvised solo line, along with Ahmed Abdul-Malik's bass line
    (mm.~201--4, 3:33). Bass sounds as written.}
  \label{ran:change-running-bass}
\end{figure}

Admittedly, this last hearing may be difficult to discern, not least because
it is so far from the typical first four bars of Rhythm changes. Another
important reason, and one we have yet to consider, is the role of the other
band members in shaping harmony. Indeed, the principal role of the rhythm
section (piano and bass) is to provide the harmonic framework for the
soloists. Since Monk does not play during Griffin's solo, we might look to
bassist Ahmed Abdul-Malik's line during these four bars, which is shown in
Figure \ref{ran:change-running-bass}.\fn{ran-3} Abdul-Malik does not seem to use the
half-note harmonic rhythm here, and instead plays a generalization in the
first two bars, walking up the \Bflat major scale (or Lydian, depending on
whether the \Eflat or E\nat\ is heard as the chromatic pitch). The strong
tonic--dominant motion from \Aflat to \Eflat in the third bar seems to imply
some \Aflat harmony, perhaps as a backdoor substitution to the downbeat \Bflat
in the fourth bar. He does gesture towards the home-key \tf in the last bar:
we might hear the \Bflat--G on beats 2--3 as a weak arpeggiation of \h{Cm7},
and the final C as a representative of \h{F7} (which resolves to \Bflat in the
next bar). While this bass line certainly provides insight into Abdul-Malik's
conception of the harmony of these four bars, it does not necessarily tell us
anything more about \emph{Griffin}'s harmonic understanding; it is entirely
possible (and common, as here) that all band members do not share exactly the
same harmonic framework, especially in a Rhythm tune.\fn{ran-3a}

Again, our job as listeners and analysts is not to decide on a set of
definitive changes. This ambiguity is a critical part of understanding exactly
what jazz harmony \emph{is}, and carries with it important epistemological
questions---questions, incidentally, which relate to my own suspicion of the
Schenkerian analysis of jazz first sketched in Section
\ref{sec:theoretical-approaches}. If we are to take Abdul-Malik's bass line as
\emph{the} harmony, are we to understand some of Griffin's note choices as
incorrect? Or vice versa, if Griffin's solo line represents the true version
of the harmony, why does Abdul-Malik choose to ignore it? In
``Rhythm-A-Ning,'' do the changes for the head (determined by whom?) hold
through all of the solos, or is the harmonic framework considered anew in
every chorus? In order to make a Schenkerian voice-leading sketch of this
passage, we would be forced contend with these issues, since determining what
pitches are consonant (or more structural) is dependent on being able to
identify the underlying harmony.\fn{ran-4} My own contention is that the
realities of jazz performance necessitate a more fluid theoretical conception
of harmony; the prismatic transformational approach developed here allows us
to make these kinds of distinctions by presenting multiple transformation
networks (representing multiple harmonic hearings) of a single passage.

Compared to his strategies for the A sections, Griffin's bridges are
much less varied. In most choruses, he uses the standard 4-cycle bridge;
Figure \ref{ran:jg-standard-bridge} gives an example from his final chorus. In
this passage, Griffin repeats the rising arpeggio, altering it in each two-bar
phrase to fit with the descending-fifths harmonic pattern.\fn{ran-5} He often
provides additional harmonic interest by changing the scale to lead more
strongly to the following harmony (not unlike the technique Joe Henderson used
in the first four bars of his solo on ``Isotope,'' examined in Section
\ref{subsec:isotope-solo}). Figure \ref{ran:jg-bridge-altered} gives a passage
from Griffin's second chorus; here, both the \h{D7} and \h{G7} gain a \flat{}9
in their last two beats, while the \h{F7} gets a \sharp{}5 (or \flat{}13) in
its final bar, which acts as a common-tone connection with the \Bflat blues
scale that follows in chorus \cnum{2A-3}. The other common alteration Griffin
makes is the tritone substitution, as shown in Figure
\ref{ran:jg-bridge-tritones}.


\begin{figure}[tbp]
  \centerGraphic{eps/ch5/ran/jg-standard-bridge.pdf}
  \caption[A 4-cycle bridge from Griffin's last chorus.]{%
    A 4-cycle bridge from Griffin's last chorus (mm.~337--44, 5:30).}
  \label{ran:jg-standard-bridge}
\end{figure}

\begin{figure}[tbp]
  \centerGraphic{eps/ch5/ran/jg-bridge-altered.pdf}
  \caption[A 4-cycle bridge, with chord-scale elaborations that lead
  more strongly toward the following harmony.]{%
    A 4-cycle bridge, with chord-scale elaborations that lead more
    strongly toward the following harmony (mm.~49--56, 1:19).}
  \label{ran:jg-bridge-altered}
\end{figure}

\begin{figure}[tbp]
  \centerGraphic{eps/ch5/ran/jg-bridge-tritones.pdf}
  \caption[The bridge from Griffin's eighth chorus, with tritone
  substitutions.]{%
    The bridge from Griffin's eighth chorus, with tritone substitutions shown
    in green (mm.~241--48, 4:08).}
  \label{ran:jg-bridge-tritones}
\end{figure}

\FloatBarrier
\subsection{Monk’s Solo}

While Griffin's tenor saxophone solo displays a number of interesting harmonic
formations, Thelonious Monk's own solo on ``Rhythm-A-Ning'' exhibits a few
more, and is worth a brief visit here. Monk only plays three choruses on the
tune, and his first is characteristically sparse. Throughout all three A
sections of this first chorus (chorus 12 in the transcription), he plays only
pitches from the \Bflat pentatonic collection---the same collection he uses to
comp behind Griffin's first three choruses.\fn{ran-6} The recurring rhythmic
motive is altered slightly so that it fits the harmonies of the standard
4-cycle bridge in mm.~369--76, before returning to the \Bflat pentatonic
collection for the final eight bars of the chorus.

Beginning in chorus 13, Monk consistently plays an 8-chord dominant cycle in
the A sections (a harmonization first seen in Figure
\ref{rcg:rc-first-four}e). Because this harmonization is so distinct from the
ordinary Rhythm A section, Monk simply arpeggiates each chord to avoid
blurring the overall progression. Playing a winding bebop line through the
dominant cycle might risk the coherence of the substitution, especially if the
bass player did not pick up on this harmonization and played a \Bflat diatonic
bass line.\fn{ran-7} Figure \ref{ran:monk-cycle-a} shows reproduces chorus
\cnum{13A-1}, showing the dominant cycle in the first four bars, followed by a
\Bflat blues harmonic generalization in the next four.

\begin{figure}[tbp]
  \centerGraphic{eps/ch5/ran/monk-cycle-a.pdf}
  \caption[Monk's dominant-cycle A section and blues generalization.]{%
    Monk's dominant-cycle A section and blues generalization from chorus
    13A$_1$ (mm.~385--92, 6:12).}
  \label{ran:monk-cycle-a}
\end{figure}

These dominant-cycle A sections are always paired with bridges that use the
whole-tone scales. The head of ``Rhythm-A-Ning'' uses the whole-tone scale in
the bridge (clearly over the \h{F7}, and implied over \h{C7} as well), and its
use in Monk's solo helps to provide coherence to the performance as a whole.
The bridge from chorus 14 is shown in Figure \ref{ran:monk-wt-bridge}; though
the whole-tone collection shifts between adjacent dominant seventh chords, the
passage is harmonically consistent. This uniformity is easy to see in the
chord-scale analysis: every change of harmony is represented by the
transformation \rtrans{$T_5$}{$f_1$}{$0$}.

\begin{figure}[tbp]
  \centerGraphic{eps/ch5/ran/monk-wt-bridge.pdf}
  \caption[Monk's bridge from chorus 14, using the whole-tone scale.]{%
    Monk's bridge from chorus 14, using the whole-tone scale (mm.~433--40, 6:53).}
  \label{ran:monk-wt-bridge}
\end{figure}

% xxx does this need some kind of wrap-up?

\section{Sonny Stitt, “The Eternal Triangle”}
\label{sec:eternal-triangle}

\subsection{Harmonic Peculiarities}

``The Eternal Triangle'' is a Rhythm tune composed by Sonny Stitt, and the
canonical recording appears on the album \emph{Sonny Side Up} (1957),
featuring Stitt on tenor along with Dizzy Gillespie and another tenor
saxophonist, Sonny Rollins.\nocite{gillespie:sonnyside} The album is widely
regarded as one of the best ``jam session'' albums in jazz, and ``Eternal
Triangle'' is often singled out as the standout performance of the
record.\fn{et-1} This two-tenor format will allow us the opportunity to
explore more deeply the role of interaction between players in shaping
harmony.

First, though, a brief analysis of the tune itself is in order. The head of
``Eternal Triangle'' is shown in Figure \ref{et:head-melody}; the A sections
are standard, featuring fast-moving bebop melodic lines. The B section,
though, is particular to this tune, and features \tf progressions descending
by half-step. We might imagine this bridge as being derived from the standard
Rhythm bridge, as shown in Figure \ref{et:bridge-derivation}. In the first
step, the typical III--VI--II--V is compressed into the second half of the
bridge. To preserve the correct length, Stitt extends the fifths cycle
backward by two chords to \h{E7}, maintaining the original harmonic rhythm of
one chord every two bars, increasing the harmonic work done by the bridge (and
consequently, the area of \tf space it traverses). In the next step of the
derivation, each dominant seventh is replaced by a \tf progression;
finally, every other \tf progression is replaced with its tritone
substitute, resulting in the ``Eternal Triangle'' bridge itself.

\begin{figure}[tbp]
  \centerGraphic{eps/xxx.pdf}
  \caption{Sonny Stitt, ``The Eternal Triangle,'' head.}
  \label{et:head-melody}
\end{figure}


\begin{figure}[tbp]
  \centerGraphic[width=\textwidth]{eps/ch5/et/bridge-derivation.pdf}
  \caption{Derivation of the bridge of ``Eternal Triangle'' from a standard
    Rhythm bridge.}
  \label{et:bridge-derivation}
\end{figure}

Because the A sections of ``Eternal Triangle'' are standard, both Rollins's
and Stitt's solos display many of the same features we saw in Griffin's solo
on ``Rhythm-A-Ning'' in the previous section. Both players use harmonic
generalizations of various types: diatonic (choruses \cnum{2A-3}, \cnum{5A-3},
\cnum{8A-2}, and \cnum{11A-1}, for example); blues (\cnum{4A-1}, \cnum{9A-2},
\cnum{11A-3}); and other scales (the half-whole diminished scale in
\cnum{9A-1} and \cnum{12A-2}). Stitt in particular emphasizes the half-note
harmonic rhythm of the A sections, often playing bebop lines that change
accidentals frequently to highlight the harmonic shifts (see choruses
\cnum{6A-3} and \cnum{10A-2}, for example). Many other common harmonic devices
can also be found, including tritone substitutions (mm.~133) and \abbrev{CESH}
(mm.~33--35, 193, and 395--96).

I do want to draw attention to one harmonic aspect of Rollins's solo that we
have not yet seen. Figure \ref{et:sr-side-slipping} shows the fifth and sixth
bars of an A section, where we would normally expect the harmonies
\h{Bb7}--\h{Eb}. Here, Rollins clearly arpeggiates a \h{Bm7} instead; this is
a feature which is often called ``side-slipping'' or
``side-stepping.''\fn{et-2} The overall harmony of this bar is \Bflat (either
with or without its minor seventh), but Rollins plays a harmony a half-step
away. The motion from \Bflat to \h{Bm7} is distant in all of our harmonic
spaces, though Dmitri Tymoczko notes that side-slipping usually demonstrates
efficient voice-leading.\footcite[xxx]{tymoczko:2011} It is also a convincing
way of playing ``outside,'' which is what jazz musicians call improvised lines
that do not seem to connect with the underlying harmony.\fn{et-3} Outside
playing becomes an important feature of more modern jazz improvisations, but
it also features prominently when Rollins and Stitt begin trading, which we
will examine in the next section.

\begin{figure}[tbp]
  \centerGraphic{eps/xxx.pdf}
  \caption[Side-slipping in Sonny Rollin's solo.]{%
    Side-slipping in Sonny Rollin's solo (mm.~45--46, 1:15).}
  \label{et:sr-side-slipping}
\end{figure}

%%% Local Variables:
%%% mode: latex
%%% TeX-master: "../diss"
%%% End:
