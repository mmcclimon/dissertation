% Front matter

\title{A Transformational Approach to Jazz Harmony}
\author{Michael McClimon}
\date{}
\makeatletter \let\Title\@title \makeatother

\pdfbookmark[0]{Front Matter}{fm:frontmatter}

% Title page {{{
\begin{center}
  \singlespacing
  \thispagestyle{empty}

  \vspace*{\fill}
  \keepthetitle
  {
    \addfontfeatures{LetterSpace=3.5}
    \MakeUppercase{\Title}
  }

  \vspace{\baselineskip}
  Michael McClimon

  \vspace{4\baselineskip}

  Submitted to the faculty of the University Graduate School \\
  in partial fulfillment of the requirements for the degree \\
  Doctor of Philosophy \\
  in the Jacobs School of Music, \\
  Indiana University \\
  month year

  \vspace{\fill}
\end{center}
\clearpage
% }}}

% Acceptance page {{{
\singlespacing
\begin{center}
  Accepted by the Graduate Faculty, Indiana University, in partial fulfillment
  of the requirements for the degree of Doctor of Philosophy.
\end{center}

{\flushleft
  \vspace{\baselineskip}
  Doctoral Committee
}

{\flushright

\vspace{5em}
\rule{20em}{0.5pt} \\ Julian Hook, Ph.D.

\vspace{5em}
\rule{20em}{0.5pt} \\ Blair Johnston, Ph.D.

\vspace{5em}
\rule{20em}{0.5pt} \\ Marianne Kielian-Gilbert, Ph.D.

\vspace{5em}
\rule{20em}{0.5pt} \\ Brent Wallarab

}


{\flushleft
defense date
}

\clearpage
% }}}

% Copyright page {{{
\begin{center}
  \vspace*{\fill}
  Copyright \copyright\ year \\
  Michael McClimon
  \vspace*{\fill}
\end{center}
\clearpage
% }}}

% Acknowledgements {{{

\phantomsection
\section*{\Large Acknowledgements}
\addcontentsline{toc}{section}{Acknowledgements}

\clearpage

% }}}

% Abstract {{{
\phantomsection
\addcontentsline{toc}{section}{Abstract}
\begin{center}
  Michael McClimon

  \vspace{\baselineskip}
  {
    \addfontfeatures{LetterSpace=3.5}
    \MakeUppercase{\Title}
  }
\end{center}

\doublespacing
(abstract goes here)

{\flushright

\vspace{4em}
\rule{20em}{0.5pt} \\ Julian Hook, Ph.D.

\vspace{4em}
\rule{20em}{0.5pt} \\ Blair Johnston, Ph.D.

\vspace{4em}
\rule{20em}{0.5pt} \\ Marianne Kielian-Gilbert, Ph.D.

\vspace{4em}
\rule{20em}{0.5pt} \\ Brent Wallarab

}
% }}}

\clearpage
\phantomsection
\addcontentsline{toc}{section}{Table of Contents}
\singlespacing
\tableofcontents

\clearpage
\phantomsection
\addcontentsline{toc}{section}{List of Figures}
\listoffigures


\clearpage

\phantomsection
\addcontentsline{toc}{section}{List of Tables}
\listoftables

\clearpage

% Notes to the Reader {{{
\phantomsection
\section*{\Large Notes to the Reader}
\addcontentsline{toc}{section}{Notes to the Reader}
\label{sec:notes-to-reader}

\singlespacing
References to recordings in this dissertation are generally made only by
giving the performer’s name and album title; complete identifying information
can be found in the discography.

When discussing a particular piece of jazz, the words “piece,” “composition,”
and “work” all seem out of place. In general, I have adopted the word “tune”
to mean roughly “the basic structure of a work, including (primarily) its
melody and chord changes.” This is in keeping with the way jazz musicians use
the word: they may refer to a ``16-bar tune,'' a “rhythm changes tune,” or “one
of my favorite tunes” (all referring to the abstract structure of the tune and
not simply the melody). When I am referencing a \emph{particular}
instantiation of a work (e.g.~Bill Evans’s recording of “Autumn Leaves” from
\emph{Portrait in Jazz}), I will make that clear.

In running text, dominant seventh chords are indicated with just a “\h{7},”
major sevenths with “\h{maj7},” minor sevenths with “\h{m7},” and
half-diminished sevenths with “\h{m7b5}”. In a minor key the tonic chord is
often played with a major seventh, which is indicated “\h{mM7}.” The
progression \h{Dm7}--\h{G7}--\h{Cmaj7} thus indicates a D minor seventh moving
to a G dominant seventh moving to a C major seventh chord. In examples, chord
symbols typically follow conventions used by \emph{The Real Book}.

There are two ways mathematicians notate function composition: left-to-right
and right-to-left. When combining two functions, $f\,$ followed $g$,
right-to-left orthography writes $g(f\,(x))$, while left-to-right orthography
writes $(fg)(x)$. Right-to-left orthography is familiar to most readers, and
we will use it when talking about transformations as functions: statements of
the form $f\,(x)$ should be read from right to left. (David Lewin uses
left-to-right notation throughout \emph{Generalized Musical Intervals and
Transformations}.) When discussing strings of transformations, however,
right-to-left orthography can be confusing: $L(P\,(x))$ indicates the $P$
operation applied to some object $x$, followed by the $L$ operation (it is
easy to see how the problem can proliferate when there are multiple operations
involved). To get around this problem, we will use left-to-right orthography
denoted by the symbol $\bullet$; the operation $P\,$ followed by the operation
$L$ is notated $P \bullet L$.\footnote{This use follows that of Julian Hook; see
\headlesscite[Section 2.3]{hook:mst}.} This combination of notation seems
intuitive, but we will be explicit in situations where the orthography may be
confusing.

\doublespacing

% }}}

% vim: fdm=marker
