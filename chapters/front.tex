% Front matter

\title{A Transformational Approach to Jazz Harmony}
\author{Michael McClimon}
\date{}
\makeatletter \let\Title\@title \makeatother

\pdfbookmark[0]{Front Matter}{fm:frontmatter}

% Title page {{{
\begin{center}
  \singlespacing
  \thispagestyle{empty}

  \vspace*{\fill}
  \keepthetitle
  {
    \addfontfeatures{LetterSpace=3.5}
    \MakeUppercase{\Title}
  }

  \vspace{\baselineskip}
  Michael McClimon

  \vspace{4\baselineskip}

  Submitted to the faculty of the University Graduate School \\
  in partial fulfillment of the requirements for the degree \\
  Doctor of Philosophy \\
  in the Jacobs School of Music, \\
  Indiana University \\
  December 2015

  \vspace{\fill}
\end{center}
\cleardoublepage
% }}}

% Acceptance page {{{
\singlespacing
\begin{center}
  Accepted by the Graduate Faculty, Indiana University, in partial fulfillment
  of the requirements for the degree of Doctor of Philosophy.
\end{center}

{\flushleft
  \vspace{\baselineskip}
  Doctoral Committee
}

{\flushright

\vspace{5em}
\rule{20em}{0.5pt} \\ Julian Hook, Ph.D.

\vspace{5em}
\rule{20em}{0.5pt} \\ Kyle Adams, Ph.D.

\vspace{5em}
\rule{20em}{0.5pt} \\ Blair Johnston, Ph.D.

\vspace{5em}
\rule{20em}{0.5pt} \\ Brent Wallarab

}

% xxx fix date before defense
{\flushleft
[ Month day, 2015 ]
}

\cleardoublepage
% }}}

% Copyright page {{{
\begin{center}
  \vspace*{\fill}
  Copyright \copyright\ 2015 \\
  Michael McClimon
  \vspace*{\fill}
\end{center}
\cleardoublepage
% }}}

% Acknowledgements {{{

\phantomsection
\section*{\Large Acknowledgements}
\addcontentsline{toc}{section}{Acknowledgements}

\doublespacing

This project would not have been possible without the help of many others,
each of whom deserves my thanks here. Pride of place goes to my advisor, Jay
Hook, whose feedback has been invaluable throughout the writing process, and
whose writing stands as a model of clarity that I can only hope to emulate.

Thanks are owed to the other members of my committee as well, who have each
played important roles throughout my education at Indiana: Kyle Adams,
Blair Johston, and Brent Wallarab. I would also like to extend my
appreciation to Frank Samarotto and Phil Ford, both of whom have deeply shaped
the way I think about music, but have no official role in the dissertation
itself. Thank you as well to Giuliano Di Bacco, who provided not only a job
for me during the early stages of this project, but many hearty laughs throughout.

I am grateful to the music faculty of Furman University, who inspired my love
of music theory as an undergraduate and have more recently served as friends
and colleagues during the writing process. Special thanks are owed to my
theory colleagues, Mark Kilstofte and Dan Koppelman, and to Matt Olson,
without whom I would not have discovered my passion for jazz.

I have the privilege of having been at Indiana University at the same time as
many talented individuals, including (and by no means limited to) Gabe Lubell,
Nathan Blustein, Jeff Vollmer, Diego Cubero, Mark Chilla, and Garrett
Michaelsen. The ideas that coalesced into this dissertation were formed in
part over many years of friendly conversation with these people; these
conversations are among my fondest memories of my time in Bloomington.

I have been lucky to have the unflagging support of my parents while I earned
three degrees in music theory, and they are to thank for untold amounts of
money and time spent encouraging my love of music from a very young age.

Finally, words cannot express my love and gratitude to my wife Carolyn for her
undying love and support, in the writing process and in life. Without her, I
would never have made it to this point (nor would I have the privilege of
knowing our two cats, who lent their own brand of moral support to this
project).


\singlespacing

\cleardoublepage

% }}}

% Abstract {{{
\phantomsection
\addcontentsline{toc}{section}{Abstract}
\begin{center}
  Michael McClimon \\
  \vspace{0.5\baselineskip}
  {
    \addfontfeatures{LetterSpace=3.5}
    \MakeUppercase{\Title}
  }
\end{center}

\doublespacing

Harmony is one of the most fundamental elements of jazz, and one that is often
taken for granted in the scholarly literature. Because jazz is an improvised
music, its harmony is more fluid and potentially more complex than that of
other, notated traditions. Harmony in common-practice jazz (c.\ 1940--1965) is
typically represented by chord symbols, which can be actualized by performers
in any number of ways, and which might change over the course of a single
performance.

This dissertation presents a transformational model of jazz harmony that helps
to explain this inherent complexity. While other theories of jazz harmony
require transcriptions into notation, the transformational approach enables
analysis of chord symbols themselves. This approach, in which chord symbols
are treated as first-class objects, is consistent with the way jazz harmony is
usually taught, and with the way jazz musicians usually discuss harmony.
Though transformational theory has been applied to later jazz, the aim of this
study is rather different: the music under consideration here might be called
``tonal jazz,'' in which functional harmonic progressions are still the rule.

After a general introduction, the first chapter introduces the
transformational approach by developing a diatonic seventh-chord space.
Chapter~2 expands this diatonic space to a fully chromatic space that focuses
on the \tfo progression, laying the foundation for much of the work that
follows. Chapter~3 extends the model to examine music in which root motion by
thirds plays an important role, paying special attention to the way in which
harmonic substitution interacts with more normative jazz harmony. Since the
pioneering work of George Russell in the 1950s, many jazz musicians have drawn
an equivalence between chords and scales; Chapter~4 develops a
transformational approach to these chord-scales, enabling analyses of
improvisations on tunes first analyzed in the preceding chapters. The final
chapter centers on a single harmonic archetype, Rhythm changes, and brings
together the theoretical framework in a series of analyses featuring solos by
Johnny Griffin, Thelonious Monk, George Coleman, Sonny Rollins, and Sonny
Stitt.

{\flushright \singlespacing

\vspace*{3em}
\rule{20em}{0.5pt} \\ Julian Hook, Ph.D.

\vspace{3em}
\rule{20em}{0.5pt} \\ Kyle Adams, Ph.D.

\vspace{3em}
\rule{20em}{0.5pt} \\ Blair Johnston, Ph.D.

\vspace{3em}
\rule{20em}{0.5pt} \\ Brent Wallarab

}
% }}}

\cleardoublepage
\phantomsection
\addcontentsline{toc}{section}{Table of Contents}
\singlespacing
\tableofcontents

\clearpage
\phantomsection
\addcontentsline{toc}{section}{List of Figures}
\listoffigures


\clearpage

\phantomsection
\addcontentsline{toc}{section}{List of Tables}
\listoftables

\cleardoublepage

% Notes to the Reader {{{
\phantomsection
\section*{\Large Notes to the Reader}
\addcontentsline{toc}{section}{Notes to the Reader}
\label{sec:notes-to-reader}

\singlespacing
References to recordings in this dissertation are generally made only by
giving the performer’s name and album title; complete identifying information
can be found in the discography.

When discussing a particular piece of jazz, the words “piece,” “composition,”
and “work” all seem out of place. In general, I have adopted the word “tune”
to mean roughly “the basic structure of a work, including (primarily) its
melody and chord changes.” This is in keeping with the way jazz musicians use
the word: they may refer to a ``16-bar tune,'' a “Rhythm tune,” or “one
of my favorite tunes” (all referring to the abstract structure of the tune and
not simply the melody). When I am referencing a \emph{particular}
instantiation of a work (e.g.~Bill Evans’s recording of “Autumn Leaves” from
\emph{Portrait in Jazz}), I will make that clear.

In running text, dominant seventh chords are indicated with just a “\h{7},”
major sevenths with “\h{maj7},” minor sevenths with “\h{m7},” and
half-diminished sevenths with “\h{m7b5}”. In a minor key the tonic chord is
often played with a major seventh, which is indicated “\h{mM7}.” The
progression \h{Dm7}--\h{G7}--\h{Cmaj7} thus indicates a D minor seventh moving
to a G dominant seventh moving to a C major seventh chord. In examples, chord
symbols typically follow conventions used by \emph{The Real Book}.

There are two ways mathematicians notate function composition: left-to-right
and right-to-left. When combining two functions, $f\,$ followed $g$,
right-to-left orthography writes $g(f\,(x))$, while left-to-right orthography
writes $(fg)(x)$. Right-to-left orthography is familiar to most readers, and
we will use it when talking about transformations as functions: statements of
the form $f\,(x)$ should be read from right to left. (David Lewin uses
left-to-right notation throughout \emph{Generalized Musical Intervals and
Transformations}.) When discussing strings of transformations, however,
right-to-left orthography can be confusing: $L(P\,(x))$ indicates the $P$
operation applied to some object $x$, followed by the $L$ operation (it is
easy to see how the problem can proliferate when there are multiple operations
involved). To avoid this problem, we will use left-to-right orthography
denoted by the symbol $\bullet$; the operation $P\,$ followed by the operation
$L$ is notated $P \bullet L$.\footnote{This use follows that of Julian Hook; see
\headlesscite[Section 2.3]{hook:mst}.} This combination of notation seems
intuitive, but we will be explicit in situations where the orthography may be
confusing.

\doublespacing

% }}}

% vim: fdm=marker
