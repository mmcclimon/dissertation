%% Chapter 1

\chapter{Introduction}
\label{chap:introduction}
\addtolof{chap:introduction}
\addtocspace

\section{Problems of Jazz Analysis}
\label{sec:problems-jazz-analysis} %{{{

When compared to a score by Beethoven (for example), the jazz lead sheet
appears strikingly bare. The Beethoven score specifies nearly everything one
might need to know in order to perform it. Though the minor details---dynamics,
articulations, phrasing marks, and the like---will differ from piece to piece,
we can usually depend on the presence of some basic information. It is rare
for traditional scores not to include the instrumentation, for example, and a
score that did not include the number of measures or which notes to play in
combination with which which other notes would be very unusual indeed.

And yet, this is the usual state affairs for the jazz lead sheet, which is
probably the most common form of a “jazz score.”\fn{pja-1} Most lead sheets
only include the basic outline of a melody, along with a set of “changes” that
prescribe the harmonic structure of a piece. Beyond this most basic
instruction, every other aspect is left up to the performers.  Of these two
elements (melody and harmony), harmony has a much larger role in determining
the course of a particular jazz performance, so it seems appropriate to focus
our analytical attention on it.

Jazz is essentially a harmonic music. In a typical jazz performance, the melody
of the piece is heard only twice (at the beginning and the end), while the
harmonic structure is heard throughout, determining the structure of the
performance. Each soloist typically plays one or more “choruses,” where each
chorus is understood as a single iteration of the piece’s harmonic structure.
In marked contrast to a Beethoven score, jazz compositions usually remain
unspecified when it comes to their contrapuntal structure: performers will
typically improvise counterpoint that fits with the underlying harmonic
framework. Harmony is the main restraining factor of a piece, and its primary
method of coherence.

The word “jazz”---which has been used at various times to describe McKinney’s
Cotton Pickers, Benny Goodman, Sun Ra, John Zorn, Tito Puente, and Brad
Mehldau, among many others---is inescapably vague, so it will be useful at
this point to delimit the terms of this study somewhat. Here I am interested
in in what might be called “tonal jazz,” which begins in the swing area and
continues through hard bop, covering roughly the years 1940--1965. In this
music, functional harmonic progressions are the norm; “tonal jazz” is
meant in opposition to “modal jazz,” where the rate of harmonic change is
slower and the harmony is mostly non-functional.\fn{pja-2} This includes much
of the music that most people think of when they hear the word “jazz,”
including big-band swing (Count Basie, much of Duke Ellington’s music), bebop
(Charlie Parker, Dizzy Gillespie, Thelonious Monk), and the mainstream jazz
that followed bebop, known variously as “hard bop” or “post-bop” (John
Coltrane, Sonny Rollins, Bill Evans, and many others). I intend the dates to
be flexible, especially on the later end; given the strong influence of the
bebop tradition on jazz and jazz pedagogy, the hard bop style continued to
exist well beyond 1965, and many players today still play in the
style.\fn{pja-2a}

Now that we have delineated “jazz,” we should explain exactly what we mean by
“harmony.” Harmony is of course one of the oldest topics in music theory, and
as such has been hotly contested throughout its history. It is often
found in opposition to counterpoint; in this view, counterpoint is concerned
with individual melodic voices, while harmony is concerned with individual
verticalities. In other traditions (most notably the Schenkerian tradition),
harmony is understood to be an outgrowth of counterpoint: verticalities arise
primarily through contrapuntal procedures. Furthermore, study of harmony is
often broken down by genre: “tonal harmony” plays a different role than does
“chromatic harmony” in both theoretical research and pedagogy.\fn{pja-2b}

When jazz musicians refer to “harmony,” they are typically referring to the
changes themselves; that is, the chord symbols given on a lead sheet or
arrangement. Even when they are not playing from sheet music, the chord symbol
is the basic unit of harmonic understanding for jazz musicians. The reason for
this is largely practical: a chord symbol is a concise way of referring to a
particular sound, and improvising musicians must be able to understand this
information quickly (when reading) and to recall it easily (when improvising).

Since the pioneering work of George Russell in the late 1950s, many jazz
musicians conceive of an equivalence between a harmony (a chord symbol) and a
scale.\footcite{russell:lcc} The chord symbol \h{Dm7} might imply a D dorian
scale, for example, rather than simply the notes D--F--A--C. Because any of
the notes of this scale will sound relatively consonant over a \h{Dm7} chord,
the chord symbol acts as a convenient shorthand for a particular “way of
playing” for a jazz improvisor. This equivalence between chords and scales
will be the focus of Chapter 4; for now it enough to note that understanding
jazz harmony often involves more than understanding relationships between
four-voice seventh chords.

When analyzing jazz harmony, it is often difficult to determine exactly what
one should be analyzing. Lead sheets as circulated in fake books can be highly
inaccurate, and often cannot be relied upon as a single source for any
particular jazz performance, since it is rare that performers play directly
from a lead sheet with no modifications.\fn{pja-3a} In the case of jazz
standards which may have originated elsewhere, we might wonder whether should
we analyze the original sources. In many cases, however, the “jazz standard”
version may be significantly different from the original version, reflecting a
history of adaptation by generations of jazz musicians.\fn{pja-3b} To make
matters worse for the hopeful academic, this knowledge is often secret
knowledge, not written down and learned only from more experienced musicians.

Many published jazz analyses rely on transcriptions of particular performances
as a way to avoid some of these issues. In general, this solution works well,
and I will certainly make use of transcriptions from time to time. This study, however,
is interested in harmony more generally, and transcriptions can confuse
matters somewhat. The kinds of questions I am interested in answering are of
the type “What can we say about harmony in the piece ‘Autumn Leaves?’” and
less often of the type “What can we say about Bill Evans’s use of harmony in
the recording of ‘Autumn Leaves’ from \emph{Portrait in Jazz}?” Furthermore,
even transcriptions are not definitive when it comes to harmony: the pianist
and guitarist might not be playing the same chord; the soloist might have a
different harmony in mind than the rhythm section; or the bass player might
play a bass line in such a way that affects our perception of the chordal
root. Even in the course of a single performance, a group might alter a tune’s
harmonic progression, perhaps preferring some substitutions during solos and
others during the head.\fn{pja-3d}

This is a problem without one clear solution, and it may make more sense to
use one method or another depending on the situation. Some compositions have
canonical recordings---Coleman Hawkins’s recording of “Body and Soul,” for
example---and in those circumstances determining the changes is usually
unproblematic. Other compositions are more fluid, and different choruses might
alter the basic structure within the course of a single performance
(substituting a \h{V7b9} for a \h{V13s11} chord, for example). In these
cases, I am interested in what Henry Martin has called the “ideal changes”; a
hypothetical set of chords that we can use as a basis for understanding the
many variations that might occur in actual performance.\fn{pja-3c} These
changes represent a sort of Platonic model of a composition; individual
performances of “Autumn Leaves” can be seen as instances of some ideal
\textsc{Autumn Leaves}. Determining these ideal changes is often a process
of mediating published lead sheets, recorded versions, and other sources;
throughout this study I have tried to clarify exactly \emph{what} harmony I am
analyzing in any given example.\fn{pja-4a}

In an attempt to answer some of these questions, this dissertation presents a
transformational model of jazz harmony. While on the surface a
transformational model may seem abstract and far-removed from the concerns of
performing jazz musicians, harmony in jazz fits together nicely with David
Lewin’s famous “transformational attitude.”\footcite[][159 (hereafter
\emph{GMIT)}]{lewin:gmit} A jazz
musician does not typically think of harmonies as a series of points in space,
but rather as a series of “characteristic gestures” between them. Rather than
focusing on an underlying tonality, a jazz musician often tries to “make the
changes”---to fully engage with the sound of each individual harmony.

There is often quite a large gap between the way jazz is most commonly taught
(in jazz studios and pedagogical books) and the way it has traditionally been
understood by music theorists. Another goal of the present study is to use
transformational methods in an attempt to narrow this gap, by bringing
theoretical and mathematical rigor to materials that are often ignored by the
theory community, and by applying established theoretical principles in a way
that corresponds closely with the understanding of jazz musicians.

While we can never claim to know what jazz musicians think, we might get
somewhat closer to an answer by examining jazz pedagogical materials. In the
late 1960s, jazz began to be accepted into the academy, and many young jazz
musicians began learned to play the music in schools, rather than exclusively
from older musicians.\fn{pja-4} To supplement this teaching, a great deal of
pedagogical material has appeared that aims to teach young musicians how to
play jazz.

Unfortunately, there is little interaction between these pedagogical materials
and music theoretical materials. Pedagogical materials, such as the recent
\emph{Berklee Book of Jazz Harmony}\nocite{berklee:harmony}, often do not have
bibliographies or mention recent work in the theoretical literature.\fn{pja-5}
Likewise, most theoretical work does not refer to these harmonic handbooks
which are staples of jazz education. Mark Levine’s \emph{Jazz Theory
Book}\nocite{levine:1995}, widely regarded in the jazz education world as
\emph{the} book on jazz theory, does not appear in any of the bibliographies
in a special jazz issue of \emph{Music Theory Online} (18.3), for
example.\fn{pja-5a}

In recent years the theory community has embraced pedagogical materials as a
means of uncovering how historical musicians might have thought about their
own music.\fn{pja-6} Though jazz pedagogues do not typically publish articles
in music theory journals or otherwise consider themselves to be “music
theorists,” per se, the goal of their harmonic textbooks is quite similar to
the goals of pedagogical books in music theory: to teach students how to think
(or hear, or perform) in a particular style of music.

David Lewin points out in his introduction to transformational theory that
when considering a particular musical passage, we often “conceptualize along
with it, as one of its characteristic textural features, a family of directed
measurements, distances, or motions of some sort.”\footcite[16]{lewin:gmit} I
certainly hear these characteristic motions when I listen to jazz, and I think
it is these motions that jazz pedagogues are emphasizing when they
teach students to “make the changes.” Despite its somewhat hostile
mathematical appearance, transformational theory is an effective means of
exploring these families of intuitions. Modeling sets of chord changes as
transformations between harmonic objects, for example, allows the theoretical
discourse to draw on the ways in which jazz musicians teach harmony, and can
bring these two disparate areas somewhat closer together.

% }}}

\section{Theoretical Approaches to Jazz Harmony}
\label{sec:theoretical-approaches} %{{{

Studies of jazz harmony in recent years have primarily taken the form of
Schenkerian analyses that seek to uncover large-scale voice-leading structures
in order to define tonality. Schenkerian analysis has proven to be an
extremely useful tool for analyzing tonal music, and Steve Larson’s pioneering
work in applying its methods to jazz has undoubtedly expanded the field of
jazz studies and brought jazz analysis into the theoretical mainstream. While
theorists may disagree on exactly \emph{how} we should apply these Schenkerian
techniques, hardly anyone seems to doubt that they are the best way to examine
tonal structures in jazz.

The touchstone of the Schenkerian jazz literature is Steve Larson’s
\emph{Analyzing Jazz: A Schenkerian Approach}, which is the culmination of his
work of the previous decades.\fn{ta-1} In this and all of his work, Larson
advocates what might be called an “orthodox” Schenkerian approach. He treats
the extended tones common in jazz (sevenths, ninths, elevenths, etc.) as
standing in for tonic members at some deeper structural
level.\footcite[6]{larson:2009} Steven Strunk’s important article on linear
intervallic patterns in jazz also uses an orthodox approach, as does the work
of Daniel Arthurs, David Heyer, and Mark McFarland.\fn{ta-2}

At levels close to the background, these orthodox analyses of jazz do not
appear significantly different than analyses of classical music, and indeed
that is part of their appeal.\fn{ta-3} Because most jazz is basically tonal
music, these theorists are interested in showing its connection to the
European classical tradition by using the same techniques to analyze both of
them. This is especially important to note, since Schenkerian analysis has a
well-known ethical component.\footcite{cook:1989} For Schenker, the
compositions that were well-described by his theory were judged to be
masterworks. By showing that jazz compositions can also be understood with
these Schenkerian techniques, the implicit conclusion is that they too should
be judged to be masterworks. Underlying these orthodox approaches, I think, is
a desire to legitimize a place for jazz in academic music theory (I
return to this point below).

An opposing group of theorists also supports the use of Schenkerian analysis,
but argues that it should be adapted to account for tonal features specific to
jazz. Principal among this group is Henry Martin, who advocates for the use of
alternative \emph{Ursätze} in jazz, including ascending or gapped
\emph{Urlinien}, non-triadic descents, and chromatic or neighbor-note
\emph{Urlinien}.\fn{ta-4} He argues that jazz pieces are “often influenced by
a more African-American aesthetic that favors repetition and rhythmic
interplay over voice-leading motion through descending linear progressions,”
and thus we should not feel obligated to adhere to traditional Schenkerian
techniques.\footcite[7]{martin:2011}

Martin draws in part on James McGowan’s “dialects of consonance” in jazz,
which describes different contextually stable notes of the tonic triad.\fn{ta-5}
McGowan argues that consonance in jazz is stylistically defined, and that in
certain styles we might hear extended tones as consonances, even though they
would be dissonant in classical music. He describes three “principal dialects”:
the added sixth, common in early jazz and Tin Pan Alley standards; the minor
seventh, particular to the blues; and the major seventh, common in later jazz
performance.\fn{ta-6} For Schenker, the background is derived from the
consonant tonic triad; Martin is able use these stylistically determined
definitions of consonance as support for his stylistically informed background
structures.

McGowan’s work is among the minority in recent years that does not feature a
Schenkerian bent; in addition to his dialects of consonance, he is interested
in applying (paleo-) Riemannian functional analysis to jazz.\fn{ta-7} Despite
this Riemannian focus, however, he is not interested in transformational
analysis: David Lewin’s name appears nowhere in his bibliography.
% xxx definitely need more on bishop
Closer to my own interests is John Bishop’s 2012 dissertation, which also
turns to transformations and mathematical group theory to close the gap
between jazz pedagogy and jazz theory.\footcite{bishop:2012} Bishop focuses
primarily on the interplay of pure triads in jazz, and connects harmony to
chord-scales via these triads.

Other important non-Schenkerian models of jazz harmony are found in earlier
works of Martin and Strunk. In his dissertation, Martin advocates a syntactic
approach based on the circle of fifths, in which chains of descending fifths
point towards a tonic pitch.\fn{ta-8} Steven Strunk’s early theory of jazz
harmony is a layered approached that draws on the Schenkerian concept of
analytical levels to uncover a single tonal center for a jazz
tune.\footcite{strunk:1979} These older models tend to fall more in line with
how jazz musicians themselves discuss harmony, and we will return to them
below.\fn{ta-9}

Given the prevalence of Schenkerian techniques in jazz analysis and its proven
explanatory power in other tonal repertories, it is reasonable to ask why a
different approach like transformational theory is useful or necessary. In
order to answer this question, I would first like to problematize the
Schenkerian focus of recent jazz analysis. Steve Larson’s “Schenkerian
Analysis of Modern Jazz: Questions about Method” will serve as a useful foil;
it is one of the fundamental articles in the field, and its titular questions
will help to guide the discussion here.\fn{ta-10}

In the article, Larson asks three main questions we must answer if we are to
take seriously the suggestion that such Schenkerian analysis is appropriate
for jazz:
\begin{enumerate}
  \singlespacing
  \item Is it appropriate to apply to improvised music a method of analysis
    developed for the study of composed music?
  \item Can features of jazz harmony (ninths, elevenths, and thirteenths) not
    appearing in the music Schenker analyzed be accounted for by Schenkerian
    analysis?
  \item Do improvising musicians really intend to create the complex
    structures shown in Schenkerian analysis?\footcite[210]{larson:1998}
\end{enumerate}

Larson’s answer to the first question is yes. Many of Schenker’s own methods
were of course developed for improvisatory music, and even if they had not
been, they have proven to be explanatory for such music.\fn{ta-11} I agree
with Larson on this point, and don’t have anything in particular to add.
Certainly we should not expect that our theories can prove useful only for the
music for which they are designed; in fact as theorists we generally hope that
the opposite is true, and that our theories have broader applications than
originally intended!

The second question has generated much more discussion in the literature
regarding exactly how we might apply Schenkerian methods to jazz, already
discussed above. While this discussion is mostly one of the mechanics of
analysis, it raises another point in which I am interested: is a jazz
musician’s conception of musical space (or musical structure) the same as a
classical musician’s? The orthodox Schenkerians argue that it is not, and that
jazz is a fundamentally triadic music (since at some deep structural level all
of the extended tones are reduced away). Those that favor a modified approach
disagree with this characterization, but still agree that Schenkerian analysis
is the best way to approach jazz harmony. My own reactions to this question
overlap with my answer to Larson’s third question, so I will return to it
shortly.

Larson’s third question regards compositional or improvisational intent. The
argument is perhaps obvious: because Schenkerian analysis depends on
uncovering long-range voice-leading plans, how could improvising musicians
possibly hold such plans in their memory while playing? At some level, we
might not even be interested in the answer to this question. Schenkerian
analysis has proven explanatory, after all, for music that was doubtlessly
composed without Schenkerian methods in mind (namely, music written before
Schenker’s birth). Nevertheless, Larson spends a great deal of time addressing
this particular point, so we should see to it here as well.


After dismissing the possible intentional fallacy of this question, Larson
turns to pianist Bill Evans as an example of how a jazz musician could produce
complicated long-range voice-leading patterns while improvising. To do so, he
relies heavily on an interview that Bill Evans gave on Marian McPartland’s
radio program, \emph{Piano Jazz}.\fn{ta-12} Evans discusses how he always has
a basic structure in mind while playing:
\begin{quoting}
  \singlespacing
  McPartland: Well, when you say structure, you mean like, one chorus in a
  certain style, another~\ldots

  \vspace{\baselineskip} \noindent
  Evans: No, I’m talking about the abstract, architectural thing, like the
  theoretical thing.\footcite[219]{larson:1998}
\end{quoting}
Evans goes on to demonstrate how he has certain structural features in mind
(harmonic and melodic arrivals). Larson then shows how Evans’s accompanying
commentary can be understood as explaining  voice-leading events like
prolongation and linear progressions, and provides voice-leading analyses of
his playing.\footcite[See especially the table in][229]{larson:1998}

I think that Larson’s use of Schenker’s methods to analyze the music of Bill
Evans is well justified, but I am less sure of the extent to which they are
applicable to jazz more generally. Larson anticipates this objection, noting
that it might be offered on two grounds: “first, that Evans was unusually
talented as an improviser; and second, that his way of thinking was radically
different from that of other jazz musicians.”\footcite[239]{larson:1998}
Certainly Evans was not a typical jazz musician: he was white, and studied at
the Mannes College of Music, then and now a cradle of Schenkerian activity in
this country.\fn{ta-13} Larson suggests that Evans was such an influential pianist
that his Schenkerian improvisational tendencies might have influenced other
musicians with whom he played. While this may be true, Evans was probably
not seen as influential until after 1960, and this study is concerned
primarily with music before that time (or at least, with musicians whose style
was well established by that time).\fn{ta-14}

Larson allows that the first objection is justified: “That Evans was an
unusually talented improviser---and that Schenkerian analysis can show
this---is a principal argument of this article.”\footcite[239]{larson:1998}
This statement is representative of the legitimizing enterprise of the
application of Schenker’s methods to jazz mentioned above. It also contains a
dangerous implication, made explicit in Larson’s closing paragraphs:
\begin{quoting}
  \singlespacing
  Much jazz improvisation lacks the relationships that reward long-range
  hearing, and consists, as [André] Hodeir observes, of “disconnected bits of
  nonsense.” \ldots\ But the fact that jazz musicians often say that “a jazz
  improvisation should tell a story” suggests that many jazz musicians are
  concerned with creating and experiencing global relationships. That they do
  not always achieve this goal in performance is not surprising---the task is
  difficult. But there are exceptions.

  Is Schenkerian analysis applicable only to jazz performances that are
  exceptions? No, Schenkerian analysis may be applied to any jazz
  performance---and it may show the shortcomings of that
  performance.\footcite[240--41]{larson:1998}
\end{quoting}

Far from “retaining Schenker’s methods but not his epistemology, his specific
insights into music but not the system of beliefs that supported them” (as
Nicholas Cook suggests that modern Schenkerians often do), Larson seems to be
using Schenkerian analysis to judge certain performances as “masterworks” and
others as inferior, much in the way done by Schenker
himself.\footcite[439]{cook:1989} Performances by musicians who do not share
Evans’s interest in the “abstract, architectural thing” may well be excellent
performances when judged by other value systems.\fn{ta-15}  Furthermore,
Schenkerian analyses of jazz often focus on what turns out to be the least
interesting part of a jazz piece. Jazz is essentially tonal music, so it is
not at all surprising (to me, at least) that it is often possible to reveal
an \emph{Ursatz} from a particular performance.

Whether or not jazz musicians are thinking in a Schenkerian manner is not
really the point, however. As Steven Rings puts it in the introduction to his
book, “any analytical act will~\ldots\ leave a surplus---a vast, unruly realm
of musical experience that eludes the grasp of [a] single analytical model.
Corners of that vast realm may nevertheless be illuminated via other
analytical approaches, but those approaches will leave their own surpluses.
And so on.”\footcite[5]{rings:2011} Schenkerian analysis typically focuses on
long-range voice leading in order to reveal an underlying diatonic framework,
while deemphasizing (some would say reducing) surface details, including much
harmonic chromaticism.

This focus on harmony as a first-class object is something that is at the
heart of much jazz pedagogical material, and might help constitute a different
set of analytical values by which we can understand jazz. Rather than having
to consign musicians who we cannot understand with Schenkerian analysis to a
second tier of appreciation, we can instead try to understand them on
something of their own terms. A Schenkerian analysis of Coltrane’s “Giant
Steps” solo might reveal that it is “lacking an artistically convincing
relationship among structural levels,” but it would likely be difficult to
find a jazz musician who did not hold the composition up as an example of Coltrane’s
supreme mastery of the music.\fn{ta-16}

% }}}

\section{Transformational Theory}
\label{sec:transformational-theory} %{{{

Given the current dominance of Schenkerian theory in the study of jazz
harmony, we might ask what transformational thinking brings to the table.
Transformational theory in recent years has focused on neo-Riemannian
analysis, with particular emphasis on efficient voice leading and
non-functional, chromatic progressions. Steven Rings has written that this
focus on neo-Riemannian theory has “led to a view that some works are divvied
up into some music that is tonal \ldots\ and some that is transformational.”
Continuing, he argues that to do so “is to misconstrue the word
transformational, treating it as a predicate for a certain kind of music,
rather than as a predicate for a certain style of analytical and theoretical
thought.”\footcite[9]{rings:2011} As he is right to point out, there is
nothing about transformational theory that necessitates its restriction to
this locally chromatic music; his book uses the theory to explain, as he says,
“specifically \emph{tonal} aspects of tonal music.” It is this use of
“transformational” that I wish to bring to bear on jazz, which is essentially
tonal music.

Fundamental to the Schenkerian approach is the relatively equal balance of
harmony and voice leading; for jazz musicians, though, this balance is heavily
weighted toward the harmonic.  Schenkerian analysis tends to deemphasize
certain harmonies with the aim of exposing an underlying diatonic framework.
This goal is in contrast with the typical goal of a performing jazz musician,
for whom individual chords have first-class status.

Transformational theory too, often treats harmonies as first-class objects,
and thus makes it especially appropriate for analyzing jazz. “Often,” only
because transformational theory can be used for more than simply examining
harmony: Lewin’s Generalized Interval Systems (explained in detail below) only
require a set of elements, a group of intervals, and a function mapping pairs
of elements of the set into the group of intervals.
Most commonly the elements of the set are harmonies, but they do not have to
be.\fn{tt-1}

Mathematical music theories have become especially widespread in recent
years.\fn{tt-2} Many of these models focus only on triads; while these
models are valuable, nearly all chords in jazz are (at least) seventh chords.
Because the neo-Riemannian literature is relatively well-known, it will be
useful here to limit our focus to those theories that deal in some way with
non-triadic music. This work falls basically into two categories: work
that deals exclusively with a single type of chord, and work that deals with
musical objects of different types.

Most of the studies dealing with a single type of chord are concerned with
members of set class (0258), the half-diminished and dominant seventh chords.
In a 1998 article, Adrian Childs develops a model for these chord types that
is closely related to standard neo-Riemannian transformations on
triads.\footcite{childs:1998} Edward Gollin’s article in the same issue of the
\emph{Journal of Music Theory} explores three-dimensional Tonnetze in
general, with special focus on the dominant and half-diminished seventh
chords.\fn{tt-3}

In general, neo-Riemannian-type operations on the (0258) tetrachords turn out
to be somewhat less useful than their triadic counterpoints, owing to the
symmetry of set class (0258).\fn{tt-4} Any one tetrachordal Tonnetz can only
show a subset of all of the (0258) tetrachords, while the familiar triadic
Tonnetz of course shows all 24 major and minor triads.  Recognizing this
limitation, Jack Douthett and Peter Steinbach present a model that also
includes minor sevenths and fully diminished seventh chords, using a digram
they refer to as the “Power Towers.”\footcite[255--56]{douthettsteinbach:1998}
While Douthett and Steinbach’s description accounts for two of the three main
types of seventh chords commonly used in jazz (it is missing the crucial
major seventh), all of these neo-Riemannian models focus on parsimonious
voice leading. While this focus is valuable, it will not prove to be terribly
useful for the functional harmony in which this study is interested.

The other group of transformational models consists of what Julian Hook has
termed “cross-type transformations”: he extends Lewin’s definition of a
transformation network to allow for transformations between objects of
different types.\fn{tt-5} This category of transformations contains the
inclusion transformations (discussed by both Hook and Guy Capuzzo) which map a
triad into the unique dominant or half-diminished seventh chord that contains
it and vice versa.\footcites{hook:2002,capuzzo:2004} Also included in this
category are more general approaches for relating set classes of different
cardinalities, including Joseph Straus’s formulation of atonal voice leading
and Clifton Callender’s split and fuse
operations.\footcites{straus:2003,callender:1998} Finally, Dmitri Tymoczko’s
continuous tetrachordal space can accommodate \emph{all} four-note chords, but,
as Hook notes, Tymoczko downplays (and sometimes ignores) the transformational
aspects of his geometric models.\fn{tt-6}

Though I have mentioned these cross-type transformational works only in
passing here, we will return to some of them in some detail below, where they
will be more relevant. Because Hook does not strictly define what constitutes
a “type” in a cross-type transformation, his formulation will allow us to
apply transformational procedures rigorously in situations where we might wish
to consider objects to be members of different types, even though they may be
identical in some other typological system.\fn{tt-7}

%}}}

% xxx I hate this title
\section{Aside: Lead Sheet Notation}
\label{sec:lead-sheets} %{{{

As mentioned in Section \ref{sec:problems-jazz-analysis}, jazz musicians often
begin learning a particular tune with a lead sheet, and the changes found
there form the harmonic foundation of a particular performance. Because this
dissertation is interested in jazz harmony generally, lead sheets serve as a
useful abstraction of the countless possible instantiations of any one tune,
and considering them them briefly here will prove fruitful for the rest of
this study.\fn{ls-1}

\begin{figure}[p]
  \centerGraphic[width=\textwidth]{eps/ch1/just-friends.pdf}
  \caption{A sample lead sheet of ``Just Friends'' (John Klenner/Sam Lewis).}
  \label{ls:just-friends}
\end{figure}

A lead sheet typically gives only a melody and a set of chord changes: Figure
\ref{ls:just-friends} gives the \emph{Real Book} lead sheet for John Klenner
and Sam Lewis's ``Just Friends.''\fn{ls-2} It is a very typical example, and
nearly everything about the page is designed to make it easy for a jazz
musician to ``fake'' a performance of the tune on the bandstand: the anonymous
compilers of the book provide the composer and lyricist's names and a sample
recording; the music is split into four-measure chunks to make the phrases and
the form of the tune clear; and there is almost no extraneous
information---even the key signature is omitted on all lines but the first.
The melody is given for the head, and the changes are provided for the rhythm
section (usually piano, bass, and drums, but sometimes other instruments like
guitar) and for solos. All other aspects of the tune need to be negotiated
prior to performance (how the tune will begin and end, for example).

\emph{The Real Book} uses standard conventions for labeling chords. A chord
symbol consists of the chord root (referred to by a letter name), and a symbol
indicating the quality: the most common of these are are the dominant seventh
(simply ``$7$''), minor seventh (``--$7$'') and major seventh (``maj$7$'').\fn{ls-3}
Thus, the opening of ``Just Friends'' (\h{G7}--\h{Cmaj7}) consists of a G
dominant seventh chord moving to a C major seventh chord. This abstraction is
extremely useful for a performing musician, but leaves something to be desired
if pressed into use as an analytical system. The chord symbols do not
explicitly tell us, for example, that \h{G7}--\h{Cmaj7} is a typical
V\tsup{7}--I\tsup{7} progression in C major; that kind of knowledge is
implicit for experienced musicians and analysts.

Complicating the problem somewhat is that chord symbols are imprecise by
design. In most situations, jazz musicians do not want to be told exactly what
notes they should play (if they did, they probably would not have become jazz
musicians); instead, they treat chord symbols only as guidelines. A \h{G7}
chord would certainly include the root, third, and seventh (G, B, and F), but
might also include the \h{s11} (C\sharp), \h{b9} (\Aflat), or \h{s5} (D\sharp),
depending on the situation: the melody might suggest certain alterations, a
performer might prefer some alterations over others, or an improvisor may work
themselves into a dissonant portion of a solo where a bare dominant seventh in
the piano would sound especially out of place.

\begin{figure}[htbp]
  \centerGraphic{eps/ch1/just-friends-realizations.pdf}
  \captionsetup{format=hang}
  \caption[Two piano realizations of ``Just Friends,'' mm.~1--8.]{%
    Two piano realizations of ``Just Friends,'' mm.~1--8. \\
    a\rightparen\ Using three-note voicings (from Mark Levine, \emph{The Jazz
      Piano Book}). \\
    b\rightparen\ Using more alterations/extensions (chord symbols reflect
    voicings used).

}
  \label{ls:just-friends-realizations}
\end{figure}

To illustrate the flexibility of chord symbols, Figure
\ref{ls:just-friends-realizations} gives two realizations of the first eight
measures of ``Just Friends.'' The first is Mark Levine's, from early in his
book on jazz piano; it uses what he calls ``three-note voicings'' (the root,
third, and seventh).\fn{ls-4} The second realization is my own, and features
many alterations to the basic outlines given by the chord symbols. Both of
these realizations are valid interpretations of the given chord symbols, and
are meant to reinforce the point that chord symbols, while only a guideline,
indeed represent something important about a given harmonic progression. For
all of their imprecision, chord symbols represent a reality for performing
jazz musicians, and as such will be foundational for our work on harmony here.


%}}}

\section{Diatonic Chord Spaces}
\label{sec:diatonic-spaces} %{{{

It will be easiest to introduce the transformational approach to jazz harmony
developed in this study by way of an example. Much of this dissertation will
be interested in the development of various musical spaces and motions that
are possible within them. This kind of approach was first developed by David
Lewin in \emph{Generalized Musical Intervals and Transformations}, and a
review of his approach will be useful before moving on to more involved
examples.

\subsection{Intervals and Transformations}
\label{subsec:inttrans} % {{{{

Figure \ref{ds:autumn-leaves-changes} shows the chord changes to the A section
on the jazz standard “Autumn Leaves.”\fn{ds-2a} Jazz musicians sometimes refer
to this progression as a “diatonic cycle”: it uses only seventh-chords found
in the G-minor diatonic collection.\fn{ds-2b} As in classical music, the leading
tone is raised in the dominant chord so that the resulting chord is \h{D7}, not
\h{Dm7}. We can easily arrange this progression around the familiar
circle of fifths, placing the tonic G minor at the top of the circle (see
Figure \ref{ds:circle-of-fifths}).

\begin{figure}[tbp]
  \centerGraphic{eps/ch1/autumn-leaves-changes.pdf}
  \caption{The changes to “Autumn Leaves” (Joseph Kosma), A section.}
  \label{ds:autumn-leaves-changes}
\end{figure}

\begin{figure}[htbp]
  \centerGraphic{eps/ch1/circle-of-fifths.pdf}
  \caption{The changes to “Autumn Leaves,” arranged around the diatonic circle
    of fifths.}
  \label{ds:circle-of-fifths}
\end{figure}

While this arrangement around the diatonic circle of fifths makes intuitive
sense, it can also represent what Lewin has called a Generalized Interval System
(\gis{}). Generalized Interval Systems are Lewin’s way of formalizing the
“directed measurements, distances, or motions” that we often understand as
“characteristic textural features” of a given musical
space.\footcite[16]{lewin:gmit} Though we will unpack this definition using
the harmonies from the A section of “Autumn Leaves” as an example, Lewin’s
formal definition is as follows:
\begin{quoting}
  \singlespacing
  A \emph{Generalized Interval System} (GIS) is an ordered triple (S, IVLS,
  int), where S, the \emph{space} of the GIS, is a family of elements, IVLS,
  the \emph{group of intervals} for the GIS, is a mathematical group, and int
  is a function mapping S $\times$ S into IVLS, all subject to the two
  conditions (A) and (B) following.
  \begin{compactenum}[(A): ]
    \item For all r, s, and t in S, \mbox{int(r,s)int(s,t) = int(r,t)}
    \item For every s in S and every i in IVLS, there is a unique t in S which
      lies the interval i from s, that is a unique t which satisfies the
      equation \mbox{int(s,t) = i}.\fn{ds-2c}
  \end{compactenum}
\end{quoting}

The first element in a \gis{} is a family of elements, S, which Lewin also calls
a musical space. Preceding this formal definition, he gives a number of
examples of musical spaces, including the familiar diatonic and pitch and
pitch-class spaces, along with less familiar musical spaces like frequency
space, time point space, and various durational
spaces.\footcite[16--25]{lewin:gmit} In our “Autumn Leaves” example, we are
interested in the (unordered) set of harmonies in the G-minor diatonic
collection: \mbox{$S =$ \{\h{Gm},} \h{Am7b5}, \h{Bbmaj7}, \h{Cm7}, \h{D7},
\h{Ebmaj7}, \h{F7}\}.\fn{ds-3}

With the first element of a \gis{} satisfied, we must then define a group of
intervals (\ivls). Though we could perhaps imagine a number of
different ways to define intervals among elements of the set S (a point to
which we will return later), the most obvious is to measure distances in
diatonic steps between chord roots. Because we are interested in abstract
chord roots and not the actual pitches played by some bass player or pianist’s
left hand, we will use diatonic pitch classes. This has the effect of
modularizing the set of possible intervals, changing \ivls{} from
$\{\ldots,-2,-1,0,1,2,\ldots\}$ (as it would be in diatonic \emph{pitch}
space) to $\{0,1,\ldots,6\}$ (diatonic pitch \emph{class} space).\fn{ds-3b}
Arithmetic in this group is mod-7, exactly in the way that arithmetic using
the more familiar chromatic pitch class space, $\{0,1,2,\ldots,11\}$, is
mod-12.

Lewin specifies that \ivls{} must be a mathematical group, and we will take care
here to show that \ivls{} = $\{0,1,\ldots,6\}$ is indeed such a group. A group
is a set of elements, $G$, and a binary operation, $\otimes$, that satisfies the
four group axioms:
\begin{compactitem}
  \singlespacing
  \item Closure: for $a, b \in  G$, then $a \otimes b$ must be an
    element of $G$.
  \item Associativity: for $a, b, c \in G$, then the equation $(a \otimes b)
    \otimes c = a \otimes (b \otimes c)$ must be true.
  \item Identity element: There exists an element $e \in G$ such that for any
    element $a \in G$, $a \otimes e = e \otimes a = a$ is true.
  \item Inverses: For any element $a \in G$, there exists a unique element
    $a^{-1} \in G$ such that $a \otimes a^{-1} = a^{-1} \otimes a = e$ is
    true.\fn{ds-4}
\end{compactitem}

To show that our \ivls{} is a group, it is sufficient to show that the set
$\{0,1,\ldots,6\}$ under some binary operation satisfies the group axioms. The
binary operation is simply addition mod-7 (which we will notate using the usual $+$
sign instead of the abstract $\otimes$ used above).  We can then show that the
set \ivls{} is closed: for any two elements $a, b \in \mathrm{\ivls{}}$, $a + b$ is
also an element of \ivls{} ($1 + 3 = 4; 4 + 4 = 1; 1 + 0 = 1$; and so on).
Modular addition, like its non-modular counterpart, is associative:
$(3 + 4) + 5 = 3 + (4 + 5)$, and likewise for any chosen elements of \ivls{}.
The identity element for addition is $0$, which combined with any element $a \in
\mathrm{\ivls{}}$ gives $a$ itself. Inverses in the group are simply complements
mod-7 (the number that when added to the given number gives $0$, mod-7):
$2^{-1} = 5; 1^{-1} = 6$; and so on. The integers modulo $n$ are labeled
$\intZ_n$, so we may also refer to \ivls{} in our “Autumn Leaves” example as
the group $\intZ_7$.

The last element of a \gis{} is an interval function that maps $S \times S$ into
\ivls{}. In other words, the interval from one element of S to another must be a
member of the group $\intZ_7$. In our “Autumn Leaves” example, the interval
from one element of S to another is simply the number of steps in the G-minor
diatonic collection (always counting upward) between the two elements. Thus,
\mbox{int(\h{D7}, \h{Gm}) = $3$}, since G, the root of the second chord, lies
3 diatonic steps above D, the root of the first. Likewise,
\mbox{int(\h{Am7b5}, \h{Bbmaj7}) = $1$}; \mbox{int(\h{F7}, \h{Ebmaj7}) = $6$}; and so
on.

We now have all of the elements of a \gis{}, but we must still prove that Lewin’s
conditions A and B are satisfied, which we will do by example. Condition A
states that for all r, s, and t in S, \mbox{int(r,s)int(s,t) = int(r,t)}. In
our “Autumn Leaves” example, \mbox{int(\h{Cm7}, \h{D7})} and \mbox{int(\h{D7}, \h{Ebmaj7})}
(both interval 1) must combine to equal \mbox{int(\h{Cm7}, \h{Ebmaj7})} (interval
2). Second, for every chord $s$ in S and every interval $i$ in $\intZ_7$,
there must be a unique chord $t$ which satisfies the equation \mbox{int$(s,t)
= i$}. For example, there must be exactly one chord that lies 2 units above
\h{Cm7}: namely, \h{Ebmaj7}. It is easy to confirm that the two conditions
are also true for any choice of $r, s, t \in$ S.

Before shifting our focus from generalized intervals to transformations, I
want to return to the structure of \ivls{}, the group $\intZ_7$. Above, we
measured intervals by the number of G-minor diatonic steps between chord
roots: a single step corresponded to the interval 1. We can also say that the
group $\intZ_7$ is a cyclic group, \emph{generated} by the interval 1. (Cyclic
groups are notated $\mathcal{C}_n$, where $n$ is the size of the group; we
could therefore label the group in question $\mathcal{C}_7$.) Counting
diatonic steps is not the only way we might consider measuring intervals in
the space, however. Given the ubiquitous descending fifths of “Autumn Leaves,”
we might instead like to measure distance by the number of descending fifths
between chord roots. Though this generation has no noticeable effect on the
abstract group structure of \ivls{}---the group is $\mathcal{C}_7$ in either
case---it does affect the last element of a \gis{}, the interval
function.\fn{ds-5} As shown in Figure \ref{ds:c7-generators},
\mbox{int(\h{Am7b5}, \h{D7}) = $3$} when measured by diatonic steps (the left
figure), but \mbox{int(\h{Am7b5}, \h{D7}) = $1$} when measured by descending fifths
(the right figure).\fn{ds-5a}

\begin{figure}
  \centerGraphic{eps/ch1/c7-generators.pdf}
  \caption[The “Autumn Leaves” \gis{}, generated by diatonic step and descending
    fifth.]{The “Autumn Leaves” \gis{}, generated by diatonic step (left) and
    descending fifth (right).}
  \label{ds:c7-generators}
\end{figure}

Generalized Interval Systems are but one part of Lewin’s project; we will now
turn our attention to the “transformations” of the book’s title. Much has been
made of the “transformational attitude” that accompanies the shift from
generalized intervals to transformations that occurs in the later chapters of
\emph{GMIT}. As Lewin has it, \gis{} thinking represents a Cartesian,
observer-oriented position, examining musical objects as points in abstract
space. This is in contrast to the transformational attitude, which Lewin
describes as “much less Cartesian” in what is perhaps the most-cited portion
of the book:
\begin{quoting}
  \singlespacing
  Given locations s and t in our space, this attitude does not ask for some
  observed measure of extension between reified “points”; rather it asks: “If
  I am \emph{at} s and wish to get to t, what characteristic gesture
  (e.g.~member of STRANS) should I perform in order to arrive
  there?\fn{ds-6}
\end{quoting}

The \gis{} perspective outlined above is an intervallic perspective: we developed
a system that allowed us to say that “the distance from \h{Am7b5} to \h{D7} is
3.” By replacing, as Lewin does, “the concept of interval-in-a-\gis{}” with “the
concept of transposition-operation-on-a-space,”\footcite[157]{lewin:gmit} we
can convert this \gis{} statement into a transformational one: “transposing the
root of \h{Am7b5} by three diatonic steps gives \h{D7}.” The transformational
statement is more active, replacing distance metrics with verbs like
“transpose.”\fn{ds-7}

Though there is certainly a difference in the language used in \gis{} statements
and that used in transformational statements, Lewin takes care to note that
the two attitudes are not diametrically opposed:
\begin{quoting}
  \singlespacing
  we do not have to choose \emph{either} interval-language \emph{or}
  transposition-language; the generalizing power of transformational theory
  enables us to consider them as two aspects of one phenomenon, manifest in
  two different aspects of this musical composition.\fn{ds-8}
\end{quoting}
While we might prefer interval-language in some contexts and
transformation-language in others, the two attitudes are quite closely
related; any \gis{} statement can be converted into a transformational one by
using the mechanism Lewin describes just before the famous passage in
\emph{GMIT}.

By combining the space S of a \gis{} with an operation-group on S, we can derive
a transformational system. This operation-group must be simply transitive on S
(hence the reference to \strans\ in the quote above): “given any elements s and
t of S, then there exists a unique member OP of STRANS such that OP(s) =
t.”\footcite[157]{lewin:gmit} Lewin then states that in any \gis{}, “there is a
unique transposition-operation T satisfying T(s) = T, namely T =
T\tsub{int(s,t)}.''\footcite[157]{lewin:gmit} In familiar chromatic pitch-class
space, the unique transposition $T_k$ is that operation where $k$ is the
interval in semitones between the two pitches: the interval between C and
\Eflat\ is 3, and the operator $T_3$ maps C onto \Eflat.

Converting our “Autumn Leaves” \gis{} into a transformational system, then, is
only slightly more complicated than this chromatic pitch-class space example, since
ordinary transposition will not work intuitively. To avoid confusion with the
traditional transposition operator, we will instead use the lowercase,
$t_k$.\fn{ds-9} This diatonic transposition operator will be used in almost
the same way, however: the operation $t_k$ is that transposition which
transposes the root of a chord $k$ steps inside the G-minor diatonic
collection. With this understanding, the conversion works as expected: the
interval between \h{Am7b5} and \h{D7} is 3, so the operator $t_3$ maps
\h{Am7b5} onto \h{D7} in the space \mbox{$S =$ \{\h{Gm},} \h{Am7b5}, \h{Bbmaj7},
\h{Cm7}, \h{D7}, \h{Ebmaj7}, \h{F7}\}.\fn{ds-10} The same statement may also be
written as \h{Am7b5} $\overset{t_3}{\lra}$ \h{D7}, if we want to emphasize the
idea of a transformation as a mapping operation.

Now that we have explored the underlying mathematics, we are in a position to
reexamine the A section of “Autumn Leaves,” (as given in
Figure~\ref{ds:autumn-leaves-changes}), as well as the circle-of-fifths
arrangement in Figure~\ref{ds:circle-of-fifths}. The progression is, quite
simply, a series of $t_3$ operations within the G-minor diatonic space; each
chord root descends by diatonic fifth. In a mod-12 chromatic space (like the
ones we will begin developing in the next chapter) it can be difficult to make
sense of this harmonic progression. The chord qualities can be confusing---two
major sevenths in a row followed immediately by a half-diminished
seventh---and it is sometimes hard to remember where the tritone in the bass
falls in the progression, given the prevalence of perfect-fifth bass motion in
chromatic space. Understood in the context of this diatonic space, however,
the harmonic motion becomes much clearer.

It is reasonable to pause at this point and ask what advantages this
transformational approach brings. After all, we began our discussion of
“Autumn Leaves” by noting that jazz musicians sometimes refer to the
progression of the A section as a “diatonic cycle,” and it may seem as though
we have gone through a great deal of mathematical rigmarole simply to arrive
back at our starting point. There would seem to be very little difference
between describing a progression as a “diatonic cycle” and describing it as “a
series of $t_3$ transformations in the G-minor diatonic set of seventh
chords.” And yet, this is almost exactly the point. Transformational theory
allows us a means to formalize what is often intuitive knowledge for jazz
musicians, thereby narrowing the gap between the way jazz musicians discuss
harmony and the way music theorists often do.


% }}}}

\subsection{Analytical Applications}
\label{subsec:analytical-applications} % {{{{
\FloatBarrier

Before concluding this chapter, I want to examine “Autumn Leaves” in a bit
more detail, then move on to a few other analytical examples. The full chord
changes to “Autumn Leaves” are given in Figure
\ref{ds:autumn-leaves-complete}. The opening A section, discussed at length
above, can be understood as a series of $t_3$ operations in the G-minor
diatonic set. The bridge modulates to the relative major, but we can still
understand this passage using the same transformational system. After
repeating the G-minor \tfo\ progression that concludes the A section in the
first four measures of the bridge, the entire progression is transposed up a
third (a larger-scale $t_2$) to repeat the \tfo\ in the key of \Bflat. Despite
this modulation, the connections from chord to chord are all $t_3$ operations,
continuing the chain that has been present since the beginning (see Figure
\ref{ds:autumn-leaves-bridge}).\fn{ds-11a}

\begin{figure}[tbp]
  \centerGraphic{eps/ch1/autumn-leaves-complete.pdf}
  \caption{The complete changes to “Autumn Leaves.”}
  \label{ds:autumn-leaves-complete}
\end{figure}

\begin{figure}[htbp]
  \vspace{\baselineskip}
  \centerGraphic{eps/ch1/autumn-leaves-bridge.pdf}
  \caption{A transformation network for “Autumn Leaves,” bridge.}
  \label{ds:autumn-leaves-bridge}
\end{figure}

There are three passages in this piece that are not simply $t_3$ operations:
the connection from \h{Gm} to \h{Am7b5} that begins the bridge; the connection
from \h{Bbmaj7} at the end of the bridge to \h{Am7b5} that follows; and the
third and fourth bars of the final section,
\h{Gm7}--\h{C7}--\h{Fm7}--\h{Bb7}. The first of these, \h{Gm}--\h{Am7b5}, is a
$t_1$ transformation. The bridge begins by retracing the same harmonic ground
as the last four bars of the A section (a \tfo\ progression in G minor); the
connection to the bridge, then, can be understood as reversing the two
descending fifths that ended the A section. This observation can be
represented algebraically ($t_3^{-1} \bullet\ t_3^{-1} = t_4 \bullet\ t_4 = t_1$) or
graphically (by taking two steps counterclockwise in the circle of fifths in
Figure \ref{ds:circle-of-fifths}).\fn{ds-11b}

The next transformation that breaks the series of $t_3$ operations is the
$t_6$ from \h{Bbmaj7} to \h{Am7b5} at the end of the bridge. This $t_6$ is
easily understood as a combination of two $t_3$ operations, by imagining that
there is a “missing” \h{Ebmaj7} chord in the last bar of the bridge.\fn{ds-11}
Interpolating this chord allows us to hear the progression (beginning on
\h{Cm7}) as identical to that of the first eight bars, now displaced to span a
formal boundary, as shown in Figure \ref{ds:displaced-cycle}.\fn{ds-12}

\begin{figure}[htbp]
  \centerGraphic{eps/ch1/displaced-cycle.pdf}
  \caption{The diatonic cycle of “Autumn Leaves,” with a hypermetrically
    displaced copy spanning the formal boundary at the end of the bridge.}
  \label{ds:displaced-cycle}
\end{figure}

The only four chords in “Autumn Leaves” that cannot be understood in the
\gis{} developed above are those in the progression
\h{Gm7}--\h{C7}--\h{Fm7}--\h{Bb7}. While the chord roots belong to the G-minor
collection, the qualities are incorrect. This progression consists of two
\mbox{ii\tsup{7}--V\tsup{7}} progressions, the first in F and the second in
\Eflat\ (a tonic that, like the one at the end of the bridge, does not
actually appear in the music). These \tf\ progressions are best situated in
chromatic, rather than diatonic, space; for now, we will pass over this
progression until we develop such a space in the next chapter.

It is relatively rare for tunes to be as systematically diatonic as “Autumn
Leaves” (though Bart Howard’s “Fly Me to the Moon” comes close); instead,
pieces often make use of diatonic cycles in only a portion of a piece before
moving on to contrasting music. A relatively straightforward example of this
is Sammy Fain and Bob Hilliard’s “Alice in Wonderland,” the changes to which
are shown in Figure \ref{ds:wonderland-changes}.\fn{ds-13} Like “Autumn
Leaves,” “Alice in Wonderland” begins with a minor diatonic cycle in mm.~1--7
(also beginning on a iv chord), shown here in A minor. We can modify our \gis{}
from above simply by changing the set ($S$ is now the set \{\h{Am}, \h{Bm7b5},
\h{Cmaj7}, \h{Dm7}, \h{E7}, \h{Fmaj7}, \h{G7}\}); the group \ivls{} and the
interval function are identical.\fn{ds-13a} The harmony in m.~7 is an \h{Am7} chord,
but because the quality of the tonic seventh chord is ambiguous in minor
keys, this chord is easily understood as tonic.

\begin{figure}[htbp]
  \centerGraphic{eps/ch1/wonderland-changes.pdf}
  \caption{Changes to “Alice in Wonderland” (Sammy Fain/Bob Hilliard), mm.\
  1--16.}
  \label{ds:wonderland-changes}
\end{figure}

The second eight measures are all diatonic in the key of C major; the linking
\h{Eb7} chord signals a shift between diatonic collections.\fn{ds-14} This
move to the relative major is a common one, and the only difference between
the two diatonic sets is the shift of \h{E7} (in A minor) to \h{Em7} (in C
major). The progression here is not as systematic as the first eight (see the
annotations in Figure \ref{ds:wonderland-transformations}). After a C-major
\tf\ in mm.~9--10, the harmony moves up a step to \h{Em7}--\h{Am7}.\fn{ds-15}
The progression \h{G7}--\h{Em7} seems to move backwards in the cycle, almost
as if realizing the phrase is in danger of arriving at C major too soon. We
can capture this intuition by choosing to understand the progression not as a
forward-directed $t_5$, but as an algebraically-equivalent combination of
three ascending fifths: ${t_3}^{-1} \bullet {t_3}^{-1} \bullet {t_3}^{-1}$ (which
we could also write as ${t_3}^{-1}$ raised to the third power: ${({t_3}^{-1})}^3$).

\begin{figure}[btp]
  \centerGraphic{eps/ch1/wonderland-transformations.pdf}
  \caption{The changes to “Alice in Wonderland,” with transformational labels
  between harmonies.}
  \label{ds:wonderland-transformations}
\end{figure}

Throughout \emph{GMIT}, Lewin is clear that transformational theory is a means
of expressing our “intuitions” about a musical passage in a mathematically
rigorous way.\fn{ds-16} As he puts it: “If I want to change Gestalt 1 into
Gestalt 2 \ldots\ , what sorts of admissible transformations in my space
(members of \strans\ or otherwise) will do the best
job?”\footcite[159]{lewin:gmit} Our explication of diatonic seventh-chord
spaces may appear to stem from the desire to label everything a $t_3$ and move
on, confident that we could justify our labels mathematically if called upon
to do so. This, though, could not be further from the truth; developing the
space allows us a powerful means to capture intuitions (or apperceptions) like
the one above. Though the operation $t_5$ does map \h{G7} to \h{Em7} in the
C-major diatonic space, hearing this connection as $t_3^{-1} \bullet t_3^{-1} \bullet
t_3^{-1}$ instead represents the idea of stepping backwards through the circle
of descending fifths (an observation that may not be easy to show using other
methods of analysis).

Both examples of outright diatonic cycles we have seen thus far have been in
minor keys. In general, minor-key cycles are easier to use than those in major
keys (a C-major diatonic cycle is given for reference in Figure
\ref{ds:major-cycle}). In minor, the raised leading tone in the V chord means
that the cycle consists of a \tfo\ in the relative major and a \tfo\ in the
tonic joined by the VImaj7 chord (as we saw in the second cycle in Autumn
Leaves, Figure \ref{ds:displaced-cycle}). In a major-key cycle, the lone
half-diminished seventh chord is followed immediately by three minor seventh
chords (in C major, \h{Bm7b5}--\h{Em7}--\h{Am7}--\h{Dm7}), which by comparison
is a relatively unusable tonal progression.

\begin{figure}[tbp]
  \centerGraphic{eps/ch1/major-cycle.pdf}
  \caption{A diatonic cycle in C major.}
  \label{ds:major-cycle}
\end{figure}

Nevertheless, Earl Zindars’s standard “How My Heart Sings” (the changes to
which are given in Figure \ref{ds:heart-sings-changes}) does indeed contain a
major-key cycle.\fn{ds-17} Unlike the previous two examples, this tune begins
on the iii chord, at which point it begins the $t_3$ cycle. Once again, it is
the \h{Am7} chord that alerts us that this progression takes place in
diatonic, rather than chromatic, space (\h{A7} would seem to make more
harmonic sense, as the dominant of the following D minor). After reaching the
tonic in m.~5, the $t_3$s continue making their way to A minor, four measures
later.

\begin{figure}[tbp]
  \centerGraphic{eps/ch1/heart-sings-changes.pdf}
  \caption{Changes to “How My Heart Sings” (Earl Zindars), mm.~1--12.}
  \label{ds:heart-sings-changes}
\end{figure}

Notably, the chord in m.~8 is an \h{E7b9}---this chord belongs not to C major,
but rather to A minor. At some point, then, the progression shifts from taking
place in a C-major cycle (the E chord is a minor seventh in the opening bar)
to an A-minor cycle. Unlike the abrupt shift to the relative major in “Alice
in Wonderland” (signalled by \h{Eb7}), this modulation is gradual, using a
traditional pivot chord.\fn{ds-18} In a transformational reading, the
progression maintains the $t_3$ sequence throughout the first nine measures,
but the underlying diatonic set changes almost imperceptibly from C major to A
minor. Zindars uses this modulation in order to negotiate the unusual
succession of chord qualities in the major diatonic cycle: by beginning the
progression on iii in a major key and modulating to the relative minor before
returning to it, he is able to have his cake and eat it too---the chord root E
appears first as a minor seventh and again later as a dominant of the relative
minor.

One more example will suffice to conclude our discussion of diatonic cycles:
Jerome Kern’s “All the Things You Are” (the changes are given in Figure
\ref{ds:all-things-changes}).\fn{ds-18a} Though this tune does not contain a
cycle as explicit as the examples we have seen so far, viewing this piece
through the lens of diatonic seventh chord space reveals relationships that may
otherwise go unnoticed.

\begin{figure}[btp]
  \centerGraphic{eps/ch1/all-things-changes.pdf}
  \caption{Changes to “All the Things You Are” (Jerome Kern).}
  \label{ds:all-things-changes}
\end{figure}

“All the Things You Are” begins in the key of F minor, and progresses through
a diatonic cycle (a chain of $t_3$ operations in an F-minor diatonic
seventh-chord \gis{}) until arriving on \h{Dbmaj7} in m.~5.\fn{ds-18b} At this point, the
chord roots continue to descend by diatonic fifth in the key of F minor, but
the qualities of the chords rooted on G and C have been altered, from
\h{Gm7b5} and \h{C7} (as they would be in the F-minor \gis{}) to \h{G7} and
\h{Cmaj7}. This arrival on a C major chord, rather than a C dominant seventh,
has the effect of a half cadence in the prevailing key of F minor. The
half-cadential C-major chord also serves as a linking chord to the next
phrase, which contains a diatonic cycle in the key of C minor. Like the first
A section, this phrase also veers away from the cycle to end in a half cadence
on G major.

The bridge of this tune is usually described as being made up of two \tfo\
progressions, the first in G major and the second in E major. While this is
true, we might also understand this progression as an alteration of a diatonic
cycle in E minor.  The bridge begins on a iv chord, which initiates a \tfo\ in
the relative major. After two bars of \h{Gmaj7}, a $t_6$ takes us to \h{Fsm7b5}.
Just as in the bridge to “Autumn Leaves,” we can understand this $t_6$ as a
combination of two $t_3$ operations, interpolating a missing \h{Cmaj7}
chord.\fn{ds-19} Though the cycle of the bridge (and its associated \gis{}) is in
the key of E minor, the final chord is an E \emph{major} seventh, a kind of
Picardy third; in this way the bridge, like the A sections, can end on a major
chord.

The final A section of “All the Things You Are” is an expanded version of the
first, now ending in the overall tonic of \Aflat. It begins, after a linking
\h{C7s5} chord, by outlining a series of $t_3$ operations that lead to
\h{Dbmaj7}.\fn{ds-20} At this point we might expect the $t_3$s to continue,
leading to \h{Gm7b5}--\h{C7}--\h{Fm} (the tonic of the cycle), but instead we
see \h{Gb7}--\h{Cm7}--\h{Bo7}. This progression is clearly not diatonic, so we
cannot say much about it at this point. It is interesting to note, however,
that the harmony four bars from the end is a \h{Bbm7}, which is exactly where the
chain of $t_3$s would have arrived, had it continued (see Figure
\ref{ds:all-things-last-a}). After this phrase expansion, the piece closes
with a \tfo\ in the key of \Aflat.

\begin{figure}[htbp]
  \centerGraphic{eps/ch1/all-things-last-a.pdf}
  \captionsetup{format=hang}
  \caption[Two versions of “All the Things You Are,” final nine bars.]{The
  final nine bars of “All the Things You Are.” \\
  a\rightparen\ The changes as written. \\
  b\rightparen\ A hypothetical version that continues the $t_3$ cycle in \mbox{F
    minor/}\Aflat\ major.}
  \label{ds:all-things-last-a}
\end{figure}

Again, we might pause to ask what is gained by hearing “All the Things You
Are” in diatonic, rather than chromatic, space. After all, there is never a
clear statement of a diatonic cycle in the manner of “Autumn Leaves” or “Alice
in Wonderland,” and music in chromatic space (as we will begin to see in the
next chapter) still tends to descend by fifth. Without the guiding influence
of F minor, though, the succession of chord qualities at the beginning is
difficult to make sense of: two minor chords, followed by a dominant seventh,
then two major chords, all seemingly unrelated to the phrase-concluding tonic
C major.\fn{ds-21} Understanding the first 8 bars as diatonic, ending with a
tonicized half cadence, makes sense of the chord qualities, and eliminates the
difficult-\hspace{0pt}to-explain third relation \h{AbM}--\h{CM} that results
from a desire to hear both \mbox{V--I} progressions in the A section as
tonic-defining.\fn{ds-22}

Hearing “All the Things You Are” diatonically allows us to listen to sections
of music at once: the first eight bars are in F minor, the next eight in C
minor, the bridge in E minor, and the last twelve return to F minor before
shifting to the relative major, \Aflat, for the final cadence. When we hear
the tune as a chain of $t_3$ operations in shifting diatonic spaces, our
attention is drawn to the connections between the spaces---key
areas---themselves, rather than the (comparatively boring) series of
descending diatonic fifths that occur within them.  While transformational
analyses are often accused of privileging chord-to-chord connections to the
detriment of long-range hearing, in this case the \gis{} framework developed
above allows us to hear over longer distances where chord-to-chord connections
may fail to do so (while, crucially, still recognizing the importance of
chord-to-chord connections for a jazz musician aiming to “make the changes”).

% xxx This will need to be rewritten once everything else is done

Though they do appear occasionally, most jazz tunes do not contain diatonic
cycles, and thus we will need to expand the transformational approach
introduced here to account for a larger portion of jazz practice. Chapter 2
will outline an approach to chromatic space that will help to understand the
\tfo\ progressions that we passed over in our discussion of diatonic space.
Transformational approaches, and neo-Riemannian theories in particular, have
flourished partly because of their ability to explain non-functional
progressions that contain primarily root motion by thirds. Chapter 3 will
draw upon this literature to approach jazz, especially common after
bebop, that is more dependent on thirds than fifths for structure. Though
harmony is crucially important to performing jazz musicians, much of the jazz
pedagogical literature equates chords with scales: a \h{Dm7} chord is
functionally equivalent to a D dorian scale, for example. Chapter 4 will
develop a transformational approach for these “chord-scales,” treating scales
as first-class harmonic objects. Finally, Chapter 5 will bring the theoretical
work of the early chapters to a close, by taking a close analytic look at
tunes based on George Gershwin’s “I Got Rhythm.”



% }}}}

%}}}

%%% Local Variables:
%%% TeX-master: "diss"
%%% End:
% vim:fdm=marker
