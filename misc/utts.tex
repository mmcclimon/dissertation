\documentclass[12pt]{article}
\usepackage{mjmdiss}
\newcommand{\blank}{\vspace{1em}}

\begin{document}
\singlespacing
\pagestyle{empty}

I've been thinking about your work on UTTs, specifically where you talk about
applying them to objects of multiple types (chapter 6 of your dissertation).
Obviously this is relevant to my interests, and I have some (fairly vague)
thoughts I wanted to ask about before I dove in on them.


I started thinking about this because my transformation definitions are a bit
clunky, and it seems like there could be a more elegant solution. For example,
the definition of TF reads like this:

\blank
    $\mathrm{TF}(X\,) = Y\text{, where}\ X = (x_r, x_t, x_s) \in \Smin
    \text{ and}\
    Y = (\,y_r, y_t, y_s) = (x_r + 5, x_s - 1, x_t) \in \Sdom$
\blank

\noindent All of the transformations are cross-type, of course, and
defined as single voice-leadings. Often they're defined in such a way that
the third of the second chord is calculated from the seventh of the first;
I like this (because it means that they can function as ii--V and V--I),
but it makes for clunky definitions. Defining transformations as
element-wise addition doesn't work, for example.

\blank

At first glance, it seems like your cross-type UTTs would be a good solution
to this. Apply a type index so that \Smin = $1$, \Sdom = $2$, and \Smaj = $3$,
and then let $\sigma$ be the permutation
$1 \rightarrow 2 \rightarrow 3 \rightarrow 1$. This seems great, except that
TF (for example) isn't defined everywhere. The ``UTT'' for TF then looks
something like $\langle \sigma; 5, 5, -\rangle$, where ``$-$'' means the function
is undefined. One of the problems I'm having is that not all of the
transformations work on this permutation; some work on the opposite,
$\sigma^{-1} = 3 \rightarrow 2 \rightarrow 1 \rightarrow 3$. The main
transformations from chapter 2 look like this:

\begin{align*}
    \text{TF}   &= \langle \sigma; 5, 5, - \rangle \\
    \text{TF}_\mathrm{T} &= \langle \sigma; 11, 11, - \rangle \\
    \text{\textsc{slide}}_7 &= \langle \sigma; -, -, 1 \rangle \\
    %
    \text{7\textsc{th}} &= \langle \sigma^{-1}; -, -, 0 \rangle \\
    \text{3\textsc{rd}} &= \langle \sigma^{-1}; -, 0, - \rangle \\
\end{align*}

This solution seems like a nice way to define the transformations more
succinctly, but it seems like that might be the only benefit. Because the
transformations aren't defined everywhere, it seems like much of the
``uniformity'' of the UTT is lost---the transformations do not obviously form
a group, for example. Unlike your permutational analysis of the
omnibus (which I like very much!), you can only apply TF twice before you run
out of room, so to speak. It also makes it more difficult to define additional
sets (as I do at the end of chapter 2). Including minor-major seventh chords,
for example, means adding another permutation to the mix.

Anyway, I was mostly curious as to your general thoughts on this use of UTTs,
or if anyone has done something similar, and whether you think is something
worth pursuing. It's not particularly urgent (just something I've had in the
back of my mind), and I think ultimately it wouldn't affect the course of the
dissertation all that much. (It is interesting, certainly, but wouldn't change
anything I already have planned; if anything, it would be a sort of aside.)

\blank

\noindent Thanks! \\
\noindent Michael

\end{document}


%%% Local Variables:
%%% mode: latex
%%% TeX-master: t
%%% End:
