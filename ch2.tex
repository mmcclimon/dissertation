%% Chapter 2

\chapter{Fifths Spaces}

\blankfootnote{Earlier versions of this chapter were presented at
  the annual meetings of the Music Theory Society of the Mid-Atlantic
  (Philadelphia, PA, March 2013) and the Society for Music Theory (Milwaukee,
  WI, November 2014). I am grateful for the many helpful comments and questions
  I received at these conferences (especially those from Stefan Love, Brian
  Moseley, and Keith Waters).}%
%
Because most jazz is not purely diatonic, we need to expand our
transformational system to account for more chromatic examples. This chapter
will begin that work, taking the very common \tfo\ progression as its basis.
Most of the spaces we will develop in this chapter will fall under the broad
category of “fifths spaces,” but at the end of the chapter we will have
occasion to return to the diatonic space of chapter 1 to see how they might be
enhanced with the chromatic spaces introduced here.

% xxx this title is stupid
\section{\tf\ Space: An Initial Pass}
%{{{

\subsection{Formalism}
%{{{

The most common harmonic progression in jazz is undoubtedly the
\mbox{ii\tsup{7}--V\tsup{7}--I\tsup{7}} progression (hereafter, just \tfo, or
often just \tf). It is the first progression taught in most jazz method books,
and the only small-scale harmonic progression to have an entire Aebersold
play-along volume dedicated to it.\fn{tf-1} The progression is so prevalent
that many jazz musicians describe tunes in terms of their constituent \tf{}s;
a musician might describe the bridge of “All the Things You Are” (shown in
Example \ref{tf:all-things-bridge}) as being “\tf\ to G, \tf\ to E, then V--I
in F.”\fn{tf-2} Given the importance of the progression for improvising jazz
musicians, it seems natural to use the progression as the basis for developing
a more general transformational model of jazz harmony.

\exBeg[tbp]
  \centerGraphic[width=20em]{eps/ch2/all-things-bridge.pdf}
  \caption{The bridge of “All the Things You Are” (Jerome Kern).}
  \label{tf:all-things-bridge}
\exEnd

\figBeg[tbp]
  \centerGraphic[width=15em]{eps/ch2/single-tfo-network.pdf}
  \caption[A transformation network for a single \tfo\ progression.]{A
  transformation network for a \tfo\ in C major: \caph{Dm7}--\caph{G7}--\caph{CM7}.}
  \label{tf:single-tfo-network}
\figEnd

Figure \ref{tf:single-tfo-network} shows a transformation network for a single
\tfo\ progression; we will begin by developing the formal apparatus for this
progression, after which we can begin to combine \tfo\ progressions to form a
larger musical space.\fn{tf-2a} The illustration, with its combination of
general Roman numerals and specific key centers, is designed to reflect how
jazz musicians tend to talk about harmony; we might read this network as “a
\tfo\ in C.” This combination of Roman numerals and key areas bears some
similarity to Fred Lerdahl’s chordal-regional space, but Figure
\ref{tf:single-tfo-network} is a transformation network, while
chordal-regional space is strictly a spatial
metaphor.\footcite[96--97]{lerdahl:2004}

We have encountered one transformation network already (in Figure
\ref{ds:autumn-leaves-bridge}), but we have yet to define the concept
formally.  Transformation networks are a major part of David Lewin’s project
in \emph{GMIT}, and have been thoroughly covered in the literature, so we will
not rehearse the formalism at any length here.\fn{tf-3} Briefly, a
transformation network consists of objects of some kind (here, they are
chords), represented as vertices in a graph, along with some relations between
them, represented as arrows. Lewin defines a transformation as “a function
from a family [set] S into S itself.”\fn{tf-4} The transformations in Figure
\ref{ds:autumn-leaves-bridge} are indeed Lewinnian transformations (mappings
in the G-minor diatonic set), but the \tfo\ transformation network is more
complex.

The transformation TF in Figure \ref{tf:single-tfo-network} is in fact a
\emph{cross-type} transformation, as defined by Julian
Hook.\footcite{hook:2007} Hook expands Lewin’s definition of a transformation
network to include objects of different types, necessary to define
transformations in the \tfo\ progression. The progression contains musical
objects of three types of diatonic seventh chords: minor, dominant,
and major sevenths (in the key of C major, the progression is
\h{Dm7}--\h{G7}--\h{CM7}). Using Hook’s relaxed definitions, we are free to
define transformations from any set of objects to any other; to understand the
figure above, we need to define the transformation TF such that it maps
ii\tsup{7} chords to V\tsup{7} chords, and V\tsup{7} chords to I\tsup{7}
chords.

\figBeg[htbp]
  \centerGraphic[width=15em]{eps/ch2/single-tfo-graph.pdf}
  \caption{The underlying transformation graph for a single \tfo\ progression.}
  \label{tf:single-tfo-graph}
\figEnd

Before defining the transformations, however, we first need to define the sets
themselves. To help with this, Figure \ref{tf:single-tfo-graph} shows the
underlying transformation graph of the transformation network in Figure
\ref{tf:single-tfo-network}. Throughout \emph{GMIT}, Lewin is careful to
distinguish transformation graphs from transformation networks: a graph is an
abstract structure, showing only relations (transformations) between
unspecified set members, while a network realizes a graph, specifying the
actual musical objects under consideration.\fn{tf-5} Because cross-type
transformation graphs contain objects of different types, a node in a
cross-type transformation graph must be labeled with the set from which the
node contents may be drawn (even in the abstract transformation
graph).\footcite[7]{hook:2007} In Figure \ref{tf:single-tfo-graph}, the nodes
are labeled simply $S_1$, $S_2$, and $S_3$, which we can understand as the
sets of ii\tsup{7}, V\tsup{7}, and I\tsup{7} chords, respectively.

While at its core the \tfo\ progression contains three types of seventh
chords, in reality a jazz musician might add any number of extensions or
alterations to this basic structure. Given this practice, defining the
archetypal progression as being composed of four-note set classes seems
unnecessarily restrictive. In order to allow for some freedom in the chord
qualities, we will consider only chordal roots, thirds, and sevenths; these
pitches are sufficient to distinguish the three chord qualities in a \tfo.

In this chapter, we will represent a chord with an ordered triple $X = (x_1,
x_2, x_3)$, where $x_1$ is the root of the chord, $x_2$ the third, and $x_3$
the seventh. The definitions of the three sets are as follows:\fn{tf-6}

% xxx do these definitions need numbers?
\vspace{-3em} % xxx not sure why there’s so much space here
\begin{align*}
  S_1 &= \{ (x_1, x_2, x_3)\ |\ x_2 - x_1 = 3; x_3 - x_1 = 10 \} &
    \text{ii\tsup{7} chords} \\ %
  %
  S_2 &= \{ (x_1, x_2, x_3)\ |\ x_2 - x_1 = 4; x_3 - x_1 = 10 \} &
    \text{V\tsup{7} chords} \\ %
  %
  S_3 &= \{ (x_1, x_2, x_3)\ |\ x_2 - x_1 = 4; x_3 - x_1 = 11 \} &
    \text{I\tsup{7} chords}
\end{align*}
\vspace{-3em}

\noindent The definitions are intuitive, and have clear musical relevance:
ii\tsup{7} chords have a minor third and minor seventh, V\tsup{7} chords have
a major third and minor seventh, and I\tsup{7} chords have a major third and
major seventh. Defining the chords this way rather than as four-note set
classes offers the great advantage of flexibility. Using the ordered-triple
representation, the progressions \h{Dm7}--\h{G7}--\h{Cmaj7} and
\h{Dm9(b5)}--\h{G7b13s9b9}--\h{Cmaj7s11} are understood as equivalent, since
the roots, thirds, and sevenths are the same: both progressions are
represented $(2, 5, 0)$--\allowbreak$(7, 11, 5)$--\allowbreak$(0, 4,
11)$. Because the sets are defined in pitch-class space, the three sets all
have cardinality $12$: each pitch class is the root of exactly one ii\tsup{7},
V\tsup{7}, and I\tsup{7} chord.

With the space of the nodes defined, we can now formulate the transformation
representing a \tf, which we will call simply ``TF'':

\vspace{-3em} % xxx not sure why there’s so much space here
\begin{alignat*}{3}
    \mathrm{TF}(X\text{\thinspace}) = Y\text{, where}\ X = (x_1, x_2, x_3) \in S_1
    & \text{ and}\ &
    Y &= (\text{\thinspace}y_1, y_2, y_3) &= (x_1 + 5, x_3 - 1, x_2) \in S_2 \\
    %
    \mathrm{TF}(Y\text{\thinspace}) = Z\text{, where}\ Y =
    (\text{\thinspace}y_1, y_2, y_3) \in S_2
    & \text{ and} &
    Z &= (z_1, z_2, z_3) &= (\text{\thinspace}y_1 + 5, y_3 - 1, y_2) \in
    S_3
\end{alignat*}
\vspace{-3em}

\noindent Again, these definitions are designed to be musically relevant; the
voice-leading diagram in Example \ref{tf:tfo-voice-leading} illustrates this
more clearly.\fn{tf-6a} The root of the second chord is a fifth below the root of the
first ($y_1 = x_1 + 5$), the third of the second chord is a semitone below the
seventh of the first ($y_2 = x_3 - 1$), and the seventh of the second chord is
a common tone with the third of the first ($y_3 = x_2$). In Lewin's
transformational language, if a jazz musician is ``at a ii\tsup{7} chord'' and
wishes to ``get to a V\tsup{7} chord,'' the transformation that will do the
best job is TF: ``move the root down a fifth and the seventh down a semitone
to become the new third.'' (When we wish to emphasize this mapping
aspect of the transformation, we may also write ii\tsup{7}
\TFarrow\ V\tsup{7}, rather than TF(ii\tsup{7}) = V\tsup{7}.)
Note that the transformation TF is also valid
between V\tsup{7} and I\tsup{7} as well (the second equation above,
involving sets $S_2$ and $S_3$). TF is both one-to-one and onto for sets of
ordered triples; it maps each ii\tsup{7} to a unique V\tsup{7}, and each
V\tsup{7} to a unique I\tsup{7}. As such, its inverse ($\mathrm{TF}^{-1}$) is
well defined, and allows motion backwards along the arrows shown in the
transformation graph in Figure \ref{tf:single-tfo-graph}.

\exBeg[htbp]
  \centerGraphic{eps/ch2/tfo-voice-leading.pdf}
  \caption{Voice leading in the \tfo\ progression.}
  \label{tf:tfo-voice-leading}
\exEnd

It is worth mentioning here that TF and \tft\ (which we will define in the
next section) are well-defined operations for any ordered triple of members of
the integers mod-12 (i.e., a member of the set $\intZ_{12} \times \intZ_{12}
\times \intZ_{12}$). There is nothing mathematically incorrect
about the statement $(0, 1, 2) \TFarrow (5, 1, 1) \TFarrow (10, 0, 1)$, for
example, but this succession has little musical relevance for the applications
under consideration here. Because Hook does not formally define what he means
by a ``type,'' the formulation allows for situations like this one, in which
the three types are all members of a single larger set.\fn{tf-7} The advantage
for defining TF as a cross-type transformation is that the content of a single
node in the transformation graph is restricted to members of a 12-element set
of specific ordered-triple configurations.

% xxx this should be a combination graphic: graph + network
\figBeg[thbp]
  \centerGraphic{eps/ch2/trans-graph-large.pdf}
  \caption[A transformation graph and network for a small portion of \tf\
    space]{A transformation graph (left) and transformation network (right)
    for a small portion of \tf\ space.}
  \label{tf:trans-graph-large}
\figEnd

With this understanding of the transformations involved in a single \tfo\
progression, we can continue to see how we might connect multiple progressions
in order to form a complete \tf\ space. Because root motion by descending
fifth is extremely common in jazz, we might consider connecting \tfo\
progressions by descending fifth as well; Figure \ref{tf:trans-graph-large}
illustrates this arrangement both as a transformation graph and a
transformation network.  This descending fifths arrangement means that all of
the chords sharing a root are aligned vertically (directly below \h{GM7} is
\h{G7}, which is itself above \h{Gm7}). This arrangement allows us to define
two more transformations, which we will call simply \textsc{7th} and
\textsc{3rd}:

\vspace{0.5\baselineskip}
\h{7}\textsc{th}$(L) = M$, where $L = (l_1, l_2, l_3) \in S_3$ and $M=(m_1, m_2,
m_3)=(l_1, l_2, l_3-1) \in S_2$

\h{3}\textsc{rd}$(M) = N$, where $M = (m_1, m_2, m_3) \in S_2$ and $N=(n_1, n_2,
n_3)=(m_1, m_2-1, m_3) \in S_1$
\vspace{0.5\baselineskip}

\noindent Like the TF transformation, the \textsc{7th} and \textsc{3rd}
transformations have clear musical relevance: each lowers the given note by a
semitone. Although adjacent progressions are connected by descending fifth,
the $T_5$ labels connecting adjacent ii\tsup{7} chords and I\tsup{7} chords
are shown in gray in the graph (and omitted in the network, and most later
examples), since these chords are not often directly connected in jazz.

\figBeg[thb]
  \centerGraphic{eps/ch2/tf-circle-fifths.pdf}
  \caption{The complete \tf\ space, arranged around the circle of fifths.}
  \label{tf:tf-circle-fifths}
\figEnd

By extending the graph of Figure \ref{tf:trans-graph-large} and including the
node contents (thereby changing it from a tranformation \emph{graph} to a
transformation \emph{network}), we arrive at the entirety of \tf\ space, as
shown in Figure \ref{tf:tf-circle-fifths}. Because \tf\ space includes
cross-type transformations, it does not easily form a Lewinnian \gis.\fn{tf-8}
Considered more generally, though, it is easy to see that by considering a
single \tfo\ progression as a unit, \tf\ space maps cleanly onto ordinary
pitch-class space.  As Figure \ref{tf:trans-graph-large} makes clear, we can
consider the \tfo\ in C as being three perfect fifths above the \tfo\ in
\Eflat\ (or put transformationally, the $T_3$ operation transforms a \tfo\ in
C to one in \Eflat). This formulation does not allow us a means to say, for
example, that “the ii chord in C is $x$ units away from the V chord in
\Eflat,” but because \tfo{}s are rarely split up, falling back on normal
pitch-class distance is sufficient in most situations.\fn{tf-9}

% xxx I could go into ii-V space considered as a graph here...I don’t think
% it’s really all that necessary?


%}}}

\FloatBarrier

\subsection{Analytical Interlude: Lee Morgan, “Ceora”}
%{{{

% xxx This whole section should probably be rewritten

Though we will return to the formalism a bit later, we have defined enough of
\tf\ space at this point to see how it might be useful in analysis. To do so,
we will look at Lee Morgan's composition ``Ceora,'' first recorded on the
album \emph{Cornbread} from 1965. The changes for the A section are given in
Example \ref{tf:ceora-changes-1}, and the accompanying moves in \tf\ space are
shown in Figure \ref{tf:ceora-space-1}.\fn{tf-10} ``Ceora'' is in the key of
\Aflat\ major, and begins with the progression I--ii--V--I in the first three
bars, staying within a single horizontal slice of \tf\ space. This is followed
immediately by a \tfo\ progression in \Dflat, a fifth lower (mm.~4--5).

\exBeg[thbp]
  \centerGraphic[width=24em]{eps/ch2/ceora-changes-1.pdf}
  \caption{Changes for the A section of ``Ceora'' (Lee Morgan).}
  \label{tf:ceora-changes-1}
\exEnd

\figBeg[htbp]
  \centerGraphic{eps/ch2/ceora-space-1.pdf}
  \caption{The A section of ``Ceora'' in \tf\ space.}
  \label{tf:ceora-space-1}
\figEnd

At this point, we might expect the \tf{}s to continue in descending fifths,
but the potential ii\tsup{7} chord in \Gflat\ is substituted with \h{Dm7}, its
tritone substitute, which then resolves as a \tf\ in C.\fn{tf-11} Instead of
resolving to C major in m.~7, this \tf\ resolves instead to C \emph{minor}:
both the seventh and third of the expected \h{CM7} are lowered to become
\h{Cm7}. (This progression is extremely common, and is one of the principal
means of maintaining harmonic motion in the course of a jazz tune.)

A similar progression in \Bflat\ follows, leading to a \h{Bbm7} chord in m.~9.
We then hear a \tf\ progression in the tonic in mm.~9--10, but the
expected \h{AbM7} does not materialize; the \h{Eb7} chord moves instead to
\h{Cm7} as ii of \Bflat\ (a northwesterly move in the space). This
progression repeats in mm.~11--13, leading once again to the \h{Dm7} chord
first heard in m.~5. The repeated upward motions in the space have the
effect of ramping up the tonal tension in the passage; not only do the
dominants fail to resolve as expected, but their stepwise rising motion take
the music far away from the tonic \Aflat. To release this harmonic tension,
the \tf\ in C resolves at m.~15 to C minor, at which point the harmonic rhythm
doubles and the progression follows the normal descending fifths pattern to
reach the tonic that begins the B section in m.~17.

The B section of ``Ceora'' (shown in Example \ref{tf:ceora-changes-2}) follows
much of the same trajectory as the A section until the last four bars; the
only differences are the addition of the \h{b5} in the \h{Cm7} and the \h{s9}
in the \h{F7} in mm.~11--12 of the section. Because we have defined chords and
transformations only in terms of chordal roots, thirds, and sevenths, neither
of these changes affect our transformational reading of the passage. Instead
of ramping up to ii\tsup{7} of C as in the A section, the \tf\ in \Bflat\
resolves to \h{Bbm7} in m.~12. This \h{Bbm7} becomes the ii chord of a \tfo\
in tonic, resolving in m.~15 of the section. A final \tf\ in the last measure
provides additional harmonic interest, and functions as a turnaround to lead
smoothly back to \h{AbM7} to begin the next chorus.

\exBeg[htbp]
  \centerGraphic[width=24em]{eps/ch2/ceora-changes-2.pdf}
  \caption{Changes for the B section of ``Ceora'' (Lee Morgan).}
  \label{tf:ceora-changes-2}
\exEnd

At this point, we have successfully mapped all of the chords in ``Ceora'' to
their associated locations in \tf\ space; it is reasonable to ask, though,
whether this mapping of chords to space locations should even count as
``analysis.'' After all, \tf\ space contains each minor, dominant, and major
seventh chord exactly once, so we did not even need to make any decisions as
to where in the space a particular chord should go. Have we, in fact, learned
anything about ``Ceora'' from our exploration of \tf\ space?

The answer, I think, is yes. Although we could criticize \tf\ space for being
simply a particular arrangement of common harmonic progressions in jazz,
similar arrangements have proven themselves useful in many areas of music
theory: the circle of fifths, the Neo-Riemannian Tonnetz, the pitch-class
``clock face,'' and countless others.\fn{tf-12} One of the benefits of \tf\
space is that it allows us to easily visualize harmonic motions in jazz. The
succession of chord symbols that make up the changes to ``Ceora'' may make
immediate sense to an experienced jazz musician, but \tf\ space allows others
to to make sense of these relationships more clearly. The fact that our
analysis in \tf\ space may seem obvious is in fact a feature, not a bug; such
a criticism reveals that \tf\ space, with all its mathematical formalism,
can clarify information that may otherwise remain hidden in the raw data of
the chord symbols.\fn{tf-13}

%}}}

%}}}


\section{\tf\ Space: Tritone Substitutions}
%{{{

\subsection{Formalism}

There is an important aspect of jazz harmony that has not yet been considered in
our discussion of \tf\ space. Crucial to harmony beginning in the bebop era is
the tritone substitution: substituting a dominant seventh chord for the
dominant seventh whose root is a tritone away.\fn{tft-1} Because
tritone-substituted dominants are functionally equivalent, both the
progressions \h{Dm7}--\h{G7}--\h{Cmaj7} and \h{Dm7}--\h{Db7}--\h{Cmaj7} may be
analyzed as \tfo\ progressions in the key of C.

This functional equivalence means that a tritone-subsituted dominant can
function as a shortcut to an otherwise distant portion of \tf\ space. In the
circle-of-fifths arrangement of Figure \ref{tf:tf-circle-fifths}, keys related
by tritone are maximally far apart (diametrically opposed on the circle), but
in jazz practice, \h{G7} and \h{Db7} are functionally identical (both
dominant-function chords in C major). To account for this progression in our
space, we need to somehow bring these chords closer together; one solution is
to connect two segments of the space by $T_6$ in a sort of ``third
dimension,'' as shown in Figure \ref{tft:complete-space}. The topology of this
space is more complicated than the ordinary circle of fifths, however. Once a
progression reaches the bottom of the ``front'' side of the figure, it
reappears at the top of the ``back'' side (\Gflat\ at the bottom is listed
again as \h{Fs} at the top); likewise, progressions disappearing off the
bottom of the back side reappear at the top of the front side (C major is
given in both locations).

\figBeg[htbp]
  \centerGraphic{eps/xxx.pdf}
  \caption{The complete \tf\ space, showing tritone substitutions.}
  \label{tft:complete-space}
\figEnd


 % xxx do I need more on topology here??

This arrangement of key centers is topologically equivalent to a Möbius strip,
which is somewhat easier to see by focusing only on the dominant seventh
chords, as shown in Figure \ref{tft:mobius-dominants}. By wrapping this figure
into a circle and gluing the ends together with a half-twist (so that \h{C7}
and \h{Gb7} match up), we arrive at the desired Möbius strip.\fn{tft-2} Though
the underlying topology is easier to visualize this way, it is difficult to
include all of the other progressions (the \tf{}s themselves) in this diagram,
so we will continue to use the ``three-dimensional'' version of Figure
\ref{tft:complete-space}, with the understanding that this topology remains in
effect. In any case, the arrangement of keys into the front and back sides is
arbitrary, and may be repositioned as necessary; it is often convenient to
have the tonic key (when there is one) centrally located on the front of the
space.

\figBeg[htbp]
  \centerGraphic{eps/xxx.pdf}
  \caption{The Möbius strip at the center of \tf\ space.}
  \label{tft:mobius-dominants}
\figEnd

While we could navigate this space using only the transformations TF and
$T_6$, it is convenient to define another transformation to help with a common
progression like \h{Dm7}--\h{Db7}--\h{Cmaj7}. We will call this
transformation \tft, to highlight its relationship to the more normative TF:

\vspace{-3em} % xxx not sure why there’s so much space here
\begin{alignat*}{3}
    \mathrm{TF}_\mathrm{T}(X\text{\thinspace}) = Y\text{, where}\ X =
        (x_1, x_2, x_3) \in S_1
    & \text{ and}\ &
    Y &= (\text{\thinspace}y_1, y_2, y_3) &= (x_1 - 1, x_3 + 5, x_2 + 6) \in S_2 \\
    %
    \mathrm{TF}_\mathrm{T}(Y\text{\thinspace}) = Z\text{, where}\ Y =
        (\text{\thinspace}y_1, y_2, y_3) \in S_2
    & \text{ and} &
    Z &= (z_1, z_2, z_3) &= (\text{\thinspace}y_1 - 1, y_3 + 5, y_2 + 6) \in
    S_3
\end{alignat*}
\vspace{-3em}

\noindent The \tft\ transformation represents a tritone substitution, but it
transforms bass motion by fifth into bass motion by semitone (\emph{not} bass
motion by tritone); the voice-leading diagram in Figure
\ref{tft:voice-leading-tft} clarifies this. Because TF and $T_6$ commute,
\tft\ can be considered as either TF followed by $T_6$, or vice versa. With
this new transformation, we can easily understand the progression
\h{Abm7}--\h{Db7}--\h{CM7} as a substituted \tfo\ in C: \h{Abm7} \TFarrow\
\h{Db7} \TFTarrow\ \h{CM7}.

\figBeg[htbp]
  \centerGraphic{eps/xxx.pdf}
  \caption{Voice leading in the TF and \tft\ transformations, compared.}
  \label{tft:voice-leading-tft}
\figEnd

% here, text about slide7s and whatnot


%}}}

%%% Local Variables:
%%% TeX-master: "diss"
%%% End:
% vim:fdm=marker
