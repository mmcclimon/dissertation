%% Chapter 3 notes

\fntext{mts-1}{%
  Turning Figure \ref{mts:m3-torus} into a torus involves gluing the
  top and bottom edges together and the left and right edges together; the
  dotted line representing perfect fifths then wraps around the surface of the
  torus in a continuous line (as though you had wrapped a barber's pole into a
  doughnut).}

\fntext{mts-2}{%
  In fact, the graphs of the note-based Tonnetz and the m3-torus here are
  isomorphic. This fact introduces some tantalizing possibilities, but none turn
  out to be terribly interesting since, as Richard Cohn has shown, the consonant
  triad is unique among trichords in its capability for parsimonious voice
  leading; \headlesscite[1--7]{cohn:1997}. As such, the m3-torus---made of $(026)$
  trichords---does not show common-tone relations, and the dual graph of the
  m3-torus gives vertices of set class $(0134678\mathrm{T})$.}

\fntext{mts-3}{%
  \cite[246--48]{douthettsteinbach:1998}. The chicken-wire torus is the dual
  graph of the more common note-based Tonnetz; both are shown in
  \cite[Figure 1]{tymoczko:2012}.}

\fntext{mts-4}{%
  The definition of \Smin $\xrightarrow{\mathrm{BD}}$ \Sdom is not given, since
  it is relatively rare in jazz. It models a progression like iv$^7$--\V,
  which is much more common in classical music.}

\fntext{mts-5}{%
  Keith Waters has called this period ``jazz's second practice'' in
  \headlesscite{waters:2013}. Waters, along with J.~Kent Williams, has explored
  jazz harmony after 1960 using the Tonnetz and hyper-hexatonic systems familiar
  from classical theory; see \headlesscite{waterswilliams:2010}.}

\fntext{mts-6}{%
  ``Isotope'' was first recorded on Henderson's album \emph{Inner Urge}, released in
  1965.\nocite{henderson:isotope}}

\fntext{mts-7}{%
  ``Isotope'' uses a slightly different set of changes for the solos than it
  does for the head (often referred to simply as the ``solo changes'' and
  ``head changes''). This is often the case when the head changes are complex,
  fast-moving, or contain unusual extensions to account for specific melody
  notes. These changes are taken from \emph{The Real Book}; the C chord in
  m.~7 is played as either a major or a dominant seventh on the \emph{Inner Urge}
  recording, so analyzing it as \h{C7} here seems reasonable.}


%%% Local Variables: %%%
%%% mode: latex %%%
%%% TeX-master: "../diss" %%%
%%% End: %%%
