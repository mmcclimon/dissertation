%% Chapter 3 notes

\fntext{mts-1}{%
  Turning Figure \ref{mts:m3-torus} into a torus involves gluing the
  top and bottom edges together and the left and right edges together; the
  dotted line representing perfect fifths then wraps around the surface of the
  torus in a continuous line (as though you had wrapped a barber's pole into a
  doughnut).}

\fntext{mts-2}{%
  In fact, the graphs of the note-based Tonnetz and the m3-torus here are
  isomorphic. This fact introduces some tantalizing possibilities, but none turn
  out to be terribly interesting since, as Richard Cohn has shown, the consonant
  triad is unique among trichords in its capability for parsimonious voice
  leading; \headlesscite[1--7]{cohn:1997}. As such, the m3-torus---made of $(026)$
  trichords---does not show common-tone relations, and the dual graph of the
  m3-torus gives vertices of set class $(0134678\mathrm{T})$.}

\fntext{mts-3}{%
  \cite[246--48]{douthettsteinbach:1998}. The chicken-wire torus is the dual
  graph of the more common note-based Tonnetz; both are shown in
  \cite[Figure 1]{tymoczko:2012}.}

\fntext{mts-4}{%
  The definition of \Smin $\xrightarrow{\mathrm{BD}}$ \Sdom is not given, since
  it is relatively rare in jazz. It models a progression like iv$^7$--\V,
  which is much more common in classical music.}

\fntext{mts-5}{%
  Keith Waters has called this period ``jazz's second practice'' in
  \headlesscite{waters:2013}. Waters, along with J.~Kent Williams, has explored
  jazz harmony after 1960 using the Tonnetz and hyper-hexatonic systems familiar
  from classical theory; see \headlesscite{waterswilliams:2010}.}

\fntext{mts-6}{%
  ``Isotope'' was first recorded on Henderson's album \emph{Inner Urge}, released in
  1965.\nocite{henderson:isotope}}

\fntext{mts-7}{%
  ``Isotope'' uses a slightly different set of changes for the solos than it
  does for the head (often referred to simply as the ``solo changes'' and
  ``head changes''). This is often the case when the head changes are complex,
  fast-moving, or contain unusual extensions to account for specific melody
  notes. These changes are taken from \emph{The Real Book}; the C chord in
  m.~7 is played as either a major or a dominant seventh on the \emph{Inner Urge}
  recording, so analyzing it as \h{C7} here seems reasonable.}

\fntext{mts-8}{%
  The word ``prolongational'' in this sentence is admittedly problematic. By
  using it here I mean only that at some deeper level the turnaround is
  harmonically superfluous, as it occurs after the main tonal conclusion of
  the chorus (an observation usually confirmed by the fact that the turnaround
  is omitted in the last head). I do \emph{not} mean to imply that the
  turnaround in the last two bars of ``Isotope'' is harmonically
  uninteresting; indeed, it is the most distinctive feature of the piece.}

\fntext{mts-9}{%
  Exactly how far apart \h{Bb7} and \h{C7} depends on how one chooses to
  measure voice-leading distance, and whether we consider major-minor sevenths
  in the usual way, as four-note chords, or in the way we have been doing so
  here, as ordered triples of root, third, and seventh. In Jack Douthett's
  Four-Cube trio, for example, \h{Bb7} and \h{C7} are maximally far apart---4
  semitones; see \cite[157--58]{cohn:2012}.}

\fntext{mts-10}{%
  The notes in these four dominant sevenths form an octatonic collection, and
  have been studied fairly extensively in the literature. See, for example,
  \cite[152--58]{cohn:2012}; \cite[245--46]{douthettsteinbach:1998}; and
  \cite[371]{tymoczko:2011}.}

\fntext{mts-11}{%
  The most obvious differences in mm.~5--12 of the tune are the ``slash chords''
  in mm.~8--9. The chord symbol \h{Em7/A} indicates an E minor seventh chord
  played with an A in the bass; the resulting sound is an \h{A7} chord with a
  suspended fourth (D replaces C\sharp). The older \emph{Real Book} gives the
  same change as \h{A7}sus\h{4}. The only other slight alteration is the addition of
  the ii chord, \h{Dm7}, in m.~10.}


% Major-third spaces
\fntext{maj3-1}{%
  % xxx need these citations
  The examples are too numerous to list here, but see, for example, [citations
  go here].}

\fntext{maj3-2}{%
  Waters's work in particular has focused extensively on this music, though he
  is hardly alone. See, for example, \cite{julien:2003}; \cite{strunk:2005};
  \cite{waters:2005}; and \cite{waterswilliams:2010}.}

\fntext{maj3-3}{%
  Many people have discussed ``Giant Steps'' in the literature; the most
  substantial work in this area is Matthew Santa's \headlesscite{santa:2003}.
  Guy Capuzzo compares Santa's analysis to one done by Pat Martino in
  \headlesscite{capuzzo:2006}. See also \cite{demsey:1991}; and
  \cite{goodheart:2001}.\nocite{coltrane:giantsteps}}

\fntext{maj3-4}{%
  Frank Samarotto has suggested to me in connection with an unpublished paper
  of his that ``Giant Steps'' is chromatically coherent, while locally diatonic.
  As such, it represents an example of his ``hypothetical'' Type 4 coherence,
  ``in which areas of diatony occur only in local isolation and in which some
  other (presumably post-tonal) coherence might be in effect.''
  \headlesscite{samarotto:2003}.}

\fntext{maj3-4a}{%
  These changes are from \cite[30]{levine:piano}. ``Infant Eyes'' appears on
  Shorter's album \emph{Speak No Evil} (1964).\nocite{shorter:evil}}

\fntext{maj3-4b}{%
  % xxx need this Jaffe page number
  Authors disagree on whether ``Giant Steps'' is in the key of B or \Eflat:
  David Demsey hears \Eflat\ (\headlesscite[171--72]{demsey:1991}), while Andy
  Jaffe hears B \parentext{\headlesscite[]{jaffe:1996}}. Given the organizing influence
  of the major-third cycle, I am not sure the question is so important; the
  tune uses tonal progressions, but may not be \emph{in} a key.}

\fntext{maj3-5}{%
  Exactly how Coltrane devised this substitution set is difficult to say:
  authors have at various times pointed to classical sources---especially
  Nicolas Slonimsky's \emph{Thesaurus of Scales and Melodic Patterns}---as
  well as the music of Thelonious Monk, Dizzy Gillespie, and Tadd Dameron,
  among others. One source that is nearly always cited is the tune ``Have You
  Met Miss Jones?'' (Richard Rodgers/Lorenz Hart), in which the bridge traces
  a descending major-third cycle much like that of ``Giant Steps.'' For a
  review of these possible origins, see \cite[148--57]{demsey:1991}; and
  \cite[145--47]{porter:1998}.}

\fntext{maj3-6}{%
  Nearly every discussion of Coltrane changes includes the
  ``Countdown''/\,``Tune Up'' pairing. See, for example,
  \cite[159--62]{demsey:1991}; \cite[359--60]{levine:1995}; or many others.
  ``Countdown'' was also recorded on the album \emph{Giant Steps}, and ``Tune
  Up'' can be heard on \emph{Cookin' with the Miles Davis
  Quintet} (1957).\nocite{davis:cookin}}

\fntext{maj3-7}{%
  Demsey provides a list of third-relations in jazz tunes in an appendix to
  \headlesscite[179--80]{demsey:1991}. Coltrane's famous take on ``Body and Soul'' is
  on the album \emph{Coltrane's Sound} (1960), which also features two original tunes
  that make prominent use of the major-third cycle: ``Central Park West'' and
  ``Satellite.''\nocite{coltrane:sound}}

\fntext{maj3-8}{%
  % xxx a sentence here on why we're citing Cohn more than anyone else?
  \cite{cohn:2012}. Hexatonic cycles are discussed primarily in
  Cohn's chapter 2, and Weitzmann regions in chapter 4; chapter 5 combines
  these models into a single system which is then used throughout the rest of
  the book. Lewis Porter mentions Weitzmann's treatise on the augmented triad
  as a possible influence on Coltrane \parentext{\headlesscite[146]{porter:1998}}.}

\fntext{maj3-9}{%
  In this figure ``$+$'' indicates major triads and ``$-$'' indicates minor
  triads. ``Cube Dance'' appears in \cite[254]{douthettsteinbach:1998}, and is one of
  Cohn's primary models of triadic space; see \headlesscite[86--109 and
  following]{cohn:2012}.}

\fntext{maj3-10}{%
  This figure is taken from Cohn's book, and has a small, but important,
  error: the `C' at the 10 o'clock position should be a C\sharp, forming a
  fully-diminished seventh chord. ``4-Cube Trio'' was originally devised by Jack
  Douthett, and is very similar to the ``Power Towers'' graphic in
  \cite[256]{douthettsteinbach:1998} (which omits the French sixth chords). For
  more on its history, see \cite[157n15]{cohn:2012}.}

\fntext{maj3-11}{%
  This is a central thesis of Cohn's chapter 7 \parentext{see especially
  \citetitle[148--58]{cohn:2012}}, and figures prominently in Tymoczko's
  geometric theory \parentext{\headlesscite[97--112]{tymoczko:2011}}.}

\fntext{maj3-12}{%
  My intent here is not to fault these approaches for failing to explain jazz
  harmony, only to demonstrate why they are not sufficient for our purposes
  here. These theories were designed primarily for music of the nineteenth
  century, in which triadic music is the rule and seventh chords the exception.
  In this goal, they generally succeed: many of the unusual harmonic techniques
  in that repertoire \emph{are} well explained by parsimonious voice-leading.
  Jazz, as a music driven by seventh chords, is a different sort of animal,
  and requires a different approach.}

\fntext{maj3-13}{%
  Santa omits the fifth of dominant seventh chords, as we have been doing
  here. The reasons, though, are different: the stated reason is to keep the
  cardinalities of the chords the same, but he does not mention why he chooses
  not to use major seventh chords and complete dominant sevenths, for example.
  Santa notes later that including the fifth involves one of the three missing
  notes from the nonatonic collection, which is acceptable because ``the fourth
  voice is not essential to the voice leading of the cycle.''
  \headlesscite[15]{santa:2003}.}

\fntext{maj3-14}{%
  The major-thirds figure represents a kind of cross-section of the
  minor-thirds torus: to see this clearly, locate the key areas C, \Aflat, and
  E on both Figure \ref{mts:m3-space} and Figure \ref{maj3:maj3-space}.}

\fntext{maj3-15}{%
  In fact, the rest of the figure is unnecessary for ``Giant Steps''; the piece
  is easier to understand using a subgraph of the complete M3-space, that
  contains only the \tfo progressions in B, G, and \Eflat.}

\fntext{maj3-16}{%
  These spaces represent all but one of the equal partitions of the octave:
  while we could easily construct a ``whole-tone space,'' it would not have
  very many applications to tonal jazz. In cases where a whole-tone space
  might be useful, it is usually not problematic to consider a whole-tone as a
  combination of two perfect fifths, which are readily shown in all of the
  other spaces.}

%%% Local Variables: %%%
%%% mode: latex %%%
%%% TeX-master: "../diss" %%%
%%% End: %%%
