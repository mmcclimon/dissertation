% Chapter 2 notes
% vim:fdm=marker

%{{{ ii-V space

\fntext{tf-1}{%
  \cite{aebersold:1974}. The Aebersold play-along series is a staple of jazz
  pedagogues; most contain a selection of tunes, along with a CD of a rhythm
  section so that students can practice with a recording. The \tf\ volume is
  number three of well over 100, and includes the phrase “the most important
  musical sequence in jazz!” on the cover.}

\fntext{tf-2}{%
  Our diatonic analysis of “All the Things You Are” in the previous chapter
  notwithstanding, the ubiquity of \tfo\ progressions means that many jazz
  musicians are apt to hear  the progression as successions of \tf{}s, even in
  cases where a diatonic pattern may be present.}

\fntext{tf-2a}{%
  The triangle on the C chord in this figure indicates a major seventh. The
  triangle (instead of ``\h{maj7}'') is intended to save space and
  reduce clutter in the graphical representations.}

\fntext{tf-3}{%
  Lewin’s definition of a transformation network is in
  \headlesscite[196]{lewin:gmit}. For a relatively concise summary, see
  \cite[110--16]{rings:2011}.}

\fntext{tf-4}{%
  \headlesscite[Definitions 9.3.1 (196) and 1.3.1 (3)]{lewin:gmit}.}

\fntext{tf-5}{%
  \headlesscite[195--96 and throughout]{lewin:gmit}. See also
  \cite[6--8]{hook:2007}.}

\fntext{tf-5a}{%
  In fact, many jazz piano texts begin with ``three-note'' or ``shell''
  voicings, consisting only of chordal roots, thirds, and sevenths; see, for
  example \cite[17--22]{levine:piano}; and \cite[211--12]{berklee:harmony}.}

\fntext{tf-6}{%
  Here and throughout this chapter, pitch classes are represented as mod-12
  integers, with C $= 0$; all calculations are performed mod-12.}

\fntext{tf-6a}{%
  This figure represents what Joseph Straus calls “transformational voice
  leadings” in his study of atonal voice leading; \headlesscite{straus:2003}.}

\fntext{tf-7}{%
  Hook himself makes this clear, noting that for any two sets S and T it is
  possible to define a single-type transformation in the union set S $\cup$ T,
  though it is not always clear how a function defined on one set should be
  extended to cover both. He also notes that even when a single-type
  transformation is possible, ``the cross-type approach is often simpler and
  more natural,'' which certainly seems to be the case here.
  \headlesscite[5n8]{hook:2007}.}

\fntext{tf-8}{%
  It would be possible to form a \gis\ by considering all ordered triples as
  the group, as suggested above. While this is possible, defining an interval
  function in this group is much more difficult: such a function would need to
  account for the 36 ordered triples in \tf\ space (\ii, \V, and
  \I chords) as well as the many more (1692) that are not included in the
  space. Such a function is conceivable, but would not in any case reflect the
  musical realities \tf\ space is interested in portraying.}

\fntext{tf-9}{%
  \tf\ space is a directed graph, so in circumstances where the pitch-class
  distance metric is somehow not sufficient, we can instead rely on the
  standard way of measuring distance in a directed graph: by counting the
  number of edges in the shortest path between two chords. The distance from
  \ii of C to \V of \Eflat\ is then 4: ii to V in C (1 edge),
  then 3 $T_5$s to V of \Eflat.}


\fntext{tf-10}{%
  These changes are taken from \emph{The Real Book}, and reflect what is
  played on the \emph{Cornbread} recording.\nocite{morgan:cornbread} In this figure,
  the circle indicates the tonic, while the numbers on the labels indicate the
  order of transformations. Because ``Ceora'' uses only a part of the space,
  the circle of Figure \ref{tf:tf-circle-fifths} has been squared off here so
  that the labels are easier to read.}

\fntext{tf-11}{%
  We will return to the concept of tritone substitutes in the next section.}

\fntext{tf-12}{%
  For a study of many different musical spaces (and a defense of their
  use), see \cite[Chapter 1 and throughout]{hook:mst}.}

\fntext{tf-13}}}

%{{{ tritone subs

\fntext{tft-1}{%
  The tritone substitution has been discussed extensively in the literature,
  so we will not discuss it at any length here. See, for example,
  \cite{biamonte:2008}; \cite[11]{martin:1988}; and \cite[360--65]{tymoczko:2011}.}

\fntext{tft-2a}{%
  A similar diagram can be found in \cite[5]{pohlert:bh}, and can be seen
  implicitly in Figure 1-1 of \cite{martin:1988}.}

\fntext{tft-2b}{%
  Determining the structure of the underlying transformation graph of this
  network is straightforward, so I have not included a figure of it here.}

\fntext{tft-3}{%
  Though we are defining chords as ordered triples in this chapter, I have
  included the fifth in this description to highlight the relationship to the
  triadic \textsc{slide}, which maintains the root and fifth of a triad while
  changing the quality of the third.}

\fntext{tft-3a}{%
  The \textsc{slide} transformation was introduced by David Lewin
  (\headlesscite[178]{lewin:gmit}), but has since become a part of of the
  standard set of Neo-Riemannian transformations. \slideS
  is defined here only as a transformation from \I chords to
  \ii chords, but of course the triadic \textsc{slide} is an involution
  (two successive applications of \textsc{slide} to any triad will result in the
  same triad).}

\fntext{tft-4}{%
  The \slideS\ transformation can be found in a chromatic sequence in the
  second movement of the Fauré string quartet, mm.~36--39. Julian Hook
  analyzes this passage from a number of different mathematical perspectives
  in \headlesscite{hook:2013}.}

\fntext{tft-5}{%
  This progression is often known as the ``Bird Blues,'' though Mark Levine
  calls it the ``descending blues'' in \headlesscite[228]{levine:1995}. Like many
  sets of Parker changes, several different versions exist; the changes here
  represent a mediation of these sources. \emph{The Real Book} gives \h{Am7} (a
  vi chord) instead of \h{Fmaj7} in m.~11; Levine's \emph{Jazz Theory Book} gives
  \h{Db7} (a tritone substitute) instead of \h{G7} in the second half of m.~3;
  and the \emph{Charlie Parker Omnibook} omits both \h{D7} chords (the first in
  the second half of m.~7, the second in m.~11) and the tonic in m.~11 is an
  \h{F7}.\nocite{aebersold:omnibook} Most of these differences are minor, and over
  the course of a recorded performance the changes might vary among all of these
  versions. Other compositions that contain this progression include Parker's
  own ``Confirmation,'' Sonny Stitt's ``Jack Spratt,'' and Toots Thielemans's
  ``Bluesette.''}

\fntext{tft-6}{%
  The major-minor seventh chord as a stable chord is characteristic of the
  blues; see, for example, \cite[158--59 and throughout]{mcgowan:2011}. This
  fact is somewhat obscured in \tf space, since major-minor sevenths appear in
  the space only as \V chords; we will return to this limitation
  later in Section \ref{sec:other-kinds-tonic}.}

\fntext{tft-7}{%
  At least, impossible if we want to stay within the three sets we are
  studying in this chapter. Like the other transformations we have defined, EC
  is an admissible transformation on the set of all mod-12 ordered triples: if
  we begin with a \h{Dm7} chord, $(2, 5, 0)$, EC gives us the triple $(7, 10, 4)$,
  which of course is not a major, minor, or dominant seventh chord.}

\fntext{tft-8}}}

%{{{ Extensions

\fntext{tfe-1}{%
  The extensions used for the dominant chord in a minor \tf is quite flexible:
  Aebersold's \emph{II--V\tsup{7}--I Progression} gives the quality as \h{7s9},
  but Mark Levine usually gives the chord symbol simply as ``alt.'' Levine
  includes the minor \tf in the category of ``melodic minor scale harmony,'' and
  ``alt.'' is short for the ``altered scale'' (the seventh mode of melodic
  minor). The G altered scale is
  G--\Aflat--\Bflat--B\nat--C\sharp--\Eflat--F\nat--G, and is sometimes called
  the “diminished whole-tone” scale, since it begins as an octatonic scale and
  ends as a whole-tone scale, or the “super-locrian” scale, the locrian mode
  with a flatted fourth. This sound could be expressed with a number of
  different chord symbols---\h{G7(b9s9s11b13)} or \h{G7(b5s5b9s9)}, for
  example---so jazz musicians typically write “alt.” See
  \headlesscite[70--77]{levine:1995}. We will return to this equivalence
  between chords and scales in Chapter 4.}

\fntext{tfe-2}{%
  The name of the transformation \textsc{ResI} is inspired by Steven Rings's use of
  the transformation ``ResC'' in the first chapter of \headlesscite[25-27]{rings:2011}.}

\fntext{tfe-2a}{%
  In practice, \textsc{ResI} will almost always be a transformation that
  moves the root of a \V down a perfect fifth. In theory, however,
  there is no limitation on the definition of \textsc{ResI}. It is possible,
  for example, to construct a space where tritone substitutes are normative by
  defining \textsc{ResI} to be equal to \tft; in this case, the gray arrows in
  Figure \ref{tfe:generic-space} would represent the transformation
  \textsc{7th} $\bullet$ $T_6$.
}

\fntext{tfe-3}{%
  In fact, we could make similar statements for all of the sets developed in
  this chapter: \Smin would then become (speaking loosely) ``the set of
  minor-minor seventh chords acting as \sd2, \sd4, and \sd1 in some key.'' In
  most cases, however, this level of precision is unnecessary, since the quality
  of the chord uniquely identifies its function.}

\fntext{tfe-4}{%
  For more on diatonic systems generally, see \cite{cloughmyerson:1985}.}

\fntext{tfe-5}{%
  These transformations are all relatively parsimonious, and seem in some way
  related to the \slideS transformation introduced above. We will delay a
  discussion of these parsimonious aspects of these transformations more
  generally until the next chapter.}

\fntext{tfe-5a}{%
  This figure has been skewed somewhat to conserve space on the page. If it
  were drawn in a manner parallel with standard \tf space, a \ii chord
  would be directly below the \I chord with the same root. As is the case
  throughout this study, the particular visual representation chosen for a given
  space does not affect the abstract structure of the space itself. }

\fntext{tfe-6}{%
  ``Ceora'' is perhaps even more diatonic than this space implies, since \h{Cmaj7}
  and \h{Bbmaj7} never appear in the music, while \h{Dbmaj7} does. Thus the
  chord qualities strongly suggest \h{Ab} major: I and IV (\Aflat\ and \Dflat)
  both appear as major sevenths, while ii and iii (\Bflat\ and C) appear only
  as minor seventh chords (unstable \ii chords).}

\fntext{tfe-7}}}

%%% Local Variables: %%%
%%% mode: LaTeX %%%
%%% TeX-master: "../diss" %%%
%%% End: %%%
