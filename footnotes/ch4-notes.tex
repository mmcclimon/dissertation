%% Chapter 4 notes

\fntext{lcc-1}{%
  \cite{russell:lcc}; hereafter, simply \emph{LCC}. References to the book in
  this dissertation will be to this final edition (Russell died in 2009)
  unless otherwise noted. Though a full reception history of the \emph{Lydian
    Chromatic Concept} is beyond the scope of this project, it is worth noting
  that the later editions focus more heavily on the theory than the earlier
  editions, which were more practical in nature. In the original edition, the
  section on theoretical foundations appears after eight initial ``lessons'';
  in the 2001 edition, this material has been moved front and center, to
  Chapter 1.\nocite{russell:1959}}


\fntext{lcc-2}{%
  \emph{Encyclopedia of Popular Music}, s.v.~``George Russell,'' last modified
  July 4, 2006, \url{http://www.oxfordmusiconline.com/subscriber/article/epm/48476}.}

\fntext{lcc-3}{%
  \cite[602]{berendt:jazzbook}. Takemitsu's fascination with the
  \emph{Concept} is discussed at length in \cite{burt:2002}.}

\fntext{lcc-4}{%
  \cite{tymoczko:1997}; and \headlesscite[366n13]{tymoczko:2011}. My intent
  here is not to single out Tymoczko, but only to note that modern theorists
  who engage directly with Russell's ideas do not always mention his work. The
  most complete treatments of the \emph{Concept} in modern scholarship are
  Burt's previously-cited article on Takemitsu and Brett Clement's work on
  Frank Zappa (who, while influenced by jazz, is not part of the jazz
  mainstream that is the focus of this study); see \headlesscite{clement:2014}.}

\fntext{lcc-5}{%
  I am using the term ``scale'' here as Russell does: he always refers to the
  Lydian as a scale rather than a mode. Dmitri Tymoczko has argued against
  this usage, saying that there is a ``widespread tendency to elide the
  difference between scale and
  mode'' \parentext{\citetitle[366n14]{tymoczko:2011}}. To call the Lydian a mode
  would imply that it is simply a reordering of a major scale, though, and for
  Russell the two are fundamentally different objects.}

\fntext{lcc-6}{%
   \citetitle[3]{russell:lcc}.}

\fntext{lcc-7}{%
  \headlesscite[8--9]{russell:lcc}.}

\fntext{lcc-8}{%
  Though Russell attributes this preference to the Lydian scale, we could
  probably point to another reason: the second chord does not contain the
  dissonant minor ninth between F\nat\ and E, the third of the chord.}

\fntext{lcc-9}{%
  My own suspicion is that Russell's work is too new to be considered
  historically, but not new enough to be taken seriously as modern
  scholarship. I discuss this idea at length in \headlesscite{mcclimon:2015};
  see also \cite{adams:2013}.}

\fntext{lcc-10}{%
  Russell defines ``ingoing'' only in passing, saying that ``all music
  conceived within the equal tempered system maintains a closer (more
  \textsc{ingoing}) relationship to one tone than to all others, regardless of
  the music's style or genre'' (\citetitle[9]{russell:lcc}).}

\fntext{lcc-11}{%
  Russell's idiosyncratic names have mostly fallen out of use, since they are
  long and difficult to remember. It is easier for an improvising musician,
  for example, to recall the ``half--whole diminished scale'' than it is to
  remember the ``auxiliary diminished blues scale'' (named, incidentally, for
  the fact that it shares \flat\sd3, \nat\sd3, and \flat\sd7 with the blues
  scale).}

\fntext{lcc-12}{%
  Exactly what Russell means by ``seventh'' is also unclear; the dominant
  seventh is not represented in the nine-tone order, so it seems likely that
  he means the major seventh chords.}

\fntext{lcc-13}{%
  \citetitle[16]{russell:lcc}. This entire paragraph is a gloss on the material in
  pp.~12--17 of the \emph{Concept}.}

\fntext{lcc-14}{%
  The objection is perhaps obvious: if the ideas about Lydian tonal
  organization have fallen by the wayside, why bother trying to resuscitate
  them? As I hope to show in the following sections, Russell's systematic
  approach is useful as a means of formalizing what has since become implicit
  knowledge.}

\fntext{lcc-15}{%
  Dmitri Tymoczko makes the argument that something like chord--scale theory
  existed in the music of the Impressionists, and suggests that jazz musicians
  may have discovered it by listening to Debussy and
  Ravel \parentext{\citetitle[152 and 173]{tymoczko:1997}}. Given the initial
  reception of the \emph{Concept} as the first real theory of jazz, I am
  skeptical of this claim, and treat Russell's ideas as their first
  appearance. Certainly, though, the tenets of chord--scale theory can be
  applied to other musics: Russell himself examines passages of Bach,
  Beethoven, and Ravel, among others.}

\fntext{lcc-16}{%
  \citetitle[20--21]{russell:lcc}. In order to avoid confusion, we will use
  the term ``chord-scale'' rather than ``chordmode'' from this point on.
  Russell's interchangeable use of ``chord,'' ``chordmode,'' and ``mode''
  tends to confuse more than it helps, and ``chord-scale'' is the generally
  accepted term today.}

\fntext{lcc-17}{%
  This process is simplified somewhat by the inclusion of a foldout chart in
  the book (in the first published edition, it was referred to as the ``Lydian
  slide rule''). The chart is somewhat difficult to understand, though, and
  including it here would likely confuse matters.}

\fntext{lcc-18}{%
  This table summarizes the material in \citetitle[23--29]{russell:lcc}.}

\fntext{lcc-19}{%
  His treatment of Mode II is similarly problematic, as Lydian Mode II (a
  vertical scale) is distinct from the Major flat seventh scale (a horizontal
  one). }

\fntext{lcc-20}{%
  There are eight \abbrev{PMG}s rather than seven because Russell needs to
  account for the sharp fifth in the Lydian augmented scale. Other altered
  scale degrees (\flat{}II, \flat{}III) are seen simply as alterations and are
  not counted among the principal genres. Russell does not explain why, but
  the reason is probably related to the fact that the +V genre gives rise to
  an important class of chords (7+5), while the others do not.}

%%% Local Variables:
%%% mode: latex
%%% TeX-master: "../diss"
%%% End:
