%% Chapter 4 notes

% Lydian Chromatic Concept
\fntext{lcc-1}{%
  \cite{russell:lcc}; hereafter, simply \lcc. References to the book in
  this dissertation will be to this final edition (Russell died in 2009)
  unless otherwise noted. Though a full reception history of the \emph{Lydian
    Chromatic Concept} is beyond the scope of this project, it is worth noting
  that the later editions focus more heavily on the theory than the earlier
  editions, which were more practical in nature. In the original edition, the
  section on theoretical foundations appears after eight initial ``lessons'';
  in the 2001 edition, this material has been moved front and center, to
  Chapter 1.\nocite{russell:1959}}


\fntext{lcc-2}{%
  \emph{Encyclopedia of Popular Music}, s.v.~``George Russell,'' last modified
  July 4, 2006, \url{http://www.oxfordmusiconline.com/subscriber/article/epm/48476}.}

\fntext{lcc-3}{%
  \cite[602]{berendt:jazzbook}. Takemitsu's fascination with the
  \emph{Concept} is discussed at length in \cite{burt:2002}.}

\fntext{lcc-4}{%
  \cite{tymoczko:1997}; and \headlesscite[366n13]{tymoczko:2011}. My intent
  here is not to single out Tymoczko, but only to note that modern theorists
  who engage directly with Russell's ideas do not always mention his work. The
  most complete treatments of the \emph{Concept} in modern scholarship are
  Burt's previously-cited article on Takemitsu and Brett Clement's work on
  Frank Zappa (who, while influenced by jazz, is not part of the jazz
  mainstream that is the focus of this study); see \headlesscite{clement:2014}.}

\fntext{lcc-5}{%
  I am using the term ``scale'' here as Russell does: he always refers to the
  Lydian as a scale rather than a mode. Dmitri Tymoczko has argued against
  this usage, saying that there is a ``widespread tendency to elide the
  difference between scale and
  mode'' \parentext{\citetitle[366n14]{tymoczko:2011}}. To call the Lydian a mode
  would imply that it is simply a reordering of a major scale, though, and for
  Russell the two are fundamentally different objects.}

\fntext{lcc-8}{%
  Though Russell attributes this preference to the Lydian scale, we could
  probably point to another reason: the second chord does not contain the
  dissonant minor ninth between F\nat\ and E, the third of the chord.}

\fntext{lcc-9}{%
  My own suspicion is that Russell's work is too new to be considered
  historically, but not new enough to be taken seriously as modern
  scholarship. I discuss this idea at length in \headlesscite{mcclimon:2015};
  see also \cite{adams:2013}.}

\fntext{lcc-10}{%
  Russell defines ``ingoing'' only in passing, saying that ``all music
  conceived within the equal tempered system maintains a closer (more
  \textsc{ingoing}) relationship to one tone than to all others, regardless of
  the music's style or genre'' (\lcc 9).}

\fntext{lcc-11}{%
  Russell's idiosyncratic names have mostly fallen out of use, since they are
  long and difficult to remember. It is easier for an improvising musician,
  for example, to recall the ``half--whole diminished scale'' than it is to
  remember the ``auxiliary diminished blues scale'' (named, incidentally, for
  the fact that it shares \flat\sd3, \nat\sd3, and \flat\sd7 with the blues
  scale).}

\fntext{lcc-12}{%
  Exactly what Russell means by ``seventh'' is also unclear; the dominant
  seventh is not represented in the nine-tone order, so it seems likely that
  he means the major seventh chords.}

\fntext{lcc-13}{%
  The remainder of this paragraph is a gloss on the material in \lcc 12--17.}

\fntext{lcc-14}{%
  The objection is perhaps obvious: if the ideas about Lydian tonal
  organization have fallen by the wayside, why bother trying to resuscitate
  them? As I hope to show in the following sections, Russell's systematic
  approach is useful as a means of formalizing what has since become implicit
  knowledge.}

\fntext{lcc-15}{%
  Dmitri Tymoczko makes the argument that something like chord--scale theory
  existed in the music of the Impressionists, and suggests that jazz musicians
  may have discovered it by listening to Debussy and
  Ravel \parentext{\citetitle[152 and 173]{tymoczko:1997}}. Given the initial
  reception of the \emph{Concept} as the first real theory of jazz, I am
  skeptical of this claim, and treat Russell's ideas as their first
  appearance. Certainly, though, the tenets of chord--scale theory can be
  applied to other musics: Russell himself examines passages of Bach,
  Beethoven, and Ravel, among others.}

\fntext{lcc-16}{%
  In order to avoid confusion, we will generally respect the distinction
  between chord and scale from this point on. Where we have occasion to refer
  to the explicit chord--scale equivalence, we will use the term
  ``chord-scale'' rather than Russell's ``chordmode.'' Russell's
  interchangeable use of ``chord,'' ``chordmode,'' and ``mode'' tends to
  confuse more than it helps, and ``chord-scale'' is the generally accepted
  term in modern scholarship and pedagogy.}

\fntext{lcc-17}{%
  This process is simplified somewhat by the inclusion of a foldout chart in
  the book (in the first published edition, it was referred to as the ``Lydian
  slide rule''). The chart is somewhat difficult to understand, though, and
  including it here would likely confuse matters.}

\fntext{lcc-18}{%
  This table summarizes the material in \lcc 23--29.}

\fntext{lcc-19}{%
  His treatment of Mode II is similarly problematic, as Lydian Mode II (a
  vertical scale) is distinct from the Major flat seventh scale (a horizontal
  one). }

\fntext{lcc-20}{%
  There are eight \abbrev{PMG}s rather than seven because Russell needs to
  account for the sharp fifth in the Lydian augmented scale. Other altered
  scale degrees (\flat{}II, \flat{}III) are seen simply as alterations and are
  not counted among the principal genres. Russell does not explain why, but
  the reason is probably related to the fact that the +V genre gives rise to
  an important class of chords (7+5), while the others do not.}

\fntext{lcc-21}{%
  \cite[31--94]{levine:1995}. Dmitri Tymoczko summarizes several textbooks'
  views on chord--scale theory in \citetitle[174--79]{tymoczko:1997}. Other
  texts that discuss chord--scale theory at length include \cite{jaffe:1996};
  \cite{berklee:harmony}; \cite{aebersold:handbook}; and
  \cite{grafnettles:1997}. I am eliding the minor differences in this
  discussion, since they will not affect the transformational system in the
  next section. John Bishop provides a good overview of the distinctions in
  \headlesscite[77--81]{bishop:2012}.}

\fntext{lcc-22}{%
  Levine and others often describe only the first choice (Russell would say
  ``most ingoing'') for matching a scale with a chord. Russell's system of
  substituting other member scales sharing the same Lydian tonic is generally
  not present in the later texts. For jazz musicians the melodic minor scale
  refers only to its ascending form, which is generally played in both
  directions in jazz.}

% Chord-scale transformations

\fntext{cst-1}{%
  \cite[Chapter 4 and throughout]{tymoczko:2011}. This work is a culmination
  of much of his earlier work that incorporates scales; see, for example,
  \headlesscite{tymoczko:1997} and \headlesscite{tymoczko:2004}.}

\fntext{cst-2}{%
  Nor will we have need of the extensive literature on the abstract structure
  of scales in general. For an introduction, see \cite{cloughetal:1999}.}

\fntext{cst-3}{%
  Bishop's primary source is Graf and Nettles's \emph{Chord Scale Theory and
    Jazz Harmony}\nocite{grafnettles:1997}; Mulholland and Hojnacki's
  \citetitle{berklee:harmony} covers much of the same material but in a more
  modern way (neither book mentions Russell at all). The Berklee method
  systematizes much of Russell's method in a way suitable for teaching
  undergraduate jazz musicians. Every tone in a scale, for example, is either
  a chord tone, an avoid note, or a ``tension'': an upper extension that
  colors the basic sound of a chord (\emph{Berklee Book of Jazz Harmony},
  xi).}

\fntext{cst-4}{%
  These triadic approaches also tend towards music that is less clearly tonal,
  and were devised partly as a means of moving beyond the standard chord-scale
  approach we are examining here.}

\fntext{cst-5}{%
  Indeed, this lack of focus on common-tone connections in the pedagogical
  literature is the main reason for Love's ``Model of Common Tone
  Connections.''\nocite{love:2009}}

\fntext{cst-6}{%
  The shift in terminology from the Lydian ``scale'' to the Lydian
  ``collection'' here is deliberate, but not theoretically significant. In
  practice, Russell treats a scale as a collection: a group of notes from
  which to generate chord tones or improvisations. In this section we will be
  more interested in scales as collections, rather than (say) their function
  as ``musical rulers.''}

\fntext{cst-7}{%
  These names have been chosen to reflect common jazz usage, so scale 3 is the
  Lydian dominant scale rather than the acoustic scale.}

\fntext{cst-8}{%
  In fact, the blues scale is probably much more common than some of Russell's
  vertical scales (especially the Lydian diminished scale).}

\fntext{cst-9}{%
  \cite{hook:2008}. In this section we will not apply the full power of Hook's
  transformations, since they operate not only on key signatures themselves
  but on pitches (what he calls ``floating diatonic forms'').}

\fntext{cst-10}{%
  \Cite[142]{hook:2008}. We could also write $f_n$ as ${s_n}^{-1}$, and the
  entire set of signatures can be generated by $s_1$, adjusting for enharmonic
  equivalence as necessary.}

\fntext{cst-11}{%
  This is apparent in his description of the tone orders of the Lydian
  Chromatic scale, as reproduced in Figure~\ref{lcc:lc-tonal-gravity}. We
  could also define a \gis using these tone orders, but such a \gis would lose
  some distinctions between scales (the Lydian flat seventh and auxiliary
  augmented are both representatives of the 10-tone order).}

\fntext{cst-12}{%
  This limitation is not as significant, and in fact is the normal state of
  affairs for diatonic intervals, where, for example, intervals of both 3 and
  4 semitones are called ``thirds.''}

\fntext{cst-13}{%
  We might consider using $-7$ rather than $1$ in the second place here, to
  reflect the intuition that the move from the blues scale to the parent
  Lydian is an ingoing motion. This seems to be a valid judgment, but affects
  the group structure of \ivls ($-7$ is not a member of $\intZ_8$). We will
  return to this limitation in the next section.}

\fntext{cst-14}{%
  Because the diatonic modes are well-known, I have substituted ``diatonic''
  for ``Lydian'' in the first place here in the hopes that it will be less
  confusing. This way, we might refer to the first collection in Figure
  \ref{cst:gis-fiveone} as \scalepair{\nat}{Mixolydian}.}

\fntext{cst-15}{%
  The dominant chords that appear in mode +V are related by tritone to those
  in mode II. The fact that these tritone-related dominants appear along with
  the whole-tone and diminished scales is no accident: because these scales
  are symmetrical at the tritone, these scales are particularly effective when
  soloing over dominant seventh chords. Dmitri Tymoczko discusses this
  practice explicitly in \headlesscite[365--68]{tymoczko:2011}.}

\fntext{cst-16}{%
  It is reasonable to wonder why we did not collapse the Lydian augmented and
  Lydian \flat{}7 scales into a single mode, as we did with the horizontal
  major scale and the vertical acoustic scale. The aim in this section is to
  show that the two scales do in fact function differently, which justifies
  their presence as separate vertical scales.}

\fntext{cst-17}{%
  The diatonic collections are shown in flatwise order traveling clockwise.
  This corresponds to the way the circle of fifths (or fourths) is usually
  presented in jazz textbooks; see, for example, \cite[v]{coker:elements}.}

\fntext{cst-18}{%
  Many authors have commented on this limitation, which Lewin first observes
  in \citetitle[27]{lewin:gmit}. Dmitri Tymoczko is probably the most vocal in
  his opposition, and proposes incorporating a distance metric into the
  definition of what he calls a ``Lewinian interval
  system'' \parentext{\headlesscite[245--46]{tymoczko:2009}}; in other places,
  he has suggested that relaxing some of the restrictions on a \gis is
  ``anti-Lewinian'' \parentext{\headlesscite{tymoczko:2008}}. For more on this
  tension, see Rachel Wells Hall's review of \emph{GMIT} (\emph{Journal of the
    American Musicological Society} 62 no.~1 [Spring 2009]:
  205-22);\nocite{hall:2009} \cite[especially 185--86]{hook:2007gmit}; and
  \cite[19--20]{rings:2011}.}

\fntext{cst-19}{%
  Admittedly, the outgoing boundary is more permeable than the ingoing:
  we could conceive of a scale that is more dissonant than the blues scale
  (the total chromatic, perhaps) and incorporate it into the system, while for
  Russell the Lydian boundary is absolute.}

\fntext{cst-20}{%
  We will generally use this new form of the $R$ transformation, but in cases
  where there is a need to distinguish between this version and the version
  used just above, ``2-element $R$\,'' versus ``3-element $R$\,'' works nicely.}

\fntext{cst-21}{%
  In fact, the situation is even more complex, since a single chord symbol can
  support only a subset of the 96 chord-scales. Thus every chord symbol would
  contain a \emph{different} partial copy of the planetary model. Only the
  chords listed in Table \ref{cst:scale-table}, for example, would be able to
  show the scales from the \nat\ diatonic wedge.}

% Chord-Scale Analysis

\fntext{csa-1}{%
  Complete transcriptions for all solos analyzed in this chapter and the next
  can be found in Appendix B.}

\fntext{csa-2}{%
  This is to say nothing of the fact that perhaps the most important aspect of
  a jazz solo is that it is improvised; transcriptions must necessarily only
  focus on a single recorded performance.}

\fntext{csa-3}{%
  Thomas Owens (among others) refers to the bebop-inspired style as jazz's
  lingua franca \parentext{\headlesscite[4]{owens:1995}}. To avoid focusing on
  Parker and Coltrane is not to say that the performers examined here are not
  also great musicians; all of them (but especially Stitt and Henderson) are
  widely regarded among jazz musicians.}

\fntext{csa-4}{%
  \Cite[46]{coker:elements}. Coker's manual is primarily for student
  improvisers (hence the wording here), but is also analytically useful as a
  handbook of common improvisational devices.}

\fntext{csa-5}{%
  There are two staves in this figure because Kirk often played multiple
  saxophones simultaneously. Here the tenor saxophone (which sounds down an
  octave) is played with his left hand and the manzello---a modified soprano
  saxophone---is played with his right.}

\fntext{csa-6}{%
  The diagram here assumes \h{Ab7}, since it is more common to substitute
  dominant sevenths than minor sevenths. An analysis with \h{Abm7} would be
  similar, except that the second chord would be in the 6\flat\ collection.}

\fntext{csa-7}{%
  Recall from Table \ref{cst:scale-table} that a diminished scale played over
  a dominant chord is always the whole--half diminished scale. Conceptually,
  the parent scale of \h{C7} is \Bflat Lydian, so the scale in question is the
  second mode of the \Bflat whole--half diminished scale (which is of course the
  same as the C half--whole diminished scale).}

\fntext{csa-8}{%
  All of the networks here are what Lewin calls ``figural''
  networks \parentext{\headlesscite[45--53]{lewin:mft}}; since the $R$
  transformations do not form a semigroup, a ``formal'' network (showing all
  possible transformations) is not possible. Hall suggests in a
  footnote \parentext{\citetitle[213n16]{hall:2009}} that the logical
  structure of the group is unnecessary in figural networks; the networks of
  this figure support this claim.}

\fntext{csa-9}{%
  The pile-driver/gnat metaphor comes from Richard G.\ Swift's response to
  Lewin's first published article \parentext{\headlesscite{swift:1960}; quoted in
    \cite[162]{hook:2007gmit}}.}

\fntext{csa-10}{%
  Figures \ref{csa:kirk-tfo-progs}c is missing beats 2--4 of the third bar
  because this \tfo leads into the turnaround in Figure
  \ref{csa:first-chorus-turnaround}, which distracts from the topic at hand.}

\fntext{csa-11}{%
  Absent any additional information we will assume Russell's most ingoing
  scale for chords. Since there is no B (flat or natural) in any of the third
  bars of Figure \ref{csa:kirk-tfo-progs}, we thus assume F Lydian.}

\fntext{csa-12}{%
  The kind of metrical shifts discussed here are much simpler than those
  usually discussed in the jazz literature. The classic work on metrical
  dissonance in jazz is \cite{folio:1995}; Stefan Love provides an overview of
  other literature in this area in \headlesscite{love:2013}.}

\fntext{csa-13}{%
  In ``double-time'' passages, the soloist plays twice the speed of the
  prevailing note value, though the underlying tempo remains constant. Here
  there is a quarter-note pulse, and the soloists play predominantly eighth
  notes; the double-time passages use sixteenth notes instead. Matthew
  Voglewede has discussed the practice of double-time from a metrical standpoint
  in \headlesscite{voglewede:2013}.}

\fntext{csa-14}{%
  The dominant bebop scale is a Mixolydian scale with a chromatic note added
  between the chordal seventh and the root, so that when played in eighth notes
  the chord tones fall on the beat. David Baker (who played with George
  Russell early in his career) is usually credited with inventing the
  term; see \headlesscite{baker:1985}.}

\fntext{csa-15}{%
  In this figure and others following, I have omitted the chord symbol from
  the chord-scale triples to save space, since it is obvious from the
  transcription itself.}

\fntext{csa-16}{%
  By tallying up all of the appearances of a given chord--scale pairing, it
  would be possible to develop something like a snapshot of a performer's
  chord--scale choices. Doing so could perhaps bring some clarity to
  discussions of jazz style (as in, for example, David Baker's \emph{Giant of
    Jazz} series) by incorporating the chord-scale transformations developed
  here. While this is an interesting possibility, we will pursue it no further
  here.}

\fntext{csa-17}{%
  The only time it is not occurs in Stitt's first chorus ($3$A\tsub{1},
  mm.~67--68), where he plays a sustained D\nat\ through all four chords.}

%%% Local Variables:
%%% mode: latex
%%% TeX-master: "../diss"
%%% End:
