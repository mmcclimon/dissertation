%% Chapter 4 notes

\fntext{lcc-1}{%
  \cite{russell:lcc}. References to the book in this dissertation will be to
  this final edition (Russell died in 2009), with the caveat that some details
  have changed from earlier editions. The early editions of the \emph{Lydian
  Chromatic Concept} were more like pamphlets---comb-bound sheaves of copy
  paper---and have not found their way into many libraries. I am thankful to
  David Baker for loaning me his original edition for examination.}

\fntext{lcc-2}{%
  \emph{Encyclopedia of Popular Music}, s.v.~``George Russell,'' last modified
  July 4, 2006, \url{http://www.oxfordmusiconline.com/subscriber/article/epm/48476}.}

\fntext{lcc-3}{%
  \cite[602]{berendt:jazzbook}. Takemitsu's fascination with the
  \emph{Concept} is discussed at length in \cite{burt:2002}.}

\fntext{lcc-4}{%
  \cite{tymoczko:1997}; and \headlesscite[366n13]{tymoczko:2011}. My intent
  here is not to single out Tymoczko, but only to note that modern theorists
  who engage directly with Russell's ideas do not always mention his work. The
  most complete treatments of the \emph{Concept} in modern scholarship are
  Burt's previously-cited article on Takemitsu and Brett Clement's work on
  Frank Zappa (who, while influenced by jazz, is not part of the jazz
  mainstream that is the focus of this study); see \headlesscite{clement:2014}.}


%%% Local Variables:
%%% mode: latex
%%% TeX-master: "../diss"
%%% End:
