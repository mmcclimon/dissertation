% Chapter 5 notes

\fntext{rcg-1}{%
  ``I Got Rhythm'' was written in 1930 and first appeared in the musical
  \emph{Girl Crazy}. Because the phrase ``Rhythm changes'' has developed a
  life beyond its initial meaning, it is rendered throughout this chapter
  without quotes but with a capital ``R.'' The phrase is normally used as a
  noun, while adjectival uses drop the ``changes'' (as in ``Oleo is a Rhythm
  tune'').}

\fntext{rcg-2}{%
  The vast majority of Rhythm tunes are in the key of \Bflat. Those
  in Table \ref{rcg:rhythm-tunes} that are not are Hoyt Curtain's theme to the
  cartoon \emph{The Flintstones} and Bud Powell's ``Wail'' (both in \Eflat),
  along with Thad Jones's ``Tiptoe'' (\Aflat) and Thelonious Monk's ``52nd
  Street Theme'' (C).}

\fntext{rcg-3}{%
  Adding to this virtuosity is the fact that many recordings of rhythm
  changes are quite fast. Of the standard bebop recordings of tunes in Table
  \ref{rcg:rhythm-tunes}, Parker's recording of ``Moose the Mooche'' is 212
  bpm, Powell's ``Wail'' is 270 bpm, while the Parker/Gillespie
  ``Anthropology'' burns along at roughly 305 bpm.}

\fntext{rcg-4}{%
  The ``mix-and-match'' metaphor comes from \cite[149]{jaffe:1996}.}

\fntext{rcg-5}{%
  These six harmonizations do not begin to constitute a complete set of
  substitutions for these four bars. Jamey Aebersold gives 43 harmonizations
  of the Rhythm A section in \headlesscite[26]{aebersold:rhythm}. Robert
  Hodson, borrowing a linguistic metaphor from Noam Chomsky, describes the
  harmony of the tune in a generative fashion: the ``deep structure'' of
  the Rhythm changes consists of prolongations of \Bflat in the A sections,
  with the bridge consisting of a motion towards \Bflat; see
  \cite[62--65]{hodson:2007}.}

\fntext{rcg-6}{%
  I have not included another copy of the complete \tf space here, but one can
  be found in Figure \ref{tft:complete-space}
  (p.~\pageref{tft:complete-space}).}

\fntext{rcg-7}{%
  This \h{Eo7} is functionally ambiguous; it could also stand in for an
  \h{A7b9} as the dominant (or \h{C7b9} as part a backdoor progression) to the
  following \h{Dm7}. It is spelled as \h{Eo7} to produce a smooth bass line
  from \Eflat in the first half of the bar.}

\fntext{rcg-8}{%
  Given the modular nature of Rhythm tunes, the genre seems particularly ripe
  for a schema theory approach, following in the footsteps of
  \cite{gjerdingen:2007}. The two- and four-bar units here (turnarounds,
  dominant cycles) are not unlike the stock phrases used in the galant style,
  and jazz pedagogical materials like the Aebersold and Jaffe texts previously
  cited could easily serve as analogs to the 18th-century Italian
  \emph{partimenti} often used by schema theorists.}

\fntext{rcg-9}{%
  A lack of understanding about the recorded history of jazz is often seen as
  something of a social mistake, especially in the case of Rhythm changes.
  Mark Levine tells a story of playing a Rhythm tune with Sonny Stitt, when
  Stitt began to play the cycle progression of Figure \ref{rcg:rc-first-four}e
  over the A sections. Levine recounts: ``after a couple of choruses, glares
  from Sonny, and a growing sense of feeling smaller and smaller, I finally
  `strolled,' or stopped playing. After the set, I asked him what were the
  changes he was playing, and he growled `just listen, man.'{}'' This story is
  accompanied by a footnote that (based on the word of saxophonist Don Byas)
  attributes the cycle progression to pianist Art Tatum.
  \cite[242]{levine:1995}. \\ \hspace{2.55em}%
  Phil Ford notes that this kind of secret knowledge is fundamental to
  understanding jazz as a part of an emerging hip culture in the 1940s--50s,
  which is incidentally the time that Rhythm tunes began to proliferate. As he
  puts it, ``{}`knowing the score' is what practically defines the hipster:
  \emph{hip}, in its original meaning, means to be
  aware'' \parentext{\headlesscite[54]{ford:2002}}. We might instead read
  ``knowing the changes,'' since to really know Rhythm changes involves knowing
  about the wide variety of harmonic possibilities that defines the genre.}

\fntext{ran-a}{%
  Robert Hodson provides a similar analysis of the \emph{Criss-Cross}
  recording (though not from a transformational perspective) in
  \headlesscite[66--74]{hodson:2007}. His analysis focuses more strongly on
  the interactive elements of the performance than the harmonic ones.}

\fntext{ran-1}{%
  The second volume of the old \emph{Real Book} does give changes during the
  A sections, but they are somewhat inaccurate; in any case, it also includes the
  indication to ``solo over Rhythm changes.''}

\fntext{ran-2}{%
  Again, this is a common occurrence for Monk. During particularly good solos,
  he would rise from the piano and dance around the stage \parentext{an aspect
    of his performance on display throughout Charlotte Zwerin's documentary of
    Monk, \headlesscite{zwerin:monk}}. In the live recording here, he can
  occasionally be heard shouting words of encouragement to Griffin.}

\fntext{ran-3}{%
  % xxx citation about walking bass lines
  A detailed account of exactly how walking bass lines project harmony is
  beyond the scope of this project; for a good overview, see \cite{}.}

\fntext{ran-3a}{%
  There is much more to be said about the role of interaction in negotiating
  harmony in Rhythm tunes; we will return to this idea in more detail in the
  analysis of ``The Eternal Triangle'' in the next section.}

\fntext{ran-4}{%
  This difficulty is perhaps one of the reason that Steve Larson's book on the
  subject \parentext{\headlesscite{larson:2009}} focuses primarily on solo
  piano recordings of ``'Round Midnight,'' a piece in which the harmonies are
  well-defined---unlike Rhythm changes---and there are no other band members
  to muddy the waters. (Larson does include a live recording from Bill Evans
  and a partial transcription of a Bud Powell recording, each of which uses a
  piano trio, plus an ``ensemble'' recording from Evans's \emph{Conversations
    with Myself}, a multi-track recording in which Evans acts as all three
  members of the ensemble.) Garrett Michaelsen critiques Larson on this same
  point, and also suggests that Larson overemphasizes the importance of
  harmony in general
  \parentext{\headlesscite[9--10]{michaelsen:2013}}.}

\fntext{ran-5}{%
  The main interest in this particular passage is metric: Griffin superimposes
  a three-beat pattern over the quadruple meter starting in m.~341.}

\fntext{ran-6}{%
  ``Comping'' is what jazz musicians call the act of accompanying (or
  complementing) a soloist; see \cite[223--34]{levine:piano}.}

\fntext{ran-7}{%
  \emph{Thelonious in Action} is a live recording made at the end of an
  eight-week run at the Five Spot Cafe in New
  York \parentext{\cite[242--43]{kelley:2009}}. Ahmed Abdul-Malik was used to
  playing with Monk by the time of the recording, and catches the
  dominant-cycle A section almost immediately; Monk's strong left-hand
  entrance on F\sharp\ at m.~385 removes any doubt as to the progression that
  will follow.}

\fntext{et-1}{%
  % xxx find better source for this?
  See, for example, Stephen Cook's review on \emph{AllMusic}, where he notes
  that on ``{}`The Eternal Triangle,' in particular, Stitt and Rollins impress
  in their roles as tenor titans \ldots\, an embarrassment of solo riches
  comes tumbling out of both these men's horns'' \parentext{\cite{cook:2015}}.}

\fntext{et-2}{%
  % xxx side-slipping citation
  Jerry Coker says this.
}

\fntext{et-3}{%
  % xxx need citation
  ``Playing outside'' is one of Coker's categories of improvisational tools;
  see \cite{}. I suspect that the ``outside'' terminology is related to
  Russell's ``outgoing'' scale choices, but I have not been able to
  corroborate this suspicion.}

%%% Local Variables:
%%% mode: latex
%%% TeX-master: "../diss"
%%% End:
