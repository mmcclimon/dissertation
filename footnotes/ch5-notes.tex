% Chapter 5 notes

\fntext{rcg-1}{%
  ``I Got Rhythm'' was written in 1930 and first appeared in the musical
  \emph{Girl Crazy}. Because the phrase ``Rhythm changes'' has developed a
  life beyond its initial meaning, it is rendered throughout this chapter
  without quotes but with a capital ``R.'' The phrase is normally used as a
  noun, while adjectival uses drop the ``changes'' (as in ``Oleo is a Rhythm
  tune'').}

\fntext{rcg-2}{%
  The vast majority of Rhythm tunes are in the key of \Bflat. Those
  in Table \ref{rcg:rhythm-tunes} that are not are Hoyt Curtain's theme to the
  cartoon \emph{The Flintstones} and Bud Powell's ``Wail'' (both in \Eflat),
  along with Thad Jones's ``Tiptoe'' (\Aflat) and Thelonious Monk's ``52nd
  Street Theme'' (C).}

\fntext{rcg-3}{%
  Adding to this virtuosity is the fact that many recordings of rhythm
  changes are quite fast. Of the standard bebop recordings of tunes in Table
  \ref{rcg:rhythm-tunes}, Parker's recording of ``Moose the Mooche'' is 212
  bpm, Powell's ``Wail'' is 270 bpm, while the Parker/Gillespie
  ``Anthropology'' burns along at roughly 305 bpm.}

\fntext{rcg-4}{%
  The ``mix-and-match'' metaphor comes from \cite[149]{jaffe:1996}.}

\fntext{rcg-5}{%
  These six harmonizations do not begin to constitute a complete set of
  substitutions for these four bars. Jamey Aebersold gives 43 harmonizations
  of the Rhythm A section in \headlesscite[26]{aebersold:rhythm}.}

\fntext{rcg-6}{%
  I have not included another copy of the complete \tf space here, but one can
  be found in Figure \ref{tft:complete-space}
  (p.~\pageref{tft:complete-space}).}

\fntext{rcg-7}{%
  This \h{Eo7} is functionally ambiguous; it could also stand in for an
  \h{A7b9} as the dominant (or \h{C7b9} as part a backdoor progression) to the
  following \h{Dm7}. It is spelled as \h{Eo7} to produce a smooth bass line
  from \Eflat in the first half of the bar.}

\fntext{rcg-8}{%
  Given the modular nature of Rhythm tunes, the genre seems particularly ripe
  for a schema theory approach, following in the footsteps of
  \cite{gjerdingen:2007}. The two- and four-bar units here (turnarounds,
  dominant cycles) are not unlike the stock phrases used in the galant style,
  and jazz pedagogical materials like the Aebersold and Jaffe texts previously
  cited could easily serve as analogs to the 18th-century Italian
  \emph{partimenti} often used by schema theorists.}

%%% Local Variables:
%%% mode: latex
%%% TeX-master: "../diss"
%%% End:
