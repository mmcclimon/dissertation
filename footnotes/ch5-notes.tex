% Chapter 5 notes

\fntext{rcg-1}{%
  ``I Got Rhythm'' was written in 1930 and first appeared in the musical
  \emph{Girl Crazy}. Because the phrase ``Rhythm changes'' has developed a
  life beyond its initial meaning, it is rendered throughout this chapter
  without quotes but with a capital ``R.'' The phrase is normally used as a
  noun, while adjectival uses drop the ``changes'' (as in ``Oleo is a Rhythm
  tune'').}

\fntext{rcg-2}{%
  The vast majority of Rhythm tunes are in the key of \Bflat. Those
  in Table \ref{rcg:rhythm-tunes} that are not are Hoyt Curtain's theme to the
  cartoon \emph{The Flintstones} and Bud Powell's ``Wail'' (both in \Eflat),
  along with Thad Jones's ``Tiptoe'' (\Aflat).}

\fntext{rcg-3}{%
  Adding to this virtuosity is the fact that many recordings of rhythm
  changes are quite fast. Of the standard bebop recordings of tunes in Table
  \ref{rcg:rhythm-tunes}, Parker's recording of ``Moose the Mooche'' is 212
  bpm, Powell's ``Wail'' is 270 bpm, while the Parker/Gillespie
  ``Anthropology'' burns along at roughly 305 bpm.}

\fntext{rcg-4}{%
  The ``mix-and-match'' metaphor comes from \cite[149]{jaffe:1996}.}

%%% Local Variables:
%%% mode: latex
%%% TeX-master: "../diss"
%%% End:
