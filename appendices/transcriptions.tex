\doublespacing
\chapter{Transcriptions}
\addtocspace
\singlespacing

Transcription is a notoriously difficult problem in jazz. As Steve Larson
(among others) has noted, any transcription is also, in some sense, an
analysis.\footcite[2]{larson:2009} In the transcriptions that follow, I have
(in general) notated only the solo line, focusing primarily on the pitches and
rhythms. This means that many of the so-called ``secondary parameters'' are
absent from the notation: dynamics, intonation, phrasing, issues of timing,
etc. To that end, the transcriptions should be seen as companions to the
source recordings, not replacements for them.

In the transcriptions of the Rhythm tunes, no chord symbols are given.
One of the principal arguments of Chapter 5 is that the chord symbols are
somewhat fluid through the course of a performance; I have simply omitted the
chord symbols rather than assigning them based on the solo line. In the
transcriptions, the notes remain more-or-less uninterpreted (though the
notation of accidentals is a question of interpretation); examples in the text
refer to a given passage's harmonic context.

\vspace{\baselineskip}
\noindent There are a few other minor things to note:
\begin{compactitem}
  \item Formal designations follow Larson's convention: $2\mathrm{A}_3$ refers
    to the third A section of the second chorus, $4\mathrm{B}$ refers to the B
    section of the fourth chorus, and so on. Timestamps from the reference
    recording are given at the beginning of each chorus.
  \item Throughout, the tenor saxophone is notated at concert pitch but sounds
    down an octave, and soprano saxophone is notated at pitch.
  \item ``X'' noteheads indicate either ghosted notes (which are much lower in
    volume than surrounding notes) or alternate fingerings for the same pitch
    (a timbral effect often used in the high register by saxophonists).
\end{compactitem}

\section*{List of Transcriptions}

\begin{compactitem}[\hspace{1.5em}]
    \item ``Autumn Leaves'' -- Gene Ammons and Sonny Stitt \DottedPage{transcription:autumn-leaves}
    \item ``Blues for Alice'' -- Rahsaan Roland Kirk \DottedPage{transcription:blues-for-alice}
    \item ``The Eternal Triangle'' -- Sonny Stitt and Sonny Rollins \DottedPage{transcription:eternal-triangle}
    \item ``Isotope'' -- Joe Henderson \DottedPage{transcription:isotope}
    \item ``Lo-Joe'' (head only) -- George Coleman \DottedPage{transcription:lo-joe}
    \item ``Rhythm-a-ning'' -- Johnny Griffin and Thelonious Monk \DottedPage{transcription:rhythm-a-ning}
\end{compactitem}
\nocite{ammons:stitt,henderson:isotope,monk:action,kirk:freekings,coleman:amsterdam}

\newpage

\phantomsection
\addcontentsline{toc}{section}{“Autumn Leaves” -- Gene Ammons and Sonny Stitt}
\label{transcription:autumn-leaves}
\centerGraphic[page=1,clip,trim=0pt .6in 0pt 0pt,width=.95\textwidth]{%
  eps/transcriptions/autumn-leaves.pdf}
\blankfootnote{%
    The head changes for this recording are as given in Figure
    \ref{ds:autumn-leaves-complete}, but the solo changes shown here are
    slightly different. The third and fourth bars of each A section consistently
    substitute \h{Bm7}--\h{E7}--\h{Bbm7}--\h{Eb7} for \h{Bbmaj7}--\h{Ebmaj7}, and the
    progression \h{Gm7}--\h{C7}--\h{Fm7}--\h{Bb7} in the third and fourth bars
    of the C sections appear here as \h{Gm7}--\h{Em7b5}.}
\includepdf[pages=2-,pagecommand={},noautoscale]{eps/transcriptions/autumn-leaves.pdf}

\phantomsection
\addcontentsline{toc}{section}{“Blues for Alice” -- Rahsaan Roland Kirk}
\label{transcription:blues-for-alice}
\centerGraphic[page=1,clip,trim=0pt 1in 0pt 0pt]{eps/transcriptions/blues-for-alice.pdf}
\blankfootnote{%
    The album gives Kirk's name only as Roland Kirk; he added
    Rahsaan to his name in 1969. Kirk often played multiple instruments
    simultaneously; the transcription tries to make this clear, providing a
    separate staff for each instrument. The ``manzello'' is a modified soprano
    saxophone (both names are given here).}
\includepdf[pages=2-,pagecommand={},noautoscale]{eps/transcriptions/blues-for-alice.pdf}

\phantomsection
\addcontentsline{toc}{section}{“The Eternal Triangle” -- Sonny Stitt and Sonny
  Rollins}
\label{transcription:eternal-triangle}
\centerGraphic[page=1,clip,trim=0pt 1in 0pt 0pt]{eps/transcriptions/eternal-triangle.pdf}
\blankfootnote{%
  Sonny Rollins is indicated in the transcription with [S.R], and Sonny Stitt
  with [S.S.].}
\includepdf[pages=2-,pagecommand={},noautoscale]{eps/transcriptions/eternal-triangle.pdf}

\phantomsection
\addcontentsline{toc}{section}{“Isotope” -- Joe Henderson}
\label{transcription:isotope}
\includepdf[pages=-,pagecommand={},noautoscale]{eps/transcriptions/isotope.pdf}

\phantomsection
\addcontentsline{toc}{section}{“Lo-Joe” (head) -- George Coleman}
\label{transcription:lo-joe}
\centerGraphic[page=1,clip,trim=0pt 2in 0pt 0pt]{eps/transcriptions/lo-joe.pdf}
\blankfootnote{%
  This transcription contains only the head of the tune; the bass line sounds
  down an octave. In many cases, the piano voicings are difficult to hear in the
  recording. I have tried to keep the transcription as faithful as possible,
  but the disclaimer on using the transcription as a supplement to (not a
  replacement for) the recording applies more strongly than usual in this case.}
\includepdf[pages=2-,pagecommand={},noautoscale]{eps/transcriptions/lo-joe.pdf}

\phantomsection
\addcontentsline{toc}{section}{“Rhythm-a-ning” -- Johnny Griffin and Thelonious Monk}
\label{transcription:rhythm-a-ning}
\includepdf[pages=-,pagecommand={},noautoscale]{eps/transcriptions/rhythm-a-ning.pdf}

%%% Local Variables:
%%% mode: latex
%%% TeX-master: "../diss"
%%% End:
