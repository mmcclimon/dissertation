\doublespacing
\chapter{Transcriptions}
\singlespacing

% xxx Write something about transcription generally here
Transcription is a notoriously difficult problem in jazz. As Steve Larson
(among others) has noted, any transcription is also, in some sense, an
analysis.\footcite[2]{larson:2009} In the transcriptions that follow, I have
(in general) only notated the solo line, focusing primarily on the pitches and
rhythms. This means that many of the so-called ``secondary parameters'' are
absent from the notation: dynamics, intonation, phrasing, issues of timing,
etc. To that end, the transcriptions should be seen as companions to the
source recordings, not replacements for them.

In the transcriptions of the rhythm changes tunes, no chord symbols are given.
One of the principal arguments of Chapter 5 is that the chord symbols are
somewhat fluid through the course of a performance; I have simply omitted the
chord symbols rather than assigning one based on the solo line. In the
transcriptions, the notes remain more-or-less uninterpreted (though the
notation of accidentals is a question of interpretation); examples in the text
refer to a given passage's harmonic context. Formal designations follow
Larson's convention: $2\mathrm{A}_3$ refers to the third A section of the
second chorus, $4\mathrm{B}$ refers to the B section of the fourth chorus, and
so on. Timestamps from the reference recording are given at the beginning of
each chorus.

There are several things to note about Kirk's performance of ``Blues for
Alice.'' First, the album gives his name only as Roland Kirk; he added Rahsaan
to his name in 1969. Second, Kirk often played multiple instruments
simultaneously; the transcription tries to make this clear, providing a
separate staff for each instrument. He played (among other things) a modified
soprano saxophone known as a manzello; both names are given in the
transcription.

\section*{List of Transcriptions}

\begin{compactitem}
    \item ``Blues for Alice'' -- Rahsaan Roland Kirk
      (p.~\pageref{transcription:blues-for-alice})
    \item ``Isotope'' -- Joe Henderson (p.~\pageref{transcription:isotope})
    \item ``Rhythm-a-ning'' -- Johnny Griffin and Thelonious Monk
      (p.~\pageref{transcription:rhythm-a-ning})
\end{compactitem}
\nocite{henderson:isotope,monk:action,kirk:freekings}

\newpage

\phantomsection
\addcontentsline{toc}{section}{``Blues for Alice'' -- Rahsaan Roland Kirk}
\label{transcription:blues-for-alice}
\includepdf[pages=-,pagecommand={},noautoscale]{eps/transcriptions/blues-for-alice.pdf}

\phantomsection
\addcontentsline{toc}{section}{``Isotope'' -- Joe Henderson}
\label{transcription:isotope}
\includepdf[pages=-,pagecommand={},noautoscale]{eps/transcriptions/isotope.pdf}

\phantomsection
\addcontentsline{toc}{section}{``Rhythm-a-ning'' -- Johnny Griffin and Thelonious Monk}
\label{transcription:rhythm-a-ning}
\includepdf[pages=-,pagecommand={},noautoscale]{eps/transcriptions/rhythm-a-ning.pdf}

%%% Local Variables:
%%% mode: latex
%%% TeX-master: "../diss"
%%% End:
